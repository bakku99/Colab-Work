\SetPartOne

\section*{Inventor}

% [Image Inserted Manually]

A gnome sits hunched over a workbench in a room cluttered with every sort of tool, carefully drawing the final lines to an intricate rune. With a crackling hum of arcane energy, the completed rune flares with power, and she watches with a smile of pride as the golem comes to life and stands.

An ogre looms large over a dwarf, saliva sloping down its chin as it considers its hearty meal. Its growling anticipation turns into horrified roars of agony as the dwarf unloads terrifying blasts of thunderous power from a small, metal tube.

With a mighty bloom and a crackle of energy, a lone figure lands in the middle of the battlefield. As the smoke clears, a towering presence stands amidst the war-torn ruins clad in glowing, mechanical plate. The luminous visor swivels to inspect the wreckage. It raises a gauntlet, and lightning surges forth.

With a beckoning gesture, the elf summons a swirl of swords leaping from their sheaths and springing to life. The shocked bandits begin to edge away, breaking into a run as the swords exuberantly give chase.

Makers of magic-infused objects, inventors are defined by their innovative nature. Like wizards, they see magic as a complex system waiting to be decoded and controlled through a combination of thorough study and investigation. Inventors, though, focus on creating marvelous, new magical objects. Spells are often too ephemeral and temporary for their tastes. Instead, they seek to craft durable, useful items.

\subsection*{Cunning Creators}

Every inventor is defined by a specific craft. Inventors see mastering the basics of a craft as the first step towards true progress, the invention of new methods, and innovative approaches. Inventors are arcane engineers, students of invention and warfare who craft deadly firearms, ingenious gadgets, magical armor, and mechanical beings that they can augment with magic.

All inventors are united by their curious and creative nature. To an inventor, magic is an evolving art with a leading edge of discovery and mastery that pushes further ahead with each passing year.

Inventors value novelty and innovation. This penchant pushes them to seek a life of adventure.

A hidden ruin might hold a forgotten magic item or a beautifully crafted mirror perfect for magical enhancement. Inventors win respect and renown among their kind by uncovering new lore or inventing new methods of creation.

\subsection*{Boundless Imagination}

An inventor is not the sort to be limited by what already exists or by conventional wisdom. They combine all the tools at their disposal to go beyond—to discover new boundaries and promptly exceed them. Few things can stand in the path of an inventor with a mind to change the world.

Seeking to push beyond all and striving for the new is the unifying call of inventors, but how jealously they guard their secrets can vary greatly. Some push their knowledge upon the worlds, while others hoard it.

\subsection*{Rivals and Challenges}

Inventors often acquire rivals—other inventors racing to be the first to unlock the secrets of the world, scrambling to be the one after whom a new magical theory or scientific breakthrough will forever be named. Theft and acquisition of stolen ideas can often become an issue when too many inventors clash over a field of expertise.

Further, inventors quickly make enemies of those who would just as soon see the world unchanged.

They are often viewed as dangerous upstarts or radical troublemakers with their newfangled ways of doing things.

These challenges rarely daunt an inventor for long and most often simply fuel their continued drive to rise above. Inventors are rarely set in their ways, and the hated rival of last week may be a research associate the next.

\subsection*{Creating an Inventor}

When creating an inventor character, think about your character’s background and drive for adventure. Does the character have a rival? What drove your character down the path of becoming an inventor? Is it about curiosity and innovation, or is it about the power their inventions bring? Did your character learn from another inventor, or did they receive a vision or flash of inspiration to learn their craft?

Consider how your character interacts with the world and what they represent. Consult with your GM regarding guilds or societies your character might belong to.

\subsection*{Quick Build}

You can make an inventor quickly by following these suggestions. For the Gadgetsmith, Thundersmith, Cursesmith, or Relicsmith specializations, make your highest ability score Dexterity, followed by Intelligence; for the Golemsmith, Infusionsmith, Potionsmith, or Runesmith specializations, make Intelligence your highest, followed by Constitution or Dexterity; and for Warsmith or Fleshsmith specializations, choose either Strength or Intelligence as your highest, with Constitution as your second-highest ability score. Second, choose a background that emphasizes your connection to a guild or society.

\section*{The Inventor}

\begin{longtable}{p{2.5cm}\toprule
|p{2.5cm}|p{2.5cm}|p{2.5cm}|p{2.5cm}|p{2.5cm}|p{2.5cm}|p{2.5cm}|p{2.5cm}|p{2.5cm}|p{2.5cm}|p{2.5cm}|p{2.5cm}|}
\midrule
\multicolumn{6}{c}{The Inventor} & \multicolumn{6}{c}{—Spell Slots per Spell Level—} \\
\midrule
Level & Proficiency Bonus & \multicolumn{4}{c}{Features} & Spells Known & 1st & 2nd & 3rd & 4th & 5th & Upgrades \\
\midrule
1st & +2 & \multicolumn{4}{c}{Inventor Specialization, Magic Item Analysis} & — & — & — & — & — & — & — \\
\midrule
2nd & +2 & \multicolumn{4}{c}{Spellcasting, Arcane Retrofit, Tool Expertise} & 3 & 2 & — & — & — & — & — \\
\midrule
3rd & +2 & \multicolumn{4}{c}{Inventor Specialization feature, Specialization Upgrade} & 4 & 3 & — & — & — & — & 1 \\
\midrule
4th & +2 & \multicolumn{4}{c}{Ability Score Improvement} & 4 & 3 & — & — & — & — & 1 \\
\midrule
5th & +3 & \multicolumn{4}{c}{Inventor Specialization Feature} & 5 & 4 & 2 & — & — & — & 2 \\
\midrule
6th & +3 & \multicolumn{4}{c}{Arcane Reconstruction, Cross-Disciplinary Knowledge} & 5 & 4 & 2 & — & — & — & 2 \\
\midrule
7th & +3 & \multicolumn{4}{c}{Wondrous Item Proficiency} & 6 & 4 & 3 & — & — & — & 3 \\
\midrule
8th & +3 & \multicolumn{4}{c}{Ability Score Improvement} & 6 & 4 & 3 & — & — & — & 3 \\
\midrule
9th & +4 & \multicolumn{4}{c}{—} & 7 & 4 & 3 & 2 & — & — & 4 \\
\midrule
10th & +4 & \multicolumn{4}{c}{Improved Magical Crafting, Wondrous Item Recharge} & 7 & 4 & 3 & 2 & — & — & 4 \\
\midrule
11th & +4 & \multicolumn{4}{c}{Study of Magic} & 8 & 4 & 3 & 3 & — & — & 5 \\
\midrule
12th & +4 & \multicolumn{4}{c}{Ability Score Improvement} & 8 & 4 & 3 & 3 & — & — & 5 \\
\midrule
13th & +5 & \multicolumn{4}{c}{—} & 9 & 4 & 3 & 3 & 1 & — & 6 \\
\midrule
14th & +5 & \multicolumn{4}{c}{Inventor Specialization Feature} & 9 & 4 & 3 & 3 & 1 & — & 6 \\
\midrule
15th & +5 & \multicolumn{4}{c}{—} & 10 & 4 & 3 & 3 & 2 & — & 7 \\
\midrule
16th & +5 & \multicolumn{4}{c}{Ability Score Improvement} & 10 & 4 & 3 & 3 & 2 & — & 7 \\
\midrule
17th & +6 & \multicolumn{4}{c}{—} & 11 & 4 & 3 & 3 & 2 & 1 & 8 \\
\midrule
18th & +6 & \multicolumn{4}{c}{Wondrous Item Mastery} & 11 & 4 & 3 & 3 & 3 & 1 & 8 \\
\midrule
19th & +6 & \multicolumn{4}{c}{Ability Score Improvement} & 12 & 4 & 3 & 3 & 3 & 2 & 9 \\
\midrule
20th & +6 & \multicolumn{4}{c}{Peerless Inventor} & 12 & 4 & 3 & 3 & 3 & 2 & 9 \\
\midrule
\end{longtable}

\subsection*{Class Features}

As an Inventor you gain the following class features.

\subsection*{Hit Points}

Hit Dice: 1d8 per inventor level 
Hit Points at 1st Level: 8 + your Constitution modifier 
Hit Points at Higher Levels: 1d8 (or 5) + your Constitution modifier per inventor level after 1st

\subsection*{Proficiencies}

Armor: Light armor, medium armor 
Weapons: Simple weapons, hand crossbows, heavy crossbows 
Tools: Thieves’ tools, one other tool of your choice 
Saving Throws: Constitution, Intelligence 
Skills: Choose three from Arcana, Deception, History, Investigation, Medicine, Nature, Religion, Sleight of Hand

\subsection*{Equipment}

You start with the following equipment, in addition to the equipment granted by your background:

\begin{itemize}
  \item (a) a light crossbow and case of 20 bolts or (b) any two simple weapons
  \item (a) scale mail, (b) leather armor, or (c) chain mail (if proficient)
  \item Thieves’ tools and a dungeoneer’s pack
\end{itemize}

If you forgo this starting equipment, as well as the items offered by your background, you start with 4d4 × 10 gp to buy your equipment.

\subsection*{Inventor Specialization}

At 1st level, you focus your craft on a particular specialization: Cursesmith, Fleshsmith, Gadgetsmith, Golemsmith, Infusionsmith, Potionsmith, Relicsmith, Thundersmith, or Warsmith, each of which are detailed at the end of the class description. Your choice grants you features at 1st level and again at 3rd, 5th, and 14th level.

\subsection*{Magic Item Analysis}

Starting at 1st level, your understanding of magic items allows you to analyze and understand their secrets. You know the detect magic and identify spells, and you can cast them as a ritual, requiring no material components.

\subsection*{Tool Expertise}

Starting at 2nd level, your proficiency bonus is doubled for any ability check you make that uses any of the tool proficiencies you gained from this class.

\subsection*{Arcane Retrofit}

Additionally at 2nd level, you gain the ability to retrofit magical gear. During a long rest you can perform a ritual using any +1, +2, or +3 weapon—excluding artifacts and sentient weapons— to transfer the magic into an inventor weapon (a weapon created by a subclass feature or upgrade). For example, you can turn a +1 longsword and your Impact Gauntlet into a +1 Impact Gauntlet . This includes weapon-like Upgrades that make attack and damage rolls, such as a Warsmith’s Force Blast.

You can’t transfer any properties from a magical weapon besides a bonus to attack and damage rolls, and the original weapon is destroyed in the process.

Additionally, you can convert a set of armor with a magical bonus to AC to a lighter armor type.

\begin{minipage}{0.48\textwidth}
Additional Retrofits

At the GM’s option, this feature can be expanded to do similar tasks. For example, you may be allowed to turn a +1 heavy crossbow into a +1 light crossbow, or the GM may choose to allow this feature to work with other magical armor or weapons, such as converting adamantine chain mail into an adamantine chain shirt.
\end{minipage}\hfill
\begin{minipage}{0.48\textwidth}

\end{minipage}

\subsection*{Spellcasting}

\begin{minipage}{0.48\textwidth}
At 2nd level, as part of your study of magic you gain the ability to cast spells. The spells you learn are limited in scope, primarily concerned with modifying creatures and objects or creating items.

Spell Slots

The Inventor table shows how many spell slots you have to cast your inventor spells of 1st level and higher. To cast one of these spells, you must expend a slot of the spell’s level or higher. You regain all expended spell slots when you finish a long rest.

For example, if you know the 1st-level spell longstrider and have a 1st-level and a 2nd-level spell slot available, you can cast longstrider using either slot.

Spells Known of 1st Level and Higher

You know three 1st-level spells of your choice from the inventor spell list.

The Spells Known column of the Inventor table shows when you learn more inventor spells of your choice. Each of these spells must be of a level for which you have spell slots. For instance, when you reach 5th level in this class, you can learn one new spell of 1st or 2nd level.
\end{minipage}\hfill
\begin{minipage}{0.48\textwidth}
Inventor’s Spellcasting

An inventor is fundamentally someone that understands and regularly interacts with magic, leading to the ability to apply this knowledge as a spellcaster, but how you want to approach your inventor’s spellcasting is up to you.

Feel free to explore other approaches in coordination with your GM. From a functional standpoint, the only requirement is that you have a defined material component or focus for a spell that requires it.
 Consider for some spells perhaps using your artisan’s tools as your focus, or perhaps a specific item you’ve made is your material component for a spell. Perhaps instead of a “component pouch” that simply contains all the material components you could need for your spells, you have an “inventor’s tool belt” that contains the various spellcasting assistance contraptions you’ve made to cast your spells. For mechanical purposes, the only thing that matters is that it functions in the same way and is used consistently.
\end{minipage}

Additionally, when you gain a level in this class, you can choose one of the inventor spells you know and replace it with another spell from the inventor spell list, which also must be of a level for which you have spell slots.

Spell save DC = 8 + your proficiency bonus + your Intelligence modifier

Spell attack modifier = your proficiency bonus + your Intelligence modifier

Spellcasting Focus

You can use an arcane focus as a spellcasting focus for your inventor spells.

\subsection*{Specialization Upgrade}

At 3rd level, choose an upgrade from the list at the end of your specialization, gaining the benefits listed in its description.

You select an additional upgrade at 5th, 7th, 9th, 11th, 13th, 15th, 17th, and 19th level. You can’t select an upgrade more than once, unless the upgrade’s description says otherwise. Whenever you gain a level in this class, you can exchange one of your existing upgrades with another upgrade of the same level requirement as the replaced upgrade.

Whenever an existing upgrade is exchanged for a replacement upgrade (either by a subclass feature or when you gain a level), the new upgrade is selected as if you’re the level you were when you originally gained that upgrade. For example, if you replace your Stormforged Weapon and reselect all of your upgrades as a 5th-level inventor, you could select one 3rd-level upgrade and one 5th-level upgrade, or two 3rd-level upgrades, but you would not be able to select two 5th-level upgrades.

\begin{minipage}{0.48\textwidth}
Customizing Inventor Upgrades

The upgrades for each Inventor Specialization are presented in a list at the end of the subclass, but invariably there will always be ideas for upgrades not included in that list. At the heart of an inventor beats an unrelenting drive for creativity, after all! Feel free to consult your GM for potential custom upgrades.
\end{minipage}\hfill
\begin{minipage}{0.48\textwidth}

\end{minipage}

The following generic upgrades are available to all subclasses, and can be selected in place of any subclass upgrade:

Shield Proficiency Unrestricted Generic Upgrade 
You gain proficiency with shields.

Tool Proficiency Unrestricted Generic Upgrade 
You gain proficiency with a tool of your choice.

\subsection*{Ability Score Improvement}

When you reach 4th level, and again at 8th, 12th, 16th, and 19th level, you can increase one ability score of your choice by 2, or you can increase two ability scores of your choice by 1. As normal, you can’t increase an ability score above 20 using this feature.

\subsection*{Arcane Reconstruction}

At 6th level, you have mastered the knowledge of using magic to repair things. You learn the mending cantrip, and you can cast it at will. Additionally, you learn the cure wounds spell. If you already know cure wounds, you learn a different spell of your choice from the inventor spell list. Constructs targeted by your cure wounds can regain hit points as normal.

\subsection*{Cross-Disciplinary Knowledge}

At 6th level, you can expand on your knowledge across fields. You can craft one of the following: any of the Gadgetsmith’s Unrestricted Upgrades, an Infusionsmith’s Animated Weapon, Blasting Rod, or Infused Weapon, a Potionsmith’s Alchemical Reagent Pouch and Alchemical Fire or Alchemical Acid, or a Thundersmith’s Stormforged Weapon.

If you select a Stormforged Weapon, you gain proficiency with that weapon and knowledge of how to make ammunition for it (if required). You can’t apply Infused Weapon to another weapon granted by this class.

If this crafted item is lost or destroyed, you can remake it following the rules for remaking it in its respective subclass.

\subsection*{Wondrous Item Proficiency}

At 7th level, your familiarity with the workings of magical items means that you ignore all class requirements on the use of magic items. Additionally, you can now attune to up to four magic items at once.

\subsection*{Improved Magical Crafting}

\begin{minipage}{0.48\textwidth}
At 10th level, your experience in creating your own wondrous inventions makes you more adept at crafting a magic item than normal spellcasters. Creating a non-consumable magic item takes you half the time it would normally take.

Additionally, you can make 1 hour of progress toward crafting a magic item, potion, or scroll during a long rest.

\subsubsection*{Wondrous Item Recharge}

Starting at 10th level, you can recharge a magic item that has charges or per rest uses, as long as those charges or uses can only be used to cast spells. To restore charges or uses, you perform a ritual that takes 1 minute and expend a spell slot of a level equal to or greater than the spell slot level of a spell cast by the item.
\end{minipage}\hfill
\begin{minipage}{0.48\textwidth}
Interaction with the Crafting System

If you are using the Crafting System presented later in the book, this reduces the time a crafting check takes to 1 hour, allowing you to make two crafting checks during a long rest, or eight during a work day.
\end{minipage}

The number of charges or uses restored to the item is equal to the number of charges or uses required to cast that spell using the item once.

\subsection*{Study of Magic}

At 11th level, your proficiency in the workings of magic has become so great you can cast the detect magic and identify spells at will without expending a spell slot.

Additionally, you have advantage on Intelligence (Arcana) checks made to understand the workings of magical traps, effects, or runes.

\subsection*{Wondrous Item Mastery}

Starting at 18th level, you can use a magic item that would normally take an action as a bonus action instead. In addition, you can now attune to up to five magic items at once.

\subsection*{Peerless Inventor}

At 20th level, your mind is always thinking of new options and clever solutions. At the end of a short or long rest, you can create a temporary version of an Upgrade from your subclass that you meet the prerequisites for but don’t already have. This upgrade must have a level requirement of 11th level or lower, and it lasts until you finish a short or long rest, at which time you can select a temporary Upgrade with this feature again.

\begin{minipage}{0.48\textwidth}
Magical or Mechanical?

One of the core debates that arises around inventors is if they are a “Magical Engineer” or what a magical engineer would even mean. Do they tinker with mechanics and gears or with magic and runes? The answer is, of course, whatever suits your world, game, and the vision of the player.
\end{minipage}\hfill
\begin{minipage}{0.48\textwidth}

\end{minipage}

\begin{itemize}
  \item Class: Inventor
\end{itemize}

\section*{Gadgetsmith}

% [Image Inserted Manually]

A gadgetsmith is an inventor whose curiosity and inventive genius have run rampant. While other inventors may spend their whole career perfecting a single-minded pursuit, a gadgetsmith believes that quantity is at least as good as quality.

Quick footed and even quicker witted, gadgetsmiths are never caught without another trick up their sleeve—their minds always jumping ahead to solve the next problem with a clever contraption.

A gadgetsmith can come from any walk of life, but they usually exemplify curiosity and a distaste for the suppression of knowledge or technology, usually favoring freedom to experiment and leaning toward more chaotic behaviors.

\subsection*{Gadgetsmith’s Proficiency}

When you choose this specialization at 1st level, you gain proficiency with nets, rapiers, whips, and tinker’s tools.

\subsection*{Essential Tools}

At 1st level, you’ve mastered the creation of the essential reusable tools for surviving the battlefield as a gadgeteer. You have the following items:

\begin{itemize}
  \item Gadgetsmith Weapon. Choose one of the following from the Upgrade section: Impact Gauntlets, Lightning Baton, Repeating Hand Crossbow, Ricocheting Weapon, or Shock Generator. You gain the chosen upgrade, and it doesn’t count against your total number of upgrades.
  \item Grappling Hook. You can forgo any attack you make as part of the Attack action to throw your grappling hook at a target within 20 feet of you. The target can be a creature, object, or surface. If the target is Small or smaller, make a grapple check against it, pulling it to you and grappling it on a success. If the target is Medium or larger, you can instead pull yourself to it (no check required); this doesn’t grapple the target. Opportunity attacks provoked by this movement are made with disadvantage.
\end{itemize}

\begin{minipage}{0.48\textwidth}
\begin{itemize}
  \item Smoke Bomb. As an action, you can use a smoke bomb to cast the fog cloud spell centered on yourself, without expending a spell slot and requiring no components. The spell lasts for a number of rounds equal to your Intelligence modifier and doesn’t require your concentration. When you cast fog cloud in this way, you can choose for it to have a radius of 5, 10, 15, or 20 feet.
\end{itemize}
\end{minipage}\hfill
\begin{minipage}{0.48\textwidth}
Unlimited Smoke Bombs?

If you or your GM are concerned about the unlimited use of Smoke Bombs, consider limiting them to a number of uses per short rest equal to your proficiency bonus. This shouldn’t mechanically impact their usage, but the reason they don’t have a limit is to encourage players to feel free to use them in non-combat situations. With limits—even ones that are not usually reached—players tend to treat their Smoke Bomb as an exclusively combat-oriented ability.
\end{minipage}

\subsection*{Additional Upgrade}

At 3rd level, you’ve mastered the essential tools and have begun to tinker with ways to expand your arsenal. The number of upgrades you can have, based on your class level, increases by one. It increases by one again at 5th level (for a total of two more than shown in the Inventor class table).

\subsection*{Recycle Gadgets}

Starting at 3rd level, you can disassemble your gadgets and create different ones during a long rest. When you finish the rest, you can replace any upgrade you have with a new one.

The chosen upgrade must still be one that is valid for the level at which you gained the replaced upgrade. For example, at 9th level, you can only have one upgrade that has a 9th-level prerequisite.

Additionally, if one of your gadgets is destroyed, you can use this feature to recreate it over the course of a long rest, which requires materials worth 20 gp.

\subsection*{Extra Attack}

Beginning at 5th level, you can attack twice, instead of once, whenever you take the Attack action on your turn.

\subsection*{Combat Gadgets}

Beginning at 14th level, you can forgo any attack you make as part of the Attack action to use a gadget that normally requires an action to use.

\section*{Gadgetsmith Upgrades}

\subsection*{Unrestricted Upgrades}

Airburst Mine 
You create a mechanical device capable of producing a devastating blast. You can use this device to cast shatter or thunderburst mine* without expending a spell slot.

Once used, this gadget can’t be used again until you finish a short or long rest.

Belt of Adjusting Size 
You create a belt with a creature size dial on it. While you are wearing the belt, you can use an action to cast the enlarge/ reduce spell on yourself without expending a spell slot.

Once used, this gadget can’t be used again until you finish a short or long rest.

Element Eater 
You create a device capable of absorbing incoming elemental damage. You can activate this device as a reaction when you take acid, cold, fire, lightning or thunder damage. The device grants you resistance to the triggering damage type until the start of your next turn. After triggering it, the first time you hit with a melee attack on your next turn, the target takes an extra 1d6 elemental damage of the triggering type.

Once use, this gadget can’t be used again until you finish a short or long rest.

Enhanced Grappling Hook 
You enhance the grappling hook gained from the Essential Tools feature, increasing its range to 30 feet. Additionally, when pulling yourself to a Large or larger target, the enhanced power of the grappling hook allows you to drag one Medium or smaller creature with you. The creature must be willing or grappled by you, and it must be within 5 feet of you.

Fire Spitter 
You create a gadget that can create a quick blast of fire. As an action, you can use the gadget to cast burning hands as a 2nd-level spell without expending a spell slot.

Once used, this gadget can’t be used again until you finish a short or long rest.

Flashbang 
You create a high-luminosity discharge device. As an action, you can use the device and target a point within 30 feet of you. Any creature within 20 feet of that point must succeed on a Dexterity saving throw against your spell save DC or be blinded until the end of its next turn.

Once used, this gadget can’t be used again until you finish a short or long rest.

Gliding Cloak 
You make a cloak that allows you to glide while falling. When you fall more than 10 feet and aren’t incapacitated, you can spread the cloak to reduce your rate of descent to 30 feet per round, taking no falling damage when you land. While falling in this manner under normal gravity, you can move up to 2 feet horizontally for every 1 foot you descend.

Gravity Switch 
You build a switch that turns off gravity. You can use it to cast fall* without expending a spell slot.

Once used, this gadget can’t be used again until you finish a short or long rest.

Impact Gauntlet
 You create a magic weapon capable of amplifying the impact of your blows. You have proficiency with this weapon, and it has the finesse and light properties, as well as the special property (described below). It deals 1d6 bludgeoning damage on a hit. You can select this upgrade up to two times, making a separate weapon each time.

Special. When you make an attack with this weapon, you can choose to forgo adding your proficiency bonus to the attack roll. If your attack hits, you can add double your proficiency bonus to the damage roll.

\begin{minipage}{0.48\textwidth}
Variant: Power Fist

Your GM may allow you to take the Power Fist upgrade from the Warsmith specialization in place of an Impact Gauntlet, which is similar but deals 1d8 damage and lacks the finesse property. For the purpose of upgrades, Impact Gauntlets and Power Fists should be considered interchangeable.
\end{minipage}\hfill
\begin{minipage}{0.48\textwidth}

\end{minipage}

Jumper Cable (Prerequisite: Shock Generator) 
Once per turn when you deal lightning damage with a cantrip from your Shock Generator, you can add your Intelligence modifier to the damage dealt.

Additionally, you can use your Shock Generator as an action to make a DC 10 Intelligence (Medicine) check, attempting to stabilize a dying creature within your reach. If you succeed on the check, the creature regains 1 hit point, becoming conscious again, and gains a number of temporary hit points equal to your inventor level. When you stabilize a creature in this way, it gains one level of exhaustion.

Jumping Boots 
You modify your boots with arcane boosters. While wearing these boots, you are under the effects of the jump spell.

Lightning Baton 
You create a magic weapon that channels lightning. You have proficiency with this weapon, and it has the finesse and light properties. It deals 1d4 bludgeoning damage plus 1d4 lightning damage on a hit. When you roll a 20 on an attack roll made with this weapon, the target must succeed a Constitution saving throw against your spell save DC or become stunned until the start of your next turn. You can choose this upgrade up to two times, making a separate weapon each time.

Mechanical Arm 
You create a mechanical arm, giving you an extra hand. This mechanical arm only functions while it is mounted on gear you are wearing, but can be operated mentally without the need for your hands. This mechanical arm can serve any function a normal hand could, such as holding things, making attacks, and interacting with the environment, but it doesn’t grant you any additional actions.

Mechanical Familiar 
You create the blueprint for a small, mechanical creature. At the end of a long rest, you can choose to create a familiar based on it, casting the find familiar spell without expending a spell slot. When cast in this way, the familiar is a construct. It stays active until you deactivate it or it is destroyed.

In either case, you can choose to reactivate it at the end of a long rest.

Net Launcher 
You build a device capable of delivering nets to their targets more effectively. While you have this upgrade, nets have a normal range of 20 feet and a long range of 60 feet for you.

Quick Essential Gadget 
You modify your essential gear for quickened use. You can use your Grappling Hook or Smoke Bomb as a bonus action.

Once you use either of them in this way, you can’t use either of them in this way again until you finish a short or long rest.

Repeating Hand Crossbow 
You build an improved hand crossbow. You have proficiency with this weapon, which has a normal range of 30 feet and a long range of 120 feet, and it has the ammunition and light properties, as well as the special property (described below). It deals 1d6 piercing damage on a hit.

\begin{itemize}
  \item Special. This weapon doesn’t require a free hand to load, as it has a built-in loading mechanism. When you make an attack with this weapon as part of the Attack action, and you don’t have disadvantage on the attack roll, you can choose to make the attack roll with disadvantage to make one additional attack with this weapon as a bonus action. The attack roll for this additional attack is also made with disadvantage.
\end{itemize}

Ricocheting Weapon
 You create a thrown melee weapon enchanted or engineered to ricochet off its targets and return to your hand. You have proficiency with this weapon, which has a normal range of 30 feet and a long range of 90 feet, and it has the finesse and thrown properties, as well as the special property (described below). It deals 1d8 bludgeoning, piercing, or slashing damage on a hit (your choice when you choose this upgrade). You can Arcane Retrofit this weapon.

\begin{itemize}
  \item Special. When this weapon is thrown, you can target two creatures within 10 feet of each other, making a separate attack roll against each target; the damage dealt by the attack is halved for each target hit after the first. In addition, the weapon flies back to your hand immediately after the attack.
\end{itemize}

Shock Generator 
You create a device capable of generating potent shocks. You can use this device to cast shocking grasp . When you cast shocking grasp in this way, you can use either your Dexterity or Intelligence modifier for the attack roll.

Shocking Hook (Prerequisite: Shock Generator) 
You can integrate your Shock Generator and your Grappling Hook. Immediately after using your Grappling Hook to pull a creature to you or to pull yourself to a creature, you can use your bonus action to cast shocking grasp from your Shock Generator targeting that creature.

Sight Lenses 
You create a set of lenses that allow you to see through darkness and obscurement, which you can integrate into a pair of goggles, glasses, or other vision assistance. While using these lenses, you can see through fog, mist, smoke, clouds, and nonmagical darkness out to a range of 15 feet.

Smoke Cloak 
You create a cloak that causes you to blend in with smoke. When you are lightly or heavily obscured by smoke at the start of your turn, you are invisible until your turn ends, or until you cast a spell, make an attack, or deal damage to a creature.

Steelweave Nets 
You thread your nets with metal reinforcement, making them tougher and more conductive. The net has an AC of 15 and resistance to slashing damage. Whenever a creature restrained by the net takes lightning damage, it takes an extra 1d6 lightning damage.
Only you have proficiency with your Steelweave Nets. You can have a number of these nets up to your Intelligence modifier at a time (a minimum of one), and you can replenish your supply of them over the course of a long rest by forging normal nets into Steelweave Nets.

Striding Boots 
You modify your boots with amplified striding speed. While wearing these boots, you are under the effects of the longstrider spell.

Trick Shots 
At the end of a short or long rest, you can add one of the following effects to a piece of ammunition or a thrown weapon, turning it into a trick shot; the effect is expended when you use the trick shot:

\begin{itemize}
  \item Arcane Trick. You can imbue a cantrip or 1st-level spell you know into the shot, casting it on impact where the shot lands. The target of the attack is also the spell’s target.
  \item Bouncing. You can attack a target within range that has total cover from you if you know the target’s location, bouncing the shot off a surface. The target has the benefit of half cover for the attack, instead of total cover.
  \item Ricochet. If you hit a target with the shot, you can make another attack roll against a different target within 10 feet of the first as part of the same attack.
  \item Smoke Shot. You can use the shot to deploy a Smoke Bomb on impact where the shot lands.
  \item Special Tip. The shot deals one of the following damage types of your choice: bludgeoning, piercing, slashing, acid, cold, fire, or lightning.
\end{itemize}

You can select this upgrade multiple times, allowing you to make multiple trick shots per rest. You can have a maximum number of trick shots equal to the number of times you have selected this upgrade.

\subsection*{5th-Level Upgrades}

Antimagical Shackle 
You create a set of antimagical shackles. As an action when you are adjacent to a creature, you can attempt to shackle the creature to yourself or a nearby object using these shackles. Make a Dexterity (Sleight of Hand) check contested by the target’s Strength (Athletics) or Dexterity (Acrobatics) check (target’s choice). A creature that is immune to being grappled or restrained automatically succeeds on this check. If you succeed, the target is shackled to you or the nearby object, and it can only move if it’s able to move what it’s shackled to.

Additionally, while shackled in this way, the target can’t teleport, travel to another plane of existence, alter its form (such as by the polymorph spell), or dematerialize. As an action, the target can make a Strength saving throw against your spell save DC, breaking the shackles on a success; otherwise, the shackles last until you remove them.

Binding Rope 
You create a rope that is capable of animating and binding a creature. As an action, choose a creature within 30 feet of you. The target must make a Dexterity saving throw against your spell save DC or become restrained until the end of your next turn. If you are currently grappling the target, it makes the Dexterity saving throw with disadvantage. The rope can only restrain one target at a time.

Crossbow Spider 
You modify a crossbow to be able to aim and fire remotely. As an action, you can deploy a Tiny construct called a crossbow spider (see the construct’s statistics below). Once deployed, you can use a bonus action to cause the construct to move up to its speed and make a crossbow attack from its location. The construct becomes inactive after 1 minute, or after it has been used to make 10 attacks.

Once you have activated the construct, you can’t activate it again until you finish a long rest, unless you expend a spell slot of 1st level or higher to activate it again. If the construct is destroyed, you can’t activate it again until you repair it or recreate it during a long rest.

The construct can’t take any actions besides the ones you direct it to take with your bonus action.

\begin{minipage}{0.48\textwidth}
\subsubsection*{Crossbow Spider}

\begin{tabularx}{\textwidth}\toprule
{}XXXXX}
\midrule
\multicolumn{6}{c}{Tiny construct, unaligned} \\
\midrule
\multicolumn{6}{c}{Armor Class 10 
Hit Points 5 
Speed 15 ft., climb 15 ft.} \\
\midrule
STR & DEX & CON & INT & WIS & CHA \\
\midrule
2 (−4) & 10 (+0) & 10 (+0) & 1 (−5) & 1 (−5) & 1 (−5) \\
\midrule
\multicolumn{6}{c}{Damage Immunities poison 
Condition Immunities charmed, exhaustion, frightened, paralyzed, petrified, poisoned 
Senses passive Perception 5 
Languages — 
Proficiency Bonus equals your bonus} \\
\midrule
\multicolumn{6}{c}{Actions} \\
\midrule
\multicolumn{6}{c}{Crossbow. Ranged Weapon Attack: your spell attack modifier to hit, range 30/120 ft., one target you can see. Hit: 1d6 + your Intelligence modifier piercing damage.} \\
\midrule
\end{tabularx}

\begin{itemize}
  \item NPC: Crossbow Spider
\end{itemize}
\end{minipage}\hfill
\begin{minipage}{0.48\textwidth}

\end{minipage}

Explosive Gauntlet (Prerequisite: Impact Gauntlet) 
Your gauntlets can exert massive kinetic force when striking. When you make an attack with your Impact Gauntlet, you can choose to be pushed 10 feet in the opposite direction of your attack. Alternatively, you can use a bonus action to attempt to push the target after the attack. The target must succeed on a Strength saving throw or be pushed 10 feet away from you. This movement doesn’t provoke opportunity attacks.

You can use the movement effect of this gauntlet even if you’re not attacking a target, allowing you to push yourself in any direction (including upwards).

Smoky Images 
Immediately after using a Smoke Bomb, you can use a bonus action to cast mirror image without expending a spell slot.

Once you cast the spell in this way, you can’t do so again until you finish a short or long rest.

Vanishing Trick 
Immediately after using a Smoke Bomb, you can cast misty step without expending a spell slot.

Once you cast the spell in this way, you can’t do so again until you finish a short or long rest.

\subsection*{9th-Level Upgrades}

Arcane Nullifier 
You make a device that nullifies the arcane through means you assure everyone else you understand. As an action, you can use this device to cast dispel magic without expending a spell slot.

Once used, this gadget can’t be used again until you finish a short or long rest.

Phase Trinket 
You create a magical gadget that manipulates ethereal magic. As an action, you can use this device to cast blink or dimension door without expending a spell slot.

Once used, this gadget can’t be used again until you finish a long rest.

Stinking Gas 
You make a more potent compound for your Smoke Bombs. When you use a Smoke Bomb, you can choose to cast stinking cloud , instead of fog cloud , following the same rules.

Once you cast stinking cloud in this way, you can’t do so again until you finish a short or long rest.

Stopwatch Trinket 
You create a magical stopwatch that manipulates time magic. As an action, you can use the stopwatch to cast haste or slow without expending a spell slot.

Once used, this gadget can’t be used again until you finish a long rest.

\subsection*{11th-Level Upgrades}

Flying Gadget 
You build a device that allows you to fly, such as deployable artificial wings. You can activate this device as a bonus action, or as a reaction when you fall. Whatever form the device takes, when it’s activated, you gain a flying speed of 30 feet, which lasts until you choose to deactivate it (no action required).

Lightning Generator (Prerequisite: Shock Generator) 
You upgrade your Shock Generator with additional lightning capabilities. You can overload your Shock Generator to cast lightning bolt without expending a spell slot.

Once you cast lightning bolt in this way, you can’t do so again until you finish a short or long rest.

Additionally, once per turn when you deal lightning damage with a spell, you can add your Intelligence modifier to the spell’s damage roll.

Truesight Lenses (Prerequisite: Sight Lenses) 
You upgrade and fine-tune your Sight Lenses, granting you truesight out to a range of 15 feet while using them.

Useful Universal Key 
You create a Universal Key to obstacles, transmuting them into not-obstacles. As an action, you can apply this key to a surface to cast knock or passwall on it without expending a spell slot.

Once used, this gadget can’t be used again until you finish a long rest.

\subsection*{15th-Level Upgrades}

Bee Swarm Rockets 
You design a type of tiny, firecracker-like device, which can release rockets in large numbers. You have a maximum number of rockets equal to your inventor level. As an action, you can release between 1 and 10 rockets. Each rocket targets a point you can see within 40 feet of you. Each creature within 10 feet of one of these points must make a Dexterity saving throw against your spell save DC.

Creatures in the area of effect of multiple rockets make a separate saving throw for each rocket. For each failed save, a creature takes 2d6 + 1 fire damage, or half as much damage on a successful one.

You can replenish your supply of rockets to your maximum over the course of a long rest.

Bracers of Empowerment 
You create bracers that can empower you. You can use these bracers to cast martial transformation* without expending a spell slot.

Once used, this gadget can’t be used again until you finish a long rest.

Dimensional Toolbox 
You build a toolbox, filling it with the many ideas you’ve had and discarded, with the magical power of making those ideas reality when you need them most. As an action, you can reach into the toolbox and withdraw an Unrestricted Upgrade of your choice (one with no level requirement) from the Gadgetsmith list. Gadgets withdrawn from the toolbox are fleeting, and disappear after 1 minute.

Once you have withdrawn an upgrade from the toolbox, you can’t withdraw another upgrade from it until you finish a long rest.

Disintegration Ray 
You create a Disintegration Ray. You can use this device to cast disintegrate without expending a spell slot.

Once used, this gadget can’t be used again until you finish a long rest.

Nexus Hive 
You create a floating metal sphere, that when activated spews a storm of metal shards in all directions around it. As an action you can deploy a small construct (see the construct’s statistics below), throwing it up to 30 feet where it will hover in place. Once deployed you can use a bonus action to move the Nexus Hive up to 20 ft. The construct becomes inactive after 1 minute.
If the construct is destroyed, you can’t activate it again until you repair it or recreate it during a long rest.

The construct can’t take any actions besides the ones you direct it to take with your bonus action.

\begin{minipage}{0.48\textwidth}
\subsubsection*{Nexus Hive}

\begin{tabularx}{\textwidth}\toprule
{}XXXXX}
\midrule
\multicolumn{6}{c}{Small construct, unaligned} \\
\midrule
\multicolumn{6}{c}{Armor Class 15 (natural armor) 
Hit Points 30 
Speed 0 ft., fly 20 ft. (hover)} \\
\midrule
STR & DEX & CON & INT & WIS & CHA \\
\midrule
2 (–4) & 10 (+0) & 12 (+1) & 1 (−5) & 1 (−5) & 1 (−5) \\
\midrule
\multicolumn{6}{c}{Damage Immunities poison, psychic 
Condition Immunities charmed, exhausted, frightened, paralyzed, petrified, poisoned 
Senses passive Perception 5 
Languages — 
Proficiency Bonus equals your bonus} \\
\midrule
\multicolumn{6}{c}{Traits} \\
\midrule
\multicolumn{6}{c}{Swarm. The nexus hive sprays metal shards in a 10-foot radius around itself. A creature takes 4d4 piercing damage when it enters this area for the first time on a turn or starts its turn there.} \\
\midrule
\multicolumn{6}{c}{Explosive Vent. When reduced to 0 hit points, the nexus hive will violently vent energy. Creatures within 10 feet of the nexus hive must succeed on a Dexterity saving throw against your spell save DC. A creature takes 4d6 fire damage on a failed save, or half as much damage on a successful one.} \\
\midrule
\end{tabularx}

\begin{itemize}
  \item NPC: Nexus Hive
\end{itemize}
\end{minipage}\hfill
\begin{minipage}{0.48\textwidth}

\end{minipage}

\begin{itemize}
  \item Inventor Specialization: Gadgetsmith
  \item Feature: Gadgetsmith Upgrades
\end{itemize}

\section*{Golemsmith}

% [Image Inserted Manually]

Golemsmiths are inventors who have committed themselves to creating a true work of artifice, forging a golem. With this painstaking life ambition, they plan and design meticulously, even if, in practice, sometimes compromises on materials must be made.

Why a golemsmith embarks on the quest to forge this artificial construct of life can vary. For many it is the pure pursuit of forging the perfect creation; while for others, it is simply so they do not have to carry around their loot, or to have a loyal companion to count on at all times.

A golemsmith is rarely chaotic, as they are people of great care and discipline—those that are not would not have succeeded where they have. However, some have been set on their path by such events that might drive them to interact chaotically with society as a whole.

\subsection*{Golemsmith's Proficiency}

When you choose this specialization at 1st level, you gain proficiency with smith’s tools and tinker’s tools.

\subsection*{Mechanical Golem}

Starting at 1st level, you forge a mechanical golem to carry out your orders and protect you. The golem is under your control and understands the languages you speak, but it can’t speak.

The golem obeys your commands as best it can. On your turn, you can verbally command it where to move (no action required by you), and you can use your action to command it to take an action. Additionally, whenever it would be able to take a reaction, you must use your reaction to command it to take that reaction. The golem acts on your commands during your turn. If you issue no commands to your golem, it takes no actions.

\begin{minipage}{0.48\textwidth}

\end{minipage}\hfill
\begin{minipage}{0.48\textwidth}
The base statistics of your golem are as follows:
\end{minipage}

\begin{minipage}{0.48\textwidth}
Your golem’s proficiency bonus increases when yours does. If the golem is killed, it can be returned to life by normal means, such as the revivify spell, or you can repair it during a long rest if you have access to its body. At the end of the long rest, it returns to life with half its hit points (rounded down).

If the golem is beyond recovery, you can recreate it exactly as it was, with four days of work (8 hours each day) and 100 gp of raw materials.

Over the course of a short rest, you can restore hit points to your golem equal to your Intelligence modifier + your inventor level, or you can fully repair it to during a long rest, causing it to regain all its hit points. Any spells you cast that cause the target to regain hit points ignore any restrictions against constructs for your golem.

\subsubsection*{Golem Chassis}

When you create your golem, you can choose one of the chassis below, adding the prefix to its type. For example, if you select Ironwrought, your golem becomes an Ironwrought Golem and gains the corresponding modifications to its base statistics.

Launcher

Your golem becomes akin to a mobile turret, taking the frame of a ballista or other launching device. Its Dexterity becomes 16 (+3), its speed becomes 25 feet, and it has the following natural weapon attack, which uses its Dexterity for the attack and damage rolls:

\begin{itemize}
  \item Shoot. Ranged Weapon Attack: +5 to hit, range 60/240 ft., one target. Hit: 1d10 + 3 piercing damage.
\end{itemize}

\begin{itemize}
  \item NPC: Launcher Golem
\end{itemize}
\end{minipage}\hfill
\begin{minipage}{0.48\textwidth}
\subsubsection*{Golem}

\begin{tabularx}{\textwidth}\toprule
{}XXXXX}
\midrule
\multicolumn{6}{c}{Medium construct, unaligned} \\
\midrule
\multicolumn{6}{c}{Armor Class 14 (natural armor)
 Hit Points (5 + (the golem's Constitution modifier +5) * your inventor level)
 Speed 30 ft.} \\
\midrule
STR & DEX & CON & INT & WIS & CHA \\
\midrule
14 (+2) & 12 (+1) & 12 (+1) & 4 (−3) & 5 (−3) & 1 (−5) \\
\midrule
\multicolumn{6}{c}{Damage Immunities poison, psychic
 Condition Immunities charmed, exhaustion, frightened, poisoned
 Senses passive Perception
 Languages understands the languages of its creator but can’t speak} \\
\midrule
\multicolumn{6}{c}{Traits} \\
\midrule
\multicolumn{6}{c}{Bound. The golem is magically bound to its creator. As long as the creator and it are on the same plane of existence, the creator can telepathically call the golem to travel to it, and the golem knows the distance and direction to its creator.} \\
\midrule
\end{tabularx}

\begin{itemize}
  \item NPC: Mechanical Golem
\end{itemize}
\end{minipage}

\begin{minipage}{0.48\textwidth}
Quadrupedal

Your golem takes on a quadrupedal design. Larger and sturdier, it is more suitable to launch into the fray or carry its creator. Its size becomes Large, its base AC becomes 16, and its speed becomes 35 feet. In addition, its Strength and Constitution become 16 (+3), and it has the following natural weapon attack, which uses its Strength for the attack and damage rolls:

\begin{itemize}
  \item Bite. Melee Weapon Attack: +5 to hit, reach 5 ft., one target. Hit: 1d10 + 3 piercing damage.
\end{itemize}

\begin{itemize}
  \item NPC: Quadrupedal Golem
\end{itemize}

Specialized

Your golem defies all expectations, its design fueled by your own rampant creativity. A little less robust and stable, it is far more extensible to your visionary plans. Your golem starts with the basic golem statistics, but you can select two free upgrades for it that don’t count against your upgrade total. In addition, it has the following natural weapon attack, which uses its Strength for the attack and damage rolls:

\begin{itemize}
  \item Slam. Melee Weapon Attack: +4 to hit, reach 5 ft., one target. Hit: 1d8 + 2 bludgeoning damage.
\end{itemize}

\begin{itemize}
  \item NPC: Specialized Golem
\end{itemize}

Specialized

Your golem defies all expectations, its design fueled by your own rampant creativity. A little less robust and stable, it is far more extensible to your visionary plans. Your golem starts with the basic golem statistics, but you can select two free upgrades for it that don’t count against your upgrade total. In addition, it has the following natural weapon attack, which uses its Strength for the attack and damage rolls:

\begin{itemize}
  \item Slam. Melee Weapon Attack: +4 to hit, reach 5 ft., one target. Hit: 1d8 + 2 bludgeoning damage.
\end{itemize}

\begin{itemize}
  \item NPC: Specialized Golem
\end{itemize}
\end{minipage}\hfill
\begin{minipage}{0.48\textwidth}
\subsubsection*{Optional Golem Chassis: Flesh Golem}

When a golemsmith is selecting their golem, some disturbed minds think “but what if I want it be a fleshy monstrosity against sanity?” Well, as usual, the inventor is here for you:

Flesh

Your golem is roughly humanoid, a lumbering terror of stitched flesh with an odd smell. Its Strength becomes 16 (+3), it gains proficiency with shields, simple weapons, and martial weapons, and it has the following natural weapon attack, which uses its Strength for the attack and damage rolls:

\begin{itemize}
  \item Slam. Melee Weapon Attack: +5 to hit, reach 5 ft., one target. Hit: 1d4 + 3 bludgeoning damage.
\end{itemize}

Additional Changes

\begin{itemize}
  \item Replace your smith’s tools proficiency with leatherworker’s tools proficiency.
  \item The upgrades are the same, but with “...but from some gross, fleshy organ.”
  \item At your GM’s discretion, you can apply Fleshsmith upgrades to your golem. This has the potential to cause balance issues, so assume the GM will say no unless it’s not too powerful and you can explain how it’s not.
\end{itemize}

\begin{itemize}
  \item NPC: Flesh Golem
\end{itemize}
\end{minipage}

Ironwrought

Your golem is roughly humanoid and comes with the robust flexibility and options that this form provides. Its Strength becomes 16 (+3), it gains proficiency with shields, simple weapons, and martial weapons, and it has the following natural weapon attack, which uses its Strength for the attack and damage rolls:

\begin{itemize}
  \item Slam. Melee Weapon Attack: +5 to hit, reach 5 ft., one target. Hit: 1d4 + 3 bludgeoning damage.
\end{itemize}

\begin{itemize}
  \item NPC: Ironwrought Golem
\end{itemize}

Winged

Your golem is modeled off a flying creature. Smaller and more lightweight than most golems, it is kept aloft by intricate wings. Its size becomes Small, it gains a flying speed of 30 feet, and it has the following natural weapon attack, which uses its Strength for the attack and damage rolls:

\begin{itemize}
  \item Talon. Melee Weapon Attack: +4 to hit, reach 5ft., one target. Hit: 1d8 + 2 slashing damage.
\end{itemize}

\begin{itemize}
  \item NPC: Winged Golem
\end{itemize}

\subsection*{Intelligent Oversight}

Starting at 3rd level, you can take the Help action as a bonus action when assisting your golem.

Additionally, when you take the Help action to aid an ally (including your golem) in attacking a creature, the target of that attack can be up to 30 feet away from you, rather than within 5 feet of you, if your ally can see or hear you.

\subsection*{Autonomous Action}

Starting at 5th level, you no longer need to spend your action or reaction to command your golem to use its action or reaction, and you can issue commands to it mentally while it is within 60 feet of you.

If the golem isn’t commanded to take any action, it will take the Dodge action in combat.

\subsection*{Magical Nature}

Additionally at 5th level, your golem’s natural weapons count as magical for the purpose of overcoming resistance and immunity to nonmagical attacks and damage.

\subsection*{Perfected Design}

Starting at 14th level, your golem can add your Intelligence modifier to all of its attack rolls, ability checks, and saving throws.

\section*{Golemsmith Upgrades}

\subsection*{Unrestricted Upgrades}

Arcane Barrage Armament 
You install a mounted armament on your golem, taking whatever form is most appropriate, which is charged with arcane power. As an action, the golem can cast magic missile through the armament, without expending a spell slot and requiring no components.
Once the golem casts the spell in this way, it can’t do so again until you finish a short or long rest.

When you reach 5th level in this class, the spell is cast as a 2nd-level spell, and it increases again when you reach 11th level (as a 3rd-level spell) and 17th level (as a 4th-level spell).

Arcane Resonance 
You craft a magical essence connector and install it into your golem’s core, allowing you and the golem to share certain magical effects. You can make any spell you cast that targets only you also target your golem.

Defender Protocol 
You build a protocol into your golem, instructing it to defend its master. The golem gains the Protection Fighting Style.

Ether Heart 
You install a magical ether heart into your golem, because that seems like a good idea. Your golem gains 2 charges. It can expend 1 charge to cast any 1st-level spell you know, and it regains all expended charges when you finish a long rest.

Flamethrower Armament 
You install an armament to your golem, taking whatever form is most appropriate, which is capable of producing powerful flames. As an action, the golem can cast burning hands through the armament, without expending a spell slot and requiring no components. The spell save DC is equal to your spell save DC.

Once the golem casts the spell in this way, it can’t do so again until you finish a short or long rest.

When you reach 5th level in this class, the spell is cast as a 2nd-level spell, and it increases again when you reach 11th level (as a 3rd-level spell) and 17th level (as a 4th-level spell).

Fine-Tuned Dexterity 
You craft improved gears and joints for your golem. Your golem’s Dexterity score increases by 2, increasing its ability to perform tasks that require fine motor skills. If this upgrade increases the golem’s Dexterity to 16 or higher, it gains proficiency with thieves’ tools; if this upgrade increases its Dexterity to 18, it gains proficiency in the Stealth skill.

You can select this upgrade multiple times, but you can’t increase the golem’s Dexterity above 18 using this upgrade.

Grappling Appendages 
You install an additional pair of grappling appendages to your golem, which take a form of your choice. For the purpose of grappling, these appendages count as two additional free hands for your golem. Additionally, as long as your golem isn’t using these appendages to grapple, it has a climbing speed equal to its walking speed.

Heavy Armor Plating 
You can incorporate a suit of heavy armor into your golem, allowing it to calculate its armor class as if it was wearing that armor. Your golem is proficient with the incorporated armor, and it has disadvantage on Dexterity (Stealth) checks until the armor is removed. You can remove the incorporated armor or incorporate a new suit of armor at any time, but it takes twice as long as it would normally take to don or doff the armor.

Magical Essence 
You infuse a fragment of magical essence into your golem, allowing it to attune to one magic item. This follows all normal attunement rules.

Structural Constitution 
You have reinforced your golem with layers of protection and redundant systems. Your golem’s Constitution score increases by 2, increasing its stability and durability. If this upgrade increases the golem’s Constitution to 16 or higher, it gains proficiency in Constitution saving throws; if this upgrade increases its Constitution to 18, it has advantage on death saving throws.

You can select this upgrade multiple times, but you can’t increase the golem’s Constitution above 18 using this upgrade.

Systematic Strength 
You build an improved frame and power source for your golem. Your golem’s Strength score increases by 2, increasing its ability to perform tasks that require raw strength. If this upgrade increases the golem’s Strength to 16 or higher, it gains proficiency in Strength saving throws; if this upgrade increases its Strength to 18, it gains proficiency in the Athletics skill.

You can select this upgrade multiple times, but you can’t increase the golem’s Strength above 18 using this upgrade.

Warfare Routines 
You advance the control routines for your golem, allowing it to fight more effectively. Your golem gains one of the following Fighting Styles of your choice: Archery, Dueling, or Great Weapon Fighting.

\subsection*{5th-Level Upgrades}

Cloaking Device 
You install an arcane cloaking device on your golem. Your golem can use this device to cast invisibility on itself, without expending a spell slot or requiring components.

Once the golem casts the spell in this way, it can’t do so again until you finish a short or long rest.

Expanded Frame 
You enlarge your golem, increasing its size by one category, up to a maximum of Large. If your golem is Large after taking this upgrade, it gains advantage on Strength checks and Strength saving throws, and its hit point maximum increases to (5 + (the golem's Constitution modifier +5) * your inventor level).

During the process, you can determine if your golem will have the appropriate anatomy to serve as a mount. You can select this upgrade multiple times, but you can’t increase the golem’s size above Large using this upgrade.

Iron Fortress Prerequisite: golem of Medium or larger size with a Constitution of 18 or higher 
You extend your golem’s shielding and stationary stability. Your golem now counts as three-quarters cover for creatures riding it or for creatures within 5 feet of it (as long as it is between the target and the attacker). Additionally, while your golem is on the ground, it can’t be moved against its will.

Reciprocity Programming 
If you use your bonus action to take the Help action to grant your golem advantage on an attack against a creature, your golem can take the Help action as a bonus action on its next turn to grant you advantage on an attack against a creature.

Shielding Bond 
After studying the arcane fundamentals of Shield Golems, you have gained insight on how they shield their controllers, and can implement it in your own golem. Your golem gains the ability to cast warding bond without expending a spell slot. When cast in this method, it doesn’t require material components.

Once the golem casts warding bond in this way, it can’t use this feature to cast it again until it finishes a short or long rest.

\subsection*{9th-Level Upgrades}

Mark of Sapience (Incompatible: Launcher Golem Chassis) 
You have attained the understanding of magic to craft a Mark of Sapience on the forehead of your Mechanical Golem, turning it into a Golem Companion. Its mental ability scores increase, granting it an Intelligence of 10, a Wisdom of 10, and a Charisma of 8. This allows it speak, remember things, and follow more complex commands without direct input. Additionallly, it gains proficiency in Intelligence and Wisdom saving throws.

\begin{minipage}{0.48\textwidth}
Roleplaying an Ironwrought Companion

If you choose the Mark of Sapience upgrade, your Ironwrought Golem becomes a sapient companion, capable of learning, thinking, and having opinions. Consider how this may impact your interactions.

Fundamentally, a Ironwrought Companion is still entirely loyal to its creator, but it can develop a personality and thoughts of its own.
\end{minipage}\hfill
\begin{minipage}{0.48\textwidth}

\end{minipage}

Overdrive 
You build in a special mode that allows your golem to push beyond its limitations. As an action, you can overcharge your golem with energy, granting it the effects of the haste spell for a number of rounds equal to your Intelligence modifier.

Once you overcharge your golem in this way, you can’t do so again until you finish a long rest.

Powered Charge 
You improve your golem’s charging capabilities, increasing its walking speed by 5 feet. If your golem has a walking speed of 40 feet or higher after taking this upgrade, it gains the Forceful Slam trait below. You can select this upgrade up to two times.

\begin{itemize}
  \item Forceful Slam. If the golem moves at least 20 feet straight toward a target and then immediately hits it with a melee attack, the target must succeed on a Strength saving throw (DC 8 + the golem’s proficiency bonus + its Strength modifier) or be knocked prone.
\end{itemize}

\subsection*{11th-Level Upgrades}

Airborne Propulsion 
You add a new method of propulsion to your golem, such as intricate mechanical wings or propellers. It gains a flying speed of 30 feet.

Multiattack Protocol 
Your golem can attack twice, instead of once, whenever it takes the Attack action on its turn.

Thundering Stomp (Prerequisite: Golem of Large size) 
Your golem can leverage its increased size and magical nature, allowing it to replace any attack it makes as part of the Attack action with a crushing stomp of magical energy as it brings down its foot. When it does so, each creature within 5 feet of the golem, other than itself, must succeed on a Constitution saving throw against your spell save DC or take 1d6 thunder damage.

Transforming Golem 
You install clever, multifunctional components into your golem, allowing it to reduce its size. Your golem can collapse itself down, decreasing its size by one category—from Medium to Small, for example. While collapsed in this way, its speed is reduced by 10 feet. It takes 1 minute for your golem to collapse itself or revert to its normal size.

\subsection*{15th-Level Upgrades}

Artificial Learning (Prerequisite: Mark of Sapience) 
Your Golem Companion begins to apply its abilities to learn new things, gaining one level in a class of your choice. Your Golem Companion gains all the 1st-level features of the chosen class. This doesn’t include hit points, Hit Dice, or class proficiencies. For example, choosing the Fighter class grants the golem the Fighting Style and Second Wind features, but nothing else.

You can select this upgrade multiple times, granting your golem another level of features from the same class or the 1st-level features from another class each time it is selected.

Brutal Armaments 
You increase the effectiveness of your golem’s natural weapons, allowing it to attack with lethal strikes. When your golem makes an attack, you can choose to subtract its proficiency from the attack roll. If you do so, and the attack hits, you can add double its proficiency bonus to the damage roll of the attack.

Shared Power 
You bind your golem’s power source to your own soul, allowing you to tap its power and it to tap your power. Your golem can use its action and expend one of your spell slots to cast a spell you know of the slot’s level or lower.

In addition, you can use your action to cause your golem to lose a number of hit points up to twice your inventor level, granting you a number of temporary hit points equal to that amount.

Finally, on your turn, you can forgo your action to grant your golem an additional action, or you can command your golem to forgo its action, granting yourself an additional action. That action can only be used to take the Attack (one weapon attack only), Dash, Disengage, Hide, or Use an Object action.

Titan Slayer 
You build into your golem one oversized weapon. This weapon can be a martial melee weapon or an improved version of any natural weapon the golem has. The weapon’s base damage dice are doubled—from 1d8 to 2d8, for example. The golem can only attack with this oversized weapon once per turn (making any additional attacks it can make using other weapons).

\begin{itemize}
  \item Inventor Specialization : Golemsmith
  \item Feature: Golemsmith Upgrades
\end{itemize}

\section*{Infusionsmith}

% [Image Inserted Manually]

An infusionsmith is, in some ways, perhaps the most quintessential type of inventor. While other inventors may delve into mechanics and tinkering, an infusionsmith tinkers with magic itself.

These inventors are on the cutting edge of magical engineering, understanding the principal applications of magic. Infusionsmiths would have ground to stand on in calling a wizard an impulsive spellslinger, for they are the inventors that work their magic through careful and meticulous study, laying down magic they may not use for hours, or painstakingly crafting a long-lasting enchantment.

An infusionsmith can be a magical swordsman, a wandslinger, or a bookish supporting character with a bagful of tricks that never quite runs dry. Some will stand within a maelstrom of animated blades, while others will plink away with a crossbow that seems unerringly accurate, but their common theme is using their prowess of infusion to make themselves formidable.

\subsection*{Infusionsmith's Proficiency}

When you choose this specialization at 1st level, you gain proficiency with calligrapher’s supplies and jeweler’s tools. Your knowledge of infusion magic also gives you a natural affinity for scribing spell scrolls. The time you must spend to create a magic spell scroll is halved.

\subsection*{Infused Armament}

Starting at 1st level, you can infuse a powerful armament to take into battle. At the end of a long rest, select one of the infusions below. The infusion lasts until the end of your next long rest.

Animated Weapon

You touch a melee weapon, causing it to spring to life. This Animated Weapon can be carried or stowed like a normal weapon, or you can ready it, causing it to float beside you. While an Animated Weapon is readied, you can make attacks with it as part the Attack action, sending it out to strike a target. All attacks made with your Animated Weapon during a turn must be made against the same target.

This special attack is a melee spell attack with which you are proficient. You can make this special attack against a target out to a range of 30 feet away from you. If the weapon has the heavy or special property, this range is halved to 15 feet; and if the weapon has the light property, this range is doubled to 60 feet. On a hit, the target takes damage equal to the weapon’s damage dice + your Intelligence modifier.

Blasting Rod

You touch a nonmagical object—a wand-blank, stick, staff, or rod—turning it into a Blasting Rod and infusing it with the ability to cast a cantrip. Select one evocation cantrip from the wizard spell list that doesn’t require concentration. Thereafter, as an action, you can use the Blasting Rod to cast that cantrip.

Once per turn, when you deal damage to a creature or object with your Blasting Rod, you can add your Intelligence modifier to the damage dealt to that target.

Infused Weapon

You touch a weapon, enchanting it. While this weapon is enchanted, you (and only you) have proficiency with it. This Infused Weapon can be wielded like a normal weapon, but it gains the following special property:

You can use your Intelligence modifier, instead of Strength or Dexterity, for the weapon’s attack and damage rolls. In addition, if the weapon has a single damage die, the size of that die increases by one (to a maximum of a d12). For example, if the Infused Weapon is a dagger, its damage die increases from a d4 to a d6.

\subsection*{Spell Manual}

At 3rd level, you have a Spell Manual containing two 1st-level wizard spells of your choice. Your Spell Manual is a repository of any non-inventor spell you know.

You can’t cast spells from your Spell Manual, and they don’t count against your number of spells known, but whenever you would learn a new inventor spell, you can select a spell from your Spell Manual instead of the inventor spell list. That spell then becomes an inventor spell for you, and you can cast it as normal. You can choose to scribe any inventor spell you know into your Spell Manual, as if copying it from a written spell.

Learning Spells of 1st Level and Higher. Each time you gain an inventor level, you can add one wizard spell of your choice to your Spell Manual for free. The chosen spell must be of a level for which you have spell slots, as shown on the Inventor table.

On your adventures, you might find other spells that you can add to your Spell Manual. For each level of the spell, the process takes 2 hours and costs 50 gp.

\subsection*{Infuse Magic}

Also at 3rd level, you can use your Spell Manual to infuse magic items, such as wands, storing magical power for later use. Over the course of 1 minute, you can perform a special ritual to cast a spell into the item, expending a spell slot as normal. The spell being stored has no immediate effect when cast in this way.

Thereafter, a creature holding the item that is aware there is magic infused in it can trigger the stored magic, casting the spell. The creature must have an Intelligence of 6 or higher to cast the spell in this way.

The spell uses your spell save DC, spell attack bonus, and spellcasting ability, but is in all other ways treated as if the creature holding it cast the spell. The spell infused in the item fades if you finish a long rest and the stored magic hasn’t been triggered.

\subsection*{Extra Attack}

Beginning at 5th level, you can attack twice, instead of once, whenever you take the Attack action on your turn.

\subsection*{Empowered Infusions}

\begin{minipage}{0.48\textwidth}
By the time you reach 5th level, your Infused Armament grows more powerful. If you infuse or animate a weapon, you can infuse or animate one additional weapon at the same time (having two Animated Weapons, two Infused Weapons, or one of each). If you have multiple Animated Weapons, you can attack the same or different targets with them during a turn.

If you cast a spell of 1st level or higher using an item from Infuse Magic or an upgrade, or from a magic wand, you can cast the cantrip from your Blasting Rod as a bonus action.

\subsubsection*{Infused Focus}

Starting at 14th level, you can anchor a powerful spell into an item. When you cast a spell that requires concentration, you can anchor it to an object you touch (such as a staff, wand, or weapon), causing the spell to not require your concentration. When you do so, the spell lasts a number of rounds equal to your Intelligence modifier, after which the spell ends.
\end{minipage}\hfill
\begin{minipage}{0.48\textwidth}
Benefits of Flexibility?

If you choose to have an Animated and Infused Weapon, you can attack twice with either, but will limit yourself to being able to only attack one target if you solely rely on the Animated Weapon. If you animate two weapons, you can split your attacks like normal using one to attack a different creature. Alternatively, you can animate a short-range heavy weapon and a long-range light weapon, giving you more options at different ranges but limiting your ability to target multiple targets at any optimal range.
\end{minipage}

Once you use this feature, you can’t use it again until you finish a short or long rest.

\section*{Infusionsmith Upgrades}

\subsection*{Unrestricted Upgrades}

Animated Archer 
You master animation enchantments, allowing you to use your Animate Weapon feature on a ranged weapon and a quiver of ammunition. An animated ranged weapon hovers near you, and you can make a ranged spell attack with it against a target within the weapon’s range.

On a hit, the target takes damage equal to the weapon’s damage dice + your Intelligence modifier. The weapon requires ammunition, and it can carry up 30 pieces of ammunition at a time. You can reload the ammunition at any time as an action.

Arcane Armament 
You master armoring yourself with magical enchantments. You learn the mage armor spell. While under the effect of mage armor , you can add your Intelligence modifier to your AC, instead of your Dexterity modifier. You can only do this if you are not adding your Intelligence modifier to your AC from another source.

Additionally, you gain resistance to force damage.

Loyal Weapon 
You can cast the returning weapon* spell without expending a spell slot, but if you cast it on a second weapon with this upgrade, the spell immediately fades from any previous use of this upgrade.

Magical Wand of... 
You create a new wand that you can infuse with a spell of 1st level or higher that you have recorded in your Spell Manual. This wand doesn’t require attunement, but can only be used by you. The spell must be of a level that you can cast as an inventor (as of when you would choose this upgrade). This wand has 3 charges.

You can expend a charge to cast the chosen spell at its lowest level. The wand regains all expended charges when you finish a long rest.

You can select this upgrade multiple times, choosing a different spell and creating a new wand each time you select this upgrade.

\begin{minipage}{0.48\textwidth}
Magic Wand Spell Level

A Magical Wand created using an upgrade slot at level 3 can always only ever cast 1st-level spells. For example, at level 5 if you have two Magical Wand upgrades, you will have one that can cast 1st-level spells, and one that can cast 2nd-level spells. Even if you later swap the upgrade, the Wand is still an Upgrade from level 3, and selected as if you are level 3 and only able to cast 1st-level spells.
\end{minipage}\hfill
\begin{minipage}{0.48\textwidth}

\end{minipage}

Skilled Infusions 
You manage to make the magic of your Animated and Infused Weapons so potent that attacks made with them gain the benefits of a fighting style. Attacks made with one-handed melee weapons gain Dueling, attacks made with two-handed melee weapons gain Great Weapon Fighting, and attacks made with ranged weapons gain Archery.

Soul-Saving Bond 
You set up a special magical bond between you and another creature. When either of the bonded creatures fails an Intelligence, Wisdom, Charisma, or death saving throw, the other creature can make the same saving throw, replacing the original result with its own. If this ability is used on a death saving throw, the roll is automatically a 20.

Once this ability has been used to replace a roll, it can’t be used again until both bonded creatures have finished a short or long rest.

This bond can be set up with a different creature at the end of a long rest.

Warding Stone 
You learn how to weave a protective enchantment on an item. That item gains a pool of temporary hit points equal to your Inventor level. Any creature carrying the item gains any temporary hit points remaining in its pool, but these hit points are lost if that creature is no longer carrying the item.

The item’s pool of temporary hit points replenishes when you finish a long rest.

Worn Enchantment 
You can enchant an item you are wearing, such as a cloak or scarf, to animate and assist you with a task, be it climbing a wall, grappling an enemy, or picking a lock. You can expend a spell slot of 1st level or higher as an action to gain proficiency in one Strength- or Dexterity-based skill of your choice until you finish a long rest.

Additionally, you can choose to use up all the magic in the item to gain advantage on one ability check using the chosen skill. After you do so, you lose the proficiency granted by this upgrade.

\subsection*{5th-Level Upgrades}

Animated Shield 
You apply your animation magic to a shield. At the end of a long rest, you can touch a shield, causing it to spring to life and protect you until you finish your next long rest. While it is protecting you, you gain the benefits of the shield as if you were wielding it. As usual, you can benefit from only one shield at a time.

While you have an animated shield, you can also send it to defend your allies. When a creature you can see attacks a target other than you that is within 30 feet of you, you can use your reaction to impose disadvantage on the attack roll. When you do so, you lose the benefits of the animated shield until the start of your next turn.

Arcane Ammunition 
You infuse a ranged weapon with special, arcane magic. The weapon no longer requires ammunition to fire, as it creates magical ammunition when you fire it, and whenever you hit a target with this magical ammunition, the attack deals force damage, instead of the weapon’s normal damage type.

Additionally, if the weapon had the loading property, it no longer has that property.

Deflecting Weapon 
While you have an animated melee weapon, you can use your reaction when a creature makes a weapon attack against you to attempt to block it, rolling a d8 and adding the number rolled to your AC for that attack. You make this choice after the creature makes its roll, but before the GM says whether the attack hits or misses. If you attempt to block a melee weapon attack in this way and the attack misses, you can immediately make one weapon attack against the attacker as part of the same reaction.
When you use this ability, if you take the Attack action on your next turn, you can only make one weapon attack as part of that action. This attack can’t be made with the animated weapon you used to attempt to block the attack.

Lesser Ring of Protection 
You make a prototype ring of protection . Any creature wearing this ring gains a +1 bonus to its AC.

Malicious Infusion 
You learn the heat metal spell.

Additionally, as a reaction to being hit by a metal weapon, you can cast heat metal targeting the weapon that struck you, expending a spell slot as normal.

Ring of Reaction 
You make a minor magic ring. Any creature wearing this ring can add your Intelligence modifier to its Dexterity saving throws and initiative rolls.

Translocation Binding 
When you make an attack with an Animated or Infused melee weapon, you can use your bonus action and expend a spell slot of 1st level or higher to teleport to an unoccupied space within 5 feet of the weapon before it returns to you. You can use this ability even if your attack misses the target.

Weapon Enchantment Expertise 
When you cast arcane weapon* , magic weapon , or vorpal weapon* , the targeted weapon deals an extra 1d4 force damage, and when you cast prismatic weapon* , the extra damage it deals increases by 1d4.

Additionally, you have advantage on Constitution saving throws made to maintain concentration on arcane weapon*, prismatic weapon* , magic weapon , and vorpal weapon* .

Weapon Enchantment Resonance 
When you cast arcane weapon* , prismatic weapon* , magic weapon , or vorpal weapon* on one of your Animated or Infused weapons, all of your Animated and Infused weapons gain the benefits of that spell. Additionally, you can target any weapon with arcane weapon*, prismatic weapon* , or magic weapon , even if the weapon is already magical.

\subsection*{9th-Level Upgrades}

Detonate Armament (Prerequisite: Arcane Armament) 
When you take damage while under the effects of mage armor , you can end the effects of that spell as a reaction to cast turbulent warp* without expending a spell slot.

Once you cast the spell in this way, you can’t do so again until you finish a short or long rest.

Dimensional Pockets 
You enchant a pocket on your gear to contain far more than it would appear. That pocket can hold up to 50 pounds, and you can put any object in it that can fit through a 6-inch-diameter opening. The pocket otherwise behaves like a bag of holding.

Invisibility Cloak 
You make an invisibility cloak. While wearing this cloak, a creature can cast invisibility on itself, without expending a spell slot or material components. When cast in this way, the spell doesn’t require the caster’s concentration. The spell ends if the caster is no longer wearing the cloak or if they choose to end the spell early (no action required).

Once a creature uses the cloak to cast this spell, it can’t be used in this way again until you finish a short or long rest.

Prepared Enchantment 
You can cast a limited version of contingency , without expending a spell slot or material components, as you bestow magical enchantments onto your gear. When you cast the spell in this way, it ends if you finish a long rest before the contingent spell is triggered, and the contingent spell must be a spell of 2nd level or lower.

Once you cast contingency in this way, you can’t do so again until you finish a long rest.

Spell-Trapping Ring 
At the end of a long rest, you can set a powerful magic into a nonmagical ring you touch. You can use this ring to cast counterspell without expending a spell slot. When you cast counterspell in this way and it succeeds, the spell you countered is stored in the ring. You can cast the stored spell from the ring without expending a spell slot, but that spell fades if you don’t cast it before you finish a long rest.

Once you cast counterspell to store a spell in this ring, you can’t do so again until you finish a long rest.

\subsection*{11th-Level Upgrades}

Empower Weapon 
When you hit a creature with a weapon, you can use your bonus action to infuse your strike with arcane energy, dealing an extra 2d6 force damage to the target. When you use this bonus action, you can also choose to expend a spell slot of 1st level or higher to further increase this extra damage. When you do so, the target takes an extra 1d6 force damage, plus another 1d6 per level of the spell slot.

Enchanted Broom 
You can enchant a broom (or broom-like object), turning it into a broom of flying . You set the command word for the broom, and it only obeys you.

Life Infusion 
You learn a potent magical infusion that suffuses a creature with life energy. You can cast regenerate , without expending a spell slot or material components.

Once you cast the spell in this way, you can’t do so again until you finish a long rest.

Magical Rod of... 
You create a new rod that you can infuse with a spell of 5th level or higher that you have recorded in your Spell Manual. This rod doesn’t require attunement, but can only be used by you. The spell must be of a level no higher than half your inventor level (as of when you would choose this upgrade), rounded down.

You can use the rod to cast the chosen spell at its lowest level. Once you cast the spell in this way, you can’t do so again until you finish a long rest. You can select this upgrade multiple times, choosing a different spell and creating a new rod each time you select this upgrade.

\begin{minipage}{0.48\textwidth}

\end{minipage}\hfill
\begin{minipage}{0.48\textwidth}
How to get high-level spells?

Astute players will note that when you add spells to your Spell Manual through leveling, the spell must be of a level for which you have spell slots, but when you gain access to the Magical Rod upgrade, you don’t yet have 5th-level spell slots. Spells for Magical Rods will primarily have to be found in the wild, in the form of scrolls or spellbooks, and copied into your Spell Manual that way.
\end{minipage}

Mixed Technique 
When you use your action to cast the cantrip from your Blasting Rod, you can make one weapon attack with an Animated or Infused Weapon as a bonus action.

\subsection*{15th-Level Upgrades}

Advanced Object Animation 
When you cast the animate objects spell, the animated objects use your spell attack bonus for their attack rolls, and they gain a bonus to their damage rolls based on their size: +1 (Tiny or Small), +2 (Medium), +4 (Large), or +8 (Huge).

Third Animated Weapon 
Your mastery of weapon animation expands to a greater breadth of control. If you have two Animated Weapons, you can animate a third weapon with your Animate Weapon feature. When you take the Attack action and make an attack with your other two Animated Weapons, you can make an attack with your third Animated Weapon as part of the same action.

\begin{minipage}{0.48\textwidth}
High-Level Rods \& Wands

While it may appear that infusionsmiths lack high-level upgrades, their Magical Rod and Wand upgrades provide a plethora of very valuable, high-level options.
\end{minipage}\hfill
\begin{minipage}{0.48\textwidth}

\end{minipage}

\begin{itemize}
  \item Inventor Specialization: Infusionsmith
  \item Feature: Infusionsmith Upgrades
\end{itemize}

\section*{Potionsmith}

% [Image Inserted Manually]

A potionsmith is an inventor who has pursued the secrets of alchemy. While many a village has an apothecary grinding odd herbs and roots into potent (or not so potent) concoctions, the careful process of mixing and brewing is just one way to achieve results. Through the use of the intricate secrets of the craft and direct infusions using magical rituals, a potionsmith can come up with explosive results in the blink of an eye... sometimes literally.

A potionsmith can be a scholar who has delved into the knowledge the world has to offer or an explorer that has unlocked the secrets of the wilderness. Their knowledge could come from being friends of the fey or from unfettered access to the royal library. As such, potionsmith’s can be good or evil, lawful or chaotic.

\subsection*{Potionsmith's Proficiency}

When you choose this specialization at 1st level, you gain proficiency with blowguns, alchemist’s supplies, and the herbalism kit.

Your knowledge of alchemy gives you a natural affinity for brewing potions. Creating a potion through normal crafting takes you only half the time it would normally take.

\subsection*{Alchemical Reagents Pouch}

At 1st level, you’ve acquired a pouch of useful basic reagents, much as a wizard might carry a component pouch. You’ve tucked away things that will come in handy, and can retrieve them as part of using an ability that might require them. As long as you have this pouch on you, you can use the potionsmith’s features. This pouch is considered to be alchemist’s supplies for the purposes of crafting.

If you lose your reagent pouch, you can spend 50 gp to reacquire the various stocks you need, or spend 1 full day gathering them from a natural environment without expense.

\begin{minipage}{0.48\textwidth}
Alchemists and Potions

Immediately on making a potionsmith, one may think of the magic potions that exist in most settings—from the simple potion of healing to the legendary potion of storm giant strength. However, someone does not need to be a potionsmith to make these, as crafting rules for them are open to all classes.

A potionsmith with their proficiency will certainly have the edge in potion crafting, but these potions are not fundamental to the class.
\end{minipage}\hfill
\begin{minipage}{0.48\textwidth}

\end{minipage}

\subsection*{Instant Reactions}

At 1st level, you know how to get instant reactions to occur without the niceties of grinding, simmering, and brewing required for more refined concoctions.

Moreover, you know enough ways to do these processes that with just a few standard supplies, you can get results out of a wide range of things you can gather in almost any locale and a pinch from your reagent pouch.

Pick three instant reactions from the following: Alchemical Acid, Alchemical Fire, Fortifying Fumes, Healing Draught, and Poisonous Gas. These instant reactions don’t count against your number of upgrades known.

\subsection*{Alchemical Infusions}

At 3rd level, you have discovered the secrets to drastically bypass the alchemical brewing process, crafting potent magical effects through the direct infusion of the potion with your own magic. When you finish a short or long rest, you can create any number of potions, choosing one of the following spells for each potion and expending a spell slot as if you were casting the spell. You must provide an empty vial for each potion you create in this way.

\begin{tabularx}{\textwidth}\toprule
{}XXXXXXXXX}
\midrule
Spell Level & \multicolumn{9}{c}{Infusion Spells} \\
\midrule
1st & \multicolumn{9}{c}{cure wounds, fog cloud, grease, heroism} \\
\midrule
2nd & \multicolumn{9}{c}{barkskin, lesser restoration, shatter, web} \\
\midrule
3rd & \multicolumn{9}{c}{blink, haste, stinking cloud, water breathing} \\
\midrule
4th & \multicolumn{9}{c}{confusion, stoneskin} \\
\midrule
5th & \multicolumn{9}{c}{cloudkill} \\
\midrule
\end{tabularx}

When you do so, the spell doesn’t take effect immediately, but is instead infused into the potion.

\begin{minipage}{0.48\textwidth}
If the spell grants an effect or restores hit points, a creature can consume it or administer it to another creature as an action. If a spell targets an area of effect, you can accurately throw the vial up to 30 feet, dealing no inherent damage, but breaking the vial on impact. The area of effect takes place when the vial is broken, with the effect centered on where the vial breaks. If the spell targets a creature, you can treat the vial as a ranged weapon with which you are proficient, and it has the thrown (20/60) property. You replace the spell attack with making a weapon attack made with the vial.

If the spell has a non-instantaneous effect that normally requires concentration, it doesn’t require your concentration when it takes effect in this way, but its duration is shortened to a number of rounds equal to your Intelligence modifier (minimum 1). A spell that doesn’t require concentration lasts its normal duration. An infused potion loses its potency if it isn’t used by the end of your next long rest.

You can gain additional spells for your Alchemical Infusions through your potionsmith upgrades.
\end{minipage}\hfill
\begin{minipage}{0.48\textwidth}
On Creativity and Catapults

While you can only accurately throw the vial containing an infused potion up to 30 feet, as per the feature, the trigger merely specifies that the effect takes place where it breaks.

If you can find another way to deliver the vial, such as the launch object* spell (or a similar spell) on the inventor spell list—or, well, an actual catapult—you can apply these methods as well.
\end{minipage}

\subsection*{Practiced Quaff}

Additionally at 3rd level, you gain the ability to consume potions and infused potions as a bonus action.

\subsection*{Empowered Alchemy}

Starting at 5th level, whenever you deal damage, grant temporary hit points, or restore hit points to a creature with an instant reaction or infused potion on your turn, you can add your Intelligence modifier to the amount of damage dealt or the number of hit points granted or restored.

\subsection*{Infusion Expertise}

Starting at 14th level, when you create an infused potion at end of a short or long rest, the first infused potion you create doesn’t require a spell slot to infuse, and you can choose a spell for which you would not otherwise have a high enough level spell slot when making this infusion.

\section*{Potionsmith Upgrades}

\subsection*{Unrestricted Upgrades}

Alchemical Acid (Instant Reaction) 
As an action, you can produce an instant reaction, throwing this combination of reagents at a point within 20 feet of you, which causes a caustic acid to splatter the area within a 5-footradius sphere centered on that point. Each creature in that area must succeed on a Dexterity saving throw against your spell save DC or take 2d4 acid damage. Damage from this acid deals double damage to objects and structures.

The damage increases by 2d4 when you reach 5th level (4d4), 11th level (6d4), and 17th level (8d4).

Alternatively, you can use a bonus action to prepare this instant reaction as a ranged weapon with which you are proficient. It lasts until the end of your turn, and it has the finesse and thrown (20/60) properties. On a hit, it deals damage equal to failing the saving throw against this effect. The damage dealt by this attack doesn’t include your Strength or Dexterity modifier.

Alchemical Fire (Instant Reaction) 
As an action, you can produce an instant reaction, tossing a quick combination of reagents at a point within 20 feet of you, which causes searing fire to flare up in a 5-foot-radius sphere centered on that point. Each creature in the area must succeed on a Dexterity saving throw against your spell save DC or take 1d8 fire damage.

This damage increases by 1d8 when you reach 5th level (2d8), 11th level (3d8), and 17th level (4d8).

Alternatively, you can use a bonus action to prepare this instant reaction as a ranged weapon with which you are proficient. It lasts until the end of your turn, and it has the finesse and thrown (20/60) properties. On a hit, it deals damage equal to failing the saving throw against this effect. The damage dealt by this attack doesn’t include your Strength or Dexterity modifier.

Delivery Mechanism 
You modify the stability of your reagents and develop a better delivery mechanism. You can target a point within 40 feet of you for your instant reactions (including ones gained from upgrades, such as Explosive Reaction) and infused potions that target a point. This additional precision enables you to better target the effects, allowing each creature of your choice in the area to automatically succeed on its Dexterity saving throw against the effects of your instant reactions and infused potions.

Additionally, when making an attack roll with an instant reaction or infused potion, you can use your Intelligence modifier, instead of your Dexterity modifier, for the attack roll.

Dragon Draught (Instant Reaction; upgrade only) 
As an action, you can quaff a vial of highly reactive liquid, allowing you to immediately belch a blast of devastating elemental energy. Choose one of the following damage types: acid, cold, fire, or lightning.

If you choose cold or fire damage, the area of effect is a 10-foot cone; if you choose acid or lightning damage, the area of effect is a 15-foot line. Each creature in the area must succeed on a Dexterity saving throw against your spell save DC or take 1d8 damage of the chosen type.

The damage increases by 1d8 when you reach 5th level (2d8), 11th level (3d8), and 17th level (4d8).

Explosive Reaction (Instant Reaction; upgrade only) 
As an action, you can produce an instant reaction, throwing a devastating combination of reagents at a point within 20 feet of you, which causes a minor explosion. Each creature within 10 feet of that point must succeed on a Constitution saving throw against your spell save DC or take 1d10 thunder damage from the shockwave of the explosion.

The damage increases by 1d10 when you reach 5th level (2d10), 11th level (3d10), and 17th level (4d10).

Alternatively, you can use a bonus action to prepare this instant reaction as a ranged weapon with which you are proficient. It lasts until the end of your turn, and it has the finesse and thrown (20/60) properties. On a hit, it deals damage equal to failing the saving throw against this effect. The damage dealt by this attack doesn’t include your Strength or Dexterity modifier.

Flaming Grease 
You tweak your grease formula to be flammable, as grease should be. When you cast grease or use it to make an infused potion, the effect becomes flammable. If the grease takes fire damage, all creatures in the spell’s area take 2d4 fire damage. The grease burns for a number of rounds equal to your Intelligence modifier (minimum 1).

Additionally, when you cast the grease spell or use it as an infused potion, you can instead use it to coat a weapon. This coating lasts for 1 hour or until it takes fire damage, igniting the grease. Once ignited, that weapon deals an extra 1d4 fire damage on a hit for the next minute.

Fortifying Fumes Reaction (Instant Reaction) 
As an action, you can produce an instant reaction, throwing a fortifying stimulus of reagents at a point within 20 feet of you, which causes fumes to erupt. Each creature within 10 feet of that point can choose to hold its breath and not inhale the fumes, but creatures that do inhale them gain 1d4 temporary hit points, deal an extra 1d4 damage on their next melee weapon attack, and have advantage on their next Constitution saving throw. Any remaining benefits from this effect fade at the end of your next turn.

Both the temporary hit points and the extra damage increase by 1d4 when you reach 5th level (2d4), 11th level (3d4), and 17th level (4d4).

Frostbloom Reaction (Instant Reaction; upgrade only) 
As an action, you can produce an instant reaction, throwing a combination of devastating, endothermically reactive reagents at a point within 20 feet of you, which causes a bloom of ice to erupt. The area within 5 feet of that point becomes difficult terrain until the end of your next turn.

Each creature in the area must succeed on a Dexterity saving throw against your spell save DC or be caught by the ice, taking 1d6 cold damage. If a creature that fails its saving throw is entirely in the area of effect, it becomes restrained until the end of its next turn. A restrained creature can use its action to make a Strength saving throw against your spell save DC, ending the effect on itself on a success.

The damage increases by 1d6 when you reach 5th level (2d6), 11th level (3d6), and 17th level (4d6).

Healing Draught (Instant Reaction) 
As a bonus action, you can produce an instant reaction that provides potent magical healing. A creature can use its action to consume this draught or administer it to another creature within 5 feet of it before the start of your next turn, causing the target to regain 1d8 hit points. If you have the Practiced Quaff feature, you can consume a Healing Draught as part of the bonus action used to make it.

A creature can benefit from a number of these draughts up to its Constitution modifier (minimum of 1), after which they provide no benefit to that creature until it finishes a long rest. A Healing Draught that isn’t consumed before the start of your next turn loses its potency.

The number of hit points the draught restores increases by 1d8 when you reach 5th level (2d8), 11th level (3d8), and 17th level (4d8).

Homunculus Familiar 
You learn the alchemical process for creating a homunculus minion. You can cast find familiar without expending a spell slot or material components other than your alchemical reagents and blood. This familiar can appear in any Tiny shape you want, though it uses the statistics of a creature that can normally be chosen by the spell, and it is a construct or monstrosity, rather than one of the types listed in the spell.

Inoculations 
You gain resistance to poison damage.

Additionally, at the end of a long rest, you can choose up to five creatures (including yourself) to become inoculated against poisonous effects you can produce that require a Constitution saving throw (such as the Poisonous Gas instant reaction or the cloudkill alchemical infusion). You and each of the chosen creatures automatically succeed on saving throws against these effects until the end of your next long rest.

Long Acting 
If the spell you infuse into an infused potion would normally require concentration, its duration is now a number of rounds equal to your Intelligence modifier + your proficiency bonus.

Persistent Reactions 
Your instant reactions that have an area of effect persist in that area until the start of your next turn. While they persist, each creature that enters the area for the first time on a turn or ends its turn there must repeat the saving throw against the effect. You can choose to make the effect not persist in this way when you take the action to produce the instant reaction.

Poisoner’s Proficiency 
You delve into the darkest, most mysterious depths of herbal lore, learning the potent secrets of poison. You gain proficiency with poisoner’s kits; if you already have this proficiency, your proficiency bonus is doubled for any ability check you make using a poisoner’s kit. Additionally, during a long rest, you can create one of the following poisons of your choice:

\begin{itemize}
  \item Contact Poison. You can apply this poison to a melee weapon or up to ten pieces of ammunition. That weapon or ammunition deals an extra 1d4 poison damage on a hit; the weapon can deal this poison damage 10 times before the poison wears off.
  \item Ingested Poison. This poison is a simple, flavorless powder. If a creature consumes a full dose of the poison, it must make a Constitution saving throw with disadvantage against your spell save DC. After 1 minute, if the creature failed its saving throw, it takes 1d10 poison damage per level you have in this class, and it becomes poisoned until it finishes a long rest.
  \item Inhaled Poison. You can use this dose of poison to increase the size and potency of your Poisonous Gas instant reaction. When you do so, the fumes erupt in a 20-footradius sphere and the effect’s damage dice are doubled.
\end{itemize}

Any unused poison loses its potency and no longer has any effect after you finish your next long rest.

Poisonous Gas (Instant Reaction) 
As an action, you can produce an instant reaction, tossing a quick combination of reagents at a point within 20 feet of you, which causes noxious fumes to erupt in a 10-foot-radius sphere centered on that point. Each creature in that area must succeed on a Constitution saving throw against your spell save DC or take 1d4 poison damage and become poisoned until the end of its next turn.
The damage increases by 1d4 when you reach 5th level (2d4), 11th level (3d4), and 17th level (4d4).

Secrets of Acid 
You learn the secrets of infusing acid into your infused potions. You add the following options to your list of available spells for your Alchemical Infusions:

\begin{minipage}{0.48\textwidth}
\begin{tabularx}{\textwidth}\toprule
{}XXXXX}
\midrule
Spell Level & \multicolumn{5}{c}{Infusion Spells} \\
\midrule
1st & \multicolumn{5}{c}{summon ooze*} \\
\midrule
2nd & \multicolumn{5}{c}{acid arrow} \\
\midrule
3rd & \multicolumn{5}{c}{erode*} \\
\midrule
\end{tabularx}
\end{minipage}\hfill
\begin{minipage}{0.48\textwidth}

\end{minipage}

Additionally, you can create one infused potion of summon ooze* at the end of a long rest without expending a spell slot.

Secrets of Fire 
You learn the secrets of infusing fire into your infused potions. You add the following options to your list of available spells for your Alchemical Infusions:

\begin{minipage}{0.48\textwidth}
\begin{tabularx}{\textwidth}\toprule
{}XXXXX}
\midrule
Spell Level & \multicolumn{5}{c}{Infusion Spells} \\
\midrule
1st & \multicolumn{5}{c}{faerie fire} \\
\midrule
2nd & \multicolumn{5}{c}{unstable explosion*} \\
\midrule
3rd & \multicolumn{5}{c}{fireball} \\
\midrule
\end{tabularx}
\end{minipage}\hfill
\begin{minipage}{0.48\textwidth}

\end{minipage}

Additionally, you can create one infused potion of faerie fire at the end of a long rest without expending a spell slot.

Secrets of Flight 
You learn the secrets of infusing the ability to escape the shackles of gravity into your infused potions. You add the following options to your list of available spells for your Alchemical Infusions:

\begin{minipage}{0.48\textwidth}
\begin{tabularx}{\textwidth}\toprule
{}XXXXX}
\midrule
Spell Level & \multicolumn{5}{c}{Infusion Spells} \\
\midrule
1st & \multicolumn{5}{c}{feather fall} \\
\midrule
2nd & \multicolumn{5}{c}{levitate} \\
\midrule
3rd & \multicolumn{5}{c}{fly} \\
\midrule
\end{tabularx}
\end{minipage}\hfill
\begin{minipage}{0.48\textwidth}

\end{minipage}

Additionally, you can create one infused potion of feather fall at the end of a long rest without expending a spell slot.

Secrets of Frost 
You learn the secrets of infusing frost into your infused potions. You add the following options to your list of available spells for your Alchemical Infusions:

\begin{minipage}{0.48\textwidth}
\begin{tabularx}{\textwidth}\toprule
{}XXXXX}
\midrule
Spell Level & \multicolumn{5}{c}{Infusion Spells} \\
\midrule
1st & \multicolumn{5}{c}{freezing shell*} \\
\midrule
2nd & \multicolumn{5}{c}{cold snap*} \\
\midrule
3rd & \multicolumn{5}{c}{flash freeze*} \\
\midrule
\end{tabularx}
\end{minipage}\hfill
\begin{minipage}{0.48\textwidth}

\end{minipage}

Additionally, you can create one infused potion of freezing shell* at the end of a long rest without expending a spell slot.

\begin{minipage}{0.48\textwidth}
Secrets of...

Almost any set of three spells (a 1st-level, 2nd-level, and 3rdlevel spell) that are either short-term buffs or deal area of effect damage can work, as long as there is a thematic connection between the spells.

Consult with your GM about additional options if you don’t see what you want present here.
\end{minipage}\hfill
\begin{minipage}{0.48\textwidth}

\end{minipage}

\begin{minipage}{0.48\textwidth}
Weapon Coating 
As a bonus action, you can apply one of your instant reactions to a melee weapon or piece of ammunition. The next time you hit a target with that weapon or ammunition before the end of your next turn, the target is subjected to the effects of the applied instant reaction. The creature automatically takes the damage or healing associated with the instant reaction, but makes a saving throw as normal against any additional effects.
\end{minipage}\hfill
\begin{minipage}{0.48\textwidth}
The Implications

There is an instant reaction that restores hit points, rather than deals damage. This can be applied via the Weapon Coating upgrade as well, though the weapon will still deal its damage on a hit. Perhaps your allies will forgive a blowgun dart coated in a healing draught...if you can hit with the attack.
\end{minipage}

\subsection*{5th-Level Upgrades}

Adrenaline Serum
 You learn to create a potent serum. As a bonus action, you can consume a dose of this serum, increasing your Strength and Dexterity modifiers by an amount equal to your Intelligence modifier (up to a maximum of +5) for a number of rounds equal to your Constitution modifier (minimum of 1), after which all effects of the serum fade at the start of your next turn. While you are under the effect of this serum, you also gain the benefits of the heroism spell. When these effects end, your speed is halved and you can’t benefit from your Adrenaline Serum again until the start of your next turn.

Explosive Powder (Prerequisite: Explosive Reaction) 
You’ve managed to stabilize and refine your Explosive Reaction. You can prepare a number of Explosive Reactions up to your Intelligence modifier (a minimum of 1). These Explosive Reactions don’t detonate instantly, but rather last for 1 minute and will only detonate when exposed to fire. A creature can be affected by the damage from no more than two Explosive Reactions simultaneously, but structures take the full damage of all Explosive Reactions in their area.

Resistance Potion 
You create a formula for making a simplified temporary resistance potion. During a long rest, you can create a potion that grants the drinker resistance to acid, cold, fire, lighting, poison, or thunder damage (selected during the long rest) for 10 minutes when consumed. If not consumed, it loses potency when you finish your next long rest.

\subsection*{9th-Level Upgrades}

Aroma Therapies 
You expand your alchemical knowledge to be able to produce incense and simmering reagents that grant benefits to those that inhale their fumes. A creature that spends a long rest surrounded by these fumes regains an additional 2d4 Hit Dice, its exhaustion level is reduced by 1d4 (instead of 1), and it is cured of any nonmagical diseases it’s suffering from.

Infusion Stone 
You use the secrets of alchemy to create an Infusion Stone. You can use this stone in the process of infusing potions in place of expending a spell slot level. The spell you infuse in this way must be of a level for which you have spell slots.
Once you use the Infusion Stone, you can’t use it again until you finish a long rest.

Mana Potion 
During a short rest, you can create a mana potion. A mana potion loses its potency if it isn’t consumed within 1 hour. As an action, a creature can consume a mana potion to regain one expended spell slot of its choice that is 3rd level or lower.

Potent Reactions 
You refine your instant reactions, increasing their potency. The die you roll to determine the damage or healing effect of your instant reactions is increased by one size: a d4 becomes a d6, a d6 becomes a d8, a d8 becomes a d10, and a d10 becomes a d12.

Rocketry (Prerequisite: Explosive Reaction and Delivery Mechanism) 
Combining your understanding of explosives with advances in delivery and stabilization, you produce the logical extreme: powering a rocket-based payload with your devastating knowledge.

You can choose any instant reaction you know as the payload for your rocket. Rockets must be prepared ahead of time, and you can prepare a number of them up to your Intelligence modifier (minimum of 1) at the end of a short or long rest. When you finish a rest, any unused rockets lose their potency and must be remade due to their volatile components.

A rocket targets a point you can see within 500 feet of you, but the DC of the instant reaction’s saving throw is reduced by 2 for every 100 feet it travels. When the rocket reaches its destination, the instant reaction takes effect, centered on that point.

Alternatively, you can load an alternate payload into your rockets, which can be an object that weighs no more than 1 pound, replacing the effect of the instant reaction. This alternate payload can be salvaged during a rest.

\subsection*{11th-Level Upgrades}

Field Infusion 
You can create an infused potion as an action.

Once you create an infused potion in this way, you can’t do so again until you finish a short or long rest.

Mutation Mixture (Prerequisite: 13th-level Inventor) 
You concoct an infused potion that can rapidly warp a creature’s body, temporarily mutating it in extreme ways. You add the polymorph spell to your list of available spells for your Alchemical Infusions. When a creature consumes this infused potion, the spell has a duration of 1 hour, but the creature that consumed it can end the effect early as an action.

Once a creature consumes this infused potion, it can’t be affected by the polymorph spell in this way again until it finishes a long rest.

Panacea 
When you create a Healing Draught, you can add a more potent concoction to it. A creature that consumes this potent draught can treat any dice rolled to determine the hit points it regains as having rolled their maximum value, and the creature gains the benefit of a greater restoration spell.

Once you create a Healing Draught in this way, you can’t do so again until you finish a long rest.

Perfect Reaction 
In a moment of perfect focus, you can create a flawless instant reaction. When you deal damage, grant temporary hit points, or restore hit points to a creature with this instant reaction, you can use the maximum possible result instead of rolling.

Once you create a Perfect Reaction, you can’t do so again until you finish a short or long rest.

\subsection*{15th-Level Upgrades}

Adrenaline Rush (Prerequisite: Adrenaline Serum) 
While under the effects of your Adrenaline Serum, you can attack twice, instead of once, whenever you take the Attack action on your turn.

Mad Alchemy 
Whenever you create an infused potion, you can infuse two spells into it, instead of one, which activate simultaneously.

Philosopher’s Stone 
You create a Philosopher’s Stone, allowing you to recreate wonders of alchemy. As long as you have a supply of non-gold metal, you can create up to 1 pound of gold each day (50 gp worth). Additionally, the Philosopher’s Stone can be used in place of a diamond worth up to 500 gp for a spell’s material component; it isn’t consumed when used in this way, but it becomes inert for 24 hours, providing no benefits.

You can brew a special potion using your Philosopher’s Stone called an Elixir of Life. Brewing this potion takes 8 hours and requires crushing a diamond worth at least 2,000 gp into it. An Elixir of Life causes a creature that drinks it to cease aging for 4d4 years. A creature that drinks this elixir also gains the effect of a death ward spell, which lasts until it is triggered, instead of 8 hours.

A more potent Elixir of Life can be created, increasing the number of years that the creature ceases aging by 1d4 for each additional diamond spent.

\begin{itemize}
  \item Inventor Specialization: Potionsmith
  \item Feature: Potionsmith Upgrades
\end{itemize}

\section*{Thundersmith}

% [Image Inserted Manually]

A Thundersmith is an inventor who harnesses the primal force of elemental power, channeling that power into their great creation: a weapon of unmatched devastation. Spectacular and terrible, these weapons bring fear to their foes and awe to their allies.

Why a Thundersmith bends their mind to the task of making such a weapon are as varied as the creations themselves. Some are coldly analytical about the destruction it causes seeking to continually improve it, tweaking it for ever more optimized destruction, while others view it merely as a tool, a means to an end; others still revel in the crash of thunder that heralds the terrifying force of their weapon.

Many look at these weapons as the dawn of a new age, in truth wielding them is a tricky and arcane art; as complex as its creation it is only truly understood and mastered by the one who forged the device—each weapon a unique piece of devastating art.

\subsection*{Thundersmith's Proficiency}

When you choose this specialization at 1st level, you gain proficiency with tinker’s tools and smith’s tools.

If your weapon requires ammunition, you gain the knowledge of how to forge and create it with smith’s tools during a long rest. You can create up to 50 rounds of ammunition during a long rest, with materials costing 1 gp per 10 rounds.

\subsection*{Stormforged Weapon}

Starting at 1st level, you harness the elemental power of thundering storms to create a powerful weapon. This weapon requires attunement, only you can attune to it, you are proficient with it while attuned, and you can only be attuned to one Stormforged Weapon at a time. If you have multiple Stormforged Weapons, you can change which one you are attuned to during a long rest.

If you lose your Stormforged Weapon or wish to create additional ones, you can do so over the course of three days (8 hours each day) by expending 200 gp worth of metal and other raw materials. When you make a new Stormforged Weapon, you can make the same or a different type, and select the same or different upgrades.

Select one of the following and consult the Stormforged Weapon table for its statistics.

\subsection*{Stormforged Weapons}

\begin{tabularx}{\textwidth}\toprule
{}XXXXXXXX}
\midrule
\multicolumn{2}{c}{Weapon Name} & \multicolumn{2}{c}{Damage} & Weight & Type & \multicolumn{3}{c}{Properties} \\
\midrule
\multicolumn{2}{c}{Thunder Cannon} & \multicolumn{2}{c}{1d12 piercing} & 15 lbs. & Ranged & \multicolumn{3}{c}{Ammunition (range 60/180), Two-Handed, Loud², Stormcharged¹} \\
\midrule
\multicolumn{2}{c}{Hand Cannon} & \multicolumn{2}{c}{1d10 piercing} & 5 lbs. & Ranged & \multicolumn{3}{c}{Ammunition (range 30/90), Light, Loud², Stormcharged¹} \\
\midrule
\multicolumn{2}{c}{Kinetic Hammer} & \multicolumn{2}{c}{1d10 bludgeoning + 1d4 thunder} & 10 lbs. & Melee & \multicolumn{3}{c}{Two-Handed, Heavy, Loud²} \\
\midrule
\multicolumn{2}{c}{Charged Blade} & \multicolumn{2}{c}{1d6 slashing + 1d4 lightning} & 3 lbs. & Melee & \multicolumn{3}{c}{Finesse, Loud²} \\
\midrule
\multicolumn{2}{c}{Lightning Pike} & \multicolumn{2}{c}{1d8 piercing + 1d4 lightning} & 10 lbs. & Melee & \multicolumn{3}{c}{Reach, Two-Handed, Loud²} \\
\midrule
\end{tabularx}

¹Stormcharged. When you use an action, bonus action, or reaction to attack with a Stormcharged Weapon, you can make only one Attack regardless of the number of attacks you can normally make. If you could otherwise make additional attacks with that action, the weapon deals an extra 3d6 lightning or thunder damage per attack that was foregone. 
²Loud. Your weapon rings with thunder that is audible within 300 feet of you whenever it makes an attack.

\begin{minipage}{0.48\textwidth}
Thunder Cannon

You use the power of thunder to launch a projectile with terrible power, if limited accuracy, over long distances. Ringing out with a booming crash, it brings fear to the battlefield.

Hand Cannon

Forgoing the guiding barrel, this pack uses the thundering power to launch a projectile with all the force of a Cannon, though its effective range is far more limited.

Kinetic Hammer

Rather than launching a projectile with the thundering force, you keep that force imbued in the weapon, allowing for devastating force to be applied to the attack.

Charged Blade

You create a bladed weapon that channels the harnessed power of the elemental storm power directly into the blade, causing it to lay waste to all it strikes. This weapon deals lightning damage instead of thunder when applying Thundermonger.
\end{minipage}\hfill
\begin{minipage}{0.48\textwidth}
Firearms in a Campaign Setting

A lot of campaign settings do not feature firearms, and in some of these the Thundersmith variant of inventor might not be the right choice, but consider that the inventor is fundamentally someone that tinkers with and explores boundaries of magic as much as or more so than technology.

A Thunder Cannon need not be a gunpowder powered device, even in a setting where gunpowder exists, but can be powered by harnessing elemental powers, bound through various carefully researched magical techniques. In most cases, the wonders of an inventor are more an engineering marvel of magic than technology, but that balance can shift depending on what is best for your setting.
\end{minipage}

Lightning Pike

You create a charged blade and stick it to the end of a pole, making it slightly more unwieldy, but giving it devastating reach. This weapon deals lightning damage instead of thunder when applying Thundermonger.

\subsection*{Thundermonger}

\begin{minipage}{0.48\textwidth}
At 3rd level, the elemental might of your weapon is so powerful its strikes deal bonus thunder damage.

When you hit a target with your Stormforged Weapon, you can deal an extra 1d6 thunder damage. After discharging this bonus damage, you can’t deal this bonus damage again until the start of your next turn.

This extra damage increases by 1d6 when you reach certain levels in this class: 5th level (2d6), 7th level (3d6), 9th level (4d6), 11th level (5d6), 13th level (6d6), 15th level (7d6), 17th level (8d6), and 19th level (9d6).
\end{minipage}\hfill
\begin{minipage}{0.48\textwidth}
Stormcharged vs. Thundermonger

A common question is why does the Stormcharged Property exist and how does it interact with Thundermonger? Functionally, the stormcharged property (like the loading property) has no effect on a single classed Thundersmith, it just exists to make Stormforged Weapons interact better with Extra Attack for multiclassing and Cross-Disciplinary Knowledge.
\end{minipage}

\subsection*{Devastating Blasts}

Beginning at 5th level, when you miss an attack with your Thundering Weapon, you can apply Thundermonger damage to the target creature you missed, but it deals only half the bonus damage. Dealing damage this way counts as applying Thundermonger damage.

\subsection*{Unleashed Power}

Starting at 14th level, when rolling damage for Thundermonger or your Stormforged Weapon, you can expend a spell slot to reroll a number of the damage dice up to your Intelligence modifier (minimum of one) and maximize a number of dice equal to the level of the spell slot expended. (You may maximize the dice rerolled this way, or the dice that were not rerolled) You must use the new rolls.

\section*{Thundersmith Upgrades}

\subsection*{Unrestriced Upgrades}

\begin{minipage}{0.48\textwidth}
How to Merge Your Weapons

If you want to call it a gunblade, I’m not here to stop you, but it can be any range of configuration, from a bayonet to something more exotic.
\end{minipage}\hfill
\begin{minipage}{0.48\textwidth}
Adaptable Weapon 
You can adapt a weapon without the ammunition property to have an alternate attack type giving it the functionality of a Hand Cannon, or give a weapon with the ammunition property an alternate attack with the properties of a Charged Blade.
\end{minipage}

Extended Range 
You extend the reach of your Stormforged Weapon. If your Stormforged Weapon has the ammunition property, its range is extended; normal range is extended by 30 feet, and maximum range is extended by 90 feet. If your Stormforged Weapon is a melee weapon with the two-handed property, it gains the reach property.

Lightning Burst 
You upgrade your Stormforged Weapon to discharge its power within a 5-foot-wide 60-foot-long line. If you have not dealt Thundermonger damage since the start of your turn, as an action, you can make a special attack. Each creature in the area must make a Dexterity saving throw against your spell save DC or take damage equal to the bonus damage of Thundermonger as lightning damage on a failed save, half as much on a successful save.

This counts as discharging your Thundermonger damage. Firing in this way doesn’t consume ammo.

Lightning Magic 
After a long study of the internal workings of your Stormforged Weapon, your mastery of lightning and thunder magic is such that you learn the following spells at the following levels and can cast them as Inventor Spells.

\begin{minipage}{0.48\textwidth}
\begin{tabularx}{\textwidth}\toprule
{}XXXXX}
\midrule
Inventor Level & \multicolumn{5}{c}{Spell} \\
\midrule
3rd & \multicolumn{5}{c}{lightning tendril*} \\
\midrule
5th & \multicolumn{5}{c}{crackle*} \\
\midrule
9th & \multicolumn{5}{c}{lightning bolt} \\
\midrule
13th & \multicolumn{5}{c}{jumping jolt*} \\
\midrule
15th & \multicolumn{5}{c}{sky burst*} \\
\midrule
\end{tabularx}
\end{minipage}\hfill
\begin{minipage}{0.48\textwidth}

\end{minipage}

You can cast lightning tendril* without expending a spell slot, after which you must finish a long rest before you can cast it without expending a spell slot again. Starting at 5th level, you can choose to cast crackle* without expending a spell slot instead.

Point Blank (Prerequisite: Hand Cannon) 
Being within 5 feet of a hostile creature doesn’t impose disadvantage on your ranged attack rolls. Additionally, you can use your Hand Cannon when making opportunity attacks.

\begin{minipage}{0.48\textwidth}
Silencer (Incompatible with Echoing Boom) 
You upgrade your Stormforged Weapon with a sound dampening module. Your Stormforged Weapon loses the loud property. Additionally, you can expend a 2nd-level spell slot to overcharge the Silencer, casting the silence spell.
\end{minipage}\hfill
\begin{minipage}{0.48\textwidth}
Note: Sound—and consequently thunder damage—can’t pass through silence or affect a creature inside its area of effect.
\end{minipage}

Shock Absorber 
You add a reclamation device to your Stormforged Weapon to gather energy from the surroundings when it is present. As a reaction to taking Lightning or Thunder damage, you can gain resistance to that damage type until the start of your next turn. When you do so, the next time you roll Thundermonger damage before the end of your next turn, it deals an extra 1d6 damage.

Sonic Movement (Prerequisite: Stormforged weapon that can deal Thunder damage) 
You recalibrate and rebalance your weapon to leverage the backdraft of the force it exerts. When you make an attack with your Stormforged weapon that would deal thunder damage on your turn, you are knocked 5 feet away from the target you attacked. This movement doesn’t provoke opportunity attacks.

Storm Blast 
You can make a special attack with your Stormforged Weapon, unleashing a storm blast in a 30-foot cone as an action. Each creature in the area must make a Strength saving throw against your spell save DC, or take damage equal to the bonus damage of Thundermonger and be knocked prone.

This counts as discharging your Thundermonger damage. Firing in this way doesn’t consume ammo.

If using a Kinetic Hammer, you can use your Strength to calculate the DC of this ability (8 + your Strength modifier + your proficiency bonus) instead of your spell save DC.

Twin Thunder 
You can attune to two one-handed Stormforged Weapons at the same time, so long as they either share the same Upgrades, or have total Upgrades equal to your maximum Upgrade count between them. Attuning to a second weapon this way doesn’t count against your maximum number of attuned items. If you make an attack with one of them while holding the other, you can attack with the other as a bonus action, however, both share the same use of Thundermonger. You don’t add your ability modifier to the damage of the bonus attack, unless that modifier is negative or you have the Two Weapon Fighting Fighting Style.

While dual wielding Stormforged Weapon, you can load a Stormforged Weapon without a free hand.

Weapon Improvement 
Your Stormforged Weapon gains a +1 to attack and damage rolls. This doesn’t stack with any benefit gained from Arcane Retrofit, and this upgrade can be replaced as part of applying a bonus to your Stormforged Weapon via Arcane Retrofit.

\subsection*{5th-Level Upgrades}

Echoing Boom (Incompatible with Silencer) 
You pack extra power into your Thundermonger, increasing the damage it deals by 1d6.

Harpoon Reel 
You devise an alternate attack method for your Stormforged Weapon that launches a harpoon attached to a tightly coiled cord. This attack has a normal range of 30 feet and a maximum range of 60 feet, and it deals 1d6 piercing damage. This attack doesn’t apply Thundermonger damage. This attack can target a surface, object, or creature.

A creature struck by this attack is impaled by the Harpoon unless it removes the Harpoon as an action, which causes it to take an extra 1d6 damage. While the Harpoon is stuck in the target, you are connected to the target by a 60 foot cord. Dragging the connected party via the attached cord causes the creature moving to move at half speed unless they are a size category larger.

While connected in this manner, you can use your bonus action to activate the Reel action, pulling yourself to the location if the target is Medium or larger. A Small or smaller creature is pulled back to you, ending the connection. Alternatively, you can opt to disconnect the cord.

This attack can’t be used again until the Reel action is taken.

\begin{minipage}{0.48\textwidth}
Connected...By a Rope

Note that Harpooning a creature means that there is a rope connecting you to them. While you can use this to move them as per the ability, they can also use this to move you, particularly if they are substantially larger. Many Thundersmiths were last seen saying “I got this” and then Harpooning a dragon. User discretion is advised.
\end{minipage}\hfill
\begin{minipage}{0.48\textwidth}

\end{minipage}

Terrifying Thunder (Prerequisite: Echoing Boom) 
You add an additional amplifier to maximize the shock and awe value of your Stormforged Weapon. The first time a target takes damage from Thundermonger, they are deafened until the end of their next turn.
Additionally, they must make a Wisdom saving throw against your spell save DC or become frightened of you for one minute. They can repeat this saving throw at the end of each of their turns.

On a successful saving throw, the creature is immune to Terrifying Thunder for 24 hours.

\subsection*{9th-Level Upgrades}

Elemental Swapping 
You upgrade the firing chamber for more adaptable damage. When you take the attack action with your Stormforged Weapon you can adjust its elemental properties, causing any bonus damage granted by Thundermonger to deal acid, cold, fire, lightning, or thunder damage instead of thunder damage.

Alternatively, you can use a vial of holy water to cause your next Thundermonger bonus damage to deal radiant damage.

Mortar Shells (Prerequisite: Stormforged weapon with the Ammunition property) 
You build an alternate fire mode allowing you to fire your Stormforged Weapon like a mortar. Pick a target point within range, and make an attack roll. Apply the attack roll to all creatures within a 5-foot radius of the target point. Creatures hit take the weapon's damage plus half of the Thundermonger bonus damage. Dealing damage this way counts as applying Thundermonger damage for the turn. If you miss all targets, you can apply Devestating Blasts as normal.

Creatures do not benefit from cover against this fire mode unless they have overhead cover as well.

Ride the Lightning (Prerequisite: Lightning Burst) 
When you use the Lightning Burst ability, you can opt to expend a spell slot of 1st level or higher to infuse yourself into the burst of power. When you do so you are teleported up to 60 feet in the direction of the Lightning Burst (ending early if the Lighting Burst is blocked by an obstacle). You can stop anywhere along the path of the Lighting Burst, but the line will stop where you do.

Shock Harpoon (Prerequisite: Harpoon Reel) 
After hitting a creature with the Harpoon fire mode, you can use a bonus action to deliver a shock. If you have not dealt Thundermonger damage since the start of your turn, you can deal damage equal to your Thundermonger bonus damage as lightning damage. This counts as discharging your Thundermonger damage. Additionally, the target must make a Constitution saving throw against your spell save DC or be stunned until the end of its next turn.

Once used, the Harpoon must be reeled in before this can be used again.

Synaptic Feedback 
You install a feedback loop into your cannon, allowing you to siphon some energy from your Stormforged Weapon to empower your reflexes.
Whenever you deal lightning damage with your Stormforged Weapon your walking speed increases by 10ft and you can take the Dash or Disengage actions as a bonus action. This effect lasts until the end of your turn.

Thunder Jump 
You build a quick release for the arcane thundering energy that fills your Stormforged Weapon. As an action you can discharge your Thunder Cannon downward, launching yourself up to 60 feet in any direction. This movement doesn’t provoke opportunity attacks, and you carry up to one willing creature of your choice within 5 feet with you. Neither you or the carried creature takes falling damage on landing.

Creatures that are within 10 feet of you when you use this ability must make a Dexterity saving throw against your spell save DC. On failure, they take damage equal to half of your Thundermonger damage. On success, they take half as much damage. This counts as discharging your Thundermonger damage. Firing in this way doesn’t consume ammo.

Transforming Weapon (Prerequisite: Adaptable Weapon) 
Select 3 of Thunder Cannon, Hand Cannon, Kinetic Hammer, Charged Blade, and Lightning Pike. As a bonus action, you can transform your Stormforged Weapon between them.

\subsection*{11th-Level Upgrades}

Backblast 
After dealing thunder damage on your turn, you can create a burst of thunderous sound that can be heard up to 100 feet away as a bonus action. Each creature within 5 feet, other than you, must make a Constitution saving throw or take 3d6 thunder damage You can do this a number of times equal to your Intelligence modifier, regaining all uses after a long rest. This damage increases by 1d6 when you reach 17th level to a total of 4d6.

Blast Radius 
Your Devastating Blasts feature now deals half your weapon damage (including your modifier) in addition to half your Thundermonger damage when you apply it to a missed target that is within 30 feet of you.

Stabilization 
When making a ranged weapon attack with your Stormforged Weapon, if neither you nor your target has moved since your last ranged attack against them, then you gain advantage on the attack. Also, being prone no longer causes you to have disadvantage when making a ranged weapon attack with your Stormforged Weapon.

\subsection*{15th-Level Upgrades}

Massive Overload (Prerequisite: Storm Blast or Lightning Burst) 
Before making an attack with your Stormforged Weapon, you can expend a spell slot of 3rd level or higher to use Storm Blast or Lightning Burst at the same time as making an attack; in this case the Storm Blast or Lightning Burst is powered by the spell slot and doesn’t count as applying Thundermonger for that turn. The direction of this secondary ability is the same as your attack.

Doing this damages your Stormforged Weapon and you must spend an action to repair it before you can fire again. You must have the secondary ability unlocked as an upgrade to use it.

Masterwork Weapon (Prerequisite: Weapon Improvement) 
The bonus to attack and damage rolls for your Stormforged Weapon increases by +2 (stacking with any existing bonus from Weapon Improvement or Arcane Retrofit), up to a maximum of +4.

\begin{itemize}
  \item Inventor Specialization: Thundersmith
  \item Feature: Thundersmith Upgrades
\end{itemize}

\section*{Warsmith}

% [Image Inserted Manually]

A Warsmith is an Inventor that has turned their wondrous talent of invention to a singular goal: making themselves a juggernaut of war. The reasons behind this could be benevolent or nefarious. Some Warsmiths seek to turn their invention into a machine of death and terror; others become the arbiter of justice and order, and others still perhaps merely seek to refine their craft in pursuit of pure innovation.

Because few individuals would pursue such a wondrous invention without a driven purpose to their endeavor, warsmiths tend to be lawful. Usually driven to their actions by a greater purpose, they seek the power to accomplish their aims, be it righting the wrongs of the world or bringing it to heel beneath their ironshod boot.

\subsection*{Warsmith's Proficiency}

At 1st level, you gain proficiency with heavy armor, tinker’s tools and smith’s tools.

\subsection*{Warplate Gauntlet}

At 1st level, when you take this specialization, you construct a Warplate Gauntlet. This is a specialized Wondrous Item that only you can attune to. When you create a Warplate Gauntlet, you can add one of the following upgrades to it: Power Fist, Force Blast, or Martial Grip. This upgrade doesn’t count against your upgrade total. You can make multiple gauntlets with different upgrades, but can only be attuned to one at a time.

If you lose your Warplate Gauntlet, you can remake it during a long rest with 25 gp worth of materials, or can scavenge for materials and forge it over two days of work (8 hours each day) without the material expense.

While wearing a Warplate Gauntlet, you can engage Artificial Strength.

Artificial Strength

When you don your Warplate Gauntlet or as an action while wearing it, you can dedicate some of your intelligence to fully controlling the power of the gauntlet. You can reduce your current and maximum Intelligence score to increase your current Strength ability score by the same amount, but you can only raise your Strength ability score up to what your Intelligence ability score was before engaging Artificial Strength. You can stop using Artificial Strength at any time, and it automatically ends if your gauntlet is removed.

\subsection*{Warsmith's Armor}

\begin{minipage}{0.48\textwidth}
At 3rd level, you’ve attained the knowledge of forging and arcane tinkering sufficient to create a set of armor that augments and expands your abilities from a standard, nonmagical, set of heavy armor using resources you’ve gathered. This process takes 8 hours to complete, requiring the use of a forge, foundry or similar and it incorporates a Warplate Gauntlet (they do not require separate attunement). While wearing your armor, your Strength ability score increases by 2, and your maximum Strength ability score becomes 22.

You can create a new set of armor by forging it from a set of gathered and purchased materials in a process that takes 2,000 gp and 8 hours. You can create multiple sets of armor, but you can only be attuned to one of them at a given time, and you can only change which one you are attuned to during a long rest. If you create a new set of Warsmith’s Armor, you can apply a number of Upgrades equal to the value on the class table, applying each at the level you get it on the class table.

When you create your armor, you can create a heavy plated Warplate, medium balanced Wargear, light flexible Warsuits, or you can integrate your changes directly into your body as Integrated Armor. If your armor is Warplate or Integrated Armor, you gain the Powerful Build trait. Powerful Build means you count as one size larger when determining your carrying capacity and the weight you can push, drag, or left. Integrated Armor can’t be removed, but doesn’t confer any penalty when resting in it.

Additionally, if you are Small, you become Medium while wearing Warplate.
\end{minipage}\hfill
\begin{minipage}{0.48\textwidth}
Warsmiths \& Magical Armor

By the rules laid out here, using magical armor as a base for your Warsmith’s armor has no additional effect. This is intentionally the rules-as-written, but there is certainly some flexibility here.

\begin{itemize}
  \item Using Adamantine or Mithral, the properties carry over to the Warsmith’s armor.
  \item Using +1/+2/+3 armor carries over, but counts as taking a free “Armor Class” upgrade for each +1 the armor has, meaning that upgrade can’t be taken to make the armor +4 or better.
  \item Armor of Resistance carries over, counting as taking “Resistance” upgrade for that damage type for free.
\end{itemize}

Other cases can be handled on a case-by-case basis. Consult with your GM and work something out that would be reasonable to combine making receiving magic armor a cool bonus, but not something that breaks the game!
\end{minipage}

\subsection*{Warsmith Armor}

\begin{tabularx}{\textwidth}\toprule
{}XXXXXX}
\midrule
Armor Name & AC & Weight & Strength Requirement & Stealth & \multicolumn{2}{c}{Properties} \\
\midrule
Warplate & 18 & 65 lbs. & — & Disadvantage & \multicolumn{2}{c}{Heavy Armor, Powerful Build¹} \\
\midrule
Warsuit & 14 + Dex Modifier (max 2) & 20 lbs. & — & — & \multicolumn{2}{c}{Medium Armor} \\
\midrule
Warskin & 12 + Dex Modifier & 13 lbs. & — & — & \multicolumn{2}{c}{Light Armor} \\
\midrule
Integrated Armor & 14 + Dex modifier (max 2) & N/A & — & — & \multicolumn{2}{c}{Medium Armor, Powerful Build¹} \\
\midrule
\end{tabularx}

¹Powerful Build. You count as one size larger when determining your carrying capacity and the weight you can push, drag, or lift.

\subsection*{Extra Attack}

Beginning at 5th level, you can attack twice, instead of once, whenever you take the Attack action on your turn.

\subsection*{Fully Customized Gear}

\begin{minipage}{0.48\textwidth}
What Does Your Armor Look Like?

Nothing in this document specifies the visual appearance of your armor beyond the type of armor, but it very likely does not look like a standard set of armor. Consider what the visual differences are: Are your enhancements more mechanical in nature or more magical in nature? Does it have geared joints, glowing runes, or both? Consider how your setting might react to someone standing around in such armor. In all but the highest magical settings, such a set of armor is likely to attract some curiosity or concern.
\end{minipage}\hfill
\begin{minipage}{0.48\textwidth}
Starting at 14th level, you’ve mastered the customization of your Warsmith’s armor. You can add one additional Upgrade to your armor that doesn’t count against your Upgrade total.

Additionally, during a long rest, you can now swap out any one upgrade for any other upgrade of the same level, so long as you don’t have an Upgrade that requires the Upgrade you are removing as a prerequisite, or an incompatible Upgrade.
\end{minipage}

\section*{Warsmith Upgrades}

\subsection*{Unrestricted Upgrades}

Accelerated Movement (You can apply this upgrade twice.)
 You reduce the weight of your armor by 15 pounds (to a minimum of 0 pounds). While wearing your armor your speed increases by 10 feet. This applies to all movement speeds you have while wearing your armor.

Adaptable Armor 
You integrate deployable hooks and fins into your armor, augmenting its mobility. While wearing your armor you gain a climbing speed equal to your walking speed, and you can move up, down, and across vertical surfaces and upside down along ceilings, while leaving your hands free.

Additionally, you gain a swimming speed equal to your walking speed.

Arcane Visor (You can take this upgrade multiple times, selecting a different option each time.)
 You magically enchant your visor. You gain one of the following effects while wearing your armor; you pick the effect when selecting the upgrade.

\begin{itemize}
  \item You gain darkvision to a range of 60 feet. If you already have darkvision, the range of that darkvision is increased by 60 feet.
  \item You can ignore Sunlight Sensitivity.
  \item Divination spells no longer require your concentration to maintain. You can only use this effect on one spell at a time.
\end{itemize}

Regardless of the selection, you have advantage on saving throws against being blinded while wearing your armor.

Collapsible (Incompatible with Integrated Armor) 
Your Warsmith’s armor can collapse into a case for easy storage. When transformed this way the armor is indistinguishable from a normal case and weighs 1/3 its normal weight. As an action you can don or doff the armor, allowing it to transform as needed.

Construct Constitution (Prerequisite: Integrated Armor) 
You gain resistance to poison damage and immunity to the poisoned condition. You have advantage on saving throws against diseases as well as spell effects that require a “humanoid” target.

Flame Projector 
You gain the ability to unleash fire energy. While wearing your Warplate Gauntlet, you can cast fire bolt , and gain access to the following spells at the following levels while wearing your Warplate Gauntlet:

\begin{minipage}{0.48\textwidth}
\begin{tabularx}{\textwidth}\toprule
{}XXXX}
\midrule
Inventor Level & \multicolumn{4}{c}{Spell} \\
\midrule
3rd & \multicolumn{4}{c}{burning hands} \\
\midrule
5th & \multicolumn{4}{c}{scorching ray} \\
\midrule
9th & \multicolumn{4}{c}{fireball} \\
\midrule
13th & \multicolumn{4}{c}{wall of fire} \\
\midrule
17th & \multicolumn{4}{c}{flame strike} \\
\midrule
\end{tabularx}
\end{minipage}\hfill
\begin{minipage}{0.48\textwidth}

\end{minipage}

Force Blast 
You upgrade your Warplate Gauntlet to deliver a special ranged attack. This attack is a ranged spell attack that deals 1d8 + your Intelligence modifier force damage, and has a range of 60 feet.

You are proficient in this weapon. When you take the attack action, you can use this ranged spell attack in place of any attack made.

Fortified Brace (Prerequisite: Warplate) 
You build in systems to maximize defensive potential. As a reaction to taking damage, you can brace your armor, locking it in a defensive position. Until the start of your next turn, you gain resistance to all damage (Including the triggering damage) and any subsequent attacks made against you are made with disadvantage. On your next turn after taking this action, your speed is halved and you can’t take an action (you can still take a bonus action).

Grappling Hook (Prerequisite: Warskin or Warsuit) 
Your Warsuit gains an integrated grappling hook set into your gauntlet. As an attack or as an action, you may target a surface, object or creature within 20 feet. If the target is Small or smaller, you can make a Strength (Athletics) grappling check to pull it to you and grapple it. Alternatively, if the target is Medium or larger, you can choose to be pulled to it (this doesn’t grapple it). Opportunity attacks provoked by this movement are made with disadvantage.

Grappling Reel (Prerequisite: Integrated Armor or Warplate) 
Your Warsmith’s armor gains an integrated grappling reel set into your gauntlet. As an attack or as an action, you may target a surface, object or creature within 30 feet. If the target is Large or smaller, you can make a Strength (Athletics) check as if grappling normally to pull it to you and grapple it on success. Alternatively, if the target is Large or larger, you can choose to be pulled to it (this doesn’t grapple it). This movement provokes opportunity attacks as normal movement would.

Iron Fortitude (Prerequisite: Integrated Armor) 
You gain an unnatural durability. When damage reduces you to 0 hit points, you make a Constitution saving throw with a DC of 5 + the damage taken, unless the damage is from a critical hit. On a success, you drop to 1 hit point instead.

Lightning Channel 
You are able to funnel your suit’s power into your attacks. You can use this upgrade to cast lightning charged* as a bonus action without expending a spell slot.

Once used, this upgrade can’t be used again until you finish a short or long rest.

Additionally, you can apply the damage from lightning charged on your Force Blast ranged spell attacks.

Lightning Projector 
You gain the ability to unleash lightning energy. While wearing your Warplate Gauntlet, you can cast shocking grasp , and gain access to the following spells at the following levels while wearing your Warplate Gauntlet:

\begin{minipage}{0.48\textwidth}
\begin{tabularx}{\textwidth}\toprule
{}XXXX}
\midrule
Inventor Level & \multicolumn{4}{c}{Spell} \\
\midrule
3rd & \multicolumn{4}{c}{lightning tendril*} \\
\midrule
5th & \multicolumn{4}{c}{lightning charged*} \\
\midrule
9th & \multicolumn{4}{c}{lightning bolt} \\
\midrule
13th & \multicolumn{4}{c}{jumping jolt*} \\
\midrule
17th & \multicolumn{4}{c}{sky burst*} \\
\midrule
\end{tabularx}
\end{minipage}\hfill
\begin{minipage}{0.48\textwidth}

\end{minipage}

Martial Grip 
Your Warplate Gauntlet grants you the ability to wield a wide variety of powerful weapons. You gain proficiency with martial weapons while wearing your Warplate Gauntlet.

\begin{minipage}{0.48\textwidth}
Power Fist (You can apply this upgrade twice.) 
Your Warplate Gauntlet becomes a melee weapon while you aren’t holding anything in it. You have proficiency in this weapon, and it has the Light and Special properties. It deals 1d8 bludgeoning damage.

\begin{itemize}
  \item Special. When you make an attack roll, you can choose to forgo adding your proficiency bonus to the attack roll. If the attack hits, you can add double your proficiency bonus to the damage roll.
\end{itemize}

You can apply this upgrade up to 2 times, making a separate weapon with the same properties the second time.
\end{minipage}\hfill
\begin{minipage}{0.48\textwidth}
Variant: Impact Gauntlet

Your GM may allow you to take the Impact Gauntlet upgrade of the Gadgetsmith in place of a Power Fist, which is similar but deals 1d6 damage and has the finesse property. For the purposes of upgrades, Impact Gauntlets and Power Fists should be considered interchangeable.
\end{minipage}

Reinforced Armor (You can apply this upgrade up to 3 times.) 
You reinforce the structure and materials that make up your Warsmith’s armor. Your Warsmith’s Armor’s Armor Class (AC) increases by 1.

Sentient Armor 
You create an artificial personality integrated into your armor, giving it limited sentience. This sentience assists you in many ways. While wearing your armor, your Intelligence ability score and maximum Intelligence ability score are increased by 2. Your armor can understand and speak any language you can speak. You can communicate telepathically with it while wearing it.

\begin{minipage}{0.48\textwidth}
Talking Armor

How much of a personality Sentient Armor has is up to the player and GM. Functionally, it only enhances things that use Intelligence as written , but many people enjoy fleshing it out much further as a sentient item as a sort of NPC. This doesn’t extend what it does (unless the GM says it does, of course) but can be fun to interact with. It can be assumed to see what you see when you’re wearing it, but can’t provide assistance with ability checks without related upgrades.
\end{minipage}\hfill
\begin{minipage}{0.48\textwidth}

\end{minipage}

Wire Acrobatics (Prerequisite: Grappling Hook) 
You can take your movement using your grappling hook instead of using it as an attack or action; you can’t move other creatures when using the Grappling Hook in this way.

The first time you use your Grappling Hook to move on a turn, the movement doesn’t provoke opportunity attacks.

\subsection*{5th-Level Upgrades}

Active Camouflage 
As an action, you can activate active camouflage, causing your Warsmith’s Armor to automatically blend into its surroundings. This lasts until deactivated. While this is active, you are considered lightly obscured, and can hide from a creature even when they have a clear line of sight to you. Wisdom (Perception) checks to find you that rely on vision are made with disadvantage.

Arcane Barrage 
Your armor collects arcane energy whenever you expend a spell slot, storing a number of charges equal to the level of the spell cast. These charges fade after 1 minute if you do not expend additional spell slots. You can store a maximum number of charges equal to half your Inventor level. You can use this upgrade to cast magic missile without expending a spell slot. This expends all charges, increasing the level of the magic missile spell by one for each charge spent. If you have an Ether Reactor, you can expend charges from it to add additional charges, up to the maximum number of charges you can have stored.

Once activated, it can’t be used again until you finish a short or long rest.

Artificial Guidance (Prerequisite: Sentient Armor) 
You upgrade the artificial personality integrated into your armor to assist with a new skill. While able to communicate with your armor, you can gain the effect of guidance when making any Intelligence or Wisdom check.

Emergency Protocol (Prerequisite: Sentient Armor) 
The intelligence in your armor will attempt to preserve your life. If you are incapacitated or unconscious and can’t take your action, it will cast a spell or take the dodge action. It can only cast spells granted by your armor’s upgrades, and can use your spell slots to do so. It can act in this way up a number of turns equal to your Intelligence modifier, but this control strains its abilities, and it can’t take control again until you finish a short or long rest. You automatically resume control if you are no longer incapacitated or unconscious.

Force Accumulator (Prerequisite: Force Blast)
 Every time you expend a spell slot of 1st level or higher, you accumulate charges to your Force Blast equal to the level of spell slot spent, up to a maximum number of charges equal to half of your Intelligence modifier (rounded down). When you deal damage with Force Blast, you can expend the accumulated charges to deal an extra 1d6 damage or to move the target 5 feet directly away from you, or any combination thereof per charge spent.

Charges not expended within 1 minute of being accumulated are lost.

Mechanical Enhancement (Prerequisite: Integrated Armor) 
You improve every aspect of yourself ever so slightly. You gain +5 feet of movement, one additional hit point per Inventor level, and +1 to Strength, Dexterity, and Constitution saving throws.

Reactive Plating (Prerequisite: Warplate) 
Your armor partially deflects incoming blows. You can use your reaction when hit by an attack that deals bludgeoning, piercing, or slashing damage to reduce the damage of that attack by an amount equal to your proficiency bonus.

Resistance (You can apply this upgrade multiple times.) 
You tune your armor against certain forms of damage. Choose acid, cold, fire, force, lightning, necrotic, radiant, or thunder damage. While wearing your armor you have resistance to that type of damage. If you apply this upgrade more than once you must choose a different damage type.

Sealed Suit (Prerequisite: Warplate) 
As a bonus action on your turn you can environmentally seal your Warplate, giving you an air supply for up to 1 hour and making you immune to poison (but not curing you of existing poisoned conditions).

In addition to the above, you are also considered acclimated to cold and hot climates while wearing your armor, and you’re also acclimated to high altitude while wearing your armor.

\subsection*{9th-Level Upgrades}

Assume Control (Prerequisite: Emergency Protocol) 
You expand the control available to your armor’s sentience. You can set additional conditions when your armor will take control of your movement and actions, and it can maintain this control for up to 1 minute.

While acting in this mode, it can take any action you could take (including attacking), but can only cast spells granted by Upgrades. It uses your ability scores.

The triggering event can be a preset condition, a verbal command, or a specified time. Your armor is immune to the charmed, blinded, frightened, paralyzed, and poisoned conditions, and doesn’t suffer from exhaustion, ignoring these effects if you are under them.

If you have Warplate, it can act in this way even when you are not wearing it (or are dead). When doing so, it uses the game statistics of animated armor with the following modifications: It has an intelligence of 12 and it can cast any spells that come from Upgrades with a DC equal to your spell save DC.

This counts as a use of your Emergency Protocol, and it can’t control the armor again (in either way) until you finish a short or long rest.

In non-combat situations in which you are not wearing your armor, your armor’s sentience can exert a lesser control of the armor to move it about and perform simple tasks. During this time it has the statistics of an unseen servant, though it looks like your suit of armor. Its ability to do this is not limited, but it takes 1 minute to assume this kind of control of your armor, and it can only do so while you are not wearing it. If reduced to zero hit points, the armor is incapacitated until you repair it during a long rest, but it can still serve as armor.

\begin{minipage}{0.48\textwidth}

\end{minipage}\hfill
\begin{minipage}{0.48\textwidth}
Optional Control

With the Assume Control upgrade, the armor taking control on Emergency Protocol becomes optional, though if you do not choose for it to activate, it can’t activate later unless the condition was previously specified (e.g. “take control if I fail two death saving throws”).
\end{minipage}

Brute Force Style 
The strength imparted by your armor gives the force of blows the devastating power of a more skilled combatant. You can select a Fighting Style from Duelist, Great Weapon Fighting, or Two-Weapon Fighting and gain the effect of that Fighting Style while wearing your armor.

Ether Reactor 
You construct an ether reactor that feeds your armor, powering its upgrades with arcane energy. The Ether Reactor has 6 charges, and can be used to power upgrades that cast spells, even upgrades that would normally only recharge after a rest can instead be cast by using charges, spending 1 charge per level of the spell you are casting. You can cast spells at a higher level by expending more charges. It regains all charges at the end of a long rest.

If your Ether Reactor has no charges left, you can overdraw your Ether Reactor for one last burst of power, but your armor is temporarily immobilized and your speed while wearing your armor becomes zero for a number of rounds equal to the level of the spell cast. Once you overdraw it, you can’t do this again until you finish a long rest.

Head-Up Display (Prerequisite: Arcane Visor, Sentient Armor) 
You can delegate displaying and tracking things in your sight to your Sentient armor, granting you the following benefits:

\begin{itemize}
  \item When a creature attempts to take the Hide action against you, you can make an active Wisdom (Perception) check to contest its Dexterity (Stealth) check as a reaction.
  \item When making a Dexterity saving throw against an attack you can see, you can make an Intelligence saving throw instead.
  \item When a creature hits you with a ranged attack roll, you can cast true strike as a reaction targeting that creature.
\end{itemize}

Phase Suit (Prerequisite: Warskin or Warsuit) 
You gain the ability to cast misty step and blink while wearing your Warsuit. Additionally, as an action, you can become intangible, and move through creatures or objects until the end of your turn. If you end your turn inside a creature or object, you are forced to the nearest unoccupied location, taking 2 force damage for each foot you are forced to move.

Once you become intangible using this upgrade, you can’t do so again until you finish a short or long rest.

Recall 
When not being worn you can hide your Warsmith’s Armor in a pocket dimension. As an action you can magically summon the armor and don it. You can use a bonus action to return the armor to the pocket dimension.

While in the pocket dimension the armor can’t be affected by other abilities and can’t be interacted with in any way.

\subsection*{11th-Level Upgrades}

Cloaking Device (Prerequisite: Active Camouflage) 
If you do not move during your turn with Active Camouflage engaged, you can use your reaction to take the Hide action using an Intelligence (Stealth) check. You make this check with disadvantage if you are within 5 feet of another creature or you attacked during your turn.

You can overload your camouflage to cast greater invisibility without expending a spell slot.

Once you do this, you can’t do this again until you finish a long rest.

Flash Freeze Capacitor (Incompatible with other Capacitors) 
You can store arcane energy, releasing it in a torrent of freezing energy. As an action, you can cast cone of cold without expending a spell slot or charges. The affected area becomes difficult terrain until the end of your next turn.

Once you use this upgrade, you can’t use it again until you finish a long rest.

Flight 
You integrate a magical propulsion system into your Warsmith’s armor. While wearing your Warsmith’s armor you have a magical flying speed of 30 feet.

Integrated Attack (Prerequisite: Integrated Armor or Warplate; incompatible with Iron Grip) 
You integrate a melee weapon into your Warsmith’s Armor. When you apply this upgrade you must have a weapon to integrate, and you must choose where on your armor the weapon is located. The weapon can’t have the heavy property. You are proficient with this weapon. As a bonus action you can activate the weapon.

You must treat it as though you are wielding it with one hand, but you can’t be disarmed of it. Once activated, you can use this weapon when you take the attack action, and it doesn’t require the use of a hand or your Warplate Gauntlet. You can apply this upgrade multiple times, selecting a new weapon and new location on your armor to install it.

When you activate your integrated weapon, you can immediately make one attack with it. While it is active, if you take the attack action on your turn, you can make an one additional weapon attack with your integrated weapon using your bonus action.

Iron Muscle (Prerequisite: Integrated Armor or Warplate) 
You reinforce the structure of your armor, giving it the strength of giants. While wearing your armor, your current Strength ability score is increased by 2, and your maximum Strength ability score becomes 24.

Lightning Rod (Prerequisite: Lightning Channel) 
Whenever you cast lightning charged* , you can treat the spell as if it was cast by a spell slot one level higher.

Powered Charge 
As an action, you activate a booster allowing you move up to 40 feet in a straight line. During this movement, you can make a Strength (Athletics) check to grapple any creature you would collide with (you still require a free hand to grapple a creature as normal; if you do not have a free hand, the check automatically fails and you collide with the target). On success, you do not stop and it is carried with you for the remaining distance or until your movement is stopped by a creature or wall.

At the end of the movement, any creature you collided with or are grappling takes 1d6 bludgeoning damage for each 5 feet you traveled. If your movement is stopped by a wall, and creature you are grappling must make Constitution saving throw. On failure, they are stunned until the end of their next turn.

Once activated, it can’t be used again until you finish a short or long rest.

Power Slam Capacitor (Incompatible with other Capacitors) 
You store up kinetic energy, and unleash it in a mighty bound. As an action, you can jump up to your entire walking speed and cast shockwave* without expending a spell slot or charges upon landing.

Once you use this upgrade, you can’t use it again until you finish a long rest.

\subsection*{15th-Level Upgrades}

Heavy Plating (Prerequisite: Warplate) 
You install special heavy plating, giving you resistance to bludgeoning, piercing, and slashing damage from non-magical sources while wearing your Warplate.

Iron Grip (Prerequisite: Iron Muscle; incompatible with Integrated Attack) 
You improve your grip strength and control of one of your gauntlets beyond the limits of flesh. While wearing your suit, you gain the following benefits:

\begin{itemize}
  \item You can wield Large-sized weapons without a penalty as a Medium-sized creature.
  \item You can wield two-handed or versatile weapons in one hand and still treat them as if wielded with two.
\end{itemize}

\begin{minipage}{0.48\textwidth}
Large Weapons

How GMs handle large-sized weapons varies. Consult with your GM for the stats of a large weapon. Typically approaches are to either double the damage dice, or increase the damage by 1d4, but other approaches exist between the two.
\end{minipage}\hfill
\begin{minipage}{0.48\textwidth}

\end{minipage}

Phase Engine (Prerequisite: Warskin or Warsuit) 
When you are attacked, you can use your reaction to become intangible, causing that attack to miss if it is a nonmagical attack, or to have disadvantage if it is a magical attack.

Once you do so, you can’t do so again until you finish a short or long rest, or until you teleport or enter the ethereal plane.

Sun Cannon 
You install a sun cannon into your Warsmith’s armor, allowing you to unleash devastating solar laser blasts. As an action, you can cast sunbeam without expending a spell slot.

Once you use this ability, you can not use it again until you finish a long rest.

Virtual Interface (Prerequisite: Sentient Armor) 
When you use Artificial Strength to raise your Strength ability score, you no longer lower your Intelligence ability score below your natural maximum (not counting Sentient Armor).

\subsection*{Optional Feature: Large Suits}

\begin{minipage}{0.48\textwidth}
Some players will want to build a larger suit, the Piloted Golem option below allows you to do that. The player and the GM should give some consideration to the consequences of being a Large-sized creature. It has advantages but considerable drawbacks that should not be ignored.
\end{minipage}\hfill
\begin{minipage}{0.48\textwidth}
Large Inconvenience 
This is an optional route that can offer substantial benefits; it would generally be advisable to not pull punches on the drawbacks of being a Large-sized creature (difficulty gaining cover, in tight areas, etc). Discuss what sort of challenges you might face with this. Consider if a Piloted Golem—essentially a walking war tank—would have trouble interacting with things created for Medium-sized creatures: ladders, potion bottles, and more.
\end{minipage}

Iron Aura (Prerequisite: Piloted Golem) 
Building defensive systems into your armor, you project a deflecting aura. Creatures of your choice within 20 feet of you gain half cover against ranged attacks that originate from outside the range of this aura.

Iron Fortress (Prerequisite: Piloted Golem) 
You count as three quarters cover for creatures within 5 feet (so long as you are between them and the attacker). Additionally, you can’t be moved against your will while in contact with the ground.

Piloted Golem (Prerequisite: Warplate, 9th-Level Inventor, incompatible with Collapsible and Flight) 
You enlarge your Warplate, turning it into a piloted mechanical golem. Your size category when wearing the armor increases by one.

You have advantage on Strength saving throws and Strength checks against creatures the same size as you or smaller than you, but you have disadvantage on Dexterity saving throws and Dexterity checks (including initiative rolls). At the end of a short or long rest, your Warplate gains temporary hit points equal to your Inventor level.

You are no longer a valid target for spells that require a humanoid target, but can be targeted by spells that require a construct target while wearing this armor.

Self-Repair Matrix (Prerequisite: Piloted Golem) 
At the start of each of your turn you gain temporary hit points equal to your proficiency bonus.

Sentry Mode (Prerequisite: Piloted Golem) 
As a bonus action deploy stabilizers, reducing your speed to zero but granting your better stability and focus. Until you end the effect as an action, your reach and range are both doubled with all weapons and you can make an opportunity attack against a creature that moves more than 5 feet within range of one of your equipped weapons (melee or ranged).

Shield Arm (Prerequisite: Piloted Golem) 
You integrate a shield into one of your armor’s arms. You have proficiency with that shield, and can deploy it as a bonus action. It requires the use of that arm while deployed.

Additionally, when you shove a Medium or smaller creature while you have the shield deployed, you can shove it by slamming it with the attached shield causing it to take damage equal to your Strength modifier.

\begin{itemize}
  \item Inventor Specialization: Warsmith
  \item Feature: Warsmith Upgrades
\end{itemize}

\section*{Fleshsmith}

% [Image Inserted Manually]

A Fleshsmith is a discomfiting presence in a group. Even when encountered in the most ideal circumstances, they simply have a lingering gaze that says "How could I improve that?" when they look at you.

A Fleshsmith is a peculiar path of Inventor that has turned their creative talents... inward. They seek to understand and improve the limitations of flesh. All too frequently they place a slightly different value on aesthetics than others might, tending to find beauty primarily in efficiency.

A Fleshsmith tends to be more eccentric than inclined to any particular alignment, though even the most benevolent ones may find that the only reason their activities wouldn’t be illegal is that no one considered them possible.

\subsection*{Fleshsmith's Proficiency}

When you choose this specialization at 1st level, you gain proficiency in the Medicine skill, as well as proficiency with Leatherworker’s tools (it’s best if you don’t think about the details on that one too much...)

Additionally, you can make all Medicine checks as Intelligence (Medicine) checks. When you take an Intelligence (Medicine) check to stabilize a creature, they regain 1 hit point.

\subsection*{Thesis of Flesh}

At 1st level, you select an approach to your work, a specialization to your art form. Select one of the following options:

\begin{minipage}{0.48\textwidth}
Perfection of Form

Nature had its chance to make your form, now it’s your turn to improve it. When you take this path, you gain the Fleshcrafted Mutation upgrade, and it doesn’t count against your upgrade total. Additionally, when an upgrade calls to use your spell save DC, you can instead use 8 + your Constitution modifier + your proficiency bonus.

When you take the Attack action, you can use your bonus action to make a single additional attack with this upgrade. You can make this additional attack a number of times equal to your Inventor level. You regain all uses at the end of a long rest.

Perfection of Creation

Why let the gods have all the fun? You’ve created life. When you take this path, you gain the Adorable Critter upgrade, and it doesn’t count against your upgrade total.

Your Adorable Critter gains temporary hit points equal to your Inventor level + your Intelligence modifier each time you finish a short or long rest. Your Adorable Critter gains a natural weapon dealing 1d6 + your Intelligence modifier piercing damage, though as normal it can’t independently take the Attack action. You can apply any upgrade you take to your Adorable Critter instead of yourself.

As an action, you can cause your Adorable Critter to move up to 10 feet and take the Attack action (requiring no action from the familiar). Additionally, you can also do this as a bonus action a number of times equal to your Inventor level and you regain all uses at the end of a long rest.
\end{minipage}\hfill
\begin{minipage}{0.48\textwidth}
Perfection of Mind

You know that perfection is an aspect of knowledge, a perfect understanding of the mechanics of the body, inside and out. But particularly inside. You gain expertise in the Medicine skill. Additionally, you gain the Dissection upgrade, and it doesn’t count against your upgrade total.

You have a pool of d8s equal to your Inventor level. When you restore hit points to a creature or use an Intelligence (Medicine) check to deal damage to a creature, you can expend these d8s to restore additional health or deal additional damage. You can spend a number equal to your proficiency bonus at a time. You regain this pool of d8s at the end of a long rest.

Perfection of Technique

Others may call themselves “flesh smiths” or “flesh crafters”, but you are a flesh artist. Flaying, dicing, deboning, you’ve mastered it all. When you select this thesis option, you gain proficiency with martial weapons and gain the Flaying Hook upgrade, and it doesn’t count against your upgrade total. You can integrate your Flaying Hook into another weapon, attaching its heavy chain to the hilt of your weapon, allowing you to attack with either weapon, and being considered to be wielding both when wielding one. Any magical bonus to attack and damage rolls the attached weapon is also applied to the Flaying Hook (if higher).

When you use your Flaying Hook to pull a target or pull yourself toward a target, you can use your bonus action to make a single weapon attack against it. You can do this a number of times equal to your Inventor level and you regain all uses at the end of a long rest.
\end{minipage}

\subsection*{Uncanny Vitality}

Starting at 3rd level, your body has such vitality that it is constantly able to restore itself. While you have 1 or more hit points, at the start of your turn, you can choose to expend a Hit Die and regain hit points equal to the value rolled + your Constitution modifier (as normal for expending a Hit Die). If you have zero hit points, you can use this feature at the end of your turn.

Additionally, you regain Hit Dice equal to your Constitution modifier on a short or long rest (in addition to normal Hit Dice recovery on a long rest).

If you are missing any limbs at the start of a long rest, at the end of the long rest the missing limbs are regenerated.

\subsection*{Arcane Bioengineering}

Additionally at 3rd level, you can use Arcane Retrofit to transmute a bonus to attack or damage rolls on a weapon to natural weapons gained from this subclass.

\subsection*{Extra Attack}

Beginning at 5th level, you can attack twice, instead of once, whenever you take the Attack action on your turn.

\subsection*{Perfection of Thesis}

Starting at 14th level, your understanding of your Thesis of Flesh grows; you regain uses of the resource it provides equal to your Intelligence modifier when you finish a short rest.

\section*{Fleshsmith upgrades}

\subsection*{Unrestricted Upgrades}

Acid Gland 
Realizing that putting something that oozes acid inside yourself could not possibly go wrong, you do just that. As an action, or as an attack as part of the Attack action, you can spew acid, making a ranged spell attack against a target within 30 feet. On a hit, you deal 3d8 acid damage.

Alternatively, as an action, or as an attack as part of the Attack action, you can spray all creatures in a 30-foot cone. Each creature in the area must make a Dexterity saving throw against your spell save DC, taking 3d4 acid damage on failed save.

Once you use either option, you can’t use this upgrade again until you finish a short or long rest. The damage increases by one die at 5th level (4d8 or 4d4), 11th level (5d8 or 5d4), and 17th level (6d8 or 6d4).

Adorable Critter 
You experiment on creating an adorable critter. You create— or modify—a Tiny CR 0 creature. This creature serves as a familiar as per the find familiar spell, but doesn’t disappear when reduced to zero hit points, and simply becomes unconscious. It can’t fully die unless destroyed. If it starts its turn unconscious, it regains its full hit points and regains consciousness. It can do this a number of times equal to your Intelligence modifier.

You can resuscitate it (or rebuild it, as necessary) at the end of a short or long rest should anything untoward happen to it without expending a spell slot or material component. Its creature type is construct.

Better Eyes (You can select this upgrade multiple times.) 
Your eyes did not see everything you wanted them to, so you replace them with eyes that do. You gain a benefit to your vision, selecting one of following enhancements:

\begin{itemize}
  \item Blindsight (10 feet)
  \item Darkvision (60 feet),
  \item The ability to see clearly twice as far as your natural vision range.
  \item Proficiency in the Perception skill.
\end{itemize}

If you select this upgrade again, you must select a different benefit.

Brimstone Bladder 
Why should dragons have all the fun? You can now exhale fire. As an action, or as an attack as part of the Attack action, you can exhale a gout of flames in a 30-foot line or 15-foot cone. Creatures in the area must make a Dexterity saving throw against your spell save DC, or take 3d6 fire damage. You can expend a spell slot to empower this fire breath, dealing a number of additional d6 equal to the level of the spell slot spent.

Once you use the option, you can’t use it again until you finish a short or long rest. The damage increases by 1d6 at 5th level (4d6) and again at 11th level (5d6), and 17th level (6d6).

Dissection 
If you are holding a melee finesse weapon you have proficiency with, as an action you can make an Intelligence (Medicine) check against a creature within reach with a DC equal to its armor class. If you succeed on the check, you can deal damage to that creature equal to your weapon's damage dice plus your Intelligence modifier. On a roll of 20, treat the damage dice as you would on a critical hit.

If you could normally make more than one weapon attack as part of the Attack action, you can deal a number of additional dice equal to the number of attacks you could normally make as part of an Attack action when taking this action (for example, if you have the Extra Attack feature and take this action with a dagger, you would deal 2d4 + your Intelligence modifier damage).

Expertise to Hit

Dissection can generate very high hit chance due to its nature, but deals less damage than attacking twice with Extra Attack would (as it only adds your modifier once), making it a trade off.

Crushing Grip 
You don’t let things go once you have them in your grip. Creatures that are grappled or restrained by you have disadvantage on attempts to escape the condition.

Additionally, you can choose to apply damage equal to your Strength modifier to any creature that starts its turn grappled or restrained by you.

Death Flail (Prerequisite: Perfection of Technique) 
Rather than cast out the Flaying Hook, you master the art of spinning your attached martial weapon by the chain in a windmill of death. Your weapon gains the Reach property when used in this way.

Additionally, instead of attacking a target with it normally, as an action, you can spin the weapon wildly in giant deadly sweeps. Up to four creatures within 10 feet must make a Dexterity saving throw against a DC of 8 + your Strength modifier + your proficiency bonus. On failure they take damage equal to your weapon’s damage dice + your Strength modifier of your weapon’s damage type.

If you have the Spiked Chain upgrade and hit one or more targets, you can select one of them to be affected by the special property of Spiked Chain.

Extra Eyes 
Why only see one direction? You add extra eyes. You gain proficiency in the Perception skill. If you are already proficient in the skill, you add double your proficiency bonus to checks you make with it. In addition, you have advantage on Wisdom (Perception) checks that rely on sight.

Flaying Hook 
A metal hook attached to a chain. You are proficient with this melee weapon, and it has the Special property. It deals 1d6 piercing damage.

\begin{itemize}
  \item Special. This weapon has a range of 20 feet. When you hit a target that is more than 5 feet away from you, if the target is Medium or smaller, you can make a Strength (Athletics) check contested by its Strength (Athletics) to pull the target toward you. On a successful check, a Small or smaller target is pulled to you, and a Medium target is pulled half the distance (rounding up) toward you. The creature takes 1 additional damage for each 5 feet it is pulled.
\end{itemize}

If the creature is Medium or larger, when you hit a target more than 5 feet away from you you can pull yourself up to 10 feet toward the target.

Fleshcrafted Enhancement (Prerequisite: Applicable Fleshcrafted Mutation) 
You use your knowledge of inhuman anatomy and the twisted powers you gain from it to formulate an unnatural enhancement of your Fleshcrafted Mutation, granting it an additional power. The table below shows which enhancements can be applied to which mutations.

Each enhancement has an empowered effect. An empowered effect automatically occurs when you roll a 20 to hit with your mutation on an attack with the enhancement, or once per turn on your turn, you can manually activate the empowered condition by expending a spell slot of 1st level or higher.

\begin{tabularx}{\textwidth}\toprule
{}XXXXX}
\midrule
Enhancement & Qualified Mutations & \multicolumn{2}{c}{Effect} & \multicolumn{2}{c}{Empowered Effect} \\
\midrule
Vampiric & Extra Fangs & \multicolumn{2}{c}{Once per turn, on your turn, when dealing damage with your Extra Fangs, you can drain the life from your victim adding 1d4 necrotic damage. If the target is a living creature over CR 1/4, you regain hit points equal to the necrotic damage dealt.} & \multicolumn{2}{c}{You add your Constitution modifier to the hit points regained.} \\
\midrule
Infernal & Extra Fangs, Extra Claws & \multicolumn{2}{c}{Once per turn, on your turn, when dealing damage with your Fangs or Claws you can ignite your claws with infernal flames, the attack deals an extra 1d6 fire damage.} & \multicolumn{2}{c}{You deal an extra 1d6 fire damage.} \\
\midrule
Venomous & Extra Fangs, Extra Claws & \multicolumn{2}{c}{Once per turn, on your turn, when you hit a creature with your natural weapon you can inject venom, dealing an extra 1d8 poison damage.} & \multicolumn{2}{c}{The target must make a Constitution saving throw against your spell save DC or become poisoned for 1 minute. They can repeat their saving throw at the end of each of their turns.} \\
\midrule
Razor & Extra Fangs, Extra Claws & \multicolumn{2}{c}{The damage die of your natural weapon becomes one step higher (for example, Extra Fangs go from 1d10 to 1d12).} & \multicolumn{2}{c}{You can maximize one die of your choice after rolling damage for this mutation.} \\
\midrule
Relentless & Extra Arm, Extra Tentacle & \multicolumn{2}{c}{You gain an additional reaction you can only use to make an opportunity attack with that Fleshcrafted Mutation.} & \multicolumn{2}{c}{You can make an additional single attack with your Fleshcrafted addition. This attack only deals the weapon damage of the attack, and doesn’t add your modifier to the damage dealt.} \\
\midrule
\end{tabularx}

One Enhancement
 Note that unless otherwise specified, an Upgrade can be taken only once. In this case, all the enhancements are effectively mutually exclusive.

Empowered Relentless Attacks
 Your Empowered Relentless attack will deal either the normal weapon die of the natural weapon, or the weapon die of a weapon held in an extra arm. This can be a two handed weapon only if you have two extra arms.

Fleshcrafted Mutation (You can select this upgrade multiple times.) 
You enhance your body by crafting or mutating a new part through grotesque experimentation. Select one of the following mutations:

\begin{tabularx}{\textwidth}\toprule
{}XXXXXXXX}
\midrule
Mutation & \multicolumn{8}{c}{Description} \\
\midrule
Extra Arm & \multicolumn{8}{c}{You get an extra arm on your body. This arm is capable of doing things an arm can do, like holding and hitting things.} \\
\midrule
Extra Claws & \multicolumn{8}{c}{You gain a natural weapon that deals 1d8 slashing damage, and counts as a light finesse weapon. You have proficiency with this weapon, and are considered to be holding this weapon in one hand for the purposes of attacking.} \\
\midrule
Extra Fangs & \multicolumn{8}{c}{Your mouth becomes a natural weapon that deals 1d10 piercing damage. You have proficiency with this weapon.} \\
\midrule
Extra Tentacle & \multicolumn{8}{c}{You gain a tentacle appendage which is a natural weapon that deals 1d6 bludgeoning damage with the reach property. You have proficiency with this weapon. Using this tentacle, you can make a Strength (Athletics) check or object interaction with a reach of 10 feet.} \\
\midrule
\end{tabularx}

Extra Arm \& Perfection of Form

When attacking with a Perfection of Form with an extra arm, you can make an attack with a weapon held in that arm. You only make a weapon attack with two hands if you have two or more extra arms to use for it.

Field Surgery 
Some say that not all medical problems require surgery. Not you. As an action, you can repair a body to its natural limits far more quickly than its natural healing. You can repair a willing creature within 5 feet of you, allowing them to expend hit dice up to half your Inventor level (rounded up). Each Hit Die spent is rolled as normal, but you can add the higher value of their Constitution modifier or your Intelligence modifier.

Fix Flesh 
Your expertise in the working of flesh makes you an artisan of fixing broken creatures. When you cast cure wounds you restore an additional amount of health equal to your Intelligence modifier.

Forbidden Knowledge 
You delve into the arcane mechanics of how bodies work. Otherwise known as necromancy. You learn additional spells as shown on the table below:

\begin{tabularx}{\textwidth}\toprule
{}XXXXXXXX}
\midrule
Inventor Level & \multicolumn{8}{c}{Spells Learned} \\
\midrule
3rd & \multicolumn{8}{c}{inflict wounds} \\
\midrule
5th & \multicolumn{8}{c}{blindness/deafness, gentle repose} \\
\midrule
9th & \multicolumn{8}{c}{invest life*, vampiric touch} \\
\midrule
13th & \multicolumn{8}{c}{blight} \\
\midrule
\end{tabularx}

You can cast inflict wounds without expending a spell slot, after which you must finish a long rest before you can cast it without expending a spell slot again. Starting at 5th level, you can choose to cast blindness/deafness without expending a spell slot instead.

Horrifying Abomination (Prerequisite: At least 3 upgrades modifying your body) 
The perfect form you have crafted intimidates inferior beings. You gain proficiency in the Intimidation skill, and if you are already proficient in it, you can add twice your proficiency bonus.

Massive Mutation 
You develop a method to suddenly mutate. As a bonus action, you can become a Large-sized creature. Creatures of your choice within 30 feet of you that witness this must make a Wisdom Save against your spell save DC or become frightened of you until the end of their turn.

While you are large, you deal 1d4 additional damage with your natural weapons and weapons sized for Medium creatures, and you have advantage on contested Strength (Athletics) checks against Large or smaller creatures. You can revert to normal size as a bonus action.

You can spend a number of rounds equal to your Constitution modifier + your Intelligence modifier as a Large creature before you must revert and can not become Large again until you finish a short or long rest. After a short or long rest, you regain all rounds of duration you can remain as a large creature.

Mutating Mastery 
After so many little adjustments, you find that your form is quite flexible to your needs. You gain the ability to cast alter self without expending a spell slot.

Once cast in this way, you can’t cast it again with this upgrade until you finish a short or long rest.

Secondary Life Organs 
Realizing the fragility of mortal life, you modify yourself with additional necessary functions. You no longer take extra damage from critical hits.

Additionally, when you make a death saving throw, you can replace the results of that roll with a 20. Once you do this, you can’t do this again until you finish a long rest.

Spiked Chain (Prerequisite: Flaying Hook) 
You lace the chain of your Flaying Hook with spikes. When you attack a creature within 5 feet with your Flaying Hook, you can opt to make a Spiked Chain attack instead. This attack deals 2d4 piercing damage on hit, and has the Special property.

\begin{itemize}
  \item Special. When you hit a creature with this weapon, that target loses 10 feet of movement (hindered by the chain) and takes 1d4 damage each time they move 5 feet, up to 2d4 damage (for 10 feet of movement). You can’t use this weapon against another creature until the start of your next turn, but have advantage on opportunity attacks against that target. If it doesn’t move before the start of your next turn, you reel your chain back in, dealing 1d4 slashing damage to that target.
\end{itemize}

Subdermal Plating 
Exoskeletons are bulky, and endoskeletons just don’t offer enough protection, so you compromise. You gain natural armor granting you a base AC of 17 (your Dexterity modifier doesn’t affect this number). This increases to 18 when you reach 5th level. If you are using a shield, you can apply the shield’s bonus as normal.

Scales \& Shells

Like with many upgrades, what this upgrade really represents is a base AC of 17. Anything else can, within reason and with consultation of your GM, be adjusted. This upgrade could be "Heavy Scales" or "Shell", or any other version of natural armor that could plausible provide heavy armor.

Toxicity 
Copying certain frog species, you make your blood poisonous. Once per turn, when a creature deals piercing or slashing damage to you while within 5 feet of you, they take poison damage equal to your Constitution modifier. If damage is from a biting attack, they take twice as much damage.

You may also opt to make yourself take on a different skin hue, to let creatures know they shouldn’t eat you.

Unnatural Health 
You’re a shining beacon of vitality. You gain an additional 1 maximum hit point for each level of Inventor you have.

Additionally, when you roll a 1 or 2 on a Hit Die for recovering health, you can reroll the die and must use the new roll, even if it is a 1 or a 2.

\subsection*{5th-Level Upgrades}

Corrosive Critter?! (Prerequisite: 5th level, Perfection of Creation) 
Your adorable critter’s natural weapon deals an extra 1d6 acid damage on hit.

Life Merchant 
You can share your considerable vitality with other creatures. You are a generous person, after all. You learn the spell invest life* , and you can cast invest life* without expending a spell slot.

Once you cast in this way, you can’t cast it again with this upgrade until you finish a short or long rest.

Pressure Points (Prerequisite: Perfection of Mind) 
Your extensive knowledge of anatomy allows you to target critical spots. When you deal damage to a target with a melee weapon, as a bonus action you can force them to roll a Constitution saving throw against your spell save DC.

If they fail, they suffer the effects of the slow spell until the end of your next turn. If you hit them again while they are under the effect of this feature, they become restrained until the end of your next turn. If the creature is already restrained they become stunned until the end of your next turn. If they are already stunned they become paralyzed until the end of your next turn.

If the target becomes paralyzed from these attacks or passes a Constitution saving throw against it, they become immune to this ability for 24 hours.

Reflexive Twitch (Prerequisite: Extra Arm) 
When you take damage from a target within 5 feet of you, you can use your reaction to reflexively attack them with your Extra Arm. This attack doesn’t add your Strength or Dexterity modifier to damage dealt, unless that modifier is negative.

Safe Revival Technique (Prerequisite: Field Surgery) 
You can cast revivify without expending a spell slot or material component. For some reason the target gains a level of exhaustion, and is frightened of you for one minute upon reviving.

\subsection*{9th-Level Upgrades}

Devouring Maw 
Due to the inefficiency of having only one intake port, you build an additional one, properly equipped with razor sharp fangs and a special property. This maw becomes a natural weapon that deals 1d10 piercing damage.

If you hit with an attack with this maw, you can choose to make a grapple against the target as a bonus action. Targets grappled by this maw take 1d4 piercing and 1d4 acid damage at the start of their turn.

Attacks with this maw are made with advantage against targets it is grappling, but it can’t be used to attack other creatures while grappling a creature.

If you have the Fleshcrafted Enhancement upgrade, you can apply a benefit from it to this Maw. This can be a different selection than the one applied to your Fleshcrafted Mutation.

Extreme Mutation (Prerequisite: Mutation Mastery) 
You can push your mutation to the limits and well beyond using a touch of magic. You learn the spell polymorph , but unless you know this spell from another source, you can only target yourself. Additionally, you can cast this spell without expending a spell slot, but once you do so, you can’t do so again until you finish a long rest.

Life Void (Prerequisite: Life Merchant) 
You hunger for missing vitality. After casting invest life* , the first time you damage a living creature within the next minute, you deal an extra 3d8 necrotic damage, and regain hit points equal to the necrotic damage dealt.

Massive Hulk (Prerequisite: Massive Mutation) 
You no longer have a limit to how long you can stay in your larger form.

\subsection*{11th-Level Upgrades}

...Adorable Critter? (Prerequisite: Perfection of Creation) 
As an action, your familiar can become Small, Medium, or Large. It can revert to its normal size as a bonus action. While Medium or larger, its natural weapon deals 1d12 + your Intelligence modifier piercing damage. Its Strength ability score increases by 4 for each size category larger it becomes, to a maximum of 18.

Dark Miracle 
When you use an action that restores hit points to another creature, you can expend up to 4d4 of your own hit points to increase the amount of health restored by that much. Creatures that receive this strange surge of vitality have advantage on their next attack or saving throw.

Unrelenting Predator (Prerequisite: Fleshcrafted Enhancement) 
Your Fleshcrafted Enhancement's Empowered effect no longer requires a spell slot, though it can still only be done once per turn.

Vivisection (Prerequisite: Perfection of Mind) 
When dissecting a creature using an Intelligence (Medicine) check granted by Perfection of Mind, if the creature’s remaining hit points are lower than the result of your Intelligence (Medicine) check, you can choose to vivisect the creature, killing it instantly instead of dealing damage.

If you kill a creature in this manner, you have advantage on a Medicine check to harvest organs, ingredients, or other items from it.

Wings Seem Useful 
You decide that wings seem useful, and install a pair on your back. You can shape them like any naturally functional wings, such as a bird, bat, or insect. These grant you a flying speed equal to your walking speed, so long as you are not wearing heavy armor.

\subsection*{15th-Level Upgrades}

Adaptive Response 
You attain full mastery of your body and its functionality, able to detect when it malfunctions. As a reaction to failing a saving throw against becoming blinded, deafened, paralyzed, poisoned, or infected by a disease as the result of failing a Constitution saving throw, you can end the effect, even if you would normally not be able to take reactions due to the effect.

Once you do this, you can’t do this again until you finish a short or long rest.

Best Eyes (Prerequisite: Better Eyes) 
Despite your high quality eyes, you’ve noticed some things still see better than you, so you take their eyes and use those instead. You gain truesight of 30 feet in addition to your normal vision.

Flesh Shaper 
You gain the ability to cast clone without expending a spell slot. You still require material components.

Once you cast this spell, you can’t cast it again until you finish a long rest.

\begin{minipage}{0.48\textwidth}
Uncanny Strength (Prerequisite: Strength 18 or higher; you can take this upgrade twice.) 
You don’t see what the big deal is with two handed weapons. After some small improvements to one of your arms and grip, you can hold a two handed weapon in one-hand in the upgraded hand.

Undying Fortitude 
When you drop to 0 hit points, rather than falling unconscious, you make a Constitution saving throw with a DC equal to the amount of damage over your current hit points taken. On a success, you drop to 1 hit point instead, and you remain conscious.
\end{minipage}\hfill
\begin{minipage}{0.48\textwidth}
Rules Tip: Two-Weapon Fighting

Note that this does not grant them the light property, so two-weapon fighting will not inherently work with them (unless you have an appropriate feat).
\end{minipage}

\begin{itemize}
  \item Inventor Specialization: Fleshsmith
  \item Feature: Fleshsmith Upgrades
\end{itemize}

\section*{Cursesmith}

% [Image Inserted Manually]

A Cursesmith’s power is invariably marked by the dark decisions they’ve made; some ignore this lingering darkness, pursuing the heights of power their reckless creation allows. Others leverage the twisted side effects themselves, believing there are no mistakes, only opportunities. Darker still are those that truly embrace the darkness of their twisted creations, becoming twisted creations themselves.

A Cursesmith is not always evil, but they walk a path that dangles precariously over evil, making it an easy thing to fall into. They may be lawful, but the laws they respect are rarely those of normal mortal society that might get in the way of their dark delving.

\subsection*{Cursesmith's Profiency}

When you choose this specialization at 1st level, you gain proficiency with one artisan’s tool of your choice, and learn one language of your choice from Infernal, Abyssal, Deepspeech, or Primordial.

Additionally, you have advantage on any check to determine the nature of a curse.

\subsection*{Forbidden Artifact}

At 1st level, you set yourself apart from other Inventors by seizing power that others dare not wield. You partake in a dark ritual investing enormous power and a part of your soul into a weapon. The weapon becomes permanently bound to you. You have proficiency with the weapon. If the weapon is lost or destroyed, you can reform it by drawing on its connection to your soul. This is a special ritual that takes 1 day, and summons the weapon to you if it was lost, or recreates it if it was destroyed.

When you perform the rite, select two of the following upgrades: Empowered Artifact, Abhorrent Life, Grasping Form, Twisting Reach, Necrotic Wounding, or Eldritch Eruption. You can select additional properties as Upgrades, but your Forbidden Artifact can have a maximum number of these Upgrades equal to your proficiency bonus (meaning the first time you can select an additional property is 5th level).

Abhorrent Life

Your Forbidden Artifact takes a life of its own, writhing and lashing on its own at your mental direction. You can apply your Intelligence modifier to the attack and damage rolls of the weapon where you usually apply your Strength or Dexterity modifier.

Eldritch Eruption

You can release magical blasts of power from the weapon that deal damage equal to the weapon’s damage dice. Select from cold, fire, lightning or necrotic for its damage type when you select this upgrade. In place of making an attack with this weapon, you can release this energy as a ranged spell attack. The ranged spell attack has a range of 60 feet. You can apply your Intelligence modifier to damage done with this attack.

Empowered Artifact

The weapon hums with great power. When you roll damage with an attack made with the weapon, you can reroll one of the weapon’s damage dice.

When attacking objects or structures, you can instead maximize its damage dice.

Grasping Form

Your Forbidden Artifact can twist around, binding targets. When you hit an attack with it, you can attempt to grapple the target as a bonus action. If you have the Abhorrent Life property, you can make an Intelligence (Athletics) check to initiate or sustain the grapple. You can’t attack other creatures while grappling a creature with this weapon.

\begin{minipage}{0.48\textwidth}
Necrotic Rot

The weapon is enshrouded in dark energy. Each time you hit a creature with it, it inflicts a stack of necrotic rot on the creature. At the end of that creature’s turn it takes 1d4 damage per stack of necrotic rot on it and makes a Constitution saving throw against your spell save DC. On a success, all stacks are removed. You can have a maximum number of stacks on a creature equal to your proficiency modifier.

Twisting Reach

If your Forbidden Artifact is a melee weapon stretches and flexes unnaturally in seeking its targets. Its range increases by 5 feet.
\end{minipage}\hfill
\begin{minipage}{0.48\textwidth}
Tracking Necrotic Rot

My favorite way to track this is just place the d4 next to the creature each time it gets a stack as it gives a great feeling of impending doom. If you have a less absurd collection of d4s on hand, simple tic marks can suffice.
\end{minipage}

\subsection*{Soul Investiture}

Due to your unique bond to your Forbidden Artifact, as a bonus action you can feed part of your soul into the weapon to empower it, weakening you and empowering it. While weakened in this way, you are under the effect of the bane spell. While empowered, the weapon deals an extra 1d6 necrotic damage on hit.

You can make a DC 10 Charisma saving throw as an action to reclaim the invested soul, ending the investiture on success. If you finish a long rest while your soul is invested into the weapon, the number of Hit Dice you regain is halved.

Drawbacks of Investiture 
Soul Investiture is usually not a good idea, and forms only part of the power budget of the weapon. When interacting with various upgrades it can be quite powerful, but it is a risky and high-cost option.

\subsection*{Cursed Path}

Starting at 3rd level, the path you’ve chosen begins to claim your body and soul: some could call it a curse, some could call further opportunity. Select a path from the following.

Curse Bearer

\begin{minipage}{0.48\textwidth}
Your soul can carry great burdens. When you are under the effect of bane, you can choose to ignore the effect for a roll. Once you choose to ignore the effect, you can’t do so again until the start of your next turn.

Additionally, when you suffer the effect of any curse from one of the upgrades granted by this class, you can suppress the effect, ignoring its negative consequences. You can do this a number of times equal to your proficiency bonus, regaining all uses on a long rest. If the curse is an ongoing effect, it is suppressed until the start of your next turn.
\end{minipage}\hfill
\begin{minipage}{0.48\textwidth}
Other Curses

At your GM’s discretion, you may be able to use this feature to shrug off the effects of other lesser curses.
\end{minipage}

When you select this path, you learn the following spells at the following levels. These spells do not count against your spells known, and are Inventor spells for you.

\begin{tabularx}{\textwidth}\toprule
{}XXXXXXXX}
\midrule
Inventor Level & \multicolumn{8}{c}{Spell} \\
\midrule
3rd & \multicolumn{8}{c}{protection from evil and good} \\
\midrule
5th & \multicolumn{8}{c}{warding bond} \\
\midrule
9th & \multicolumn{8}{c}{protection from energy} \\
\midrule
13th & \multicolumn{8}{c}{death ward} \\
\midrule
17th & \multicolumn{8}{c}{dispel evil and good} \\
\midrule
\end{tabularx}

Curse Bringer

\begin{minipage}{0.48\textwidth}
You can force your soul's burdens onto others. When you make an attack roll while under the effect of bane, if you hit the creature after rolling a 3 or 4 on the d4 from bane, the target suffers the effect of bane until the end of your next turn. This effect is extended to the end of your subsequent next turn if you hit them your Forbidden Artifact again while they are afflicted (regardless of the roll of the bane die)

Additionally, when you suffer the effect of any curse of one of the upgrades granted by this class, as a reaction, you can force another creature within 30 feet to make a Charisma saving throw against your spell save DC or suffer the same effect. You can do this a number of times equal to your proficiency bonus, regaining all uses on a long rest.
\end{minipage}\hfill
\begin{minipage}{0.48\textwidth}
When you select this path, you learn the following spells at the following levels. These spells do not count against your spells known, and are Inventor spells for you.

\begin{tabularx}{\textwidth}\toprule
{}XXX}
\midrule
Inventor Level & \multicolumn{3}{c}{Spell} \\
\midrule
3rd & \multicolumn{3}{c}{bane} \\
\midrule
5th & \multicolumn{3}{c}{disorient*} \\
\midrule
9th & \multicolumn{3}{c}{bestow curse} \\
\midrule
13th & \multicolumn{3}{c}{phantasmal kille r} \\
\midrule
17th & \multicolumn{3}{c}{killing curse*} \\
\midrule
\end{tabularx}
\end{minipage}

Curse Eater

You consume your Forbidden Artifact, destroying it and afflicting yourself with a terrible curse. You gain a melee natural weapon as the manifestation of this cursed power. On hit, this weapon deals 1d8 damage of your choice of bludgeoning, piercing, or slashing damage (selected when you gain this feature), and 1d4 necrotic damage. All effects that apply to a Forbidden Artifact apply to this natural weapon. You can apply Soul Investiture to this natural weapon.

Additionally, you can absorb any item you create with this specialization, innately gaining its effect and curse. At the start of your turn while you are not incapacitated you gain temporary hit points equal to 1 + the number of cursed items you have consumed with this ability. If you are under the effect of bane, this becomes 1d4 + the number of cursed items you have consumed with this ability.

Furthermore, when you gain temporary hit points from this feature, you can instead regain hit points equal to the temporary hit points you would have gained. You can regain hit points this way a number of times equal to your proficiency bonus, regaining all uses on a long rest.

\begin{minipage}{0.48\textwidth}
When you select this path, you learn the following spells at the following levels. These spells do not count against your spells known, and are Inventor spells for you.

\begin{tabularx}{\textwidth}\toprule
{}XXXXXX}
\midrule
Inventor Level & \multicolumn{6}{c}{Spell} \\
\midrule
3rd & \multicolumn{6}{c}{inflict wounds} \\
\midrule
5th & \multicolumn{6}{c}{darkness} \\
\midrule
9th & \multicolumn{6}{c}{mutate*} \\
\midrule
13th & \multicolumn{6}{c}{ichorous blood*} \\
\midrule
17th & \multicolumn{6}{c}{devouring darkness*} \\
\midrule
\end{tabularx}
\end{minipage}\hfill
\begin{minipage}{0.48\textwidth}
Unnatural Natural Weapons

The form of your natural weapon is largely whatever you want it to be. Slashing razor claws, chomping vampiric fangs, flailing tendrils, protruding spikes, integrated weapons, or anything you can imagine. Simply select the most appropriate damage type and pick a thematic element that suits your character.
\end{minipage}

\subsection*{Damned Affinity}

Additionally at 3rd level, magical items granted by this subclass that require attunement do not count against your attunement total if that item is cursed.

\subsection*{Extra Attack}

Beginning at 5th level, you can attack twice, instead of once, whenever you take the Attack action on your turn.

\subsection*{Unlimited Power}

Starting at 14th level, when you use Soul Investiture, you can double the amount of bonus damage gained, but take 1 necrotic damage for each turn it has been active at the end of each of your turns while it is active. This damage bypasses immunity and can’t be resisted in any way.

\section*{Cursesmith Upgrades}

\subsection*{Unrestricted Upgrades}

Abhorrent Split (Prerequisite: Forbidden Artifact with damage die of d8 or higher) 
When awakened, your Forbidden Artifact splits into multiple writhing branches. The weapon's damage dice convert to smaller increments based on the table below:

\begin{tabularx}{\textwidth}\toprule
{}X}
\midrule
Damage Dice & Awakened Damage Dice \\
\midrule
1d12 or 2d6 & 3d4 \\
\midrule
1d10 & 1d6 + 1d4 \\
\midrule
1d8 & 2d4 \\
\midrule
\end{tabularx}

It can attack other creatures even while grappling or restraining a creature with the Grasping Form artifact property, but loses one damage die (starting with its smallest) for each creature it is grappling or restraining.

Blood Rites 
Delving your dark path, you formulate a way to form temporary containers of magical power using blood. Select three ritual spells from the Wizard spell list; you gain the ability to cast these spells as Ritual Spells. You can only take spells of a level you could normally cast at the level you take this Upgrade.

To cast these spells as rituals, you require fresh blood; this can be provided by a creature that has been slain in the last hour, or by a creature (including yourself) providing it at a cost of 1d4 slashing damage (This damage can’t be reduced or prevented).

Cursing Rod 
You make a twisted magical rod that can dispense curses. Select one of rotting curse* , binding curse* , or befuddling curse* when you create this item. You can cast that curse once using it without expending a spell slot.

Once you do so, you can’t do so again until you finish a short or long rest.

Dark Magic 
You record some of the foulest effects your experiments have had, codifying them into magical spells. You learn the following spells. They are Inventor spells for you, but do not count against your spells known.

\begin{tabularx}{\textwidth}\toprule
{}XXXXXXXXXXX}
\midrule
Spell Level & \multicolumn{11}{c}{Spell} \\
\midrule
1st & \multicolumn{11}{c}{crippling agony*} \\
\midrule
2nd & \multicolumn{11}{c}{blindness/deafness} \\
\midrule
3rd & \multicolumn{11}{c}{rain of spiders*} \\
\midrule
4th & \multicolumn{11}{c}{blight} \\
\midrule
5th & \multicolumn{11}{c}{contagion} \\
\midrule
\end{tabularx}

You can cast crippling agony* without expending a spell slot, after which you must finish a long rest before you can cast it without expanding a spell slot again. Starting at 5th level, you can choose to cast blindness/deafness without expending a spell slot instead.

Form of the Fiend 
As an action, you can expend a spell slot to assume the form of a fiend from the following list. The transformation lasts for a number of rounds equal to your spellcasting ability modifier + your proficiency bonus and requires your concentration to maintain, as if concentrating on a spell, or until you drop to 0 hit points or die. The new form can be a fiend from the following table based on the level of spell slot spent. Your game statistics, excluding mental ability scores, are replaced by the statistics of the chosen fiend. You retain your alignment and personality.

You assume the hit points of your new form. When you revert to your normal form, you return to the number of hit points you had before you transformed. If you revert as a result of dropping to 0 hit points, any excess damage carries over to your normal form. As long as the excess damage doesn’t reduce your normal form to 0 hit points, you aren’t knocked unconscious.

You do not gain any spells the form can cast, legendary actions, or legendary resistances it might have. Your gear melds into the new form. The creature can’t activate, use, wield, or otherwise benefit from any of its equipment.

\begin{tabularx}{\textwidth}\toprule
{}XXXXXXXX}
\midrule
Spell Slot Level & \multicolumn{8}{c}{Fiend Options} \\
\midrule
1st & \multicolumn{8}{c}{Imp} \\
\midrule
2nd & \multicolumn{8}{c}{CR 2 Devil¹} \\
\midrule
3rd & \multicolumn{8}{c}{Bearded Devil} \\
\midrule
4th & \multicolumn{8}{c}{CR 4 Devil¹} \\
\midrule
5th & \multicolumn{8}{c}{Barbed Devil or CR 5 Devil¹} \\
\midrule
\end{tabularx}

¹ Consult with your GM for an appropriate devil that exists in your setting for these levels.

Helm of Invulnerability (You can only wear 1 Helm at a time.)
Attunement, Cursed. 
You forge a helmet, making yourself invulnerable... almost. As a reaction to taking bludgeoning, piercing, or slashing damage, you can become immune to bludgeoning, piercing and slashing damage (Including the triggering damage) until the start of your turn.

Once you use this upgrade, you can’t use it again until you finish a short or long rest.

\begin{itemize}
  \item Curse. Whenever this helm prevents damage, you take necrotic damage equal to half the damage you would have taken. This damage can’t be reduced or prevented in any way.
\end{itemize}

Helm of Madness (You can only wear 1 Helm at a time.)
 Attunement, Cursed. 
With great madness comes great power. As a bonus action, you can allow the madness to take your mind. You can immediately move up to your speed and make a single weapon attack.

Once you use this upgrade, you can’t use it again until you finish a short or long rest.

\begin{itemize}
  \item Curse. When you activate this helm, your AC and bonus to saving throws is reduced by 5 until the start of your next turn.
\end{itemize}

Helm of Omniscience (You can only wear 1 Helm at a time.)
Attunement, Cursed. 
You make a helm granting you unlimited knowledge. As a bonus action, you can gain the effect of the foresight spell until the start of your next turn.

Once you use this helm, you can’t use it again until you finish a short or long rest.

\begin{itemize}
  \item Curse. When you activate this helm, your mind becomes overloaded, causing you to take 1d6 psychic damage when you make a roll that benefits from this foresight until the start of your next turn.
\end{itemize}

Ring of Dark Investment
 Attunement, Cursed. 
You create a ring that can store a fragment of corrupted power. As an action, you can infuse an Inventor spell you know (including spells gained from upgrades) with a casting time of 1 action into the ring. You cast the spell as normal (expending the spell slot if it is 1st level or higher) but the spell doesn’t take effect, and is stored in the ring for later use. When you take the Attack action while a spell is stored in your ring, you may replace one attack with the stored spell. You may also release the stored spell directly, casting it at its normal casting time. A stored spell fades if you remove the ring, or if it is unused at the end of a long rest.

\begin{itemize}
  \item Curse. When you store a spell in your ring, your current and maximum hit points are reduced by 1 + the level of the spell (1 for a cantrip).
\end{itemize}

Ring of Gilded Lies
 Attunement, Cursed. 
You create a ring that magically enhances your social abilities—some of them. When you make a Charisma (Deception or Performance) check, you gain the effects of the guidance spell for the roll.

\begin{itemize}
  \item Curse. When you make a Charisma (Persuasion) check, roll a d4 and subtract it from the result.
\end{itemize}

Shadowed Shades 
Attunement, Cursed
 You make a set of spectacles. While wearing them, you can see normally in darkness, both magical and nonmagical, to a distance of 120 feet.

\begin{itemize}
  \item Curse. You treat all light as dim light for the purposes of vision while wearing them.
\end{itemize}

Vampiric Infusion 
You develop a dark infusion you can work upon a weapon. As a bonus action, you can expend a spell slot to infuse a weapon you touch with vampiric thirst for the next minute. When the wielder of the weapon rolls damage with a weapon attack using the infused weapon, they deal 1d8 additional necrotic damage, and regain hit points equal to the necrotic damage dealt + your Intelligence modifier. This effect can occur a number of times equal to the spell slot used on the infusion.

Whispers of the Night 
After tinkering with the properties of your Artifact, you can hear it... whisper. You can hear that too, right? You gain the ability to cast guidance and message , but the message is always delivered in a haunting, unearthly, or creepy voice.

You can amplify these whispers to cast terrifying visions* without expending a spell slot. Once you do so, you can’t do so again until you finish a short or long rest.

\subsection*{5th-Level Upgrades}

Eldritch Magic (Prerequisite: Artifact with the Eldritch Eruption upgrade) 
Select a cantrip with a spell attack roll from the Wizard spell list that deals the same damage type as your weapon (when awakened). Once per turn, you can apply the secondary effect of that cantrip to your weapon’s attack roll (for example, applying chill touch’s secondary effect to a weapon that deals necrotic damage).

Ghostgrasp Gloves (You can only wear 1 set of Gauntlets at a time.)
Attunement, Cursed. 
When you pull on these pale white gloves, your hands become ethereal, ghostlike appendages that drift and float from your wrists. You can interact with objects with your hands with a range of 10 feet, including picking up objects, opening doors, or grappling foes; you can attack with these hands at that range only if wielding light weapons.

\begin{itemize}
  \item Curse. You have disadvantage on grappling checks and the amount you can lift or drag with your hands is halved as things slip through them.
\end{itemize}

Mantle of the Beast Attunement, Cursed 
You forge a magical mantle by seeping it in the blood beasts. By dipping it in the blood of a beast slain within the past week, you can gain one trait of that beast for an hour while wearing the mantle (such as Keen Senses or Spider Climb).

\begin{itemize}
  \item Curse. When you activate its power by dipping it in the blood of a beast, the languages you can speak become the languages the beast could speak (you still can understand any language you could previously understand) until the effect ends.
\end{itemize}

Skeletal Gauntlets (You can only wear 1 set of Gauntlets at a time.)
Attunement, Cursed 
You forge a pair of gauntlets that cause your hands to appear as skeletal appendages of dark metal. While wearing these gauntlets, if you grapple a creature, at the start of its turn it takes 1d6 necrotic damage, and can’t regain hit points until the start of its next turn.

Additionally, you have advantage against any check or save that would disarm you. While wearing these gauntlets, you can cast the grip of the dead* spell.

\begin{itemize}
  \item Curse. The death grip of these gloves make it difficult to let go of things. You have disadvantage on attacks made by throwing weapons or with weapons with the ammunition property. Additionally, you have disadvantage on any ability check to throw an item.
\end{itemize}

Soul Ring 
You create a ring to store your soul. While wearing this ring, you have advantage on death saving throws. You can absorb yourself, your body, and equipment into the ring. When you do so, it gains weight equal to one tenth of your weight (including your equipment).

While another creature is wearing the ring, you can attempt to possess them. They make a Charisma saving throw. Your body vanishes and you become incapacitated for the duration, and they fall under the effect of a dominate monster spell as if you had cast it. A creature can choose to fail their save against this.

Vicious Effigy 
You learn the cruel puppetry* spell and can cast it once without expending a spell slot.

Once you use this upgrade, you can’t do so again until you finish a long rest.

\subsection*{9th-Level Upgrades}

Amulet of Exiling 
You forge a magical amulet that causes creatures to phase from reality to varying degrees of permanency. You can use this amulet to cast banishment or blink without expending a spell slot.

Once you use the amulet to cast a spell, you can’t use it to cast a spell again until you finish a long rest.

Aspect of the Damned (Prerequisite: Curse Eater, 2 upgrades consumed) 
The powers you have pilfered for your soul have tainted it... naturally this gives you great power. In addition to the powers you have gained from your upgrades, select two of the following powers to permanently gain:

\begin{itemize}
  \item Aberrant Life: You no longer count as a humanoid creature. Your creature type becomes Aberration.
  \item Creature of Darkness: Your eyes become inky black pools. You can see normally in darkness, both magical and nonmagical, to a distance of 120 feet.
  \item Hungering Soul: Once per turn on your turn while you are grappling or grappled by another creature, you can inflict 1d8 necrotic damage to them as you sap their life force through the contact. Any healing they receive that turn is reduced by half.
  \item Physical Mutation: The darkness within warps your body. You can choose an unrestricted Fleshsmith upgrade (it must be one that affects your body).
  \item Unlife: You have found yourself in an odd half-living state. You gain resistance to necrotic damage, and you no longer need to consume food.
\end{itemize}

Ring of Nightmares Attunement, Cursed. 
You forge a ring of, well, pure evil. It can bring horrifying nightmares to life in a way that you don’t quite understand, but is very effective. When you invoke a nightmare, you can cast blindness/deafness , rain of spiders* , phantasmal killer , or black tentacles without expending a spell slot.

Once you use the ring to invoke a nightmare, you can’t use it until you finish a long rest.

\begin{itemize}
  \item Curse. You can’t choose the nightmare invoked. When you invoke a nightmare, roll 1d4 and consult the following table:
\end{itemize}

\begin{tabularx}{\textwidth}\toprule
{}XXXXXXXXXXXXXX}
\midrule
d4 & \multicolumn{14}{c}{Spell Cast} \\
\midrule
1 & \multicolumn{14}{c}{blindness/deafness} \\
\midrule
2 & \multicolumn{14}{c}{rain of spiders*} \\
\midrule
3 & \multicolumn{14}{c}{phantasmal killer} \\
\midrule
4 & \multicolumn{14}{c}{black tentacles} \\
\midrule
\end{tabularx}

Shadow Vessel 
You create a container filled with an inky shadow. You can open the container to cast summon horror* without expending a spell slot. You can expend a spell slot of a higher level to cast the spell at the level of the expended spell slot plus one level.

Once you use this Upgrade, you can’t use it again until you finish a long rest.

Spell-Eating Ring 
You make a magical item that consumes magic around you. You can activate the item, casting counterspell or dispel magic against a target within 15 feet of you. When you cancel a spell in this way, you can consume the spell to recover 1d4 hit points per level of the spell.

Alternatively, if the spell would have had a non-instantaneous duration and would have affected a creature, you can transfer one spell to yourself. A spell stolen in this way doesn’t require concentration to maintain, but lasts only a number of turns equal to your Intelligence modifier or the spell’s duration, whichever is shorter.

Once you use this Upgrade, you can’t use it again until you finish a long rest.

\subsection*{11th-Level Upgrades}

Blood Cloak (You can only wear 1 cloak at a time.)
Attunement, Cursed. 
You forge a blood-red cloak that drinks blood to empower you. Unfortunately, mostly your blood. When you take damage from slashing or piercing damage while below half of your maximum hit points, you gain temporary hit points equal to half the damage taken (after taking the damage). If you hit a creature while you have temporary hit points, you can expend them, adding them to your damage roll.

\begin{itemize}
  \item Curse. If you are healed while you have temporary hit points, the temporary points are lost and the healing is reduced by the amount of temporary hit points lost (to a minimum 1)
\end{itemize}

Consuming Power Attunement, Cursed. 
You create a worn magical item—such as a set of bracers, bands or a ring—that leverages a dark bargain. Souls for power... starting with bits of your own.

As a bonus action, you allow allow it to feed off your own soul. You can decrease your Charisma ability score by up to your proficiency bonus, increasing your Strength ability score by an amount equal to your decreased Charisma. You can’t increase your Strength ability score beyond its normal maximum in this way. This lasts until you use an action to reclaim your Charisma from the item.

Whenever you slay a creature of CR 1/4 or higher, you gain +1 current and maximum Strength ability score. Both of these effects fade after 1 minute without killing a creature.

\begin{itemize}
  \item Curse. At the end of each turn your strength is increased by weapon, you take necrotic damage equal to 1 + the amount of additional maximum Strength you have gained. This damage can’t be resisted.
\end{itemize}

Ghost Cloak (You can only wear 1 cloak at a time.)
Attunement, Cursed. 
You craft a cloak that chips away at your bonds to the material world around you. At the start of your turn while wearing the cloak, you can choose to slip those bonds entirely, stepping into the ethereal plane. Moving through creatures doesn’t impose a movement penalty, and you can move through objects up to 5 feet thick as difficult terrain. If you end your turn in another creature or object's space, you are moved to the nearest available space, taking 1d10 force damage per five feet moved.

Once you activate this item, you can’t activate it again until you finish a short or long rest.

\begin{itemize}
  \item Curse. After leaving the Ethereal plane, until the start of your next turn you appear vaguely translucent, all damage you deal is reduced by half, and you have resistance to nonmagical bludgeoning, piercing and slashing damage.
\end{itemize}

Weapon Apotheosis 
When you activate Soul Investiture, your weapon deals an extra 1d6 necrotic damage.

\subsection*{15th-Level Upgrades}

Curse Numbness (Prerequisite: Curse Bearer) 
You become immune to the effects of bane , bestow curse , and hex , even when self-inflicted.

Exude Darkness 
You allow the power of your artifact to leak out. Whenever you cast a spell of 1st level or higher, you become heavily obscured by dark shadowy flames for a number of rounds equal to the level of the spell (for example, a 1st-level spell the flames shroud you until the start of your next turn). While you are shrouded in these flames, any creature within 5 feet of you that hits you with an attack takes 2d8 necrotic damage.

Pandemic of Despair (Prerequisite: Curse Bringer) 
If a target becomes affected by bane (either the spell or the effect from Curse Bringer) you can force a creature within 5 feet of it to make a Charisma saving throw against the same effect, repeating indefinitely as long as you are affecting any creature (besides yourself) with bane . If a target fails their save, they become affected by the same effect as the initial target.

True Artifact 
Long ago you forged great power; now you realize its full potential. Your cursed weapon becomes a true artifact. Any bonus to attack and damage rolls lower than +3 it has is replaced by a bonus of +3, and it gains two additional properties from the following list:

\begin{itemize}
  \item While attuned to this artifact, you are immune to disease.
  \item While attuned to this artifact, you can’t be charmed or frightened.
  \item While attuned to this artifact, you can cast a 1st or 2nd level spell of your choice from any spell list once per short rest without expending a spell slot (choose the spell when this upgrade is selected). • While attuned to this artifact, you can treat a 1 on a death saving throw as a 20.
  \item While attuned to this artifact, one of your ability scores (chosen when this upgrade is selected) increases by 1, to a maximum of 24.
\end{itemize}

When you select this upgrade, you can select the conditions by which your weapon can be destroyed. Your weapon can no longer be permanently destroyed by any other means. This means can be absurd or implausible, but must be something potentially possible.

Undying Creature (Prerequisite: Curse Eater) 
You gain twice as many temporary hit points at the start of your turn from Curse Eater (or recover twice as many hit points when recovering hit points with it). You gain temporary hit points from the feature even when incapacitated.

\begin{itemize}
  \item Inventor Specialization: Cursesmith
  \item Feature: Cursesmith Upgrades
\end{itemize}

\section*{Runesmith}

% [Image Inserted Manually]

A Runesmith is an Inventor that has narrowed their focus to working a specific language of magic - powerful runes that can channel long lasting power. These runes come in many shapes and forms, the lore behind them from many sources.

A Runesmith can be a knight, their runes splashed with gleaming power across their armor, or a scholar lending their power to their companions, marking potent runes on their weapons before standing back, or even a strange tattooed mystic, their runes tattooed across their very body.

Runesmiths are not inherently good or evil, though their rigorous attention to detail and patience tend to make them more inclined to a lawful perspective.

\subsection*{Runesmith Proficiency}

When you choose this specialization at 1st level, you gain proficiency with martial weapons as well as smith's tools or calligrapher's supplies (based on your preferred method of marking runes). You learn one additional exotic language of your choice.

\subsection*{Runic Marks}

Starting at 1st level, you can mark magical runes, imbuing the things they are marked onto with magic. A rune can be marked on a weapon, on a suit of armor, a shield, or directly onto yourself, as a type of runic tattoo.

The affected creature of a rune is the creature wielding the weapon or armor it is marked on, or you if the rune is directly marked on you. A weapon or set of armor can only bear one rune, but there is no limit to the number of runes that can be marked on a creature.

If a rune is marked on a creature, select either the weapon effect (affecting its natural weapons or unarmed strikes) or armor effect (affecting its natural armor).

You know the following runes, and can learn more from upgrades. You can mark your runes during a long rest, and can have two runes marked at a time, gaining the ability to have an additional rune marked at 3rd level (to three runes), 5th level (to four runes), and 14th level (to five runes). These last until you mark a new rune, with the oldest rune fading when you mark a new one after reaching your maximum number of runes. Upgrades that grant new runes do not increase this number, but only expand the range of runes you can mark.

Runic Effects

Each rune grants a passive ability based on what it is marked on that enhances the ability of what it is marked on, and has an active ability that can be activated by you as an action or in place of an attack as part of the Attack action while it is marked. A rune can only be activated in this way once per turn.

\begin{minipage}{0.48\textwidth}
Rune of Power

An imposing rune that speaks to raw power.

It has the following effects:

\begin{tabularx}{\textwidth}\toprule
{}XXXXX}
\midrule
Target & \multicolumn{5}{c}{Effect} \\
\midrule
Weapon & \multicolumn{5}{c}{The base damage of the weapon becomes 1d6 (if it was lower), and the weapon adds +1 to damage rolls.} \\
\midrule
Armor & \multicolumn{5}{c}{The base AC of the armor becomes 12 (if it was lower), and the armor’s AC bonus is increased by +1.} \\
\midrule
Active (Empower) & \multicolumn{5}{c}{The next attack of the affected creature before the start of your next turn deals an extra 1d8 + your Intelligence modifier force damage.} \\
\midrule
\end{tabularx}
\end{minipage}\hfill
\begin{minipage}{0.48\textwidth}
Rune of Lightning

A jagged rune that manipulates energy and controls lightning.

It has the following effects:

\begin{tabularx}{\textwidth}\toprule
{}XXXXX}
\midrule
Target & \multicolumn{5}{c}{Effect} \\
\midrule
Weapon & \multicolumn{5}{c}{The first time the weapon deals damage on a turn, it deals an extra 1d4 lightning damage.} \\
\midrule
Armor & \multicolumn{5}{c}{The affected creature’s speed increases by 5 feet.} \\
\midrule
Active: (Hasten) & \multicolumn{5}{c}{During the affected creature’s turn, it gains an additional action that can be used to attack (one weapon attack with a light weapon or unarmed strike only), Dash, Disengage, Hide, or Use an Object. The creature can’t gain multiple actions from this. Actions not used by the end of the creature’s turn are lost.} \\
\midrule
\end{tabularx}
\end{minipage}

\begin{minipage}{0.48\textwidth}
Rune of Fire

A swirling rune that represents fire and burning.

It has the following effects:

\begin{tabularx}{\textwidth}\toprule
{}XXXXX}
\midrule
Target & \multicolumn{5}{c}{Effect} \\
\midrule
Weapon & \multicolumn{5}{c}{The first time the weapon deals damage on a turn, it deals an extra 1d4 fire damage.} \\
\midrule
Armor & \multicolumn{5}{c}{When the affected creature is hit by a melee weapon attack, they can use their reaction to deal 1d4 fire damage to the attacker.} \\
\midrule
Active (Explode) & \multicolumn{5}{c}{Creatures within a 5-foot radius of the affected creature must pass a Dexterity saving throw, or take 1d8 fire damage.} \\
\midrule
\end{tabularx}
\end{minipage}\hfill
\begin{minipage}{0.48\textwidth}
Rune of Warding

A stalwart rune that wards off harm.

It has the following effects:

\begin{tabularx}{\textwidth}\toprule
{}XXXXX}
\midrule
Target & \multicolumn{5}{c}{Effect} \\
\midrule
Weapon & \multicolumn{5}{c}{If the affected creature deals damage with this weapon on their turn, they gain 1d4 temporary hit points.} \\
\midrule
Armor & \multicolumn{5}{c}{Damage taken by the affected creature is reduced by 1.} \\
\midrule
Active (Protect) & \multicolumn{5}{c}{The next time the affected creature takes damage before the start of your next turn, the damage is reduced by 1d4 + your Intelligence modifier.} \\
\midrule
\end{tabularx}
\end{minipage}

\subsection*{Runic Flare}

Starting at 3rd level, you can activate a rune as a bonus action. You can do this a number of times equal to your proficiency bonus, and regain all uses after a short or long rest.

\subsection*{Runic Path}

Additionally at 3rd level, you can specialize in a certain application of runes, selecting from the following Runic Paths:

Runic Knight

You have proficiency with weapons or armor that are marked with your runes. Your speed is not reduced by wearing heavy armor, as long as that heavy armor bears one or more of your runes.

Runic Mystic

When you mark a rune on yourself, you gain both the weapon and armor effects of it. You add your Intelligence modifier, instead of your Strength modifier, to the attack and damage rolls when you attack with unarmed strikes, and your Intelligence modifier, instead of your Dexterity modifier, to AC calculations for armor.

Runic Sage

You gain the ability to invoke spells with the magical power of the runes. You can use the runes you’ve marked to channel magic. You gain the following spells based on which runes you currently have marked.

\begin{tabularx}{\textwidth}\toprule
{}XX}
\midrule
Rune & Cantrip & Spell \\
\midrule
Rune of Fire & burn* & burning hands \\
\midrule
Rune of Lightning & shocking grasp & lightning tendril* \\
\midrule
Rune of Power & light & magic missile \\
\midrule
Rune of Warding & N/A & freezing shell* \\
\midrule
Rune of Gravity & N/A & fall* \\
\midrule
Rune of Blood & N/A & cure wounds \\
\midrule
\end{tabularx}

\subsection*{Extra Attack}

Beginning at 5th level, you can attack twice, instead of once, whenever you take the Attack action on your turn.

You can activate a rune using its active property in place of one or both attacks.

\subsection*{Twin Flares}

Starting at 14th level, once per turn, when you use the active property of a rune or glyph, you can cause the active effect of any two runes or glyphs .

\section*{Runesmith upgrades}

\subsection*{Unrestricted Upgrades}

Animate Inscription 
You mark a special rune on yourself or an item in your possession to represent a familiar. You can spend a Hit Die to give it life and cast find familiar using this rune. When you cast find familiar in this way it doesn’t require a spell slot or require material components, and the casting time is one action.

Arcane Glyph 
You learn how to translate a spell you know with a range of self into a glyph that can be marked on a creature or piece of armor. As a special 1 minute ritual, you can cast a spell you know with a casting time of 1 action, casting the spell as normal. The spell doesn’t take immediate effect, but instead becomes imbued as a special glyph on a creature or armor you are touching. This glyph has no passive effect, but when activated with an action, it casts the stored spell, originating from the creature or armor marked with this glyph, and using your spellcasting ability, spell attack, and spell save DC.

Alternatively, the creature marked with this glyph or wearing armor marked with this glyph that is aware of its presence can use this glyph to cast the spell marked within as if they cast the spell stored using a magical item. The spell originates from the creature or armor marked with this glyph, but the casting uses your spellcasting ability, spell attack, and spell save DC.

The glyph immediately fades after the spell has been cast (the spell lasts as normal). The glyph fades when the Runesmith that marked it finishes a long rest.

Channel Magic 
You can cast a spell with a range of touch on a creature marked by one of your runes, wearing armor marked by one of your runes, or wielding a weapon marked by one of your runes, regardless of the distance between you.

Additionally, you can cast any spell the rune grants, the creature marked by the rune is the point of origin of the spell

Example: Runic Lightning Bolts

This would mean that when you cast lightning bolt from Greater Rune Magic, you could cast that lightning bolt either from your current position, or from the Rogue's position, giving you great flexibility.

Glyph Magic 
You gain the ability to cast glyphs. You learn the following spells at the following levels. These spells are Inventor spells for you, but do not count against your spells known.

\begin{tabularx}{\textwidth}\toprule
{}XXXXX}
\midrule
Type & \multicolumn{4}{c}{Effect} & Level Requirement \\
\midrule
Absorption & \multicolumn{4}{c}{You learn the glyph of absorption* spell.} & — \\
\midrule
Fire & \multicolumn{4}{c}{You learn the glyph of fire* spell.} & — \\
\midrule
Frost & \multicolumn{4}{c}{You learn the glyph of frost* spell.} & — \\
\midrule
Translocation & \multicolumn{4}{c}{You learn the glyph of translocation* spell.} & 5 \\
\midrule
Nullification & \multicolumn{4}{c}{You learn the glyph of nullification* spell.} & 9 \\
\midrule
Gravity & \multicolumn{4}{c}{You learn the glyph of gravity* spell.} & 13 \\
\midrule
\end{tabularx}

Mark of Messaging 
You can place a special runic mark on a creature or item. This mark lasts until you mark it again or take a long rest. Subsequently you can cast the message spell targeting the marked creature, or a creature carrying the marked item, regardless of distance between you and the target.

Mark of Shielding 
As an action, you can place a special mark on a creature or armor. This mark lasts until you mark it again or take a long rest. While the mark is on that creature, you can cast the shield spell, affecting the creature marked, if the affected creature is hit by an attack.

\begin{minipage}{0.48\textwidth}
Rune of Blood
 An esoteric rune that binds the essence of life. It has the following effects:

\begin{tabularx}{\textwidth}\toprule
{}XXXXX}
\midrule
Target & \multicolumn{5}{c}{Effect} \\
\midrule
Weapon & \multicolumn{5}{c}{Each time the weapon deals damage to a creature, it stores one charge, up to a maximum number of charges equal to your Intelligence modifier.} \\
\midrule
Armor & \multicolumn{5}{c}{If the affected creature is hit by a critical hit or reduced to zero hit points, the rune stores one charge, up to a maximum number of charges equal to your Intelligence modifier.} \\
\midrule
Active (Revitalize) & \multicolumn{5}{c}{The affected creature can immediately expend 1 Hit Die, rolling it and regaining hit points as normal. All charges of the rune are consumed, restoring an extra 1d4 per charge consumed.} \\
\midrule
\end{tabularx}
\end{minipage}\hfill
\begin{minipage}{0.48\textwidth}
Rune of Gravity 
A complex rune that manipulates the interactions of objects to change gravity. It has the following effects:

\begin{tabularx}{\textwidth}\toprule
{}XXXXX}
\midrule
Target & \multicolumn{5}{c}{Effect} \\
\midrule
Weapon & \multicolumn{5}{c}{Marking this rune on a weapon adds or removes the heavy property. If you remove the heavy property from a melee weapon with this rune, you can also remove the two-handed property.} \\
\midrule
Armor & \multicolumn{5}{c}{Any time the affected creature would be moved against their will, they can use their reaction to stay where they are, as long as the movement is not falling.} \\
\midrule
Active (Pull) & \multicolumn{5}{c}{All creatures within 20 feet of the affected creature are dragged 5 feet toward the affected creature.} \\
\midrule
\end{tabularx}
\end{minipage}

Rune of Returning 
A weapon you mark with this rune is under the effect of the returning weapon* spell. This rune doesn’t count against the runes you can mark, though you can only have it marked on one weapon at a time.

Surging Flare 
Whenever you activate a rune’s active effect, the next attack roll you make before the end of your turn gains advantage.

\subsection*{5th-Level Upgrades}

Duplicate Rune 
For one of the runes you mark, you can mark it on two different items or creatures counting it as a single instance of marking that rune. This applies to a single application (not a single type of rune).

Gravity Brand (Prerequisite: Rune of Gravity) 
When you hit a creature with a weapon marked with a Rune of Gravity, until the start of your next turn, any movement they make moving away from you costs an additional foot of movement.

Mark of Size 
As an action, you can place a special runic mark on a creature or armor. This mark lasts until you mark it again or take a long rest. While the rune is on that creature, you can cast enlarge/reduce targeting that creature as a bonus action.

You can cast it this way without expending a spell slot once, after which you can't cast that spell in this way again until you finish a short or long rest. You can cast it again using spell slots.

Mystic Flare (Prerequisite: Runic Mystic) 
The first time you use a Runic Flare on your turn, it doesn't require a bonus action.

Paired Effects 
You can mark two runes on a single item.

Runic Aegis (Prerequisite: Runic Knight) 
You gain temporary hit points equal to your Intelligence modifier when you use a Runic Flare.

Rune Magic (Prerequisite: Runic Sage) 
You gain access to more powerful magic through your runes.

\begin{tabularx}{\textwidth}\toprule
{}XXX}
\midrule
Rune & \multicolumn{3}{c}{Spells} \\
\midrule
Rune of Fire & \multicolumn{3}{c}{scorching ray} \\
\midrule
Rune of Lightning & \multicolumn{3}{c}{lightning charged*} \\
\midrule
Rune of Power & \multicolumn{3}{c}{star dust*} \\
\midrule
Rune of Warding & \multicolumn{3}{c}{warding bond} \\
\midrule
Rune of Gravity & \multicolumn{3}{c}{fling*} \\
\midrule
Rune of Blood & \multicolumn{3}{c}{hold person} \\
\midrule
\end{tabularx}

You can cast one spell granted by this feature without expending a spell slot. Once you cast a spell in that way, you can't do so again until you finish a long rest (you can still cast them normally by expending spell slots).

Thunder Mine 
You learn the thunderburst mine* spell and can cast it without expending a spell slot.

Once you cast it in this way, you can't cast it this way again until you finish a short or long rest.

\subsection*{9th-Level Upgrades}

Efficient Language 
The number of runes you can mark at the same time increases by one.

Fire Mine 
You learn the fireburst mine* spell and can cast it without expending a spell slot.

Once you cast it in this way, you can’t cast it this way again until you finish a short or long rest.

Greater Rune Magic (Prerequisite: Runic Sage) 
The power you can access through runes grows more advanced.

\begin{tabularx}{\textwidth}\toprule
{}XXX}
\midrule
Rune & \multicolumn{3}{c}{Spells} \\
\midrule
Rune of Fire & \multicolumn{3}{c}{fireball} \\
\midrule
Rune of Lightning & \multicolumn{3}{c}{lighting bolt} \\
\midrule
Rune of Power & \multicolumn{3}{c}{aether lance*} \\
\midrule
Rune of Warding & \multicolumn{3}{c}{protection from energy} \\
\midrule
Rune of Gravity & \multicolumn{3}{c}{crushing singularity*} \\
\midrule
Rune of Blood & \multicolumn{3}{c}{invest life*} \\
\midrule
\end{tabularx}

You can cast one spell granted by this feature without expending a spell slot. Once you cast a spell in that way, you cannot do so again until you complete a long rest (you can still cast them normally by expending spell slots).

Mark of Proficiency 
You can place a special runic mark on a weapon, armor, shield, or tool. Any creature that is holding or wearing that item gains proficiency with it. This mark lasts until you mark it again or take a long rest.

Painted Bulwark 
You gain two additional maximum hit points for each rune marked on you or your gear.

\subsection*{11th-Level Upgrades}

Linguistic Structure (Prerequisite: Paired Effects) 
You can mark up to three runes on a single item.

Mark of the Hidden 
You can mark a special runic mark on a creature or item. This mark lasts until you mark it again or take a long rest.

As an action, you can cause the effect of the invisibility, arcanist’s magic aura , or nondetection spells on the marked creature or a creature carrying the marked item This doesn’t require concentration, but otherwise has the normal limitations of the spell.

You can have one of these effects active on the target at a time, and the effect ends if the target is no longer marked by the rune or carrying the item marked with the rune.

Perfected Form 
You master the language of runes allowing you to get more from less. The number of runes you can mark at the same time increases by one.

Primal Emphasis 
The damage die size of runes that deal cold, fire, or lightning damage increases by one step (for example from a d4 to a d6). This applies to both their passive and active effect.

Runic Formation 
As a reaction to a creature within 120 feet of you that is marked by one of your runes taking damage, you can expend a 2nd level or higher spell slot to transfer that damage to another willing creature marked by your one of your runes within 120 feet.

\subsection*{15th-Level Upgrades}

Glyph of Force 
You learn the spell wall of force . It is an Inventor spell for you and doesn’t count against your spells known.

Mark of Stone 
As an action, you can place a special runic mark on a creature or armor. This mark lasts until you mark it again or take a long rest. While the rune is on that creature, you can cast stoneskin targeting that creature as a bonus action without requiring any material components.

You can cast it this way without expending a slot once, after which you can’t cast it without a spell slot again until you finish a short or long rest.

Reckless Flare 
As an action, you flare all of your marked runes and glyphs of your choice, causing their active effects to take place. Once you use this ability, you can’t use the active effect of a rune flared in this way again for 1 minute.

You can’t use this ability again once used until you finish a long rest.

\begin{itemize}
  \item Inventor Specialization: Runesmith
  \item Feature: Runesmith Upgrades
  \item Glyphs
\end{itemize}

\section*{Multiclassing}

Should you want to multiclass into Inventor, the prerequisites and proficiencies are listed below:

\begin{itemize}
  \item Prerequisite: Intelligence 13
  \item Proficiencies gained: Arcana skill, one artisan’s tool of your choice, light armor, simple weapons
\end{itemize}

For the purpose of multiclassing and spell slots, add half your inventor levels (rounded down) when calculating your Spell Slots on the Multiclassing Spells Slots table (like paladin or ranger)

\section*{Psion}

% [Image Inserted Manually]

A huge orc flies across the room, smashing through tables and chairs before hitting the wall with a thunderous crash, collapsing dazed. A human, her eyes still glowing with unearthly power, tosses a few extra coins on the bar. “Sorry for the mess. If he wakes up, tell him to try picking on someone his own size next time.”

The prisoner’s expression turns from defiant to puzzled as the interrogator asks no questions, merely stares at them silently. His puzzled expression turns worried as he finds himself, unbidden, recalling where he stashed the loot. “Under the stables behind the Rusty Hook Inn” the interrogator finally speaks, as the thief’s expression becomes terrified.

“Ain’t you... cold or somethin’?” the dwarf asks the elf as they trudge through the snow, eying her simple robe and bare feet. “I just think ‘warm’” the elf replies with a distracted air. The dwarf snorts, a puff of chilled air. Typical elf nonsense. Except... they haven’t frozen to death yet. “Say... could you think some ‘warm’ this way?” the dwarf asks hopefully. To the dwarf’s surprise, the chill of the snow fades away completely. “That’s downright creepy... hey don’t stop now! Was jus’ sayin’ it was creepy.”

Psions are those who have tapped into a special otherworldly force, with the ability to actualize the power of their mind to accomplish impossible feats. Reading minds, lifting vast weights, and transcending physical limitations, they tend to inspire awe and terror in equal measure.

The exact nature of what psionic power is might be a question answered in your setting, or it might not be. Most view psionic power as coming from within, though other answers exist—anything from the leaking power of the realms beyond, to an alternate way to express magic, to an actualization of an individual’s will upon the world.

\subsection*{Powerful Minds}

Since a psion’s weapon is their mind, they are always a careless thought away from harming those around them. For a psion, the line between thinking and doing can be very thin, forcing them to keep their thoughts and emotions in check, lest their powers run amok. This leads to many of them developing odd behaviors or mannerisms to help them control their state of mind, frequently seeming quite eccentric to an outside observer.

Consider how your psion keeps their powers in check, and how much their power bleeds into their everyday life. How reading minds may affect their body language or habits. Do they add wood to the campfire from the comfort of their bedroll, or do they restrain themselves from using their power for trivial matters lest they slip up?

\subsection*{Misunderstood Power}

In a world of widespread magic, many will assume a psion to be a type of Sorcerer or Wizard, mistaking their powers for the common applications of magic. The common villager keeps only one mental category for the supernatural and creepy people that bend reality to their will. But even among magic users, a psion’s powers will frequently be found unnerving.

It is often those who do understand magic that find a psion’s abilities the most aberrant. A force that intersects the magical weave rather than obeys it, the nature of the power that the psion wields one can only contemplate with unease.

Psionics \& Magic

The mechanical interactions between magic and psionics are covered in the Psionics feature but may vary depending on your GM and setting. Psionics are still an expression of the supernatural and may still be affected by effects that suppress magical effects but are often mysterious and baffling to those familiar with magic, being a more raw interaction of the mind and weave.

\subsection*{Creating a Psion}

When creating a psion, consider your character’s background. How did you become a psion? Were you born with latent powers? Or did something happen to you granting your powers? Or did you intentionally train yourself through rigorous mental exercise to leverage them? Consider how developing a rare and mysterious power would have affected your interactions with people.

Consider what your purpose is with your newfound powers, or how you plan to use your powers. Do you intend to be taken for a Wizard casting magic, or are you freely open about being of a... different nature?

\subsection*{Quick Build}

You can make a psion quickly by following these suggestions. First, make Intelligence your highest ability score, followed by Dexterity or Constitution. Second, choose a background that lends itself to intelligence skills.

\section*{The Psion}

\begin{longtable}{p{2.5cm}\toprule
|p{2.5cm}|p{2.5cm}|p{2.5cm}|p{2.5cm}|p{2.5cm}|p{2.5cm}|p{2.5cm}|}
\midrule
Level & Proficiency Bonus & Psi Points & Psi Limit & \multicolumn{3}{c}{Features} & Psionic Talents \\
\midrule
1st & +2 & 1 & 1 & \multicolumn{3}{c}{Psionic Archetype, Psionics} & — \\
\midrule
2nd & +2 & 2 & 1 & \multicolumn{3}{c}{Psionic Talents} & 2 \\
\midrule
3rd & +2 & 3 & 2 & \multicolumn{3}{c}{Secondary Discipline, Psionic Archetype feature} & 2 \\
\midrule
4th & +2 & 4 & 2 & \multicolumn{3}{c}{Ability Score Improvement} & 2 \\
\midrule
5th & +3 & 5 & 3 & \multicolumn{3}{c}{Psionic Mastery} & 3 \\
\midrule
6th & +3 & 6 & 3 & \multicolumn{3}{c}{Psionic Archetype feature} & 3 \\
\midrule
7th & +3 & 7 & 4 & \multicolumn{3}{c}{—} & 4 \\
\midrule
8th & +3 & 8 & 4 & \multicolumn{3}{c}{Ability Score Improvement} & 4 \\
\midrule
9th & +4 & 9 & 5 & \multicolumn{3}{c}{—} & 5 \\
\midrule
10th & +4 & 10 & 5 & \multicolumn{3}{c}{Psionic Archetype feature} & 5 \\
\midrule
11th & +4 & 11 & 6 & \multicolumn{3}{c}{Innate Psionic Ability (6th level)} & 5 \\
\midrule
12th & +4 & 12 & 6 & \multicolumn{3}{c}{Ability Score Improvement} & 6 \\
\midrule
13th & +5 & 13 & 7 & \multicolumn{3}{c}{Innate Psionic Ability (7th level)} & 6 \\
\midrule
14th & +5 & 14 & 7 & \multicolumn{3}{c}{Psionic Archetype feature} & 6 \\
\midrule
15th & +5 & 15 & 8 & \multicolumn{3}{c}{Innate Psionic Ability (8th level)} & 7 \\
\midrule
16th & +5 & 16 & 8 & \multicolumn{3}{c}{Ability Score Improvement} & 7 \\
\midrule
17th & +6 & 17 & 9 & \multicolumn{3}{c}{Innate Psionic Ability (9th level)} & 7 \\
\midrule
18th & +6 & 18 & 9 & \multicolumn{3}{c}{Third Discipline} & 8 \\
\midrule
19th & +6 & 19 & 10 & \multicolumn{3}{c}{Ability Score Improvement} & 8 \\
\midrule
20th & +6 & 20 & 10 & \multicolumn{3}{c}{Ascension} & 8 \\
\midrule
\end{longtable}

\subsection*{Class Features}

As a Psion you gain the following class features.

\subsection*{Hit Points}

Hit Dice: 1d6 per psion level
 Hit Points at 1st Level: 6 + your Constitution modifier
 Hit Points at Higher Levels: 1d6 (or 4) + your Constitution modifier per psion level after 1st

\subsection*{Proficiencies}

Armor: Light armor
 Weapons: Simple weapons
 Saving Throws: Intelligence, Wisdom
 Skills: Psionics, and choose two from Deception, History, Insight, Intimidation, Investigation, Medicine, Perception, or Religion.

\subsection*{Equipment}

You start with the following equipment, in addition to the equipment granted by your background:

\begin{itemize}
  \item (a) a quarterstaff, (b) a dagger, or (c) a martial weapon (if proficient)
  \item (a) leather armor or (b) scale mail (if proficient)
  \item (a) a scholar’s pack or (b) an explorer’s pack
\end{itemize}

If you forgo this starting equipment, as well as the items offered by your background, you start with 3d4 × 10 gp to buy your equipment.

\subsection*{Psionic Archtype}

At 1st level, you pick the archetype of psion you embody, choosing from Awakened Mind, Unleashed Mind, Transcended Mind, Shaper’s Mind, Wandering Mind, or Consuming Mind, each of which are detailed at the end of the class description. Your choice grants you features at 1st level, and again at 3rd, 6th, 10th, and 14th level.

\subsection*{Psionics}

\begin{minipage}{0.48\textwidth}
Psionic Disciplines

You are granted access to a psionic discipline (such as Telepathy or Telekinesis) by your chosen archetype. A psionic discipline comes with a passive feature that expands your character’s capabilities and an active psionic power that can be modified and empowered with psi points. Additionally you can use your discipline in more detailed applications to recreate the effect of certain spells, listed at the end of the discipline description. At 3rd level you can select a second discipline, and 18th level you can select a 3rd discipline.

Psionic powers are suppressed by antimagic fields and can be dispelled with dispel magic, but are only affected by counterspell if recreating the effect of a spell. Any check required to dispel magic, counterspell or to identify a spell being cast with psionics is made with disadvantage unless the caster also has the Psionics feature. The detect magic spell will detect the usage of psionics, but not their nature: it will show up as a mysterious untyped power, even if being used to generate the effect of a spell.

When recreating a spell through a psionic effect (using a psionic discipline to cast the spell) the spell has no material or verbal components, but using any psionic ability requires somatic components and causes the psion to vibrantly glow with the otherworldly psionic energies they are controlling.
\end{minipage}\hfill
\begin{minipage}{0.48\textwidth}
Psi Points

Starting at 1st level, you gain access to psi points used to fuel psionic discipline powers and effects. You have a number of psi points equal to your psion level, and you regain all spent points when you finish a short or long rest. You can spend a number of psi points equal to half your Psion level (rounded up) at a time. For example, if you’re a 5th-level Psion, you can spend 3 psi points on a psionic power or cast a spell with a cost of 3 psi points.

Psionic Ability

Psionic powers, Psionic Talents, and spells gained through this class use your psionic ability.

Psionic ability save DC = 8 + your proficiency bonus + your Intelligence modifier

Psionic ability attack modifier = your proficiency bonus + your Intelligence modifier
\end{minipage}

\subsection*{Psionic Talents}

Starting at 2nd level, you gain access to a psionic talent allowing you to further specialize. Pick two talents from the list of psionic talents presented at the end of the class description. You can pick a new psionic talent at 5th, 7th, 9th, 12th, 15th, and 18th level. When you level up, you can replace a psionic talent you have previously selected with a different option.

\subsection*{Second Discipline}

When you reach 3rd level, you can select a second psionic discipline from the list of psionic disciplines. You can’t select a Discipline you already know. You gain all features of a psionic discipline when selecting it.

\subsection*{Ability Score Improvement}

When you reach 4th level, and again at 8th, 12th, 16th, and 19th level, you can increase one ability score of your choice by 2, or you can increase two ability scores of your choice by 1. As normal, you can’t increase an ability score above 20 using this feature.

\subsection*{Psionic Mastery}

When you reach 5th level, you gain mastery of your psionic powers. At the start of your turn you get 1 free psi point. This can be spent to empower psionic disciplines, but not to recreate spells or fuel Psionic Talents. If you have any unspent free psi points granted by this feature left at the end of your turn, then they are lost.

At 11th level, this is increased to 2 free psi points, and at 17th level, this is increased to 3. Points can be split between different abilities.

\subsection*{Innate Psionics}

At 11th level, you gain the ability to exert great feats of psionic power. Choose one 6th-level spell from the psion spell list as an innate ability. You can use this innate ability to cast that spell once. You must finish a long rest before you can do so again. At higher levels, you gain more innate abilities of your choice that can be used in this way: one 7th-level spell at 13th level, one 8th-level spell at 15th level, and one 9th-level spell at 17th level. You regain all uses of your Innate Psionics when you finish a long rest.

Innate Psionics are well beyond the normal scope of your powers and are not restricted by what disciplines you have. Unlike psionic disciplines, they require any component the spell requires.

\subsection*{Third Discipline}

When you reach 18th level, you can select a third psionic discipline from the list of psionic disciplines. You can’t select a discipline you already know. You gain all features of a psionic discipline when selecting it.

\subsection*{Ascension}

Starting at 20th level, if you die you can choose to become an incorporeal entity on the Border Ethereal at the spot you died in the Material Plane at the start of your next turn. This entity has the physical ability scores (Strength, Dexterity, Constitution) and abilities of a ghost, but has your mental ability scores (Intelligence, Wisdom, Charisma) and your psion class abilities.

When you take this form your current and maximum hit points becomes that of the ghost while you are in the form, if you have less than 10 psi points when you would assume this form, you have 10 psi points. While in this form you can’t rest to regain spent abilities. If you exhaust all Psi Points or the ghost is destroyed, you die as a normal creature would.

\begin{itemize}
  \item Class: Psion
\end{itemize}

\section*{Awakened Mind}

% [Image Inserted Manually]

An Awakened Mind is a psion who had their psionic power awakened within them by an encounter, event, or circumstance. Perhaps their powers were latent or perhaps their destiny had been that of a normal creature until powers were suddenly thrust on them. Perhaps they brushed the realms beyond in a vivid dream, perhaps they wandered the Feywild and came back warped, perhaps they merely glimpsed the eyes of a creature that did not belong walking down the street.

An awakening is often somewhat traumatic. Psions are not well understood and the first power they manifest—telepathy— is one of the most feared and shunned abilities.

An Awakened can be either empathetic or cruel, good or evil. Knowing what is on the minds of others affects people in different ways, driving some away from society as they see its true face, while others feel drawn to help the problems only they can see.

\subsection*{Opened Mind}

At 1st level when you select this archetype, your mind awakens the ability to directly connect to the minds of other creatures, granting the psionic discipline of Telepathy.

\subsection*{Mental Awareness}

Additionally, starting at 1st level, you can use Intelligence instead of Wisdom when making an Insight check against a creature with an Intelligence ability score of 6 or higher.

Further, if you have telepathically communicated with a willing creature, you know their general location (direction and rough estimate of distance) for the next hour as long as you are on the same plane as them.

\subsection*{Mind Reader}

Starting at 3rd level, when you use Telepathic Intrusion, you can force the target to make an Intelligence saving throw instead of a Wisdom saving throw against the power (deciding when you use the ability).

When a creature fails a saving throw against your Telepathic Intrusion, you gain a d4. Until the end of your next turn, if you make an attack roll against the creature or the creature makes a saving throw against one of your psionic powers other than Telepathic Intrusion, you can add or subtract the d4 from the roll (deciding to roll before you roll the attack or use the power).

\subsection*{Empowered Psionics}

Starting at 6th level, when you deal damage with a psionic discipline power you can add your Intelligence modifier to the damage dealt.

\subsection*{All Seeing Eye}

Starting at 10th level, you can see a creature by its mind. You gain a mindsight of 60 feet, allowing you to see creatures with an Intelligence of 6 or higher within range as if by blindsight. A creature you are unaware of can still be hidden from you, but you can use your Intelligence modifier instead of your Wisdom modifier when making Perception checks to detect creatures.

\subsection*{Full Awakening}

Starting at 14th level, you can briefly fully awaken your expanded mind to true comprehension. At the start of your turn, you can expend 2 psi points to gain advantage on all saving throws and attack rolls until the start of your next turn.

\begin{itemize}
  \item Psionic Archtype: Awakened Mind
\end{itemize}

\section*{Unleashed Mind}

% [Image Inserted Manually]

An Unleashed Mind’s power is most often an innate force they struggle to understand or control, a reflection of their state of mind and mood projected onto the world around them, often to devastating results. The life of an Unleashed Psion is frequently their journey to control their unruly powers... or one of embracing the destructive powers fully.

What the wellspring of their innate power is varies: they could be a psionic race, born under the influence of distant twisted powers, warped by fey blood, or a herald of something more sinister brushing against the material world. Sometimes their power will show itself in simple applications, but most often their power becomes known and feared when their temper flares.

Their alignment frequently depends on the approach to their powers they have taken. An Unleashed Psion who emphasizes control and restraint, keeping their emotions from running wild and their powers from running rampant will most often be lawful, following strict personal codes. On the other hand, an Unleashed Psion who lets their power run rampant will almost always be chaotic in nature, acting on impulse... with great emphasis.

\subsection*{Unshackled Power}

At 1st level when you select this archetype, you gain the ability to unleash your mind to physically interact with the world around you, granting you the psionic discipline of Telekinesis.

\subsection*{Overwhelming Power}

Additionally at 1st level, you gain the ability to cast thaumaturgy with your psionic powers. When you cast it in this way, you have an additional options:

\begin{itemize}
  \item You cause up to 10 pounds of loose objects within 10 feet of you to start floating for 1 minute.
  \item You can force all targets within 5 feet to make a Strength saving throw, or be pushed 5 feet away from you.
\end{itemize}

\subsection*{Rampaging Power}

Starting at 3rd level, you gain a d4 rampage die. Once on each of your turns when making a damage roll, you can add this rampage die to the damage roll. If you dealt damage during your last turn, your rampage die becomes a d6, increasing with each subsequent turn you deal damage by one step, up to a d12; if you did not deal damage during your last turn or become incapacitated, it becomes a d4 once more. If you maintain a continuous d12 rampage die for more than one minute, you gain one level of exhaustion.

\subsection*{Empowered Psionics}

Starting at 6th level, when you deal damage with a psionic discipline power you can add your Intelligence modifier to the damage dealt.

\subsection*{Uncontrollable Mind}

Starting at 10th level, the strength of your rampaging mind is such that others’ attempts to control it are futile. You gain immunity to the charmed and frightened conditions, as well as effects that would control your mind while your rampage die is a d8 or more.

\subsection*{Unstoppable Rampage}

Starting at 14th level, when an attack reduces you to zero hit points, you can roll your rampage die. If your roll + your Constitution modifier is higher than the excess damage you took, you are instead reduced to one hit point. You can expend 2 psi points to roll an additional rampage die, adding it to the result.

\begin{itemize}
  \item Psionic Archtype: Unleashed Mind
\end{itemize}

\section*{Transcended Mind}

% [Image Inserted Manually]

A Transcended psion is most often one that has found their path to psionic powers through a mental epiphany, realizing the place of their mind within the multiverse and how it ties to everything else, seeing the keys and levers to the world laid out before them in their mind’s eye.

Most often achieved either intentionally or accidentally through meditation and ritual, this can sometimes even be a passed on technique to reliably transcend the strictly material concerns, and explore the more cognitive realm and powers within.

Transcended Psions tend to have the best grasp on their powers in control and principle, having come to understand the greater cosmos, but lack the reckless abandon some other Psions may have. Unfortunately less transcendent minds frequently cannot understand the greater truths they have learned and may view a Transcended Psion as a bit... peculiar.

\subsection*{Enlightened}

At 1st level when you select this archetype, you gain the ability to empower your body with the power of your mind, granting you the psionic discipline of Enhancement.

\subsection*{State of Mind}

Additionally at 1st level, you can ignore the effects of extreme heat or cold, hold your breath twice as long as normal, and can go twice as long without eating or sleeping before suffering exhaustion.

\subsection*{Balance of Power}

Starting at 3rd level, when you use a psionic ability (power or spell) to restore hit points or grant temporary hit points, you can add equivalent hit points granted to a stored pool. The maximum value of the pool is your psion level, and any points beyond the maximum are lost. The next time you make a damage roll, you can expend the pool to add damage equal to the stored value to one target affected by the damage roll. This stored damage is lost if not spent within 1 minute.

\subsection*{Perfected Enhancement}

Starting at 6th level, when you grant temporary hit points with a psionic power you can add your proficiency bonus to the temporary hit points gained by one creature.

\subsection*{Mental Control}

Starting at 10th level, when you have to make a Constitution saving throw to avoid losing concentration on an effect from a psionic discipline, you can add your Intelligence modifier to the saving throw.

\subsection*{Mind Over Matter}

Starting at 14th level, when you would roll a Strength, Dexterity, or Constitution saving throw, you can expend 2 psi points to roll an Intelligence saving throw instead.

Additionally, when you roll a Death saving throw, you can expend 4 psi points before rolling to treat the roll as 20.

\begin{itemize}
  \item Psionic Archtype: Transcended Mind
\end{itemize}

\section*{Shaper's Mind}

% [Image Inserted Manually]

A Shaper is a Psion that specializes in the materialization of their imagination, projecting it out into the world. No mere conjurers borrowing the powers of other planes, a Shaper manifests things from nothing but their own mind, weaving their creations into existence through the exertion of raw psionic power and imagination.

A Shaper’s mind is a tool of nearly unrivaled power possessing both boundless creativity and inexorable will, but if that will is overrun, the world would do well to fear a Shaper’s nightmares.

\subsection*{Creator’s Mind}

At 1st level when you select this Archetype, you gain the ability to will the contents of your mind into the world, granting you the psionic discipline of Projection.

\subsection*{Boundless Imagination}

Additionally at 1st level, when you conjure your Astral Construct, you can apply one of the following powers to it:

\begin{itemize}
  \item Devastating Weapons. You imagine more deadly armaments—from a greatsword to vicious fangs—causing your Astral Construct to grow more deadly. The Astral Construct’s damage becomes 1d12
  \item Psionic Conduit. You can use your psionic powers, spells, talents through your Astral Construct, as if you were in its space.
  \item Vivid Existence. Your Astral Construct fully materializes and automatically uses Solidify at the start of your turn without requiring a command to do so.
\end{itemize}

You can change which benefit you grant it for the duration of the effect as a bonus action.

\subsection*{Astral Metastability}

Starting at 3rd level, when you use your psionic powers to create an Astral Construct, it doesn’t require concentration to maintain, and lasts until dismissed, but you can’t summon another one while you have one summoned.

\subsection*{Empowered Construct}

Starting at 6th level, when you deal damage with an Astral Construct or a weapon created by Project Item, you can add your Intelligence modifier to the damage dealt.

\subsection*{Astral Guardian}

Starting at 10th level, when you would take damage while you have an Astral Construct within 30 feet of you, you can use your reaction to conjure it to you (sharing your space) and expend 1 psi point to Solidify it, transposing it between you and the source of damage. It takes the damage instead of you. If the attack deals more damage than it has hit points (from Solidify), you take the remainder of the damage.

\subsection*{Imaginary Army}

Starting at 14th level, you can allow your mind to run wild, letting an astral army spring forth into reality. When you use Replicate, you can create one additional Astral Construct that can be controlled with the same action (commands still only affect one Astral Construct of your choice). This additional construct lasts until the start of your next turn and can’t be sustained. Once you create an additional duplicate, you can’t do so again until you finish a short or long rest.

\begin{itemize}
  \item Psionic Archtype: Shaper's Mind
\end{itemize}

\section*{Wandering Mind}

% [Image Inserted Manually]

A Wandering Mind is among the more mysterious incarnates of psions, these are individuals that just aren’t quite rooted in the same reality everyone else is... they find it more pliable and slippery, and prone to jumping right through it on occasion.

How they get their powers varies. Some were born in the ethereal plane or went through a portal while young. Sometimes it is just that their mind just works along an axis most people cannot understand. No matter its source, they manifest a truly unique ability to treat certain aspects of the metaphysical with a certain mundanity, and are prone to treating their powers as absolutely natural things as one might treat an arm or leg.

\subsection*{Spatial Manipulation}

At 1st level when you select this archetype, your mind grows a greater perspective on the nature of space and dimensions, allowing you to manipulate it and your relation to it, granting you the psionic discipline of Transposition.

Additionally, your connection to your transdimensional powers are such that you can slide through the spaces you see in dimensions as another might slip through a tight space, without conscious thought or effort. You can use your Dexterity ability score in place of your Intelligence ability score calculating the DC or attack roll modifier of Transpositional powers, alternate effects, or talents that require the Transpositional discipline.

\subsection*{Nomad’s Gear}

Additionally at 1st level, you gain proficiency with martial weapons and medium armor.

\subsection*{Cunning Strikes}

\begin{minipage}{0.48\textwidth}
Starting at 3rd level, you gain the Rift Strike talent. If you already have the Rift Strike talent, you can gain one other Psionic Talent of your choice. The Rift Strike talent doesn’t count against your Psionic Talents known, but can’t be switched out on leveling up.
\end{minipage}\hfill
\begin{minipage}{0.48\textwidth}
Psionic Talent: Rift Strike Prerequisite: Transposition Discipline

When you use your Phase Rift power as an action, you can make a single weapon attack as a bonus action.
\end{minipage}

\subsection*{Curious Mind}

Starting at 3rd level, your wandering mind adapts to new situations constantly. Whenever you finish a long rest, you can select two skills you lack proficiency in. Until the end of your next long rest, you can add half your proficiency bonus, rounded down, to ability checks made with those skills.

\subsection*{Phase Dancer}

Starting at 6th level, once per turn, you automatically gain one illusory duplicate as per the blurring modifier when you use your Phase Rift power. Additionally, your first attack roll before the end of your turn after using Phase Rift gains advantage.

\subsection*{Flickering Presence}

Starting at 10th level, your unspent Psionic Mastery points are not lost until the start of your next turn, but can only be used to cast flicker.

Additionally, when you roll for the effect of flicker* or blink you can expend 1 psi point to reroll the result. You can select which of the two results you would like to use.

\subsection*{Planeswalker}

Starting at 14th level, your understanding of how to traverse space expands to a previously incomprehensible scale. You gain the ability to cast plane shift and teleport . You can cast one of these spells per day once without expending a spell slot or use of Innate Psionics. Once you do this, you must finish a long rest before you can use them again, or use your 7th level or higher use of Innate Psionics instead of the spell selected for that level to cast these spells again before finishing a long rest. If you already have one of these spells selected for your 7th level Innate psionic power, you can select a new 7th-level spell from the psion list for that slot. Material components are still required for casting plane shift this way.

\subsection*{Winding Paths}

Additionally at 14th level, your phase rift no longer has to go in a straight line, though it can only pass through a creature’s space once.

\begin{itemize}
  \item Psionic Archtype: Wandering Mind
\end{itemize}

\section*{Elemental Mind}

% [Image Inserted Manually]

An Elemental Mind is a Psion that can manifest and control elements as an extension of their will.

\subsection*{Elemental Power}

At 1st level when you select this archetype, you gain a deep intrinsic tie to elemental power, granting you the ability to manipulate it. You gain the psionic discipline of Psychokinesis.

\subsection*{Primordial Aspect}

Additionally at 1st level, as an action or when you deal fire, cold, or lightning damage (no action required), you take on an aspect of that element until the end of your next turn or until you take a new primordial aspect.

\begin{itemize}
  \item Cold. You gain an icy shell, reducing any nonmagical bludgeoning, piercing or slashing damage taken by your proficiency bonus.
  \item Fire. You gain a fiery aura. Once per turn, a creature within 5 feet that hits you with a melee attack takes fire damage equal to your proficiency bonus.
  \item Lightning. You flicker with lightning. Your walking speed is increased by 5 feet for the duration.
\end{itemize}

You can additionally spend 1 psi point to gain resistance to the element to the start of your next turn. If you are specialized in an element, you can do this without spending a psi point for the element you are specialized in.

Aspect Appearance

The appearance of the aspect manifesting itself may vary, perhaps your hair becomes fire and radiates heat, or your skin takes on an icy sheen, but whatever the nature and consequence of the aspect is apparent to observers.

\subsection*{Living Power}

\begin{minipage}{0.48\textwidth}
Starting at 3rd level, your powers become a living extension of your mind, weaving an extension of your will into reality, allowing you to manipulate your powers in more advanced ways. When you use a power or alternate effect of Psychokinetics, you can apply one of the following modifiers:
\end{minipage}\hfill
\begin{minipage}{0.48\textwidth}
Controlled Power

When you use a spell or power that targets an area, you can select a number of creatures equal to the psi points spent in the area of effect for the spell or power to be ignored. The power passes harmlessly around these creatures, they automatically succeed on their saving throw against the effect, and they take no damage if they would normally take half damage on a successful save against the effect.
\end{minipage}

\begin{minipage}{0.48\textwidth}
Shaped Power

When you use a power or spell that makes a ranged spell attack, you can convert it to a melee spell attack (such as forming it into a weapon shape), and when you would make an attack that would make a melee spell attack, you can instead form it into a shape and hurl it at a target within 15 feet making a ranged spell attack.
\end{minipage}\hfill
\begin{minipage}{0.48\textwidth}
Raging Power

When you use a spell or power, you can let it rage out of control. When you roll damage for the power or spell, you can reroll a number of dice up to 1 + the number of psi points spent. You must use the new roll.
\end{minipage}

\subsection*{Empowered Psionics}

Starting at 6th level, when you deal damage with a psionic discipline power you can add your Intelligence modifier to the damage dealt.

\subsection*{Full Manifestation}

Starting at 10th level, when you enter a primordial aspect, you can expend 1 psi point to fully manifest that element, replacing the primordial aspect with the following effect. If you are specialized in that element, it doesn’t require a psi point to fully manifest the power.

\begin{itemize}
  \item Cold. Your speed becomes 0 until the end of your next turn, but you have resistance to all damage types except fire and force.
  \item Fire. Fire fills a 10-foot radius around you. When a creature enters the area for the first time on a turn or starts its turn there, it takes fire damage equal to 1d4 + your proficiency bonus.
  \item Lightning. You can use your movement to teleport a distance up to your speed for the duration.
\end{itemize}

If you are not in a primordial aspect, you can fully manifest a power as a bonus action for 2 psi points (or 1 psi point if you are specialized in that power).

\subsection*{Elemental Form}

Starting at 14th level, you can expend 5 psi points to cast shapechange to assume elemental forms. When you cast shapechange in this way it has no verbal or material components, but you can only assume the form of a Water Elemental, Fire Elemental, or Air Elemental. If you are specialized in an element, the cost is reduced to 3 psi points, but you can only assume the form associated with your specialization (Water Elemental for cold, Fire Elemental for fire, or Air Elemental for lightning). A Water Elemental assumed with this form doesn’t have the Freeze property.

Once you cast shapechange this way, you can’t do so until you finish a short or long rest.

\begin{itemize}
  \item Psionic Archtype: Elemental Mind
\end{itemize}

\section*{Consuming Mind}

A Consuming Mind is a master of a dangerous branch of psionics, born from the ability to sap energy from other creatures. This power can tear away thoughts, minds, and ultimately vitality for the psion’s own consumption. A feared branch of psionics, it is sometimes believed to be one of the primeval roots of all psionic power, as it can be found among mind-eating monsters and ancient psionic space whales... which might have other names in various settings.

It has been refined by those that seek greater power either from ambition or desperation, and unlocks a terrible but effective path. One example of mortal creatures walking this path are the houses of dark elves, who developed their dark powers from their study of feral, vampiric, brain-eating monsters.

\subsection*{Psionic Predator}

At 1st level when you select this archetype, you gain the ability to consume the psionic power of others. You gain the psionic discipline of Consumption.

\subsection*{Dark Lurker}

Additionally at 1st level, your powers grant you intuition that allows you to better adapt and survive as a mind hunter. You gain proficiency in Stealth and Deception. If you are already proficient in either skill, you instead become proficient in a skill of your choice.

When you use psionic abilities, you can make an Intelligence (Deception) check contested by a target’s Wisdom (Insight) to conceal your use of psionic powers from them, suppressing the ability’s usual visible indicators. If you spend psi points on the ability, you must subtract the psi points spent from your Intelligence (Deception) roll result (concealing greater uses of psionics being more difficult).

\subsection*{Ravenous Powers}

Starting at 3rd level, you gain the Psionic Talent Mind Devourer; this talent ignores the normal level restriction, and doesn’t count against your Psionic Talents known, but can’t be switched out on leveling up.

Additionally, you can gain the benefit of this Talent from a range of 30 feet when the creature is killed by one of your psionic powers.

\subsection*{Empowered Psionics}

Starting at 6th level, when you deal damage with a psionic discipline power you can add your Intelligence modifier to the damage dealt.

\subsection*{Mind Vampire}

\begin{minipage}{0.48\textwidth}
Starting at 10th level, you can trigger Mind Devourer anytime you deal psychic damage to a target within 30 feet, whether it kills the target or not.

Further, you can have additional psi points (over your normal limit) equal to half your Intelligence modifier (rounded down) when gaining psi points from Mind Devourer, but any additional psi points are lost when you finish a short or long rest.
\end{minipage}\hfill
\begin{minipage}{0.48\textwidth}
Limitations

Even when used with Mind Vampire, Mind Devourer still requires a reaction to use, meaning that it can still effectively trigger once per round. All normal restrictions of Mind Devourer still apply (such as a minimum Intelligence ability score of the target.)
\end{minipage}

\subsection*{Shattered Husks}

Starting at 14th level, your Mind Leech ability always gains the Shredding modifier, and it doesn’t cost a psi point to add the modifier.

Further, you can use additional power to leave their mind further vulnerable. You can spend additional psi points on the Shredding modifier to further reduce their next Intelligence, Wisdom, or Charisma saving throw (up to 2 additional points to reduce it by a total of 3d4). Once this effect has been applied to a creature, you can’t spend additional psi points on the Shredding modifier against the creature until 1 hour has passed.

\begin{itemize}
  \item Psionic Archtype: Consuming Mind
\end{itemize}

\section*{Psion Spell List}

\begin{minipage}{0.48\textwidth}
\subsubsection*{6th Level}

\begin{itemize}
  \item Blade Barrier
  \item Chain Lightning
  \item Disintegrate
  \item Eyebite
  \item Find the Path
  \item Harm
  \item Mass Suggestion
  \item Mind Blast*
\end{itemize}

\subsubsection*{7th Level}

\begin{itemize}
  \item Etherealness
  \item Geas †
  \item Plane Shift
  \item Regenerate
  \item Reverse Gravity
  \item Teleport
\end{itemize}
\end{minipage}\hfill
\begin{minipage}{0.48\textwidth}
\subsubsection*{8th Level}

\begin{itemize}
  \item Antimagic Field
  \item Demiplane
  \item Dominate Monster
  \item Earthquake
  \item Feeblemind
  \item Inner World*
  \item Mind Blank
  \item Power Word Stun
\end{itemize}

\subsubsection*{9th Level}

\begin{itemize}
  \item Astral Projection
  \item Foresight
  \item Gate
  \item Power Word Kill
  \item Time Stop
\end{itemize}
\end{minipage}

Special Cases 
Spells with an † can be selected and cast as the level they are listed at only

\section*{Multiclassing}

Should you want to multiclass into Psion, the prerequisites and proficiencies are listed below:

\begin{itemize}
  \item Prerequisite: 13 Intelligence
  \item Proficiencies gained: Psionics
\end{itemize}

Psionic powers can be weird, strange, and rare. Your GM may place additional restrictions or requirements into multiclassing into Psionic classes

\section*{Enhancement Discipline}

Enhancement is the ability to interact with a creature’s nature and abilities with your psionic power.

\subsection*{Enhancing Skill}

You can focus your psionics to enhance your abilities. Whenever you make an ability check using Strength or Dexterity, you can add 1d4 to the result.

\subsection*{Enhancing Surge}

Psionic Power

Casting Time: 1 action
 Range: 60 feet
 Components: S
 Duration: 1 round

You empower the body of a target creature you can see with your psionics. The target gains 1d6 temporary hit points and the next time the target deals damage, it deals 1d6 additional damage to one target of that damage roll. Any remaining temporary hit points from this power fade when you use it again.

You can spend psi points up to your per use limit to add the following modifiers to Enhancing Surge (you can add multiple modifiers). The points must be spent when choosing the target of the power.

Fortifying (1+ psi points): The target gains an extra 1d6 temporary hit points for each point spent.

Resilient (3 psi points): The target gains resistance to all damage until the start of your next turn.

Savage (1+ psi points): The target’s next weapon attack deals 1d6 additional damage for each point spent.

Swift (2 psi points): The target gains an additional action. That action can be used only to take the Attack (one weapon attack only), Dash, Disengage, Hide, or Use an Object action.

\subsection*{Alternate Effects}

Additionally, when you learn the Enhancement psionic discipline you can use your Psionics feature to cast the following spells as per the rules defined in the feature:

\begin{minipage}{0.48\textwidth}
\begin{tabularx}{\textwidth}\toprule
{}XXXXXXX}
\midrule
Point Cost & \multicolumn{7}{c}{Alternate Effects} \\
\midrule
1 & \multicolumn{7}{c}{heroism, longstrider, unlocked potential*} \\
\midrule
2 & \multicolumn{7}{c}{alter self, enlarge/reduce, lesser restoration} \\
\midrule
3 & \multicolumn{7}{c}{haste, protection from energy} \\
\midrule
4 & \multicolumn{7}{c}{freedom of movement, stoneskin} \\
\midrule
5 & \multicolumn{7}{c}{greater restoration, invested competency*} \\
\midrule
\end{tabularx}
\end{minipage}\hfill
\begin{minipage}{0.48\textwidth}

\end{minipage}

If a spell can be cast at a higher level, you can spend additional psi points to cast it at a level equal to the psi points spent.

\section*{Enhancement Talents}

The following talents can be selected if you have the Enhancement Discipline:

Body Control (Prerequisite: 5th-level Psion) 
You can cast alter self at will, without expending a spell slot or psi points. Additionally, when you cast enlarge/reduce on yourself, you may expend 1 psi point instead of 2.

Enhanced Regrowth 
You gain the cure wounds spell, and can cast it as a 1st-level spell by expending 1 psi point. You can cast it at a higher level by spending an additional psi point for each level above first. When you cast cure wounds on a creature, you can use Enhancing Surge on that creature as a bonus action.

Metamorphosis (Prerequisite: Body Control) 
The mutate and polymorph spells are added to your Enhancement Alternate Effects list.

You can only target yourself when casting polymorph this way.

Physical Surge 
When you use Enhancing Surge targeting yourself, you can choose to make your Strength or Dexterity ability score equal to your Intelligence ability score until the start of your next turn.

Surging Power 
When you target only yourself with Enhancing Surge, you can use the power as a bonus action instead of an action, but the damage and temporary hit points the base power grants are reduced to 1d4 when using the power this way.

Transcendent Life (Prerequisite: 9th-level Psion) 
The mass cure wounds and reincarnate spells are added to your Enhancement Alternate Effects list. Reincarnate requires its normal material components when cast this way.

\begin{itemize}
  \item Psionic Discipline: Enhancement
\end{itemize}

\section*{Projection Discipline}

Projection is the ability to project what is in your mind to the outside world with your psionic power.

\subsection*{Project Item}

As an action, you can use your powers to project an inanimate object you imagine into your hands. It can’t be larger than 3 feet on any side or weigh more than 10 pounds, and is clearly ethereal in nature. The item behaves as a solid object. Weapons created with this feature deal force damage.

Projected items fade after 1 minute, and you can have no more than 3 projected items at a time.

\subsection*{Astral Construct}

Psionic Power

Casting Time: 1 action 
Range: 60 feet
 Components: S
 Duration: Concentration, up to 1 minute

You project an ethereal creation from your mind at a space you can see within range, taking the shape of a Medium or smaller creature, weapon, or other object. This creation is clearly ethereal in nature. When you create it, and on subsequent turns using your action to mentally control it, you can move it up to 30 feet in any direction and attack, so long as it doesn’t move beyond the range of the power. To attack with it, make a melee spell attack. On a hit, the target takes 1d8 force damage. If the Astral Construct moves out of range of the spell, it fades away into nothing, ending the spell.

While the Astral Construct is active, you can spend psi points up to your per use limit to issue commands that affect and empower your construct. Commands require no action, but you can’t exceed your psi point limit in total commands issued, and each command can only be issued once per turn.

Grab (1 psi point): Can only be used while your construct is Solidified. A target creature within 5 feet of your construct that is no more than one size larger than the construct must make a Strength saving throw or become restrained by the construct. At the end of each of the restrained creature’s turns it can repeat the saving throw. The condition ends if the construct becomes ethereal again, or becomes more than one size smaller than the target.

Grow (1 psi point): Your construct increases by one size, and its number of damage dice increases by one. It returns to its normal size at the start of your next turn. It can only gain one additional die (for example, up to 2d8).

Relocate (1 psi point): The construct disappears and reappears anywhere within 60 feet of you.

Replicate (3 psi points): You use Relocate, but the original doesn’t disappear. Until the end of your turn, you have another construct and control both with the same action (commands only affect one of your choice). Pick one construct to fade at the start of your next turn or use Sustain.

Solidify (1 psi point): Your construct becomes solid, blocking its space until the start of your next turn. You can only use this command if it is in an unoccupied space. It has an armor class of 16 and hit points equal to your Intelligence modifier + your psion level. The construct becomes ethereal again if it drops to zero hit points.

Strike (2 psi points): The construct makes an attack (even if it has already attacked).

Sustain (1+ psi points): At the start of your turn, you can sustain the effect of a Solidify, Grow or Replicate command for an additional round. An effect that has been replicated can be sustained across all replicated constructs with a single use, but each effect can only be sustained once per turn (for example, you can never sustain more than one replicated construct). This costs 1 psi point for each effect sustained. Sustaining Solidify refreshes the construct’s hit points to its maximum.

\subsection*{Alternate Effects}

Additionally, when you learn the Projection psionic discipline you can use your Psionics feature to cast the following spells as per the rules defined in the feature:

\begin{minipage}{0.48\textwidth}
\begin{tabularx}{\textwidth}\toprule
{}XXXXXXX}
\midrule
Point Cost & \multicolumn{7}{c}{Alternate Effects} \\
\midrule
1 & \multicolumn{7}{c}{floating disk, unseen servant} \\
\midrule
2 & \multicolumn{7}{c}{mirror image} \\
\midrule
3 & \multicolumn{7}{c}{phantom steed} \\
\midrule
4 & \multicolumn{7}{c}{arcane eye} \\
\midrule
5 & \multicolumn{7}{c}{creation} \\
\midrule
\end{tabularx}
\end{minipage}\hfill
\begin{minipage}{0.48\textwidth}

\end{minipage}

If a spell can be cast at a higher level, you can spend additional psi points to cast it at a level equal to the psi points spent.

\section*{Projection Talents}

The following talents can be selected if you have the Projection Discipline:

Animated Projections (Prerequisite: 10th-level Psion) 
You learn to project a swarm of ethereal objects. As an action, you can expend 5 psi points to cast animate objects. When cast in this way, the spell creates new ethereal objects rather than animating existing ones.

Astral Swap 
As a bonus action, you can expend 1 psi point to swap places with your Astral Construct.

Life Link 
When a solidified construct takes damage, you can choose to take that damage instead of the construct (no action required). You have resistance to the damage taken.

Matter Made Real (Prerequisite: 9th-level Psion) 
You gain the ability to solidify some of your projections into real objects. You can cast fabricate and wall of stone by expending psi points equal to the spell level.

Projected Weaponry 
When you project a weapon with your Project Item feature, it gains the following benefits:

\begin{itemize}
  \item You can project it as a bonus action and you can project up 3 weapons or pieces of ammunition at the same time.
  \item You are proficient with any projected weapon.
  \item It is no longer restricted to a maximum of 3 feet when it takes the form of a weapon (for a Medium-sized creature).
  \item You can use your Intelligence modifier in place of your Strength or Dexterity modifier for attack and damage rolls with it.
  \item If it has the thrown property, its throwing range is doubled. If it doesn’t have the thrown property, it gains the thrown (10/30) property.
  \item You can apply the Astral Construct modifiers Grow and Sustain to Projected Weaponry. Grow adds 1d8 to the weapon’s damage dice. You can expend Psionic Mastery on these modifiers.
\end{itemize}

Unlimited Imagination (Prerequisite: Shaper’s Mind) 
When you manifest a construct, you can select two options from Boundless Imagination.

\begin{itemize}
  \item Psionic Discipline: Projection
\end{itemize}

\section*{Telekinesis Discipline}

Telekinesis is the ability to interact with physical objects and energy with your psionic powers.

\subsection*{Telekinetic Hands}

When you gain this feature, you can manipulate small objects within 30 feet with your mind as if using your hand to interact with it. You can use this power to manipulate an object, open an unlocked door or container, stow or retrieve an item from an open container, or pour the contents out of a vial. You can’t Attack, activate magical items, or carry more than 10 pounds in this manner. You can move an item you are controlling in this way up to 30 feet during your turn.

You can spend 1 or more psi points to increase how much you can move this way by 100 pounds per psi point spent for 1 turn.

\subsection*{Telekinetic Force}

Psionic Power

Casting Time: 1 action
 Range: 60 feet
 Components: S
 Duration: Instantaneous

You smash a target creature or object you can see with your mental power. The target must succeed on a Strength saving throw or take 1d10 bludgeoning damage, and either be shoved 5 feet in a direction of your choosing or be knocked prone.

You can spend psi points up to your per use limit to add the following modifiers to Telekinetic Force (you can add multiple modifiers). The points must be spent when choosing the target of the power.

Crushing (2 psi points): The target is restrained until the end of its next turn if it fails its saving throw.

Hammering (1+ psi points): The target takes 1d10 bludgeoning damage for each point spent if it fails its saving throw.

Hurling (1–3 psi points): The target is shoved 10 feet in a direction of your choosing for each point spent if it fails its saving throw.

Zone of (1–3 psi points): You can target all creatures in a 5-foot radius of a point within range. The radius doubles for each point spent (5 feet, 10 feet, 20 feet).

\subsection*{Alternate Effects}

Additionally, when you learn the Telekinesis psionic discipline you can use your Psionics feature to cast the following spells as per the rules defined in the feature:

\begin{minipage}{0.48\textwidth}
\begin{tabularx}{\textwidth}\toprule
{}XXXXX}
\midrule
Points Cost & \multicolumn{5}{c}{Alternate Effects} \\
\midrule
1 & \multicolumn{5}{c}{jump, launch object*, thunderwave} \\
\midrule
2 & \multicolumn{5}{c}{fling*, levitate, shatter} \\
\midrule
3 & \multicolumn{5}{c}{fly, vortex blast*} \\
\midrule
4 & \multicolumn{5}{c}{orbital stones*, resilient sphere} \\
\midrule
5 & \multicolumn{5}{c}{telekinesis, wall of force, shockwave*} \\
\midrule
\end{tabularx}
\end{minipage}\hfill
\begin{minipage}{0.48\textwidth}

\end{minipage}

If a spell can be cast at a higher level, you can spend additional psi points to cast it at a level equal to the psi points spent.

\section*{Telekinesis Talents}

The following talents can be selected if you have the Telekinesis Discipline:

Kinetic Slam 
When you use the Telekinetic Force power, you can unleash it as a blast of kinetic power. You can use your power as a ranged spell attack (applying damage and effects on hit) instead of requiring a Strength saving throw. You can only select the Hammering modifier when you use it in this way.

Mental Might (Prerequisite: 5th-level Psion) 
You learn to focus your mental grip. You can use your Intelligence modifier instead of other ability modifiers when you make an Athletics check.

Additionally, you may attempt to grapple or shove a creature within 30 feet, using your mind. When grappling a creature in this way, you can’t move them. Maintaining a grapple at a range greater than your natural reach requires concentration, as if concentrating on a spell, and the grapple ends if concentration is lost.

Precise Power (Prerequisite: 9th-level Psion; incompatible with Unchecked Power) 
The animate objects spell is added to your Telekinesis Alternate Effects list, which you can cast expending 5 psi points. Additionally, as a reaction to a ranged weapon attack you can see being made against a target within 60 feet of you, you can expend 1 psi point to add or subtract 2d4 to or from that attack roll. You can do this after the attack is rolled, but before you know the outcome of the roll.

Telekinetic Movement 
You can expend 1 psi point to gain 10 additional feet of movement and the effect of spider climb , feather fall , or levitate until the start of your next turn (no action required). At the start of your next turn, you can expend 1 psi point to maintain the effect before it would end.

Telekinetic Weapons 
You gain the Telekinetic Weapons psionic power as part of your Telekinetic Discipline (in addition to your other psionic powers), giving you the ability to fling weapons as per the power.

Telekinetic Barrier 
You focus your telekinetic power, spreading it thin to create a large barrier. As an action you can expend 2 psi points to create a wall of weak telekinetic force 15 feet high and up to 40 feet long, or as a 15 radius around you. This barrier is visible as small objects hover and float within it. If you create it around yourself, it moves with you when you move. Ranged attacks that pass through this barrier are made with disadvantage as their trajectory is deflected. You can maintain this barrier by concentrating, as if concentrating on a spell, for up to 1 minute after creating it.

Unchecked Power (Prerequisite: 9th-level Psion; incompatible with Precise Power) 
The fissure* spell is added to your Telekinesis Alternate Effects list. Additionally, as a reaction to a melee attack being made against you, you can expend 1 psi point to shove the creature away. The creature must make a Strength saving throw, or be knocked 10 feet backwards. If this puts the attack out of reach, the attack automatically misses.

\begin{itemize}
  \item Psionic Discipline: Telekinesis
\end{itemize}

\section*{Telepathy Discipline}

Telepathy is the ability to interact with the minds of other creatures using your psionic abilities.

\subsection*{Telepathic Communication}

When you gain this feature, you can communicate telepathically with any creature you can see within 30 feet of you. You don’t need to share a language with the creature for it to understand your telepathic utterances, but the creature must be able to understand at least one language. Creatures you communicate with can reply in kind.

\subsection*{Telepathic Intrusion}

Psionic Power

Casting Time: 1 action
 Range: 60 feet
 Components: S
 Duration: Instantaneous

You assault the mind of a creature you can see directly. The target must succeed on a Wisdom saving throw, or take 1d8 psychic damage. If the target fails the saving throw, it has disadvantage on attacks made against you until the start of your next turn. You can choose to deal no damage to the creature when it fails its saving throw.

You can spend Psi Points up to your per use limit to add the following modifiers to Telepathic Intrusion (you can add multiple modifiers). The points must be spent when choosing the target of the power.

Meddling (2 psi points): You make one creature invisible to the target creature or cause the creature to see something that is not there with the effect of minor illusion until the start of your next turn if it fails its saving throw.

Overwhelming (3 psi points): The target is stunned until the end of its next turn if it fails its saving throw.

Rending (1+ psi points): The target takes 1d8 psychic damage for each additional point spent on a failed save.

Terrifying (1 psi point): The target is frightened of you until the end of your next turn if it fails its saving throw.

\subsection*{Alternate Effects}

Additionally, when you learn the Telepathy psionic discipline you can use your Psionics feature to cast the following spells as per the rules defined in the feature:

\begin{minipage}{0.48\textwidth}
\begin{tabularx}{\textwidth}\toprule
{}XXXXX}
\midrule
Point Cost & \multicolumn{5}{c}{Alternate Effects} \\
\midrule
1 & \multicolumn{5}{c}{compelled query*, command, frighten*} \\
\midrule
2 & \multicolumn{5}{c}{detect thoughts, suggestion} \\
\midrule
3 & \multicolumn{5}{c}{delve mind*, fear} \\
\midrule
4 & \multicolumn{5}{c}{dominate beast, compulsion, confusion} \\
\midrule
5 & \multicolumn{5}{c}{dominate person, modify memory, telepathic bond} \\
\midrule
\end{tabularx}
\end{minipage}\hfill
\begin{minipage}{0.48\textwidth}

\end{minipage}

If a spell can be cast at a higher level, you can spend additional psi points to cast it at a level equal to the psi points spent.

\section*{Telepathy Talents}

Attuned Argument 
Your telepathic insight allows to adapt and tailor your influence on others. When you make a Charisma check, after you roll but before you know the outcome, you can choose to roll an additional d20 and take the higher result. You can do this a number of times equal to your proficiency bonus, regaining all uses on a long rest.

Empathy 
You can psionically link yourself to other creatures. As a reaction to a creature you can see within 120 feet of you taking damage, you can grant them resistance to the damage taken, but you take psychic damage equal to the damage they take (after resistance). The damage you take can’t be reduced or prevented in any way.

If the damage would inflict any further negative status effect on the target, you can choose for that effect to affect you instead.

If you have any form of telepathic bond that allows long-distance telepathic communication affecting the creature taking this damage, the range is unlimited and you do not need to be able to see them.

Mental Image 
You gain the ability to use your Telepathy to project images into the minds of creatures with perfect clarity, interposing it over their reality. The spells silent image , major image , hallucinatory terrain , and seeming are added to your Telepathic Discipline Alternate Effects list costing psi points equal to their spell level.

Mental Influence 
You specialize in the subtle touch, picking stray thoughts and soothing concerns without overtly intruding upon a mind. You gain expertise in the Persuasion skill, though you are reduced to normal proficiency against creatures that are immune to mental influence or charm (such a creature under the effect of mind blank ).

Reflected Agony 
When a creature within 120 feet that you can see deals damage to you, you can use your reaction to spend 1 or more psi points (up to your psi limit) to share the pain you experienced back at them, dealing 1d10 psychic damage per point spent. The damage this deals can’t exceed the damage taken from the attack.

Tactical Opening (Prerequisite: 5th-level Psion) 
You can communicate the openings in a creature’s defenses to your allies. When a creature fails a saving throw against your Telepathic Intrusion, you can take the Help action targeting that creature as a bonus action, helping another creature that you can telepathically communicate with. When you take the Help action in this way, the range of it becomes 30 feet.

Telepathic Link 
Your Telepathic Communication gains an unlimited range as long as you have communicated with the creature within the last day, and the target willingly maintains the link. However, it takes an action to focus to communicate over distance if you can’t see the target (or for the target to communicate with you if they can’t see you). You can maintain a link with a number of people equal to your Intelligence modifier in this way. You can convey as much in a turn as you could by speaking normally.

\begin{itemize}
  \item Psionic Discipline: Telepathy
\end{itemize}

\section*{Transposition Discipline}

Transposition is the ability to modify the properties of space and manipulate dimensional boundaries with your psionic powers.

\subsection*{Flicker Step}

On your turn, you can replace your movement by teleporting 5 feet in any direction to a space you can see. You can pass through creatures but can’t pass through objects, buildings or terrain more than 4 inches thick. This distance increases by 5 feet at 5th level (to 10 feet), at 11th level (to 15 feet), and becomes equal to your speed at 17th level. This replaces all movement for your turn when used.

\subsection*{Phase Rift}

Psionic Power

Casting Time: 1 action
 Range: Self
 Components: S
 Duration: 1 round

You step through space, traveling up to 10 feet in a straight line leaving a spatial tear behind. You can pass through creatures but can’t pass through objects, buildings or terrain more than 4 inches thick. Any creature in the path of this tear must make a Dexterity saving throw or take 1d8 force damage.

You can spend psi points up to your per use limit to add the following modifiers to Phase Rift (you can add multiple modifiers). The points must be spent when choosing the target of the power.

Blurring (1–3 psi point): You gain an illusory duplicate, as per the mirror image spell. You gain 1 duplicate per psi point spent (up to a maximum of 3). One remaining image fades at the start of each of your turns.

Disruptive (1+ psi points): Each target that fails their saving throw takes an extra 1d8 force damage for each point spent.

Echoing (2 psi points): You immediately use Phase Rift again with the same action.

Ethereal (2 psi points): You can pass through solid objects, buildings, and terrain as long as you end your Phase Rift in a space you can occupy. If your Phase Rift would end inside a space you can’t occupy, the power fails.

Long (1–3 psi points): You can travel an additional 10 feet for each point spent.

\subsection*{Alternate Effects}

Additionally, when you learn the Transposition psionic discipline you can use your Psionics feature to cast the following spells as per the rules defined in the feature:

\begin{minipage}{0.48\textwidth}
\begin{tabularx}{\textwidth}\toprule
{}XXXXX}
\midrule
Point Cost & \multicolumn{5}{c}{Alternate Effects} \\
\midrule
1 & \multicolumn{5}{c}{expeditious retreat, flicker*} \\
\midrule
2 & \multicolumn{5}{c}{misty step, blur, pass without a trace} \\
\midrule
3 & \multicolumn{5}{c}{blink, nondetection, turbulent warp*} \\
\midrule
4 & \multicolumn{5}{c}{banishment, dimension cutter*, dimension door} \\
\midrule
5 & \multicolumn{5}{c}{flickering strikes*, spatial manipulation*} \\
\midrule
\end{tabularx}
\end{minipage}\hfill
\begin{minipage}{0.48\textwidth}

\end{minipage}

If a spell can be cast at a higher level, you can spend additional psi points to cast it at a level equal to the psi points spent.

\section*{Transposition Talents}

The following talents can be selected if you have the Transposition Discipline:

Rift Strike 
When you use your Phase Rift power as an action, you can make a single weapon attack as a bonus action.

Flickering Escape (Prerequisite: 11th-level Psion) 
Whenever you roll a d4 for flicker* , you can teleport 5 feet after the attack resolves.

Lingering Rifts 
When you use Phase Rift you can choose to leave a 5 foot wide tear in reality behind, forming a line between your starting location and ending location until the start of your next turn. Any creature that enters this area for the first time or ends their turn in it must make a saving throw against the effect of Phase Rift as if it passed through them. If a creature is in the area of multiple lingering rifts, they are affected only once.

Phase Shot (Prerequisite: Rift Strike) 
When use your Phase Rift power, you can instead empower a piece of ammunition you touch, granting it the following special properties when you fire it:

\begin{itemize}
  \item It can be fired through all cover, including total cover, that is less than 4 inches thick.
  \item It deals 1d8 additional force damage.
\end{itemize}

Additionally, applying Phase Rift modifiers grant special effects on this attack:

\begin{itemize}
  \item Disruptive: The attack deals an extra 1d8 force damage on hit.
  \item Blurring: You can roll an additional attack roll per psi point spent, selecting the highest roll.
  \item Long: The effective range of the attack is increased by 10 feet per point spent.
  \item Echoing: A second projectile is created on firing, using a separate attack roll. (You may target any creature in range with this attack, including the same creature)
  \item Ethereal: The attack can travel through up to 10 feet of cover.
\end{itemize}

Phase Slash (Prerequisite: Rift Strike, 11th-level Psion) 
You can expend 1 psi point to make a single melee weapon attack against any targets you pass through with Phase Rift as part of the same action. You must spend 1 psi point per creature attacked this way. You can make this attack against a target once per turn. You can spend Psionic Mastery points on this talent.

Phase Shroud 
After using Phase Rift, you gain temporary hit points equal to your proficiency bonus + the psi points spent on the power.

\begin{itemize}
  \item Psionic Discipline: Transposition
\end{itemize}

\section*{Psychokinesis Discipline}

Psychokinesis is the mental art of spontaneously creating and controlling energy; sapping energy to freeze or spontaneously creating bursts of fire or arcs of lightning. Dangerous and destructive, it provides devastating power.

\subsection*{Energy Manipulation}

When you gain this feature, you can manipulate energy in minor ways at will. As an action, you can cause a light that emits 30 feet of bright light and 30 feet of dim light, cause objects you touch to catch fire as if touched by a match, cause small arcs of electricity, or freeze or thaw up to 5-foot cube of water. Any ongoing effect you create lasts 1 minute. You can have a number of simultaneous effects active up your Intelligence modifier, after which creating another ends the oldest ongoing effect.

\subsection*{Elemental Blast}

Psionic Power

Casting Time: 1 action
 Range: 30 feet
 Components: S
 Duration: 1 round

You can use your mind to create a burst of elemental power, blasting a target you can see within range. Make a ranged spell attack against the target. On a hit, the target takes 1d8 cold, fire, force or lightning damage.

For each die of cold damage dealt, the creature’s speed is reduced by 5 feet until the end of their next turn. For each die of fire damage dealt by the original attack, the creature takes 1d4 fire damage at the start of its next turn. For each die of lightning damage dealt, an arc of lightning strikes another creature of your choice within 20 feet, dealing 1d4 lightning damage (multiple arcs may strike the same creature). Damage from these secondary effects doesn’t add any bonuses (such as Empowered Psionics).

You can spend psi points up to your per use limit to add the following modifiers to (you can add multiple modifiers). The points must be spent when choosing the target of the power.

Amplified (1+ psi points): The target takes an extra 1d8 cold, fire, force or lightning damage on a hit for each additional point spent.

Lasting (1 psi point): Your Elemental Blast leaves a 5-foot radius sphere of devastation behind where it strikes until the start of your next turn. Creatures that enter this zone for the first time or end their turn in it must make a Constitution saving throw or suffer the secondary effects (slow, burn, arc) of the blast as if they had been hit by it.

Massive (1–3 psi points): You unleash a massive blast; rather than making an attack roll, all creatures within a 15-foot cone must make a Dexterity Saving throw. On failure, they take the effect as if hit by the Elemental Blast. The size of the cone is doubled for each point up to 3 spent (2 points for 30 feet, 3 points for 60 feet).

Overcharged (0 psi points): You take 1d4 force damage and do not add your proficiency bonus from the attack or spell save DC for your Elemental Blast, but increase the damage it deals by twice your proficiency bonus.

\subsection*{Alternate Effects}

Additionally, when you learn the Psychodkinesis psionic discipline you can use your Psionics feature to cast the following spells as per the rules defined in the feature:

\begin{minipage}{0.48\textwidth}
\begin{tabularx}{\textwidth}\toprule
{}XXXXX}
\midrule
Point Cost & \multicolumn{5}{c}{Alternate Effects} \\
\midrule
1 & \multicolumn{5}{c}{burning hands, lightning tendril*} \\
\midrule
2 & \multicolumn{5}{c}{flaming sphere, scorching ray} \\
\midrule
3 & \multicolumn{5}{c}{aether lance*, fireball} \\
\midrule
4 & \multicolumn{5}{c}{jumping jolt*, wall of fire} \\
\midrule
5 & \multicolumn{5}{c}{aether storm*, cone of cold} \\
\midrule
\end{tabularx}
\end{minipage}\hfill
\begin{minipage}{0.48\textwidth}

\end{minipage}

If a spell can be cast at a higher level, you can spend additional psi points to cast it at a level equal to the psi points spent.

\subsection*{Specializations}

When you take this Discipline, you may (but do not have to) select a specialization from the following list. You may only select a specialization at the time of gaining the Discipline, and can’t change or remove a specialization once selected.

Cryokinetic

You can only deal cold damage with Elemental Blast. When a creature is hit by an attack of your Elemental Blast or fails a saving throw against this power, it becomes frozen until the end of its next turn, giving it disadvantage on its next attack roll or Dexterity saving throw, and reducing its speed by an additional 5 feet (to a total of 10 feet).

If the attack roll is a critical hit or the target fails their saving throw by 5 or more, the target becomes restrained while frozen.

The following list of spells replace your Alternate Effects list:

\begin{minipage}{0.48\textwidth}
\begin{tabularx}{\textwidth}\toprule
{}XXXXX}
\midrule
Point Cost & \multicolumn{5}{c}{Alternate Effects} \\
\midrule
1 & \multicolumn{5}{c}{arctic breath*, entomb*} \\
\midrule
2 & \multicolumn{5}{c}{cold snap*} \\
\midrule
3 & \multicolumn{5}{c}{flash freeze*, sleet storm} \\
\midrule
4 & \multicolumn{5}{c}{ice storm, ice spike*} \\
\midrule
5 & \multicolumn{5}{c}{cone of cold} \\
\midrule
\end{tabularx}
\end{minipage}\hfill
\begin{minipage}{0.48\textwidth}

\end{minipage}

Electrokinetic

You can only deal lightning damage with Elemental Blast, but the size of the damage dice of lightning damage (both the initial damage and arc damage ) is increased by one step (to 1d10 initial damage and 1d10 additional damage per point spent on Amplified, and 1d6 arc damage to a nearby target per die of initial damage).

The following list of spells replace your Alternate Effects list:

\begin{minipage}{0.48\textwidth}
\begin{tabularx}{\textwidth}\toprule
{}XXXXX}
\midrule
Point Cost & \multicolumn{5}{c}{Alternate Effects} \\
\midrule
1 & \multicolumn{5}{c}{lightning tendril*, thunder punch*} \\
\midrule
2 & \multicolumn{5}{c}{crackle*, lightning charged*} \\
\midrule
3 & \multicolumn{5}{c}{electrocute*, lightning bolt} \\
\midrule
4 & \multicolumn{5}{c}{jumping jolt*} \\
\midrule
5 & \multicolumn{5}{c}{sky burst*} \\
\midrule
\end{tabularx}
\end{minipage}\hfill
\begin{minipage}{0.48\textwidth}

\end{minipage}

Pyrokinetic

You can only deal fire damage with Elemental Blast, but the size of the damage dice of fire damage (both the initial damage and burn damage) is increased by one step (to 1d10 initial damage and 1d10 additional damage per point spent on Amplified, and 1d6 burn damage at the start of their turn per die of initial damage).

The following list of spells replace your Alternate Effects list:

\begin{minipage}{0.48\textwidth}
\begin{tabularx}{\textwidth}\toprule
{}XXXXX}
\midrule
Point Cost & \multicolumn{5}{c}{Alternate Effects} \\
\midrule
1 & \multicolumn{5}{c}{burning hands, hellish rebuke} \\
\midrule
2 & \multicolumn{5}{c}{flaming sphere, scorching ray} \\
\midrule
3 & \multicolumn{5}{c}{fireball, fire cyclone*} \\
\midrule
4 & \multicolumn{5}{c}{fire shield, wall of fire} \\
\midrule
5 & \multicolumn{5}{c}{flame strike} \\
\midrule
\end{tabularx}
\end{minipage}\hfill
\begin{minipage}{0.48\textwidth}

\end{minipage}

\section*{Psychokinesis Talents}

Elemental Aegis 
As an action, you surround yourself with a swirling shield of fire, ice, or lightning. You gain temporary hit points equal to your psion level + your Intelligence modifier. Creatures that strike you while you have these temporary hit points take 1d4 damage of the shield type chosen.

Once you use this talent, you can’t use it again until you finish a short or long rest.

Elemental Emotions 
The powers you wield affect your state of mind, empowering you based on how you wield them. When you deal elemental damage, you gain the related mental property until the start of your next turn.

\begin{minipage}{0.48\textwidth}
\begin{tabularx}{\textwidth}\toprule
{}XXXXX}
\midrule
Element & \multicolumn{5}{c}{Effect} \\
\midrule
Cold & \multicolumn{5}{c}{You can add 1d4 to Wisdom saving throws.} \\
\midrule
Fire & \multicolumn{5}{c}{You can add 1d4 to Constitution saving throws.} \\
\midrule
Lightning & \multicolumn{5}{c}{You can add 1d4 to Dexterity saving throws.} \\
\midrule
\end{tabularx}
\end{minipage}\hfill
\begin{minipage}{0.48\textwidth}

\end{minipage}

If you are specialized in an element, you can instead add 1d6 to the related saving throw.

Elemental Shield (Prerequisite: 9th-level Psion) 
You gain the ability to cast fire shield without expending psi points.

Once you cast it this way, you can’t cast it again until you finish a long rest, unless you spend 4 Psi Points to cast it again.

Manifested Emotions (Prerequisite: Elemental Emotions) 
You can manifest your emotions into elemental powers that take physical shape. While you are in an elemental emotion, you can expend 2 psi points to manifest that emotion into the world as a bonus action; this takes the form of a mephit (ice mephit for cold, magma mephit for fire, and dust mephit for lightning).

As your emotion, given form, it acts immediately after your initiative at your directive. If not destroyed, it lasts for up to 1 minute, after which it naturally fades away. You can’t manifest more than one emotion at a time, and if it is destroyed, you take 2d6 psychic damage from the backlash.

This manifestation is not a sentient creature of its own, but simply a manifestation of your emotion.

Emotional Actions 
There are no mechanical limitations on the actions of the manifested emotion, but it is recommended you consider the emotion from which it is manifest when determining its actions. Magma mephits will act rashly and aggressively, dust mephits will be restless and easily distracted, and ice mephits will be cold and calculating.

\begin{itemize}
  \item Psionic Discipline: Psychokinesis
\end{itemize}

\section*{Precognition Discipline}

% [Image Inserted Manually]

Precognition is the ability to see what lies ahead, piercing the veil of the future. Seeing things that most people cannot a Psion with Precognitive abilities can, to a limited extent, know the future; such a future is known by tracing its roots from the present, and grows more mysterious and cloudy as such roots grow distant.

Such a future can be changed by the actions of the present, particularly by knowledge of said future and action to prevent or change it. This is the function most often leveraged by Precognitive Psions, peering into the future to steer around courses they seek to prevent or steer toward a possible outcome they seek.

\subsection*{Prescience}

Your eyes wander to events before they happen. You can add your proficiency to Perception and initiative rolls; if you already are able to add your proficiency to Perception, you can add twice your proficiency.

If you concentrate on keeping an eye on the future (as if concentrating on a spell), you can use your Intelligence modifier for Wisdom (Perception) checks (instead of Wisdom) and initiative rolls (instead of Dexterity), and you make Intelligence saving throws in place of Dexterity saving throws. You can initiate this concentration at any time.

\subsection*{Seeing}

Psionic Power

Casting Time: 1 action 
Range: Self 
Components: S 
Duration: Instantaneous

You concentrate and peer into the stream of future possibilities, gaining insight into what will happen next; you can select one of the following:

\begin{itemize}
  \item You can grant advantage to yourself or to a creature that can see or hear you on their next attack roll made before the start of your next turn; if an attack that gained advantage from this feature hits, it deals an extra 1d6 damage.
  \item You can grant disadvantage on the next attack roll against yourself or a creature that can see or hear you before the start of your next turn; if that attack hits despite the disadvantage, the damaged creature rolls 1d6 and subtracts that from the damage dealt.
\end{itemize}

You can spend psi points up to your per use limit to add the following modifiers to Seeing (you can add multiple modifiers).

Omniscient (1 psi point): The beneficiary of your Seeing is under the effect of bless and guidance until the start of your next turn.

Piercing (1+ psi points): You see through the target’s defenses when granting advantage, increasing the damage of the attack benefiting from advantage by 1d8 per psi point spent if it hits.

Positioning (1+ psi points): The beneficiary of your Seeing can immediately move 5 feet per psi point spent (up to their speed) without provoking opportunity attacks.

Thwarting (2 psi points): The beneficiary of your Seeing has advantage on the next saving throw they make before the start of your next turn.

Withheld (0 psi points): Rather than granting advantage or disadvantage on the next attack, you can grant advantage or disadvantage as a reaction to an attack being made until the start of your next turn.

\subsection*{Alternate Effects}

Additionally, when you learn the Precognition psionic discipline you can use your Psionics feature to cast the following spells as per the rules defined in the feature:

If a spell can be cast at a higher level, you can spend additional psi points to cast it at a level equal to the psi points spent.

\begin{minipage}{0.48\textwidth}
\begin{tabularx}{\textwidth}\toprule
{}XXXXX}
\midrule
Point Cost & \multicolumn{5}{c}{Alternate Effects} \\
\midrule
1 & \multicolumn{5}{c}{detect good and evil, future insight*} \\
\midrule
2 & \multicolumn{5}{c}{augury†, glimpse the future*} \\
\midrule
3 & \multicolumn{5}{c}{clairvoyance} \\
\midrule
4 & \multicolumn{5}{c}{death ward†, divination} \\
\midrule
5 & \multicolumn{5}{c}{scrying} \\
\midrule
\multicolumn{6}{c}{†Augury has the same effect, but doesn’t consult a specific entity when cast in this way.
 †Death ward has the same effect, but gives a forewarning to a creature allowing them to avoid death when cast in this way.} \\
\midrule
\end{tabularx}
\end{minipage}\hfill
\begin{minipage}{0.48\textwidth}

\end{minipage}

\section*{Precognition Talents}

Glimpsed Future 
When you use Seeing with yourself as the beneficiary, you can use it as a bonus action. When you use Seeing as a bonus action in this way, it doesn’t grant additional damage to the attack, or reduce damage taken when hit.

One Step Ahead 
When you are subjected to a saving throw, you can expend 1 psi point and glimpse into the future. You know what the effect you are saving against is if it is a spell or psionic effect as if you passed an Intelligence (Arcana) or Intelligence (Psionics) check to identify it. Additionally you can add your Intelligence modifier to your saving throw against it regardless of its nature.

Precognitive Dreams (Prerequisite: 11th-level Psion) 
When you finish a long rest, your dreams have prepared you for the day to come. Upon waking, you issue reassuring words and advice to your companions to help them survive the day, giving them temporary hit points equal to your Intelligence modifier. During the next 24 hours, you can’t be surprised.

\begin{itemize}
  \item Psionic Discipline: Precognition
\end{itemize}

\section*{Nullification Discipline}

Nullification is the ability to interfere with the supernatural effects of the world, reverting reality back to its original state.

\subsection*{Disruptive Touch}

When you gain this feature, you can create an aura of interference, wreathing yourself in the disruptive power that asserts reality. You can end minor magical or psionic effects (such as the result of cantrips or zero point psionic powers) by touching them, and have resistance to illusions or magical damage from things you touch (gaining advantage on the saving throw against them, if applicable).

If you are grappling or otherwise touching a spellcasting or psionic creature, at the start of your turn you can expend 1 or more psi to interfere with their abilities until the start of your next turn. In order to cast a spell or use a psionic power, they must succeed on a saving throw of their spellcasting or psionic ability score against your psionics DC, unless they are casting a spell with a higher level or using a power with more psi points than the psi points spent on this feature.

\subsection*{Denial}

Psionic Power

Casting Time: 1 action
 Range: 30 feet
 Components: S
 Duration: Instantaneous

You release a burst of raw psionic nullification at a creature you can see within range. The creature must make a Charisma saving throw. On failure, it takes 1d4 force damage as its existence is disrupted.

If the target is an aberration, celestial, construct, elemental, fey, fiend, undead, or a creature with the ability to cast spells or use psionic powers, it takes an extra 1d4 force damage, and becomes disoriented; until the end of its next turn, it rolls a d4 and subtracts the number rolled from all its attack rolls and ability checks, as well as its Constitution saving throws to maintain concentration. A creature can choose to fail the saving throw.

You can spend psi points up to your per use limit to add the following modifiers to Denial modifying its functionality (you can add multiple modifiers). The points must be spent when choosing the target of the power.

Aura of (3 psi points): Instead of targeting a creature, the power becomes an effect around you with a radius of 20 feet until the start of your next turn; any creature of your choice that enters or starts its turn in the area of effect must make a saving throw against the power.

Existential (1+ psi points): You deal an additional 1d4 initial and bonus (if applicable) force damage to the target creature on a failed save.

Firm (2 psi points): The target creature has disadvantage on the saving throw against this ability.

Lingering (1+ psi points): You can apply the effect of Disruptive Touch to an affected creature.

Supernatural (1+ psi points): A supernatural effect of or on the creature is ended; if the effect is a magical or psionic power, it is only ended if the psi points spent on this modifier equals or exceeds the spell level or psi points spent on the effect.

If the property is an innate property of a creature (either of the target creature, or affecting the target of this power), it is only ended if the psi points exceed the CR (or class levels) of the target creature, and they return at the end of that creature’s next turn. The supernatural effect can’t be its existence, unless it has less than 10 hit points and it would otherwise qualify.

\subsection*{Alternate Effects}

Additionally, when you learn the Nullification Psionic Discipline you can use your Psionics feature to cast the following spells as per the rules defined in the feature:

\begin{minipage}{0.48\textwidth}
\begin{tabularx}{\textwidth}\toprule
{}XXXXX}
\midrule
Point Cost & \multicolumn{5}{c}{Alternate Effects} \\
\midrule
1 & \multicolumn{5}{c}{protection from evil and good} \\
\midrule
2 & \multicolumn{5}{c}{nullify effect*} \\
\midrule
3 & \multicolumn{5}{c}{counterspell, dispel magic, remove curse} \\
\midrule
4 & \multicolumn{5}{c}{banishment} \\
\midrule
5 & \multicolumn{5}{c}{dispel evil and good} \\
\midrule
\end{tabularx}
\end{minipage}\hfill
\begin{minipage}{0.48\textwidth}

\end{minipage}

If a spell can be cast at a higher level, you can spend an additional psi point to cast it at that higher level.

\section*{Nullification Talents}

Deadspot (Prerequisite: 15th-level Psion) 
You gain the ability to expend 8 psi points to cast antimagic field . Once you cast it this way, you can’t cast it again until you finish a long rest.

Iron Templar 
You gain proficiency with medium armor and shields. If you already have proficiency with medium armor, you gain proficiency with heavy armor. You can perform somatic components for psionic powers with a hand carrying a weapon.

Additionally, when you hit a creature with a melee weapon attack or with a spell attack with a range of touch, you can use your Denial power as a bonus action targeting that creature.

Magical Anathema 
You gain resistance to damage dealt by spells or magical effects. The effect of all magical healing effects (including healing potions) on you is halved.

Magical Resistance (Prerequisite: 9th-level Psion) 
You have advantage on saving throws against spells and other magical effects.

\begin{itemize}
  \item Psionic Discipline: Nullification
\end{itemize}

\section*{Psionic Talents}

Astral Arms 
As a bonus action, you can expend 1 psi point to create psionic constructions serving as additional appendages. These arms last for 10 minutes. You determine the arms’ appearance, and they vanish early if you are incapacitated or die. You can use the astral arms to make unarmed strikes. The unarmed strikes you make with the arms use your Intelligence modifier in place of your Strength modifier for the attack and damage rolls. If you hit with one of them, you deal force damage equal to 1d6 + your Intelligence modifier, instead of the bludgeoning damage normal for an unarmed strike.

When you create them, or by spending 1 psi point as a bonus action while they are manifested, you can make a single unarmed strike with these arms as a bonus action. You can use Psionic Mastery points on this.

Aura Sight 
As an action, you can spend 1 psi point to psionically see the aura of a creature of your choice within 30 feet. When you see the creature’s aura in this way, you can determine if there are any spells or magical effects affecting the creature, and you learn their schools of magic, if any. You can also determine if the creature is under the influence of psionics. A shapeshifter or creature that is transformed or disguised by magical or nonmagical means must make a Charisma (Deception) check against your Psionics save DC. On a failure, you can perceive their original form in their aura.

Awaken Mind (Prerequisite: 9th-level Psion) 
You can cast awaken once without expending a spell slot or psi points.

You can’t do so again until you finish a long rest.

Beam of Annihilation (Prerequisite: 11th-level Psion, Elemental Mind subclass) 
You gain the ability to cast a beam of annihilation* for 6 psi points. If you have a specialization of Psychokinesis, you can only select the related elemental damage type, but the beam’s damage ignores resistance to that damage type.

Controlled Power 
You gain the ability to suppress the glow and somatic component of your psionic powers. You can expend 2 psi points to use a power without a visual sign or somatic component. Each time you use this talent, the cost of doing so doubles until you finish a short or long rest.

Divided Mind (Prerequisite: 9th-level Psion) 
You learn the divide self* spell, and can cast it by expending 5 psi points. When you gain access to the Innate Psionics feature, you may expend a use of Innate Psionics to cast divide self* at the level of the use of Innate Psionics expended. For example, if you choose the teleport spell for your Innate Psionics feature at 13th level, you could expend a casting of teleport to instead cast divide self* as a 7th-level spell, and can’t cast teleport or divide self* as a 7th-level spell in this way until you finish a long rest.

You do not require the material components of the spell when you cast it by expending psi points or a use of Innate Psionics.

Dreamwalker (Prerequisite: 9th-level Psion) 
You gain the ability to cast dream . You can cast the spell without expending a spell slot, but once cast, you can’t cast it again until you finish a long rest.

Elemental Penetration (Prerequisite: Elemental Mind subclass) 
When you use a psionic power or spell that deals cold, fire, or lightning damage, you can expend 1 psi point to make the power ignore resistance to that elemental damage type.

If you have a specialization of Psychokinesis, if the target has immunity to the damage type chosen, this instead turns immunity into resistance for that power.

You can spend psi points granted by Psionic Mastery on this ability.

Empowered Strike (Prerequisite: Psychokinesis or Telekinetic Discipline) 
Once per turn, as part of making a weapon attack as part of the Attack action, you can empower a melee weapon you are holding with psionic power. When you hit a creature with a weapon, you can apply Elemental Blast or Telekinetic Force modifiers (you can only select a power you know) to the attack (you can use Psionic Mastery on this). This doesn’t deal the base damage of the power, but any added damage causes the additional effects of the power to occur.

When applying a modifier that would make it target an area of effect, only the target takes the weapon damage, but other creatures in the radius become a target of the attack as if using the power normally.

Life Wielder (Prerequisite: Enhancement or Consumption) 
You learn the spell invest life* . You can spend Psionic Mastery points to cast this spell.

Mind Devourer (Prerequisite: 5th-level Psion) 
You gain the ability to cast psychic drain* for 2 psi points. Additionally, whenever a creature within 10 feet of you with an Intelligence score of 6 or higher dies, you can expend your reaction to draw in its psionic power, regaining 1d4 hit points and 1 expended psi point.

Mind Rider 
As an action, you can touch a willing creature to see through its eyes and hear what it hears for the next hour, gaining the benefits of any special senses that the creature has.

During this time, you are deaf and blind with regard to your own senses. You can end this effect at any time. While this is active, the creature has advantage on Intelligence, Wisdom, and Charisma saving throws.

Perfect Focus (Prerequisite: 10th-level Psion) 
You can enter a state of extreme focus. Your concentration is no longer interrupted by using a second ability that requires concentration, but your speed is reduced to 0 while concentrating on more than one effect; you have a -5 penalty to any Constitution saving throw to maintain concentration. If you move or fail a save to maintain concentration, one of the spells you are concentrating on ends. If you fail the saving throw to maintain concentration by 5 or more, you lose concentration on both spells.

Personal Truth 
Your power of conviction allows you to believe what you choose to. If you spend at least one minute convincing yourself of something—no matter how absurd—spells and effects to determine if you are telling the truth will register that you believe what you have convinced yourself of.

Additionally, your psionic powers allow you to impose your will upon reality to a certain extent. When you expend 1 psi point, you can perform a minor alteration to reality:

\begin{itemize}
  \item You can conjure any Tiny object you can imagine that is worth 1 sp or less. This creates an object as you envision it, but doesn’t grant you knowledge you do not otherwise have (for example, you would be unable to conjure a key to a lock unless you had a perfect mental image of the correct key).
  \item As an action, you can change the color or taste of a Small or smaller object with 10 feet. This change lasts for 1 minute.
  \item As a reaction to a creature within 30 feet taking damage (including yourself), you can change the damage type they take.
\end{itemize}

Potent Psionics 
When a target passes the saving throw against a damaging Psionic Power (granted by a psionic discipline), they still take half the damage, but suffer no other effects.

Psi Crystal 
You gain the ability to impart part of your mind into crystal. You can expend 2 psi points to cast the find familiar spell, but your familiar takes on the statics of a psi crystal (below) and the material component required is a crystal worth 10 gp instead of the normal material components. The Psi Crystal gains your Intelligence, Wisdom, and Charisma scores. You can use Psionic Disciplines with a range greater than self through your Psi Crystal as if you were standing in its location. If the psi crystal is destroyed, you gain its memories as your own. While you have a Psi Crystal active, as a bonus action, you can deactivate it to regain 2 expended psi points.

\subsection*{Psi Crystal}

\begin{tabularx}{\textwidth}\toprule
{}XXXXX}
\midrule
\multicolumn{6}{c}{Tiny construct, unaligned} \\
\midrule
\multicolumn{6}{c}{Armor Class 20 (natural armor)
 Hit Points 2 (1d4)
 Speed 0 ft., fly 20 ft. (hover)} \\
\midrule
STR & DEX & CON & INT & WIS & CHA \\
\midrule
1 (-5) & 10 (+0) & 10 (+0) & 10 (+0) & 10 (+0) & 10 (+0) \\
\midrule
\multicolumn{6}{c}{Skills Perception +4
 Damage Vulnerabilities bludgeoning
 Damage Resistances piercing, slashing
 Condition Immunities blinded, charmed, deafened, frightened, paralyzed, petrified, poisoned, stunned
 Senses blindsight 60 ft. (blind beyond this radius), passive Perception 14
 Languages understands the languages of its creator but can’t speak} \\
\midrule
\end{tabularx}

\begin{itemize}
  \item NPC: Psi Crystal
\end{itemize}

When you summon a Psi Crystal, you can store a fragment of your personality in it that you can then release by shattering the crystal. Select one of the following when summoning a psi crystal.

Courage. When you make a saving throw against the frightened condition, you can use your reaction to shatter the crystal, releasing that emotion to gain advantage on the save.
 Cowardice. When your Psi Crystal is within 30 of you and a creature comes within 5 feet of you, you can use your reaction to shatter the crystal, releasing that emotion and immediately moving up to your speed away from the creature without taking any opportunity attacks.
 Cruelty. When your Psi Crystal is within 30 feet of a creature that takes damage, you can use your reaction to shatter the crystal, releasing that emotion and causing the creature to take additional damage equal to your psion level.
 Sympathy. When your Psi Crystal is within 30 feet of you and another creature, if that creature takes damage, you can use your reaction to shatter the crystal, releasing that emotion and granting the creature resistance to that damage. You take an equal amount of damage to the damage they take.

Psionic Defenses 
You gain a way to defend yourself using your psionic powers. While you are not wearing any armor or carrying a shield, your AC equals 13 + your Intelligence modifier.

Psionic Weapon 
As a bonus action, you can expend 1 psi point to imbue a weapon you are holding with psionic energy. For 1 minute, once per turn when you deal damage with that weapon, you can deal an extra 1d6 psychic damage.

At higher levels you can expend additional psi points to further enhance the Psionic Weapon; 2 points to enhance it to 2d6 at 5th level, 3 points to enhance it to 3d6 at 11th level, and 4 points to enhance it to 4d6 at 17th level.

Projected Nightmares (Prerequisite: Shaper’s Mind Subclass) 
You gain an additional option for Boundless Imagination to apply to your Astral Construct:

\begin{itemize}
  \item Horrifying Nightmare: Creatures of your choice that start their turn within 5 feet of your Astral Construct must make a Wisdom saving throw against your Psionics DC or become frightened of your Astral Construct until the start of their next turn. On a successful save, they are immune to the effect for the next 24 hours or until you summon a new Astral Construct.
\end{itemize}

Propelled Bound (Prerequisite: Telekinesis or Psychokinesis) 
When you move on your turn, you can expend movement, up to your speed, in single bounding leap, propelled by telekinetic power or psychokinetic force.

Schism (Prerequisite: 5th-level Psion) 
You can spend 1 psi point to temporarily divide your mind to do two things at once until the end of your turn. While dividing your mind, if you use your action on a psionic power or spell granted by a Psionic Discipline, you can use your bonus action to use a psionic power that would normally take an action. The two powers share your per use psi point limit between them.

Tantrum (Prerequisite: Unleashed Mind subclass) 
Your anger boils just beneath the surface. When you roll initiative, you can instantly increase your rampage die by one step (from a d4 to a d6, for example). Additionally, if you take damage while your rampage die is a d6 or lower, your rampage die increases by one step.

\begin{itemize}
  \item Feature: Psionic Talents
\end{itemize}

\section*{Special Psionic Powers}

\subsection*{Telekinetic Weapons}

Psionic Power

Casting Time: 1 action
 Range: 30 feet
 Components: S
 Duration: Instantaneous

You telekinetically fling a weapon at a creature or object. Choose a weapon within 15 feet that isn’t being worn or carried, or choose a weapon under your control. Make a ranged spell attack. On hit the target takes damage equal to the weapon’s damage dice. The range of the attack decreases to 15 feet if the weapon has the heavy or special property, and increases to 60 feet if the weapon has the light property.

You can use Psionic Mastery points on this power, and this power counts as a Discipline Power of the Telekinetic Discipline (for example, for the purpose of Empowered Psionics).

You can spend psi points up to your per use limit to add the following modifiers to Telekinetic Weapons (you can add multiple modifiers). The points must be spent when choosing the target of the power.

Multiple (1+ psi points): For each additional psi point spent, you can fling an additional weapon, making a separate attack and damage roll for each weapon flung.

Whirling (2+ psi points): You can replace one throwing of a weapon with casting cloud of daggers . It is cast at a level equal to the psi points spent.

\begin{itemize}
  \item Psionic Power: Telekinetic Weapons
\end{itemize}

Flinging Magic Weapons

Strictly ruled-as-written, most magic weapon properties wouldn’t apply to Telekinetic Weapons, but personally I would grant a flung +1 weapon its +1 to attack and damage rolls. However, if you allow this, this should not be allowed to stack with any sort of focus or magic item that grants bonuses to spell attack rolls. Only one bonus should apply to a given roll.

Keep in mind that there’s not inherently a way to get your weapon back, though you may be able to with other Psion abilities (such as Telekinetic Hands), so flinging magic weapons may not be an ideal tactic for longer ranges.

\section*{Class Toolbox}

The class toolbox is just what the title implies. It’s an expanded set of tools for Inventor and Psion that provide additional upgrades, talents, and more, as well as some tips and tricks to help you build your own content for the classes.

The class toolbox is devided into two sections. First is optional variant options—additional subclass upgrades that represent ideas folks love, but aren’t suitable for all games in terms of balance or complexity. Some of these include the 19th-level upgrade series for Inventor, while others are upgrades that have large permenant effects on the character. These are optional content; they should always be considered to be at the discretion of the GM.

The second section is my guidance on how to create your own additional content for the classes—guidelines that will help you think about making upgrades, talents, and more, and what sort of considerations they should take into account if you want to maintain a balanced game.

\section*{Inventor Content}

\subsection*{Generic Upgrades}

9th-Level Upgrades

Cross-Disciplinary Expertise
 You can select an upgrade that applies to your Cross Disciplinary Knowledge selection. This can be an upgrade that directly improves it, or that requires it, of 5th level or lower. An upgrade selected in this way that does not normally have a limit (such as selecting Healing Draught or Frostbloom Reaction) can be used a maximum number of times equal to your proficiency bonus.

Alternatively, you can directly upgrade the benefit gained from your Cross-Disciplinary Upgrade, giving it a +1 to its attack rolls or save DC (up to a maximum total bonus of +3).

Lateral Studies
 You can select an additional selection from Cross Disciplinary Knowledge.

\subsection*{Gadgetsmith Upgrades}

19th-Level Upgrades

Flash of Brilliance
 As an action, you can sacrifice any three of your or gadgets (upgrades from this subclass) or essential tools to create create a new gadget on the spot. This gadget can recreate any other gadget provided by upgrades of this class, or can cast any spell of 5th level or lower once without expending a spell slot. This gadget lasts 1 minute or until used (if it casts a spell).

Your original gadgets can be rebuilt as part of a short rest. You can’t use this ability again until you finish a short or long rest.

\subsection*{Golemsmith Upgrades}

19th-Level Upgrades

Soul Puppet 
You bind your soul to your golem. Your consciousness is transferred to the golem, and you gain complete control of the golem, retaining both its action and your own, but you are only able to take any action the golem could take with its action.

Your golem’s mental stats are replaced by your mental stats, and your golem gains all of your spell slots, attunement slots, and class features. Your former body dies and can’t be resurrected, unless a wish or similar magic returns your soul to your body.

\subsection*{Potionsmith Upgrades}

19th-Level Upgrades

Replicated Legend 
Combining your mastery of reactions and alchemical infusions, you create an unstable replica of a very rare or legendary potion. You can create this potion as an action, which lasts 1 minute or until consumed, having the effect of the potion it replicates.

Once you do this, you can’t do so again until you finish a long rest.

\subsection*{Infusionsmith}

5th-Level Upgrades

Wondrous Infusion 
You infuse a magical item with your magic, creating a wondrous item. This item is an imperfect creation powered by your own magical power rather than a magical essence, and loses its magical power if you lose this upgrade. You can reclaim your power from the item by switching out this upgrade, or in a process taking 8 hours (after which you can infuse a new item). You can create a non-consumable wondrous item of Common or Uncommon rarity that is worth 500 gp or less.

19th-Level Upgrades

Investiture of Soul 
You learn the spell animate objects . It does not count against your spells known, and if you already know animate objects, you can learn a new spell of 5th level or lower that does not count against your spells known.

When you cast animate objects, you can expend a Hit Die in addition to the spell slot. When you do so, the spell is invest with your soul, and no longer requires concentration.

When cast in this way, it lasts until you cast animate objects again, or until all the animated objects are destroyed.

Size Matters Not
 When you animate a weapon, you can animate the weapon of a Large-sized creature. You can only animate one Largesized weapon at a time. This weapon deals twice the damage dice of a weapon for a Medium-sized creature. The weapon is always considered to have the heavy property, regardless of weapon type.

\subsection*{Thundersmith Upgrades}

19th-Level Upgrades

Lightning Saber (Prerequisite: Charged Blade) 
You generate enough energy with your charged blade to form the blade from pure scintillating lighting. Your weapon casts bright light in a color of your choice for 30 feet, and dim light for additional 30 feet.

If your target is wearing nonmagical armor or natural armor, your attacks treat their armor class as 10 + their Dexterity modifier; only magical effects can add additional AC to the target’s defenses.

Railgun (Prerequisite: Thunder Cannon) 
The range of your Thunder Cannon is doubled, and creatures can no longer gain cover from objects that are less than 2 feet thick. You Thunder Cannon attacks now pierce the target, and can attack a second target within range behind the first target, though Thundermonger damage can still only be applied once.

\subsection*{Warsmith Upgrades}

5th-Level Upgrades

Armor Entombed (Warplate or Integrated Armor) 
If your armor is perfect, why would you ever need to remove it? Your armor is considered both Integrated Armor and Warplate, and you can select the base AC and properties of either armor set, and select armor with either prerequisite as a prerequisite (though you can’t select upgrades that are incompatible with either type).

Your creature type becomes construct. You are immune to disease, and you gain immunity to poison damage and the poisoned condition. You no longer need to eat, drink, or sleep, though you must remain inactive for a minimum of 4 hours during a long rest to allow your systems to recharge. You can absorb certain types of healing magic, allowing you to recover hit points from magical effects that do not otherwise affect constructs (such as the cure wounds spell).

19th-Level Upgrades

Hyperspace Arsenal (Prerequisite: Recall)
 You expand the pocket dimension to store multiple sets of Wargear, as well as up to 100 pounds of weapons or items. As a bonus action, you can conjure and equip any weapon or item stored in the pocket dimension. Additionally, you can swap out your attuned Wargear for another set of Wargear with this upgrade in the pocket dimension during this bonus action, as long as it also has this upgrade, instantly ending your attunement to the set you were wearing and attuning to the new set.

Once you swap your Wargear using this feature, you can’t do so again until you finish a short or long rest.

\subsection*{Fleshsmith Upgrades}

19th-Level Upgrades

Coursing Vitality
 You mend with extraordinary speed. When you expend a Hit Die with Uncanny Vitality, you can expend more than one Hit Die (up to the number you have available).

If you have less than half your maximum Hit Dice, at the end of your turn, you regain a Hit Die.

\subsection*{Cursesmith Upgrades}

19th-Level Upgrades

Soul Anchor
 You bend all of your knowledge of soul devouring artifacts to create the pinnacle of dark magic. You bind your soul to an item. When you die, your body and equipment crumble to ash, reforming after 1 hour within 5 feet of the item to which you bound your soul. Once you revive in this way, you can’t revive in this way again for 1d10 days.

After the first time you have revived, your creature type becomes undead (if it was not already). The item ceases to work if you have not slain a creature in the last 10 days. The item has 25 hit points and an AC of 10. When you bind your soul to it, the item gains immunity to all damage types beside one, to which it gains vulnerability. You select this damage type when you create it.

\subsection*{Runesmith Upgrades}

19th-Level Upgrades

Living Rune
 When you cast a concentration spell, you can mark a rune that maintains concentration on the spell. The rune be can marked on a willing creature, or into the air (in which case it can’t move). The rune can be targeted and has an AC of 10 and 10 hit points, with resistance to nonmagical damage. It lasts for the duration of the spell unless destroyed.

You can only have one Living Rune marked at a time.

\subsection*{Relicsmith Upgrades}

19th-Level Upgrades

Consecrated Weapon
 You gain the ability to focus divine energy into a weapon. As a bonus action, you can touch a weapon, consecrating it. For 10 minutes, the weapon emits bright light in a 30-foot radius and dim light for an additional 30 feet. It deals an extra 1d8 radiant damage on hit while under this effect. If the weapon houses your Divine Relic, the damage is instead increased by 2d8 radiant damage. While consecrated, the wielder of the weapon is under the effect of the spell protection from evil and good.

Once you use this ability, you can’t use it again until you finish a long rest, unless you expend a 5th-level spell slot to, allowing you to use it again.

\begin{itemize}
  \item Feature: Optional Inventor Upgrades
\end{itemize}

\section*{Special Inventor Weapons}

The following are a series of weapons specially created for the Inventor. These are not part of the class, and not balanced against the class power budget, but are rather standalone magic items that are uniquely tailored to Inventors...mostly Thundersmiths, being that they are focused so keenly on cannons.

\subsection*{Adra'vark}

Weapon (hand cannon), rare (requires attunement)

A single piece of worked stone, it has twisting alien runes worked across the length. When fired, this Hand Cannon fires bolts of pure force energy and doesn’t consume ammunition, and lacks the loud property.

Any Thundermonger damage applied by this weapon becomes force damage. When applying Thundermonger damage to undead, the dice become d8s.

\subsection*{Arm Cannon}

Weapon (thunder cannon), common
 Can only be used by a character missing an arm.

One might hope to see this mostly on Warforged characters, but the world is a broad place. This weapon serves as a Thunder Cannon, but loses the two-handed property. The arm that this takes the place of can’t be used for anything besides holding and firing the thunder cannon or activating its upgrades.

\subsection*{Evolving Multiweapon}

Weapon, uncommon/rare/very rare/legendary (requires attunement)

This is a special weapon that leverages an Inventor’s mastery of creation to craft an evolving weapon that represents their developing skill. This weapon can only be created by an Inventor, and only the Inventor that creates it has proficiency in this weapon.

When you create this item, you select any two simple weapons. As a bonus action, you can convert the weapon between the two simple weapons. As your mastery of invention grows, you can evolve the weapon further.

As you progress in the Inventor class, you can evolve this weapon further, selecting from the following properties at the following levels:

\begin{tabularx}{\textwidth}\toprule
{}XXXXXXX}
\midrule
Level & \multicolumn{7}{c}{Property} \\
\midrule
5th & \multicolumn{7}{c}{The weapon gains a +1 to attack and damage rolls.} \\
\midrule
7th & \multicolumn{7}{c}{The weapon gains an additional simple weapon type to which it can transform.} \\
\midrule
9th & \multicolumn{7}{c}{The weapon gains an additional simple or martial weapon it can transform to.} \\
\midrule
11th & \multicolumn{7}{c}{The weapon adds +1d6 elemental damage to attack rolls. Select from acid, cold, fire, lightning, or thunder when you select this upgrade.} \\
\midrule
13th & \multicolumn{7}{c}{The weapon adds a +2 to attack and damage rolls.} \\
\midrule
15th & \multicolumn{7}{c}{The weapon gains an additional simple or martial weapon to which it can transform.} \\
\midrule
17th & \multicolumn{7}{c}{The weapon adds +3 to attack and damage rolls.} \\
\midrule
20th & \multicolumn{7}{c}{The weapon gains an additional simple or martial weapon to which it can transform., and adds a +4 to attack and damage rolls. As an action, you can change the elemental damage type of the weapon to another option of the 11th-level feature} \\
\midrule
\end{tabularx}

\subsection*{Rolling Thunder}

Weapon (thunder cannon), rare (requires attunement by an Inventor)

A lesser mind might look at this weapon and be led to believe that it is just five Thunder Cannons strapped together in a terrifying swirl of lunacy, but a true connoisseur can see that it is just awesomely five Thunder Cannons strapped together in an awesome swirl of brilliance.

This Thunder Cannon weighs 75 pounds and has the Heavy property.

When you fire this weapon with the Attack action, you can choose to fire up to five times, rolling a d20 for each attack roll at the same time. For each attack roll made, subtract 2 from the value rolled for all attack rolls. For example, when firing once this weapon takes a −2 penalty to its attack roll, when firing five times, it takes a −10 penalty to its attack rolls.

\subsection*{Storm Herald}

Weapon (thunder cannon), rare (requires attunement)

A massive, elegantly wrought cannon, it has many fine gears and sliding plates. In its compact mode, it functions as a Thunder Cannon. This weapon can add +1 to its attack and damage rolls.

As an action, you can deploy it into its "Herald" configuration, greatly extending the length of the weapon. While it is extended, the weapons range becomes 300/900 and your speed is halved, and any attack roll you make after moving on your turn is made with disadvantage.

You can collapse it back to its compact mode as a bonus action.

\subsection*{Thunder Shotgun}

Weapon (thunder cannon), uncommon (requires attunement)

A special cannon made to devastate at close range. This weapon has a range of 30/90, but doesn’t suffer disadvantage when there is a creature within melee range of you. It deals 4d4 damage to targets within 10 feet, 3d4 damage to targets within 20 feet, 2d4 damage to targets within 30 feet, and 1d4 damage to targets beyond 30 feet. This weapon has advantage on targets within 10 feet of you so long as there is nothing between you and the target.

This weapon must be reloaded with your bonus action once fired, and has disadvantage on all attacks if you have a Strength of 12 or less.

\subsection*{Special Inventor Weapons}

\begin{itemize}
  \item Item: Adra'vark
  \item Item: Arm Cannon
  \item Item: Evolving Multiweapon
  \item Item: Rolling Thunder
  \item Item: Storm Herald
  \item Item: Thunder Shotgun
\end{itemize}

\section*{Psion Content}

Like Inventor, Psion is nearly infinitely expandable. It can be expanded in Disciplines, Talents, Spells, and Feats. The toolbox will go over the creation rules of each of those in the next section, but as with the Inventor, this section will provide an example—something that is both content that can be used, and vehicle for discussing how extensions to the system work and what the considerations are.

\subsection*{Psionic Synthesis}

One of the ideas of psionics that has been presented several times through the iterations of the class, is the idea of seamlessly melding two Disciplines to create new effects. synthesized abilties. There are a few things that have to be considered here.

\subsection*{The Entry Point}

The first question is, “How does this system unlock?” There are three obvious routes: this can be a new functionality to the system without any investment, this can be a new talent, or this can be a feat.

Each presents a different budget for what this feature can do. If a feature doesn’t have any cost to unlock, it consequently shouldn’t have any added power. The trouble here is that new options almost inherently have added power, because flexibility is power. When there’s no investment into the system, you have no budget to work with, and consequently I wouldn’t recommend this route unless balance is not a strong consideration.

A talent and feat both provide a barrier to entry—and a very similar one, but with some notable differences. A feat can grant a talent, and only grants one talent when it does, so they are very similar in power. The difference comes in opportunity cost and accessibility. A talent can be taken with little opportunity cost to the character, while a feat is much harder to take in most cases, and will consequently raise the level of Inventor investment needed to unlock these options.

Generally, attaching an option to a feat will make fewer people engage with the system. However, this can make it a more unique decision point for the character, one that is entered with greater intentionality.

We’ll make Psionic Synthesis a feat for this reason. It results in inherently more complicated characters, and consequently it is not a system I want characters to stumble into. It represents another level of mastery and understanding of the system; not one that makes you better or stronger than characters that don’t use it, but one that allows you to explore new concepts within the system

\subsection*{Psionic Synthesis}

Prerequisite: 4th-level Psion, 2 or more psionic disciplines known

You gain the ability to meld your Psionic abilities together to produce potent new effects. When you select this feat, you can select one fusion talent for free. You can only select fusion talents when you have all the Disciplines in their prerequisite.

\section*{Fusion Talents}

Here are some ideas for fusion talents that you might make available to a character. The power budget here is similar to what a feat would grant (since they had to take Psionic Synthesis to get access). More importantly, they try to provide a good answer for "If a character has both these disciplines, what are they trying to do?"

Astral Rift (Prerequisite: Psionic Synthesis, Transposition Discipline, Projection Discipline, Astral Swap Talent) 
As a bonus action, you can Phase Rift to where your Astral Construct is; this movement counts as a Phase Rift, and you can apply modifiers to it as normal. Your Astral Construct is moved to where you started this movement along the Phase Rift; during the movement you can command your Astral Construct to attack a creature it passes with its action.

Elemental Phasing (Prerequisite: Psionic Synthesis, Psychokinetics Discipline, Transposition Discipline) 
As you step between planes, you can tap into the Elemental Planes, bringing their power into the material with you. You can use Psychokinetic modifiers on Phase Rift (excluding Massive).

Additionally, whenever you deal damage to an area with a Psychokinetics ability that would include yourself, you can phase yourself out of reality to take no damage from the effect.

Kinetic Mastery (Prerequisite: Psionic Synthesis, Psychokinetics Discipline, Telekinesis Discipline) 
Your ability to manipulate energy becomes a single blended prowess, freely swapping between manipulating force and energy. You can apply Telekinetic Force modifiers to Kinetic Blast and Elemental Blast modifiers to Telekinetic Force (though to do not gain any benefits from from a Specialization while doing so).

Parasitic Nightmare (Prerequisite: Psionic Synthesis, Consumption Discipline, Telepathy Discipline) 
You gain the ability to add Telepathic Intrusion modifiers to Mind Leech.

Additionally, when you deal psychic damage to a creature that is frightened of you, you can render yourself invisible to that creature until the start of your next turn; you can immediately (no action required) roll an Intelligence (Stealth) check affecting only creatures you are invisible from as a result of this talent.

If you are hidden from a creature that is frightened of you, when it moves, you can use your reaction to teleport to an unoccupied space within 5 feet of that creature (at the completion of their movement).

Phantom Blade Barrage (Prerequisite: Psionic Synthesis, Projected Weaponry, Telekinetic Weapons) 
You can create a weapon with Projected Weapon as part of making an attack with it using Telekinetic Weapons, allowing you to project as many weapons in this manner as you make attacks (no additional action required). When you use weapons created by Projected Weaponry as the projectiles for your Telekinetic Weapons, the range you can fling the weapons is doubled.

Additionally, if you expend 6 psi points on the Whirling modifier of Telekinetic Weapons, you can cast blade barrier instead of cloud of daggers , generating a much larger number of ethereal blades as part of the casting.

Physical Telekinesis (Prerequisite: Psionic Synthesis, Enhancement Discipline, Telekinesis Discipline) 
When you use an Enhancing Surge to empower a creature, you can expend a psi point to augment it with your Telekinesis to further assist them, optionally granting one of the following benefits:

\begin{itemize}
  \item You can move the creature 10 feet (this movement doesn’t provoke opportunity attacks).
  \item You can telekinetically deflect the first weapon attack against them before the start of your next turn, giving that attack disadvantage.
  \item You can telekinetically boost their next Strength (Athletics) check or weapon attack before the start of your next turn, granting that check or attack advantage.
\end{itemize}

Addititionally, while under the effect of Enhancing Surge, the amount you can lift, drag, or carry is doubled with assistance of your telekinesis, and you can apply Telekinetic Force modifiers to your weapon attacks.

Reality Warper (Prerequisite: Psionic Synthesis, Matter Made Real, Mental Image)
 You can bring the illusions you make people see to life. When you are concentrating on an illusion spell, as a bonus action you can expend psi points during your turn to bring aspects of the illusion into reality. For each psi point spent you can bring a 5-foot-cube section of the illusion into reality until the start of your next turn, and it can have the following effects:

\begin{itemize}
  \item If the effect would do damage, each 5-foot cube can damage one target. That target must pass a Dexterity saving throw, or take damage equal to a number of d8 dice equal to the level of the spell.
  \item If the effect would stop a creature, they can’t move through it. If the effect would restrain a creature, that creature must pass a Dexterity saving throw or become restrained until the start of your next turn.
  \item Each 5-foot cube can affect only one creature that is either within it or adjacent to it. The GM may allow other effects at their discretion.
\end{itemize}

The Void (Prerequisite: Psionic Synthesis, Consumption Discipline, Nullification Discipline) 
You devour all supernatural effects near you. Whenever a creature (including yourself) within 5 feet of you takes damage from a magical source, you can expend 1 or more psi points to use your reaction to reduce the damage taken by 1d8 per psi point spent. You can divide this amount between multiple creatures within range if they take damage at the same time.

Additionally, whenever you successfully cast counterspell , dispel magic , or remove curse , you regain hit points equal to the level of spell stopped or ended.

\begin{itemize}
  \item Feat: Psionic Synthesis (\& Fusion Talents)
\end{itemize}

\section*{Reflavoring Content}

Before we start building brand-new content, there’s another option to consider first: reflavoring content. This involves reskinning a feature, keeping its mechanics the same while changing the description to better fit your fiction. It provides an easier alternative: if all you’re changing is flavor and theme, you don’t have to worry about testing for balance nearly as much.

This isn’t always possible—after all, this whole book exists because new options do extend the kinds of characters you can play. But it is always worth considering, and so I’ll walk through an example of a popular reflavored option during testing... the "Dragonsmith". This isn’t necessarily offered as a unique option for play, but to show how something can be built from reflavored elements to achieve a unique outcome.

\subsection*{The "Dragonsmith"}

Playing some form of dragon-inspired content is always popular, and this could be a candidate for some custom upgrades, but I’ve accomplished this by using upgrades largely already present. Let’s walk through how I’d build a Dragonsmith out of an inventor.

First of all, we want to be as tanky as possible, and most of our upgrades are going to be about biting, breathing, or doing general dragon things.

We’ll start with Fleshsmith. While it tends to be somewhat on the horrifying end in its default flavor, let’s look at what we can get from it, based on what we need.

\begin{itemize}
  \item A dragon must breathe fire. We can accomplish this with Brimstone Bladder , giving us a fairly serviceable dragon breath.
  \item A dragon should bite and/or claw things. We have our choice here, but Extra Fangs is a good option for ensuring our chompers will be online. We can always pick up Extra Claws later.
  \item We’d like some natural armor , though we can make do with regular armor. Subdermal Plating reflavored to scales can fit the bill.
  \item We want some elemental damage on that bite. Fleshcrafted Enhancement with the Infernal Modifier on our Extra Fangs gives us just the fire damage we are looking for.
\end{itemize}

We are only missing one iconic things: Wings. I would generally never recommend flight as a custom upgrade before level 11, and fortunately as level 11 rolls around, we can grab Wings Seem Useful to give us the dragon wings we were missing. Optional upgrades include Better Eyes to give you that draconic blindsight, the aforementioned Subdermal Plating, and we can give kick it up a notch with Massive Mutation giving us a "Dragonform"... it even has a built in "Frightful Presence"!

This is absolutely not a bid to say you don’t need custom upgrades. This just shows an example of how you can adapt what is already there for the basis of your build. The class is highly modular, and consequently reflavoring can get you much further than it might with less modular content.

\section*{Building Custom Content}

Invariably no matter how many options are presented, the first thing people will want to do is make more. This is fine, and expected. I cannot give you a bulletproof way to make new content because it is an equation with infinite variables, but what this section will do is arm you with the best tools and advice I have for your endeavor in this effort!

\subsection*{Upgrades}

To speak generically about upgrades is challenging, but I can give some broad strokes of advice.

\begin{itemize}
  \item The easiest way to make a new upgrade is to look at another upgrade that does similar things, and adapt the logic and limitations of that upgrade to the new one you want to make. There’s one important caveat: if your upgrade provides a stacking bonus, make it mutually exclusive with other upgrades it might stack with.
  \item As a rule of thumb, if it grants a 1st- or 2nd-level spell, the upgrade can grant a free use per short rest if that’s all the upgrade grants. The exception is spells of exceptional power that should be limited or avoided... the classic example of this being the shield spell. I generally do not recommend giving an upgrade that grants the shield spell.
  \item Spells that grant 3rd-level spells or higher should never grant the spell at a level earlier than a full caster would get the spell. If the spell only grants a single spell, that can be granted at the level a full caster would get it, as a single use per long rest.
  \item Upgrades that grant additional spell lists can grant a single use of its 1st-level spell, and scale to a single use of a 2nd-level spell at 5th level. They should not grant more than one spell per spell level.
  \item Upgrades that grant static bonuses to rolls or armor class should be carefully considered. These can define builds, and should only play to the strengths of the subclass in a build-defining way.
  \item Upgrades that do not grant damage or defenses typically have more freedom to be powerful. You can be generous with movement, particularly if it doesn’t ignore opportunity attacks, but you can reference the existing movement speed and movement upgrades for reference (Accelerated Movement, Grappling Hook, and Boots of Striding as some examples).
  \item And finally for the Grand Secret . The key to balancing options is opportunity cost . Consider what they give up to take this upgrade, and if those things are of roughly equal value. An upgrade can only be made stronger by having a higher opportunity cost that is relevant to that character. Giving up heavy armor, for example, is not an opportunity cost to a Dexterity-based character, but giving up a flat amount of AC would be.
\end{itemize}

Keeping those general guidelines in mind covers a surprisingly high number of the use cases. The game has infinite possibilities, but those possibilities generally fall into the categories above.

\subsection*{Talents}

A lot of the guidance for upgrades is relevant to talents. If you skipped that section to get here, I’d recommend going back for it! But they do have some differences. The first and most important difference is that the budget for talents is lower than the budget for upgrades, and this has to be kept in mind. Talents also can be swapped out as you level for higher-level talents, so you cannot safely make new versions of high-level talents without removing existing ones (that are relevant to the character) from the pool of options.

Here’s a list of the advice in broad strokes:

\begin{itemize}
  \item Talents should, by and large, not deal additional damage. They can give additional ways to deal damage, but should rarely increase the damage of things you are already doing, unless they are taking the place of a talent that deals damage, and are exclusive with it. If a character taking your talent is adding damage without removing it elsewhere, you should exercise caution.
  \item Talents that break the rule above or otherwise provide a potent effect should take a bonus action to activate. The Psion has many potent bonus action uses, and this limitation can serve as a good opportunity cost to gate the combination of features.
  \item Talents should generally not grant free uses of combat relevant spells, and if they do, should recharge on a long rest.
  \item Talents that grant spells should either grant no more than two spells, or grant a thematic list of mixed combat value.
  \item Psion is a short rest caster. Any effects that last more than 1 hour should be tightly curtailed or limited per long rest. For example, a talent should not grant the animate dead spell without limiting it to 1/long rest, and generally any healing effects should be similarly limited, if they go beyond the strength of cure wounds.
\end{itemize}

These general guidelines cover most of the instances I can think of... and hopefully cover most of the instances you’ll think of, but there’s always room for more. Keeping these guidelines in mind, in particular the first one, will usually keep you fairly safe.

The three limitations of Psions are short rest resources, bonus actions being tied to most of the potent effects, and, of course, limited talent selection. Keep these in mind and you should be fine to expand the tapestry into infinite possibility.

\subsection*{Disciplines}

This is where things get a bit more meaty, and is something I would recommend with caution. If you are creating a new Disicipline, you are engaging in a major undertaking, and taking balance into your own hands.

But I’m happy to steer you the best you can, and consequently, I’ll give you the best advice I got in this endeavor, so you’ll at least be able to proceed with that!

\begin{itemize}
  \item All Disciplines are constructed in three parts: a minor passive feature that serves the function of a utility cantrip, a psionic power that serves the role of a flexible damage cantrip (with a list of modifications for spending psi points), and an alternate spell list. All three of these are balanced together.
  \item The utility portion of the discipline should be more powerful and flexible than a typical utility cantrip. It often requires no action or replicates the effect of multiple cantrps, but as a rule, it should not deal damage, and should be tightly constrained to the concept of the discipline.
  \item The psionic power of the discipline is the bulk of its power, and should be the thing you have in mind when setting out to make a Discipline. It should always take an action. If it deals damage, it should deal 1d8 or 1d10, and have a secondary effect, called a "rider effect". This can vary greatly, but if it deals damage it shouldn’t exceed a total damage of 1d12 without further investment. The rider effect should be consistently useful, but tied to the thing the power is doing.
  \item The list of modifications should be phrased in such a way as to construct the name of the power, generally as prefixes to the power (for example the modifier "massive", building to "massive elemental blast").
  \item There should be one option that scales the damage, such that if all psionic mastery points were used on that modifier, the damage of the power would scale the same way as a cantrip (i.e. 1d8 at level 1, 2d8 at level 5, etc).
  \item Modifiers should be more expensive than a spell that costs the same number of points. Remember that (a) psionic mastery points can reduce the cost of these, and (b) that these are more action economy efficient than a spell, as you are still getting the basic effect of the power on top of the modifier effect.
  \item Crowd control effects should cost as much as a spell that does that thing, but should not last more than 1 round, and should be used with some caution. For example, setting the cost of an ability that paralyzed at 2 points would be far too strong, despite the existence of hold person , even if the affect only lasted one round. An effect that paralyzed should cost 4–5 points, and be prohibitively expensive.
  \item If the power targets weak saves (Strength, Intelligence, Charisma) its effects should be considered carefully. Powers uniquely can target weak saves, but that should be accounted for as part of the power budget.
  \item The alternate spell list should include a mix of good spells and weak spells, but should generally not include shield.
  \item If the power is exceptionally useful, the alternate effects should focus more on utility spells if possible.
\end{itemize}

Hopefully this can serve as a basis on your journey for making a new discipline. It is a complicated endeavor, but taking it one step at a time, and comparing it to existing disciplines can get you pretty far.

\section*{Setting Notes}

\begin{minipage}{0.48\textwidth}
\subsubsection*{Inventors in Your Setting}
\end{minipage}\hfill
\begin{minipage}{0.48\textwidth}
\subsubsection*{Psionics in Your Setting}
\end{minipage}

\begin{minipage}{0.48\textwidth}
Adding Inventors to your setting might be a large change... or it might be no change at all. Inventors and their inventions can be the herald of a new method of scientific and systematic thinking about magic into more organized and useful structures—the forebearers of a magi-tech revolution... or they might simply be an eccentric wizard with strange methods others find completely unreproducible.

While Inventors are fully at home in any number of settings that end in the word "punk", they can also fit into almost any classical fantasy setting, simply with the assumption that their creations are ultimately fueled by their own power.

All of which is to say that the impact of inventors in your setting comes down to the reproducible nature of their creations, and the default assumption is that their class features are not reproducible at all.

What is more likely to have wide implications for a setting are the ways of thought of an Inventor, and this can make an interesting element or plot. What would the powers that be think of magical street lamps? Are magical streetlamps already widespread in your setting? Inventors are sometimes called an aspect of a "wide magic" (as opposed to a "high magic" vs. "low magic") dichotomy. They don’t bring higher magic to the table than a Wizard, but they are tied to the idea of "useable" magic.

Here are some considerations a GM and Inventor can consider:

\begin{itemize}
  \item Are Inventors a recognized profession? Are they unique or rare, or is the setting in full swing of tinkering with the laws of nature through magic technology?
  \item Are their goals focused purely on creating their own tools and maximizing their ability as an adventurer, or are they trying to create reproducible effects? This can often tie into the crafting system.
  \item What do the systems of power in your setting (kingdoms, guilds, gods) think of invention? Is there a force that maintains the status quo?
  \item Are their inventions the cutting edge of the setting? Are there guilds competing for who can make the biggest, baddest, thunder cannon?
\end{itemize}

Inventors are not inherently different or disruptive—at the end of day they are just adventurers doing adventurer stuff. But they play to many themes that can have setting wide implications if that’s the route you want to go down.

In many fantasy settings, there is the implication that there is a "grand age of magic" in history where magic items and things were made. An Inventor with these abilities may be someone that is delving into the ancient past, an inheritor of what once was. They might be the reason that assumption isn’t true in your setting. They might be why your setting is in a great age of magic right now, building airships powered by crystals or elementals or fiendishly complicated metal contraptions.

As with many things, the best (and only) course of action is for the player and GM to discuss and align their expectations. It’s the GM’s world, and they may have put a lot of work into it! Consider what works best for the world, and what themes you want to play towards. An Inventor in your party can be a plot point... but it doesn’t have to be.
\end{minipage}\hfill
\begin{minipage}{0.48\textwidth}
Psionics in your setting can be whatever you want them to be. They can be the same force as magic, or they can be something entirely different. Here’s what I generally assume them to be:

\begin{itemize}
  \item Psionics are supernatural powers that operate on the same mechanism and principles of magic, but is a different way to tap into and frame that power.
  \item Psions draw their power from raw mental rigor that comes in different forms based on their subclass. It is a power that can be taught, but may rely on some underlying talent. While most similar to the powers of a Wizard, their structures of magic are internal, mental disciplines that are forged into their mind.
  \item In my setting, the power behind ki and psionics is the same. They are powers of focus and mental discipline that manipulate the supernatural weave of the world.
  \item They can be any of these in your setting, or none of these. The simplest and least impactful way to add psionics to a setting is to simply call it another way to use magic, and the system presented here supports that approach, but it also supports the extreme other end, while tending to assume something in the middle.
\end{itemize}

Here are some considerations for how you might treat it, as well as how I answer these questions in my settings:

Do Psionics work in areas that magic does not? Dead spots, antimagic zones, etc? 
For my setting, they do not. It is far simpler to just transpose all the rules of magic onto psionics. The base game does not assume the existence of player-psionics, and consequently features like "magic resistance" do not mention psionics. If you would like psionics to be a completely different power, you can certainly allow them to not interact directly with magic, but it will require more work, as you’ll need to decide which creatures should have unique resistance to psionics, etc.

Are Psionics widely recognized? Do people understand what they are? 
This same question can be posed to magic in a setting. In my setting, magic is widely recognized, and most people understand it. Psionics are far more rare and considered aberrant abilities, but scholars, adventurers, and the well informed will recognize them... often with some wariness or superstition. Folks with an understanding of magic may treat psionics the same way mundane folks might treat those that use magic—with fascination and distrust.

Is the Psion player character an anomaly? Is this a new outbreak of new powers that heralds the shifting of something beyond the material, or are they part of an ancient tradition? What is their place in the setting? 
These are all really the same question. A psion in a setting can be the introduction of psionics, or they can be an odd curiosity, a unique magic user whose abilities work with a subtle difference.

In my setting, psionics are an ancient power that predates the current systems of magic, introduced to mortal populations by interactions with ancient eldritch powers. These powers can be taught from preserved ancient traditions, or burst forth wildly from some new brush with the unknown. They are rare, but the level of rare to generate gossip, rumor, and curiosity. As with all things, set expectations with your player if their choices will draw them special attention.
\end{minipage}

\chapter{Chapter 2: Subclasses}

\section*{Path of the Raging Mind}

% [Image Inserted Manually]

The Path of the Raging Mind is a Barbarian that expresses their fury as a telekinetic manifestation. Their mind is so powerful that in their rage they warp the world around them. This is a psionic manifestation of a barbarian’s ability to focus their fury, and in battle this fury is weaponized as a deadly force that can fling about weapons and enemies alike.

They have to struggle with control... until they unleash themselves upon the battlefield, wreaking havoc on all that stands in their way.

Mental Strength

This subclass often represents strength of mind, rather than raw brawn, but does not use Intelligence. This is because applications of your mental strength are generally via telekinesis, not academic application, and Strength continues to better mechanically represent how it functions. Consider if your character’s physical strength is more of a representation of their telekinetic abilities when using their Strength ability score.

Even prior to taking this path, perhaps the character’s strength came from the instinctive use of their telekinesis at short range, only learning to manifest it beyond the range of their touch as they take this path.

If you would like to play this as an Intelligence class, replacing all references to Strength with Intelligence, it doesn’t really break anything. Consult your GM.

\subsection*{Telekinetic Fury}

Starting when you choose this path at 3rd level, when you enter a rage, you can telekinetically project your rage onto the world around you. All melee weapons gain a thrown 30/60 property for you. This range is reduced to 15/45 if the weapon has the heavy property and increased to 60/120 if the weapon has the light property.

You can apply your Rage bonus damage to damage rolls made by thrown weapons, and can gain advantage from Reckless Attack on attacks made with thrown weapons.

At the start of your turn while you are raging, you can call a number of weapons or Tiny objects equal to your proficiency bonus within 60 feet to you, as long as they are not being worn or carried. You call these to your hands or to telekinetically orbit around you until the end of your turn (after which they fall to the ground). Weapons telekinetically orbiting you can only be used when throwing that weapon. At the end of your turn (or if your rage ends) all orbiting weapons drop to the ground around you.

\subsection*{Distant Grasp}

Additionally at 3rd level, you can manipulate small objects within 30 feet with your mind as if using your hand to interact with them, though you have disadvantage on any checks relating to Dexterity when doing so. You can use this power to manipulate an object, open an unlocked door or container, stow or retrieve an item from an open container, or pour the contents out of a vial. You can’t Attack, activate magical items, or carry more than 10 pounds in this manner. You can move an item you are controlling in this way up to 30 feet during your turn.

While raging, the amount of pounds you can lift with this ability increases to 20 times your Barbarian level.

\subsection*{Forceful Mind}

Starting at 6th level, you gain the ability to use the psionic power Telekinetic Force. The DC for your Telekinetic Force is 8 + your Strength modifier + your proficiency bonus. When you take the attack action, you can use Telekinetic Force in place of one attack. While you are raging, you can add your Strength modifier and Rage bonus damage to damage done with Telekinetic Force.

\subsection*{Telekinetic Force}

Psionic power

Casting Time: 1 action
 Range: 60 feet
 Components: S
 Duration: Instantaneous

You smash a target creature or object you can see with your mental power. The target must succeed on a Strength saving throw, or take 1d10 bludgeoning damage and be shoved 5 feet in a direction of your choosing or be knocked prone.

\subsection*{Interference Field}

Additionally at 6th level, while you are raging, you and all allied creatures within 10 feet of you are considered to have half cover against ranged weapon attacks, as your telekinetic powers distort the area around you.

\subsection*{Collateral Damage}

Starting at 10th level, when you use Reckless Attack with a thrown weapon, you can ricochet to try and hit a second target within 5 feet of the first. Use the lower of the two dice you rolled for the Reckless Attack as the attack roll against the second creature. If disadvantage would cancel out the advantage from Reckless Attack, you may not use this feature.

\subsection*{Fling Foes}

Starting at 14th level, you become the epicenter of the battlefield, flinging targets about. When you hit a target with a weapon while raging, you may choose to knock them 5 feet directly away from you on hit. When you use Telekinetic Force while raging, you can always choose to apply the Hurling modifier (as if using 2 psi points), shoving the target an additional 20 feet in a direction of your choice.

Additionally, all Medium and smaller objects and creatures that you are holding become improvised weapons for you, dealing 1d6 bludgeoning damage, with the thrown 20/60 property. To throw an unwilling creature, you must have it grappled, and you can’t throw it beyond its normal range.

Whenever you shove or throw a creature into another creature, both creatures take 1d6 bludgeoning damage.

\begin{itemize}
  \item Barbarian: Path of the Raging Mind
\end{itemize}

\section*{Path of the Exosuit}

A truly unique Path, this is often defined by its exact nature, but such a nature varies widely. Perhaps it is a lost piece of technology, such as long dead Warsmith’s cast off armor that has developed strange properties, or perhaps it is an alien symbiote of some kind, with its own goal or objective (perhaps as simple as harmlessly feeding off your rage or your foes).

You can select one of the options from the table below or work with your GM to make an entirely new concept. It can be biological or mechanical in nature, or could even be part of your race or origin. Perhaps it is something you find or something you make, but whatever its origin, it grants you terrifying capabilities on the battlefield.

\begin{minipage}{0.48\textwidth}
Example Exosuit Table

\begin{tabularx}{\textwidth}\toprule
{}XXXXXX}
\midrule
d6 & \multicolumn{6}{c}{Suit Description} \\
\midrule
1 & \multicolumn{6}{c}{A complex machine that expands from a small metal ball.} \\
\midrule
2 & \multicolumn{6}{c}{A strange symbiote that envelops you, stored in your blood.} \\
\midrule
3 & \multicolumn{6}{c}{A crystalline suit that phases into existence from a gem-like pendant.} \\
\midrule
4 & \multicolumn{6}{c}{A set of mystical runes that cover your body.} \\
\midrule
5 & \multicolumn{6}{c}{A mechanical suit of armor that becomes integrated into your body} \\
\midrule
6 & \multicolumn{6}{c}{A sentient set of armor that walks around, equipping itself to you in combat.} \\
\midrule
\end{tabularx}
\end{minipage}\hfill
\begin{minipage}{0.48\textwidth}
Invention or Artifact?

In general, this path represents a character coming into contact with a highly advanced and powerful exosuit. Increases in powers represent the character becoming proficient with its abilities, recovering them, or the character becoming strong enough to support those systems being used. Invention would typically be more represented by an Inventor Warsmith.

But this is a roleplaying game! Do what fits for your character! A combination of backgrounds, ability scores, feats, and subclass can represent a huge range of characters.
\end{minipage}

\subsection*{Exosuit}

Starting when you choose this path at 3rd level, you become bonded to the exosuit, though you need not necessarily always be wearing it. You are permanently connected to it in some way, and even if it is lost or destroyed, it recovers itself through its bond to you.

In its dormant state, it has a small collapsed form, recedes into your body, or has no form at all (based on its nature). When you enter a rage, it is activated, becoming a complete suit of armor. It subsumes whatever armor you are wearing when it activates, adopting the AC of that armor (if you are not wearing armor, it has the AC of your Unarmored Defense). In addition, it grants you the following properties:

\begin{itemize}
  \item Regenerating Armor. Your exosuit protects you from blows, reducing the damage you take by your proficiency bonus.
  \item Iron Fists. If you are not carrying anything in your hand, that hand is treated as a weapon that deals 1d8 bludgeoning damage with the light property.
  \item Smashing Force. Whenever you succeed on an Athletics (Strength) check to shove or grapple a creature, you can deal 1d4 + your Strength modifier bludgeoning damage to the target creature.
\end{itemize}

\begin{minipage}{0.48\textwidth}
Damage Reduction and Resistance

Note that Damage Reduction is tallied before resistance.
\end{minipage}\hfill
\begin{minipage}{0.48\textwidth}

\end{minipage}

\subsection*{Empowered Movement}

Starting at 6th level, your integration with your exosuit grants you new movement options. You gain a climbing speed equal to your walking speed, and the distance you can jump is tripled.

\subsection*{Power Fist}

Also at 6th level, when you make an attack roll with your Iron Fists, you can choose to forgo adding your proficiency bonus to the attack roll. If the attack hits, you can add double your Proficiency bonus to the damage roll.

Additionally, your Iron Fists count as magical for the Purpose of overcoming resistance and immunity to nonmagical attacks and damage.

\subsection*{Heavy Lifting}

Starting at 10th level, while you are raging, when you are moving a grappled creature, your speed is no longer halved.

\subsection*{Slam}

Additionally at 10th level, once per turn during your turn, when you are moving a creature you have grappled, you can smash them into various features of the battlefield. If the grappled creature’s movement is stopped by colliding with a Large or larger object, a Small or larger creature, or immobile terrain, the grappled creature and the object it collides with both take 1d6 bludgeoning damage for each 15 feet you had moved prior to the collision.

\subsection*{Infinite Leap}

Starting at 14th level, while you are raging, you gain a flying speed equal to your walking speed.

\begin{itemize}
  \item Barbarian: Path of the Exosuit
\end{itemize}

\section*{College of Thunder}

% [Image Inserted Manually]

Some bards grow in talent, weaving beautiful ballads that move their listeners to tears. Some bards just turn up the volume. The College of Thunder is a path for those who want to be heard—to convey their message with great emphasis to as many people as possible. They often are keenly interested in innovating their instruments, finding new improvements that allow them to make new, louder, sounds... even if the musicality of those sounds is sometimes debated by more classical artists.

Lacking in subtlety, these bards are viewed as innovators and pioneers by themselves... and generally a large pain in the ears by most others, especially their preceding generation.

\subsection*{Dramatic Cantrips}

At 3rd level, you become quite adept at getting loud. You learn the thaumaturgy and thunder noteK cantrips, which don’t count against the number of bard cantrips you know.

\subsection*{Harsh Chord}

Starting at 3rd level, as an action you can expend one use of your Bardic Inspiration to play a powerful chord that sweeps outward. Each creature of your choice within 20 feet of you must make a Constitution saving throw. A creature takes thunder damage equal to your bardic inspiration die + your bard level on a failed saving throw, and half as much damage on a successful one. Constitution saving throws to maintain concentration on spells triggered by this damage are made with disadvantage.

\subsection*{Thundering Magic}

\begin{minipage}{0.48\textwidth}
Additionally at 3rd level, you gain access to an expanded spell list. You learn the following spells at the following levels, and they do not count against the number of bard spells known. The following spells are considered bard spells for you even if they don’t normally appear on the bard list.
\end{minipage}\hfill
\begin{minipage}{0.48\textwidth}
\begin{tabularx}{\textwidth}\toprule
{}XXXXX}
\midrule
Bard Level & \multicolumn{5}{c}{Spells Learned} \\
\midrule
3rd & \multicolumn{5}{c}{thunder punch*, shatter} \\
\midrule
5th & \multicolumn{5}{c}{thunder pulse*} \\
\midrule
7th & \multicolumn{5}{c}{echoing lance*} \\
\midrule
9th & \multicolumn{5}{c}{sonic shriek*} \\
\midrule
\end{tabularx}
\end{minipage}

\subsection*{Louder}

Starting at 6th level, when you cast a spell that deals thunder damage, you can add your Charisma modifier to the damage dealt.

\subsection*{Amplified Antics}

Additionally at 6th level, when you cast a bard spell with a range greater than 5 feet, the range increases by 30 feet (this doesn’t affect the area of spells, including spells with their area listed in their range such as thunderwave).

\subsection*{Reverberating Echo}

At 14th level, when you expend a spell slot to cast a spell that deals thunder damage, you can cast that spell again on your next turn without expending a spell slot, but the damage it deals is halved.

\subsection*{College of Thunder Quirks}

The following are some optional quirks for a player of the College of Thunder. You can optionally roll or select a quirk that suits your character.

\begin{minipage}{0.48\textwidth}
\begin{tabularx}{\textwidth}\toprule
{}XXXXX}
\midrule
d6 & \multicolumn{5}{c}{Quirk} \\
\midrule
1 & \multicolumn{5}{c}{All performance checks are better when made at maximum volume.} \\
\midrule
2 & \multicolumn{5}{c}{Your fashion taste is memorable.} \\
\midrule
3 & \multicolumn{5}{c}{You plunge into things recklessly. Old age is for boring people.} \\
\midrule
4 & \multicolumn{5}{c}{You prefer odd instruments that you tinker with endlessly.} \\
\midrule
5 & \multicolumn{5}{c}{Everything is a stage if you believe hard enough.} \\
\midrule
6 & \multicolumn{5}{c}{You sometimes get very involved in mimed performances.} \\
\midrule
\end{tabularx}
\end{minipage}\hfill
\begin{minipage}{0.48\textwidth}

\end{minipage}

\begin{itemize}
  \item Bard: College of Thunder
\end{itemize}

\section*{Mystery Cult}

% [Image Inserted Manually]

While most clerics follow defined gods who seek to convert others to the cause, those of the Mystery Cults often intentionally obscure their faith with esoteric references and cloaked intentions. No less devoted than their more classically divine peers, they are not merely warlocks, but people of true faith in their dark and unknown deities.

Standing at the blasphemous confluence of divine and psionic power, they are almost universally condemned and feared, often shrouding their nature and abilities with mystery and working toward esoteric goals.

\subsection*{Domain Spells}

You gain domain spells at the cleric levels listed in the Mystery Cult Domain Spells table.

Mystery Cult Domain Spells

\begin{tabularx}{\textwidth}\toprule
{}XXXX}
\midrule
Cleric Level & \multicolumn{4}{c}{Feature} \\
\midrule
1st & \multicolumn{4}{c}{frighten*, terrifying visions*} \\
\midrule
3rd & \multicolumn{4}{c}{augury, detect thoughts} \\
\midrule
5th & \multicolumn{4}{c}{fear, nondetection} \\
\midrule
7th & \multicolumn{4}{c}{compulsion, summon horror*} \\
\midrule
9th & \multicolumn{4}{c}{dominate person, modify memory} \\
\midrule
\end{tabularx}

\subsection*{Silent Language}

Starting when you choose this domain at 1st level, you can communicate telepathically with any creature you can see within 30 feet of you, and they can reply in kind. You don’t need to share a language with the creature for it to understand your telepathic utterances, but the creature must be able to understand at least one language.

\subsection*{Uncanny Insight}

Additionally at 1st level, as a reaction to being hit by an attack, you can add your Wisdom to your AC until the start of your next turn, including against the triggering attack.

You can use this feature a number of times equal to your proficiency bonus. You regain all expended uses when you finish a long rest.

\subsection*{Channel Divinity: Invoke Betrayal}

Starting at 2nd level, you can use your Channel Divinity to influence the mind of a target within range. When a creature within 30 feet makes an attack roll, as a reaction you can force that creature to make a Wisdom saving throw. On failure, you can direct the attack at another creature within range.

\subsection*{Esoteric Rites}

Starting at 6th level, you can develop your faith in one of the following ways, unlocking special powers as a reward for your conviction. You select one of the following features. You can select additional Esoteric Rites at 10th and 14th levels.

Gift of Tongues

You learn Deep Speech. If you already know Deep Speech, you gain one additional language of your choice. You can use this esoteric language as your verbal components, and creatures can no longer successfully identify a spell you are casting.

Spell casting checks to interfere with your magic are made with disadvantage (such as counterspell), and must make a check even if they would normally automatically succeed.

Gift of Prophecy

When you close your eyes, you can see flickers of the future. You regain one expended use of Uncanny Insight when you finish a short rest.

Gift of Mystery

You learn the associated psionic power of one Psionic Discipline of your choice. The psionic power of the Discipline becomes a 1st-level spell for you. When you cast it using a spell slot, you cast it as if expend psi points equal to the spell slot level spent (for example, when casting as a 1st-level spell, you cast it as if empowering it with 1 psi point).

\subsection*{Potent Spellcasting}

Starting at 8th level, you add your Wisdom modifier to the damage you deal with any cleric cantrip.

\subsection*{Dark Truth}

Starting at 17th level, your mind is inured to the horrors in which you deal, the veil cast aside from your eyes.

\begin{itemize}
  \item You are immune to psychic damage.
  \item You gain a truesight of 60 feet.
\end{itemize}

\subsection*{Mystery Cult Quirks}

The following are some optional quirks for a player of the Mystery Cult. You can optionally roll or select a quirk that suits your character.

\begin{minipage}{0.48\textwidth}
\begin{tabularx}{\textwidth}\toprule
{}XXXXX}
\midrule
d6 & \multicolumn{5}{c}{Quirk} \\
\midrule
1 & \multicolumn{5}{c}{You keep secrets about secrets.} \\
\midrule
2 & \multicolumn{5}{c}{You perform strange rites at odd hours.} \\
\midrule
3 & \multicolumn{5}{c}{You have a bright, cheerful attitude that never wavers no matter the situation.} \\
\midrule
4 & \multicolumn{5}{c}{You prefer to keep your face covered with a mask at all times.} \\
\midrule
5 & \multicolumn{5}{c}{You give obtuse answers when people ask about your god.} \\
\midrule
6 & \multicolumn{5}{c}{You frequently refer to the impending doom of creation.} \\
\midrule
\end{tabularx}
\end{minipage}\hfill
\begin{minipage}{0.48\textwidth}

\end{minipage}

\begin{itemize}
  \item Cleric: Mystery Cult
\end{itemize}

\section*{Circle of Nightmares}

% [Image Inserted Manually]

Druids are often thought of as the guardians of the natural order and having a deep connection to the creatures of the land. Sometimes that connection goes horribly wrong. A druid of this circle has found their mind entangled with the horrors of the beyond, their natural connection to nature and its beasts becoming warped by the dark dreams.

Druids of this circle may be haunted by the terrors they see, but many simply find their minds inoculated against the horror (perhaps by the touch of madness). Some even treat twisted abominations with the same care other druids might take to vicious and wild animals, treating them with pragmatic understanding and empathy.

Sometimes their connection even seems to be mutual.

\subsection*{Circle Spells}

\begin{minipage}{0.48\textwidth}
At 2nd level, you learn the message cantrip. At 3rd, 5th, 7th, and 9th level you gain access to the spells listed for that level in the Circle of the Nightmares Spells table. Once you gain access to a circle spell, you always have it prepared, and it doesn’t count against the number of spells you can prepare each day. If you gain access to a spell that doesn’t appear on the druid spell list, the spell is nonetheless a druid spell for you.
\end{minipage}\hfill
\begin{minipage}{0.48\textwidth}
\begin{tabularx}{\textwidth}\toprule
{}XXXXX}
\midrule
Druid Level & \multicolumn{5}{c}{Circle Spells} \\
\midrule
3rd & \multicolumn{5}{c}{alter self} \\
\midrule
5th & \multicolumn{5}{c}{mutate*} \\
\midrule
7th & \multicolumn{5}{c}{black tentacles} \\
\midrule
9th & \multicolumn{5}{c}{contact other plane} \\
\midrule
\end{tabularx}
\end{minipage}

\subsection*{Eldritch Mutation}

When you choose this circle at 2nd level, you can expend a use of your Wild Shape feature as a bonus action to take on an eldritch mutation born of your twisted nightmares, rendering a piece of them into reality through the warping of your flesh.

When you use this feature, you gain temporary hit points equal to your druid level + your Wisdom modifier and grow your choice of an Eye Stalk or Grasping Tentacles. The selected eldritch mutation lasts for 10 minutes. It ends early if you dismiss it as an action, are incapacitated, or use this feature again.

Eye Stalk

\begin{minipage}{0.48\textwidth}
When you manifest this mutation, and as a bonus action on your subsequent turns while it lasts, you can target a creature you can see within 60 feet and make a roll on the following table to shoot a random eye ray at them.

\begin{tabularx}{\textwidth}\toprule
{}XXXXXXXX}
\midrule
1d6 & \multicolumn{2}{c}{Eye Ray} & \multicolumn{6}{c}{Effect} \\
\midrule
1 & \multicolumn{2}{c}{Fear Ray} & \multicolumn{6}{c}{The target creature must make a Wisdom saving throw or become frightened of you until the start of your next turn.} \\
\midrule
2 & \multicolumn{2}{c}{Telekinetic Ray} & \multicolumn{6}{c}{The target creature must make a Strength saving throw or be moved 10 feet in a direction of your choice.} \\
\midrule
3 & \multicolumn{2}{c}{Slowing Ray} & \multicolumn{6}{c}{The target must succeed a Wisdom saving throw or be affected by the slow spell until the start of your next turn.} \\
\midrule
4 & \multicolumn{2}{c}{Petrification Ray} & \multicolumn{6}{c}{The target must succeed a Constitution saving throw or be restrained until the start of your next turn.} \\
\midrule
5 & \multicolumn{2}{c}{Enervation Ray} & \multicolumn{6}{c}{The target must succeed a Constitution saving throw or take 1d8 + your Wisdom modifier necrotic damage.} \\
\midrule
6 & \multicolumn{2}{c}{Disintegration Ray} & \multicolumn{6}{c}{The target must succeed a Dexterity saving throw or take 1d10 + your Wisdom modifier force damage.} \\
\midrule
\end{tabularx}
\end{minipage}\hfill
\begin{minipage}{0.48\textwidth}
Controlled Madness. Each time you use a Wild Shape to assume this mutation, you can select the effect of the eye ray instead of rolling a number of times equal to your proficiency bonus.
\end{minipage}

Grasping Tentacles

\begin{minipage}{0.48\textwidth}
When you manifest this mutation, and as a bonus action on your subsequent turns while it lasts, you can lash out with these tentacles. You can make a melee spell attack against a creature within 10 feet. On hit, the target takes 1d6 + your Wisdom modifier bludgeoning damage. In place of making this attack, you can initiate a grapple against a target, making a Wisdom (Athletics) check with a range of 10 feet. If the check succeeds, you can continue to use Wisdom (Athletics) to maintain and contest grapples targeting that creature. You have a number of tentacles equal to your Wisdom modifier, and need at least one free tentacle to take the bonus action attack.
\end{minipage}\hfill
\begin{minipage}{0.48\textwidth}
Note: Vulnerable Limbs!

As per the Sages, a creature you are grappling can attack you regardless if you are in its normal attack range by attacking the limb grappling you.
\end{minipage}

\subsection*{Aberrant Mutations}

Starting at 6th level, you can cast mutateK without expending a spell slot. You can do so a number of times equal to your proficiency bonus, and you regain all expended uses when you finish a long rest.

\subsection*{Faceted Forms}

Starting at 10th level, you can gain both an Eye Stalk and a Grasping Tentacle when expending a use of Wild Shape on Eldritch Mutation. While you have both mutations manifested, you can use the effect of either of them as an action or bonus action.

Additionally, you deal an additional die of damage with your Enervation Ray (2d8 + your Wisdom modifier), Disintegration Ray (2d10 + your Wisdom modifier), and Grasping Tentacle (2d6 + your Wisdom modifier).

\subsection*{Piercing Eyes}

Starting at 14th level, while you have an Eye Stalk from Eldritch Mutation, you have truesight within range of 60 feet.

\subsection*{Mutation Mastery}

Additionally at 14th level, when you cast mutateK, you can select 1 additional property (as if casting the spell 1 level higher).

\subsection*{Circle of Nightmares Quirks}

The following are some optional quirks for a player of the Circle of Nightmares. You can optionally roll or select a quirk that suits your character.

\begin{minipage}{0.48\textwidth}
\begin{tabularx}{\textwidth}\toprule
{}XXXXX}
\midrule
d6 & \multicolumn{5}{c}{Quirk} \\
\midrule
1 & \multicolumn{5}{c}{You sometimes wake up with different features.} \\
\midrule
2 & \multicolumn{5}{c}{You forgot what you originally looked like long ago.} \\
\midrule
3 & \multicolumn{5}{c}{You refer to horrifying things as "cute".} \\
\midrule
4 & \multicolumn{5}{c}{You try to speak to animals in deep speech.} \\
\midrule
5 & \multicolumn{5}{c}{You refer to going to sleep as "visiting the other side".} \\
\midrule
6 & \multicolumn{5}{c}{You refer to eyes watching you. No one else can see these eyes.} \\
\midrule
\end{tabularx}
\end{minipage}\hfill
\begin{minipage}{0.48\textwidth}

\end{minipage}

\begin{itemize}
  \item Druid: Circle of Nightmares
\end{itemize}

\section*{Tech Knight}

% [Image Inserted Manually]

A Tech Knight believes that it is neither brains nor brawn that determine the best fighter, but the combination thereof.

Why limit yourself to what nature has provided you when you can supplement your combat superiority in unique and inventive ways that give you the edge? You build and innovate on the cutting edge... and use that edge to hew through your foes. There are few things more dangerous than weaponized creativity in the hands of someone that knows how to use it.

A Tech Knight could be called an inventor that opened the door of innovation, found the deadliest thing they could invent, and closed it once more, but that wouldn’t be accurate. They innovate continuously, improving their art, just with a highly specialized focus.

Some perhaps adventure and fight to test their weapons, others perhaps turned to innovation to overcome some obstacle or seek to change the world, and others still care little for the science and engineering of the weapon, and merely cobbled together something to annihilate their foes.

\subsection*{Brutal Invention}

When you choose this archetype at 3rd level, you build a devastating new weapon to help you dominate the battlefield in a unique way. You can build one weapon.

Select one weapon from the below list:

\begin{minipage}{0.48\textwidth}
\begin{tabularx}{\textwidth}\toprule
{}XXXX}
\midrule
\multicolumn{2}{c}{Weapon} & Damage & \multicolumn{2}{c}{Properties} \\
\midrule
\multicolumn{2}{c}{Chainblade} & 2d4 slashing & \multicolumn{2}{c}{Versatile (3d4), Special} \\
\midrule
\multicolumn{2}{c}{Ramming Gauntlet} & 1d8 & \multicolumn{2}{c}{bludgeoning Light, Special.} \\
\midrule
\multicolumn{2}{c}{Repeating Hand Crossbow} & 1d6 piercing & \multicolumn{2}{c}{Ammunition (30/120), Light, Special} \\
\midrule
\multicolumn{2}{c}{Ricocheting Weapon} & 1d8 bludgeoning & \multicolumn{2}{c}{(30/90), Special Finesse, Thrown} \\
\midrule
\multicolumn{2}{c}{Transforming Weapon} & Varies & \multicolumn{2}{c}{Varies, Special} \\
\midrule
\end{tabularx}
\end{minipage}\hfill
\begin{minipage}{0.48\textwidth}
Variant Weapon Types

If a Tech Knight wants to change the damage type of a weapon to another, this is generally balanced. For example, Ramming Gauntlets that are Bladed Gloves that deal slashing damage, or a Ricocheting Weapon that does slashing damage.

The damage type of weapons in the games are not a primary factor in balance.
\end{minipage}

You gain proficiency with the selected weapon. Only you have proficiency with this weapon. If your weapon is lost or destroyed, you can remake it over the course of 4 hours spending 25 gp of materials.

\begin{minipage}{0.48\textwidth}
Special Properties

Chainblade. When you roll damage dice for this weapon, you can reroll as many damage dice as you would like once per attack, but you must use the new roll for any dice rolled this way. After hitting an attack with this weapon as part of the Attack action, if you have additional attacks you can make as part of the action, you can forgo them to deal an extra 4d4 slashing damage per attack forgone.

Ramming Gauntlet. When you make an attack roll, you can choose to forgo adding your proficiency bonus to the attack roll. If the attack hits, you can add double your proficiency bonus to the damage roll.

Repeating Hand Crossbow. This weapon doesn’t require a free hand to load, as it has a built-in loader. Once per turn, if you make an attack with this weapon as part of the Attack action, if you do not have disadvantage on that attack, you can give yourself disadvantage to make a single additional weapon attack with this weapon as a bonus action (also with disadvantage).

Ricocheting Weapon. When this weapon is thrown you can target two creatures within 10 feet of each other, using the same attack roll and damage roll for both targets; the damage dealt is halved for targets hit after the first. This weapon returns to your hand after you make an attack with it using the Thrown property. When you create this weapon, you can choose for it to deal slashing damage instead of bludgeoning damage.

Transforming Weapon. When you gain this weapon, select 3 simple or martial weapons. This weapon can transform between any of those weapons as a bonus action. When you convert the weapon, it becomes overcharged with power, and the next time you roll damage with it before the end of your turn you deal an extra 1d4 lightning damage.
\end{minipage}\hfill
\begin{minipage}{0.48\textwidth}
Rev Up

When hit an attack with your Brutal Invention, you can overcharge with an effect based on your Brutal Invention selection, granting it an additional benefit:

\begin{itemize}
  \item Reckless Power (Chainblade): The weapon’s damage dice become d6’s for that attack (including the special property if activated).
  \item Excessive Force (Ramming Gauntlet): The target is knocked 10 feet backwards.
  \item Full Auto (Repeating Hand Crossbow): You immediately make a single additional attack as part of the same action, making the attack with disadvantage; this disadvantage can’t be canceled out.
  \item Overcharge (Transforming Weapon): You deal an extra 2d4 lightning damage.
  \item Hypervelocity (Ricocheting Weapon): The weapon can target creatures within20 feet of each other, and it can bounce to an additional target.
\end{itemize}

You can use this ability a number of times equal to your Intelligence modifier (minimum 1), regaining all uses on a short or long rest.

Weapon Improvement

Additionally at 3rd level, over the course of a long rest, you can destroy a +1/+2/+3 magic weapon to transfer its bonus to attack and damage rolls to your Brutal Invention weapon. At the GM’s discretion, other properties can be transferred (it is recommended that most cannot, but a final decision is up to the GM).
\end{minipage}

Changing Your Invention

If a player wishes to change their Brutal Invention, it takes 1 full day of down time, working for at least 8 hours. You can change any option that requires a selection in this manner.

\subsection*{Tinker’s Proficiency}

Additionally at 3rd level, you gain proficiency with tinker’s tools. If you already have proficiency in tinker’s tools, you can select another type of artisan’s tools to gain proficiency in.

\subsection*{Contingent Options}

At 7th level, you extend your innovation of the art of war to give yourself a unique advantage from your gear. For any option that has a spell save DC, your save is 8 + your Intelligence modifier + your proficiency bonus. Select one of the following options:

\begin{itemize}
  \item Charged Armor. As a bonus action, you can juice your armor with power with the effect of the lightning charged* spell.
  \item Rocket Boots. As a bonus action you can activate these to give yourself a jumping distance equal to your walking speed until the end of the turn. You take no fall damage from this movement, but creatures of your choice within 5 feet of where you land take 2d4 fire damage. Alternatively, you can use this to cast feather fall targeting only yourself.
  \item Shifting Belt. You can use this to cast enlarge/reduce without expending a spell slot.
  \item Flame Thrower. As an action or in place of one attack as part of the Attack action, you can use this to cast burning hands as a 2nd-level spell without expending a spell slot.
  \item Force Shield. You can deploy a temporary force field, with the effect of casting shield .
  \item Thunder Grenade. As an action or in place of one attack as part of the Attack action, you can use this to cast shatter without expending a spell slot.
\end{itemize}

You can use the selected item once, after which you must finish a short or long rest before you can use it again.

\subsection*{Chains of War}

Starting at 10th level, you can select one of the following, or one additional selection from Brutal Invention or Contingent Options.

\begin{itemize}
  \item Returning Chain. When you throw your Brutal Invention weapon at a target within 30 feet of you, you can return the weapon to your hand. If the weapon doesn’t have the thrown property, it gains the Thrown (10/30 property).
  \item Grappling Hook. As an action, or in place of an attack as part of the Attack action, you may target a surface, object or creature within 20 feet. If the target is Small or smaller, you can make a Strength (Athletics) grappling check to pull it to you and grapple it. Alternatively, if the target is Medium or larger, you can choose to be pulled to it, however, this doesn’t grapple it. Opportunity attacks provoked by this movement are made with disadvantage.
\end{itemize}

\subsection*{Reactive Armor}

Starting 15th level, you can tune your armor during a long rest to grant specialized defense. At the end of a long rest, select one damage type and gain resistance to that damage type until the end of your next long rest.

\subsection*{Arms Race}

Starting at 18th level, you can select two additional options from any of Brutal Invention, Contingent Options, and Chains of War (two options total, selected from any of those features).

\subsection*{Tech Knight Quirks}

The following are some optional quirks for a player of the Tech Knight. You can optionally roll or select a quirk that suits your character.

\begin{minipage}{0.48\textwidth}
\begin{tabularx}{\textwidth}\toprule
{}XXXXX}
\midrule
d6 & \multicolumn{5}{c}{Quirk} \\
\midrule
1 & \multicolumn{5}{c}{You devote time each day to improving your weapons.} \\
\midrule
2 & \multicolumn{5}{c}{You think any problem can be solved with sufficient firepower.} \\
\midrule
3 & \multicolumn{5}{c}{You have disdain for simple weapons.} \\
\midrule
4 & \multicolumn{5}{c}{You daydream about explosions.} \\
\midrule
5 & \multicolumn{5}{c}{You revel in new chances to test your destructive wares.} \\
\midrule
6 & \multicolumn{5}{c}{You offer to show everyone the features of each new invention with great joy.} \\
\midrule
\end{tabularx}
\end{minipage}\hfill
\begin{minipage}{0.48\textwidth}

\end{minipage}

\begin{itemize}
  \item Fighter: Tech Knight
\end{itemize}

\section*{Way of the Soul Blade}

% [Image Inserted Manually]

Monks of the Way of the Soul Blade are monks who have learned to harness and focus their ki, using their inner will and focus to control psionic powers—primarily into a blade of pure psionic power: a Soul Blade.

\subsection*{Soul Blade}

Starting when you choose this tradition at 3rd level, you’ve learned to focus your ki into a psionic blade. As a bonus action, you can create a blade of pure scintillating psionic energy. The blade you create this way most typically takes the form of a knife-like blade projecting from your fist, but you can shape it however you choose. You can choose to create multiple blades, but any blade you are not touching vanishes at the end of your turn and must be resummoned.

Regardless of the form it takes, the weapon is a monk weapon for you, deals 1d8 psychic damage, and has the light, finesse, and thrown(20/60) properties.

\subsection*{Psionic Ki}

Additionally at 3rd level, you gain the Telekinetics Discipline; this can be found under the Psionic Disciplines list of the psion class. You can use ki points as psi points, with a limit of 1 ki point. This limit increases to 2 ki points at level 5, 3 ki points at level 9, 4 ki points at level 13, and 5 ki points at level 17. The DC for your psionic abilities is equal to your ki save DC.

If your character has both psi points and ki points, those are added together into one pool and can be used interchangeably. Your psi limit (and the limit you can use ki points as psi points) becomes your psi limit + one third of your monk levels rounded down.

When you use your action on a psionic power or to cast a spell using this feature, you can make one attack with an unarmed strike or monk weapon as a bonus action before the end of your turn.

\subsection*{Art of the Soul Blade}

Starting at 6th level, your expertise with the blade allows you to control it in unique and powerful ways

\subsection*{Extended Blade}

You can expend 1 ki point to give your Soul Blade the Reach property until the end of your turn.

\subsection*{Psionic Flurry}

When you make a Flurry of Blows, you can make the additional attacks with your Soul Blade.

\subsection*{Soul Strike}

When you take the Attack action, you use your soul blade to make a single piercing strike. Make a single attack using your action (forgoing any additional attacks gained from Extra Attack) to ignore all armor a creature has and treat its AC as 10 + its Dexterity. On a hit, the creature takes additional damage equal to your Wisdom modifier.

\subsection*{Power of the Mind}

Starting at 11th level, your psionic abilities manifest more completely, giving you greater control and power in your psionic abilities.

\subsection*{Ethereal Sweep}

When you use Extended Blade, you can sweep or stab through multiple creatures with a single blow. Once per turn, when you make an attack with your Soul Blade, if you have activated Extended Blade you can make a single additional weapon attack with your Soul Blade against a number of creatures equal to your Wisdom modifier within range.

\subsection*{Consumptive Blade}

Whenever you kill a creature with your Soul Blade that has an Intelligence of 6 or higher, you can use your reaction to draw in part of their psionic essence. You regain 1d4 hit points and 1 expended ki point.

\subsection*{Empowered Discipline}

When you use a Psionic Discipline, you can expend 1 ki point for free without exhausting the ki point on empowering the psionic power granted by the Discipline, even if you do not have any remaining ki (this can’t be used on the spells granted by the Discipline).

\subsection*{Transcendent Blade}

Starting at 17th level, your Soul Blade becomes a peerless weapon. You can add a +1 to its attack and damage rolls. You can choose for your blade to affect inanimate material, causing it to gain the Siege property and deal force damage to it when you choose. Reactions that parry or block to add Armor Class against an attack are ineffective against attacks made with the Soul Blade.

Additionally, critical hits from your Soul Blade rend the soul of the target. If a creature would have less than 50 hit points after taking damage from your critical strike, the creature must make a Charisma saving throw. On failure, its Charisma score becomes zero and it dies.

\subsection*{Soul Blade Quirks}

The following are some optional quirks for a player of this Way.

\begin{minipage}{0.48\textwidth}
\begin{tabularx}{\textwidth}\toprule
{}XXXXX}
\midrule
d6 & \multicolumn{5}{c}{Quirk} \\
\midrule
1 & \multicolumn{5}{c}{You carry around a bladeless sword hilt for your Soul Blade’s blade.} \\
\midrule
2 & \multicolumn{5}{c}{You occasionally attempt to cut fruit with your Soul Blade.} \\
\midrule
3 & \multicolumn{5}{c}{You refer to your actions as the will of the living Ki.} \\
\midrule
4 & \multicolumn{5}{c}{You view killing things with your psionic powers as evil, but killing things with your Soul Blade as perfectly okay.} \\
\midrule
5 & \multicolumn{5}{c}{You have endless platitudes about temperance and control.} \\
\midrule
6 & \multicolumn{5}{c}{You practice obscure martial arts stances every morning.} \\
\midrule
\end{tabularx}
\end{minipage}\hfill
\begin{minipage}{0.48\textwidth}

\end{minipage}

\begin{itemize}
  \item Monk: Way of the Soul Blade
\end{itemize}

\section*{Oath of Sanity}

The following oath is a new subclass option for the Paladin, for those that wish to battle the psionic horrors from beyond.

\subsection*{Oath of Sanity}

The Oath of Sanity is the oath of someone that has survived exposure to the horrors that lay beyond the veil, or perhaps someone that has prepared for them through a mystical order, a forgotten tradition... or particularly vivid and over active imagination.

While most who have glimpsed into the realms beyond suffer instant and irrevocable madness, these Paladins have glimpsed beyond and sworn to remain sane, no matter what the cost. They reject shielding their fragile mind from what they’ve seen in a comforting shroud of madness.

An Oath of Sanity Paladin will usually seem to others as true neutral or lawful neutral following obscure codes incomprehensible to those who haven’t experienced a brush with the beyond, their morality coming in blue and orange rather than black and white. Almost universally they are dedicated to preventing the threats from beyond from consuming the material world and shattering the fragile minds within.

\subsection*{Tenets of Sanity}

\begin{itemize}
  \item Order. Never act on random or chaotic impulses. These are the cracks of madness.
  \item Vigilance. Never let your attention wander, lest you miss the signs of madness.
  \item Discipline. Never indulge in a lapse of behavior, your habits keep you safe from madness.
  \item Solemnity. Your work is terrible. Never take pleasure in it. That leads to the comfort of madness
  \item Sanity. No matter what, never give into madness.
\end{itemize}

\subsection*{Oath Spells}

You gain oath Spells at the paladin levels listed.:

\begin{minipage}{0.48\textwidth}
\begin{tabularx}{\textwidth}\toprule
{}XXXXX}
\midrule
Level & \multicolumn{5}{c}{Spells} \\
\midrule
3rd & \multicolumn{5}{c}{detect magic, protection from evil and good} \\
\midrule
5th & \multicolumn{5}{c}{nullify effect*, see invisibility} \\
\midrule
9th & \multicolumn{5}{c}{dispel magic, remove curse} \\
\midrule
13th & \multicolumn{5}{c}{banishment, dimension door} \\
\midrule
17th & \multicolumn{5}{c}{dispel evil and good, hold monster} \\
\midrule
\end{tabularx}
\end{minipage}\hfill
\begin{minipage}{0.48\textwidth}

\end{minipage}

\subsection*{Channel Divinity}

When you take this oath at 3rd level, you gain the following two Channel Divinity options.

Turn the Aberrant. As an action, you can make any aberration or undead, provided that it’s within 30 feet and that can see or hear you, make a Wisdom saving throw. If the creature fails its saving throw, it is turned for 1 minute or until it takes damage. A turned creature must spend its turns trying to move as far away from you as it can, and it can’t willingly move to a space within 30 feet of you. It also can’t take reactions. For its action, it can use only the Dash action or try to escape from an effect that prevents it from moving. If there’s nowhere to move, the creature can use the Dodge action.

If the creature’s true form is concealed by an illusion, shapeshifting, or other effect, that form is revealed while it is turned.

Deny the Supernatural. Immediately after you deal damage to a creature with your Divine Smite feature, you can use your Channel Divinity as a bonus action to attempt to strip the target of its powers. The target must make a Charisma saving throw or be unable to cast spells or use psionic abilities until the end of its next turn.

\subsection*{Active Mind}

Beginning at 7th level, you can save against effects that allow you to save at the end of your turn at the start of your turn instead.

Starting at 18th level, you can use your reaction to extend this benefit to another creature you can see within 30 feet that starts their turn under an effect they can save against at the end of their turn.

\subsection*{Constant Vigilance}

Starting at 15th level, you can no longer be surprised, and your passive perception remains the same even while unconscious so long as you have at least 1 hit point.

Additionally, when you roll for initiative, if the total result of your roll is less than your passive perception, you can replace the result with your passive perception value.

\subsection*{Clarity of Purpose}

At 20th level, as an action, you can fully perceive the world around you for what it is, shattering the foul magics that bind and piercing the veils that hide, unrelenting and focused, gaining the following benefits for 1 minute:

\begin{itemize}
  \item You can gain truesight with a range of 120 feet.
  \item You are immune to psychic damage.
  \item You have advantage on saving throws against magical or psionic effects.
  \item When you strike an enemy with a Divine Smite you can choose to cast dispel magic on the target as a bonus action without expending a spell slot. If the level of the spell slot spent on Divine Smite was higher than 3rd level, you cast dispel magic at the level of spell slot spent.
\end{itemize}

Once you use this feature, you can’t use it again until you finish a long rest.

\subsection*{Oath of Sanity Quirks}

The following are some optional quirks for a player of this Oath to choose from. These can be either preexisting, signaling their inevitable path toward this Oath, or appear over time due to the influence of their experiences. It is recommended that you have two quirks from this table... This Oath rarely results in individuals that would be considered normal.

\begin{minipage}{0.48\textwidth}
\begin{tabularx}{\textwidth}\toprule
{}XXXXX}
\midrule
d12 & \multicolumn{5}{c}{Quirk} \\
\midrule
1 & \multicolumn{5}{c}{You refer to others as sheltered little lambs. They have no idea.} \\
\midrule
2 & \multicolumn{5}{c}{You talk to yourself to help you stay sane in stressful times. The second personality can be reassuring to have around.} \\
\midrule
3 & \multicolumn{5}{c}{You compare how bad anything could be to eldritch monsters ending reality.} \\
\midrule
4 & \multicolumn{5}{c}{You occasionally bark out "constant vigilance!".} \\
\midrule
5 & \multicolumn{5}{c}{You keep a list of people you suspect will betray you.} \\
\midrule
6 & \multicolumn{5}{c}{You frequently change your plan to throw off mind readers.} \\
\midrule
7 & \multicolumn{5}{c}{You never turn your back on a portal.} \\
\midrule
8 & \multicolumn{5}{c}{You are deeply suspicious of anyone casting spells on you, even if they are buffs or heals.} \\
\midrule
9 & \multicolumn{5}{c}{You don’t believe in coincidence. It’s always cosmic forces conspiring. You might be right.} \\
\midrule
10 & \multicolumn{5}{c}{You have one hundred and one superstitions.} \\
\midrule
11 & \multicolumn{5}{c}{You frequently try to explain the danger of perfectly normal activities in great depth.} \\
\midrule
12 & \multicolumn{5}{c}{You keep a tin lining on your helmet.} \\
\midrule
\end{tabularx}
\end{minipage}\hfill
\begin{minipage}{0.48\textwidth}

\end{minipage}

\begin{itemize}
  \item Paladin: Oath of Sanity
\end{itemize}

\section*{Specialist}

% [Image Inserted Manually]

Specialists are rangers that innovate their technique and equipment in equal measure. Deadly and clever, they keep their opponents guessing what is coming next.

\subsection*{Knack}

At 3rd level when you select this archetype, you gain proficiency with Tinker’s Tools. If you already have proficiency with Tinker’s Tools, you can gain proficiency with another artisan tool of your choice.

\subsection*{Trick Shots}

At 3rd level, you gain the ability to create special pieces of ammunition. You can have a number of these special shots equal to your proficiency bonus, recreating expended ones during a short or long rest.

The DC for your Trick Shots is equal to your spell save DC.

You can select from the following options for your special shots. You can make multiple copies of one shot, or select multiple different shots, as long as the total equals your proficiency bonus. You can swap out any unspent ones for other options during a short or long rest.

\begin{minipage}{0.48\textwidth}
Binding Shot

Instead of doing damage, this shot turns into a net when striking the target. The net gains additional hit points equal to your Ranger level.

Flash Shot

When this shot hits a target, it emits a brilliant flash of light. The target and creatures within 15 feet of the target must make a Dexterity saving throw or become blinded until the start of your next turn.

Explosive Shot

When this shot hits a target, it emits an explosion. The target and creatures within 10 feet of the target must make a Dexterity saving throw or take 1d6 fire damage and 1d6 thunder damage.
\end{minipage}\hfill
\begin{minipage}{0.48\textwidth}
Guided Shot

When fired, this shot tracks its target. This attack is made with advantage, and ignores cover, including complete cover, as long as there is a path to the target you are aware of that requires the shot to travel less than the maximum range of the weapon used.

Rocket Shot

The attack’s range is doubled, and on a hit the target takes an extra 1d12 damage from the attack.
\end{minipage}

\subsection*{Spell Shots}

\begin{minipage}{0.48\textwidth}
Starting at 7th level, you gain the ability to infuse special shots with magical abilities. You have a pool of these shots equal to your Wisdom modifier, recreating any expended shots during a long rest. As a bonus action, you can expend a spell slot infusing one of the spells from the following table into one of these shots.

The spell doesn’t trigger immediately, but triggers on impact. You can shoot the shot at a point within range or a creature. Regardless of whether the attack hits or misses the creature, the effect imbued in the shot is triggered at a point of your choice adjacent to the creature.

If the spell affects a line or cone, you can select the projection of the area from the point. If a spell requires concentration, it uses your concentration as normal.
\end{minipage}\hfill
\begin{minipage}{0.48\textwidth}
\begin{tabularx}{\textwidth}\toprule
{}XX}
\midrule
Spell Slot Level & \multicolumn{2}{c}{Infused Spell Options} \\
\midrule
1st & \multicolumn{2}{c}{burning hands, fog cloud} \\
\midrule
2nd & \multicolumn{2}{c}{darkness, shatter} \\
\midrule
3rd & \multicolumn{2}{c}{fireball, sleet storm} \\
\midrule
4th & \multicolumn{2}{c}{black tentacles, ice storm} \\
\midrule
5th & \multicolumn{2}{c}{cloudkill, acid rain*} \\
\midrule
\end{tabularx}
\end{minipage}

\subsection*{Empowered Shots}

Starting at 11th level, whenever you make either a Trick Shot or Spell Shot to deal damage to a creature, that creature takes an extra 1d8 damage.

\subsection*{Grapple Shot}

Additionally at 11th level, you gain the ability to fire special shots with a range of 30 feet that carry a cord that immediately pulls to that location. As an action or attack as part of the Attack action, you can make this shot against a creature, surface, or object. This shot does no damage, but on a hit, if the creature or object is Small or smaller, it is immediately pulled up to 30 feet toward you. If the creature is Medium or larger you are pulled 30 feet toward it.

\subsection*{Endless Innovation}

Starting at 15th level, if you start your turn without a Trick Shot, you can create a Trick Shot as a bonus action.

\subsection*{Specialist Quirks}

The following are some optional quirks for a player of the Specialist. You can optionally roll or select a quirk that suits your character.

\begin{minipage}{0.48\textwidth}
\begin{tabularx}{\textwidth}\toprule
{}XXXXX}
\midrule
d6 & \multicolumn{5}{c}{Quirk} \\
\midrule
1 & \multicolumn{5}{c}{You often unnecessarily abbreviate things to abbreviations you then have to explain.} \\
\midrule
2 & \multicolumn{5}{c}{Most of your plans involve complicated traps.} \\
\midrule
3 & \multicolumn{5}{c}{You never like to use the same plan twice.} \\
\midrule
4 & \multicolumn{5}{c}{All problems can be solved with the right arrow.} \\
\midrule
5 & \multicolumn{5}{c}{You provide highly specific details about simple things.} \\
\midrule
6 & \multicolumn{5}{c}{You always prefer to have the high ground.} \\
\midrule
\end{tabularx}
\end{minipage}\hfill
\begin{minipage}{0.48\textwidth}

\end{minipage}

\begin{itemize}
  \item Ranger: Specialist
\end{itemize}

\section*{Mind Reaper}

Mind Reapers are an incarnation of fear. Terrible creatures seen more often in stories told to frighten children or ancient legends than the material plane, these are predatory creatures most mortals would be right to fear, no matter what their intentions.

Some use these powers to spread fear, savoring the terror of their victims. Others use their powers to dispense what they view as justifice, stalking down and reaping only those they believe have earned such a grim fate.

\subsection*{Mind Reaper Spells}

\begin{minipage}{0.48\textwidth}
Starting at 3rd level, you learn an additional spell when you reach certain levels in this class, as shown in the Mind Reaper Spells table. The spell counts as a ranger spell for you, but it doesn’t count against the number of ranger spells you know.

\subsubsection*{Psychic Reaper}

At 3rd level, when you attack a creature that is frightened, you gain advantage on the attack roll.

Additionally, you learn to imbue your attacks with lingering psionic energy. When you make an attack as part of the attack action, your attack deals an extra 1d4 psychic damage.
\end{minipage}\hfill
\begin{minipage}{0.48\textwidth}
\begin{tabularx}{\textwidth}\toprule
{}XXX}
\midrule
Ranger Level & \multicolumn{3}{c}{Spell} \\
\midrule
3rd & \multicolumn{3}{c}{frighten*} \\
\midrule
5th & \multicolumn{3}{c}{detect thoughts} \\
\midrule
9th & \multicolumn{3}{c}{fear} \\
\midrule
13th & \multicolumn{3}{c}{confusion} \\
\midrule
17th & \multicolumn{3}{c}{dominate person} \\
\midrule
\end{tabularx}
\end{minipage}

\subsection*{Telepathic Communication}

Additionally at 3rd level, you can communicate telepathically with any creature you can see within 30 feet of you. You don’t need to share a language with the creature for it to understand your telepathic utterances, but the creature must be able to understand at least one language.

\subsection*{Reaper’s Intrusion}

Starting at 7th level, you gain the Telepathic Intrusion psionic power. You gain a number of psi points equal to your proficiency bonus, regaining them on a short or long rest. You can use these points to empower your Telepathic Intrusion or cast spells granted by the subclass, spending 1 psi point per level of the spell in place of a spell slot.

\subsection*{Telepathic Intrusion}

Psionic power

Casting Time: 1 action
 Range: 60 feet
 Components: S
 Duration: Instantaneous

You assault the mind of a creature you can see directly. The target must succeed on a Wisdom saving throw against your spell save DC, or take 1d8 psychic damage. If the target fails the saving throw, it has disadvantage on attacks made against you until the start of your next turn. You can choose to deal no damage to the creature when it fails its saving throw.

The points must be spent when choosing the target of the power.

Rending (1+ psi points): The target takes 1d8 psychic damage for each additional point spent on a failed save.

Terrifying (1 psi point): The target is frightened of you until the end of your next turn if it fails its saving throw.

Meddling (2 psi points): You make one creature invisible to the target creature or cause the creature to see something that is not there with the effect of minor illusion until the start of your next turn if it fails its saving throw.

Overwhelming (3 psi points): The target is stunned until the end of its next turn if it fails its saving throw.

\begin{minipage}{0.48\textwidth}
Reaper vs Intrusion

Psychic Reaper offers no saving throw, and consequently deals guaranteed damage in the cases you can use it. Telepathic Intrusion deals the same amount of damage and causes additional effects, but gives the target a saving throw. One is not always better than the other, though when spending psi points the Telepathic Intrusion can offer far more damage and utility at the corresponding higher risk that it might fail.
\end{minipage}\hfill
\begin{minipage}{0.48\textwidth}

\end{minipage}

\subsection*{Amplified Anguish}

Starting at 11th level, whenever you deal psychic damage to a creature that is frightened of you, you deal an extra 1d8 psychic damage.

\subsection*{Mind Hunter}

Additionally at 11th level, when you’ve dealt psychic damage to a creature, you gain the ability to see that target as if by blindsight as long as it is within 120 feet of you until the end of your next turn.

\subsection*{Taste of Fear}

Starting at 15th level, when you deal psychic damage to a frightened creature, you gain temporary hit points equal to your Wisdom modifier + your proficiency bonus. If a creature dies while frightened of you, you regain 1 expend psi point.

\subsection*{Psionic Adaptation}

Additionally at 15th level, you can expend your spell slots in place of psi points when using a psionic power, with a spell slot counting as an equal number of psi points.

\subsection*{Mind Reaper Quirks}

The following are some optional quirks for a player of the Mind Reaper. You can optionally roll or select a quirk that suits your character.

\begin{minipage}{0.48\textwidth}
\begin{tabularx}{\textwidth}\toprule
{}XXXXX}
\midrule
d6 & \multicolumn{5}{c}{Quirk} \\
\midrule
1 & \multicolumn{5}{c}{You discuss the taste of thoughts.} \\
\midrule
2 & \multicolumn{5}{c}{You enjoy spooking people.} \\
\midrule
3 & \multicolumn{5}{c}{You avoid speaking out loud.} \\
\midrule
4 & \multicolumn{5}{c}{You have a code of conduct for who gets hunted.} \\
\midrule
5 & \multicolumn{5}{c}{Your eyes glow for a few minutes after using psionic powers.} \\
\midrule
6 & \multicolumn{5}{c}{You refuse to enter particularly deranged minds. A healthy diet must draw some lines.} \\
\midrule
\end{tabularx}
\end{minipage}\hfill
\begin{minipage}{0.48\textwidth}

\end{minipage}

\begin{itemize}
  \item Ranger: Mind Reaper
\end{itemize}

\section*{Gadgeteer}

% [Image Inserted Manually]

Swinging from a grappling hook before dropping a smoke bomb, rigging a stubborn lock to blow, or rejiggering traps to spring on those that set them, the Gadgeteer lives by fast paced innovation and thinking outside the box on the fly... sometimes literally.

Specializing tools that let themselves get in and out of trouble, they don’t fight fair, relying on things that go "boom", "zap" or "whirr". They live on the edge of their own explosions, though often rewriting the story in later retellings that they doffed their darkened goggles as they walked away slowly, omitting the singed seat of their trousers and yelping.

\subsection*{Gadget DC}

Your gadgets are a combination of innovative mechanical parts and magic. You use your Intelligence for gadget-based effects when they refer to a spell attack roll or spell save DC.

Gadget DC = 8 + your proficiency bonus + your Intelligence modifier

Gadget modifier = your proficiency bonus + your Intelligence modifier

\subsection*{Tinker’s Knack}

When you choose this presence archetype at 3rd level, you gain proficiency with tinker’s tools. If you already have proficiency with tinker’s tools, you gain expertise with tinker’s tools (if you already have expertise with tinker’s tools, you gain proficiency with another tool of your choice, or expertise with another proficient tool).

Grappling Hook

Starting at 3rd level, you’ve mastered both the creation and use of your most important gadget, the grappling hook. As an action or with the bonus action granted by your Cunning Action, you may target a surface, object or creature within 20 feet. If the target is Small or smaller, you can make a Strength (Athletics) check to pull it to you and grapple it (automatically succeeding against objects that are not being worn or carried unless the GM sets a difficulty for a particularly complicated scenario).

Alternatively, if the target is Medium or larger, you can choose to be pulled to it. This automatically succeeds, but this doesn’t grapple it, though if it is a surface or large object, you can choose to hold onto at the point you grappled if there is something to grab onto. Opportunity attacks provoked by this movement are made with disadvantage.

\subsection*{Pyrotechnic Gadgets}

Additionally at 3rd level, you gain access to an important arsenal of things that explode. Forgoing some subtlety for firepower, you gain the following options:

Smoke Bomb. As an action, you can use this to cast fog cloud centered on yourself without expending a spell slot. It lasts a number of rounds equal to your Intelligence modifier and doesn’t require concentration. When you cast fog cloud in this way, you can set the radius at 5, 10, 15, or 20 feet.

Explosive Surprise. As an action you can toss explosives at a point within 30 feet. Creatures within 5 feet of the target point must make a Dexterity saving throw against your Gadget DC or take thunder damage equal to your Sneak Attack damage. If you are hidden from a creature, it makes the Dexterity saving throw with disadvantage. This counts as dealing Sneak Attack damage and reveals your location if you are hidden.

Shaped Charge. As an action, you can place a charge on an object, building or surface within 5 feet. At the start of your next turn, the charge detonates, dealing thunder damage equal to your Sneak Attack damage to the object, building or surface it was placed on, and half as much damage to any creature within 5 feet of it.

\begin{minipage}{0.48\textwidth}
Unlimited Uses?

Pyrotechnic Gadgets are balanced around the assumption of unlimited use and rogues typically speaking do not have rest gated resources, but practically speaking their supply is not truly infinite, just higher than generally makes sense to be worth tracking. If you would like a number to track, you can use proficiency bonus + Intelligence modifier uses of Pyrotechnic Gadget per short rest.
\end{minipage}\hfill
\begin{minipage}{0.48\textwidth}

\end{minipage}

\subsection*{Clever Inventions}

Starting at 9th level, you expand your selection. Select two of the following options:

Gliding Cloak. You make a cloak that allows you to glide when falling. When you fall more than 10 feet and aren’t incapacitated, can spread this cloak to reduce your falling speed to 30 feet a round and take no falling damage. While falling in this manner toward the ground under normal gravity, you can move horizontally 2 feet for every 1 foot you descend.

Mechanical Arm. You create a mechanical arm, giving you an extra hand. This mechanical arm only functions while it is mounted on gear you are wearing, but can be operated mentally without the need for your hands. This mechanical arm can serve any function a normal hand could, such as holding things, making attacks, and interacting with the environment, but it doesn’t grant you any additional actions.

Mechanical Familiar. You can create the blueprint for a small mechanical creature. At the end of a long rest, you can choose to create a mechanical familiar based on it, and cast find familiar spell without expending a spell slot. The familiar’s type is Construct. This construct stays active until you deactivate it or it is destroyed. In either case, you can choose to reactivate it at the end of a long rest.

Sight Lenses. You create a set of lenses you can integrate into a set of goggles, glasses, or other vision assistance that allow you to see through darkness and obscurement. You can see through fog, mist, smoke, clouds, and nonmagical darkness as normal sight up to 15 feet.

\subsection*{Among the Blasts}

Additionally at 9th level, when you successfully avoid damage from an area of effect with your Evasion feature (including from your own effects), you can use your reaction to move up to your speed to the edge of the effect. This movement doesn’t provoke opportunity attacks.

\subsection*{Deadly Surprise}

Starting at 13th level, your technical expertise expands. The radius of your explosive surprise becomes 10 feet, and you can select one additional option from Clever Inventions.

\begin{minipage}{0.48\textwidth}

\end{minipage}\hfill
\begin{minipage}{0.48\textwidth}
\subsubsection*{Gadgetsmith Integration}

At your GM’s discretion, you can select an upgrade from the Gadgetsmith list that requires 9th level or lower for your additional gadget.
\end{minipage}

\subsection*{Ultimate Improvisation}

Starting at 17th level, you can solve a wide array of problems with your Gadgets at a moment’s notice. Over the course of 1 minute you can construct a gadget capable of casting any spell of 4th level or lower on the Wizard or Inventor list. This gadget lasts until used, or you construct another gadget (dissembling it in the process).

You can do this a number of times equal to your Intelligence modifier before you must finish a long rest to create additional improvised gadgets with the feature.

\subsection*{Gadgeteer Quirks}

The following are some optional quirks for a player of the Gadgeteer You can optionally roll or select a quirk that suits your character.

\begin{minipage}{0.48\textwidth}
\begin{tabularx}{\textwidth}\toprule
{}XXXXX}
\midrule
d6 & \multicolumn{5}{c}{Quirk} \\
\midrule
1 & \multicolumn{5}{c}{You have no fear of heights.} \\
\midrule
2 & \multicolumn{5}{c}{Everything can be improved.} \\
\midrule
3 & \multicolumn{5}{c}{Plans are for after you get caught.} \\
\midrule
4 & \multicolumn{5}{c}{You use smoke bombs to escape awkward conversations.} \\
\midrule
5 & \multicolumn{5}{c}{Walking is for chumps that don’t have grappling hooks.} \\
\midrule
6 & \multicolumn{5}{c}{For some reason everywhere you’ve lived too long has exploded. Strange that.} \\
\midrule
\end{tabularx}
\end{minipage}\hfill
\begin{minipage}{0.48\textwidth}

\end{minipage}

\begin{itemize}
  \item Rogue: Gadgeteer
\end{itemize}

\section*{Aether Heart}

% [Image Inserted Manually]

Sorcerers of this origin are fueled by a deep connection to raw arcane energy, their soul touched by the raw stuff of the weave itself. Most of them don’t understand their connection to this power, but they can wield it with innate ease, manipulating magic by force of will alone.

Sometimes this is artificial in nature—an ironwrought drawing power from their arcane heart, for example—while other times this marks an individual that was exposed to the raw essence of aether, becoming infused and intrinsically linked to it. They are characterized by an easy affinity of warping magic and ease of manipulating it in its raw and purest forms.

\subsection*{Aetherborn}

When you choose this origin at 1st level, your aether heart gives you several advantages. You have resistance to force damage, and can serve as your own arcane focus.

When you reach 3rd level, you automatically gain the Empowered Spell metamagic option, and can use it without spending sorcery points a number of times equal to your proficiency bonus, regaining all uses on a long rest.

\begin{minipage}{0.48\textwidth}
Variant: Origin Spells

If you allow Sorcerers to take additional spells based on their subclass, the following are the recommended Origin Spells for the Aether Heart Origin.
\end{minipage}\hfill
\begin{minipage}{0.48\textwidth}
\begin{tabularx}{\textwidth}\toprule
{}XX}
\midrule
Sorcerer Level & \multicolumn{2}{c}{Spell} \\
\midrule
1st & \multicolumn{2}{c}{magic missile} \\
\midrule
3rd & \multicolumn{2}{c}{star dust*} \\
\midrule
5th & \multicolumn{2}{c}{aether lance*} \\
\midrule
7th & \multicolumn{2}{c}{dimension door} \\
\midrule
9th & \multicolumn{2}{c}{wall of force} \\
\midrule
\end{tabularx}
\end{minipage}

\subsection*{Overcharged Metamagic}

Starting at 6th level, you can unleash the power that rages within your heart. When you modify a spell that deals damage with your Metamagic options, you can add your Charisma modifier to one damage roll of that spell.

\subsection*{Arcane Fuel}

At 14th level, you gain immunity to force damage. When you would take force damage, you regain health equal to half the damage you would have taken. You can gain a maximum amount of hit points this way equal to your sorcerer level, after which you no longer gain additional hit points until you finish a short or long rest.

Additionally, you learn the spell dispel magic , or another sorcerer spell of your choice if you already know it. Dispel magic doesn’t count against your spells known. When you successfully cast dispel magic or counterspell , you regain an expanded sorcery point (if you are not currently at your maximum) as you draw in the remnants of the broken magic.

\subsection*{Tap Power}

At 18th level, you can tap the power that surges through to replenish your magic. As a bonus action, you can expend one Hit Die and regain sorcery points equal to the number rolled + your Charisma modifier. Once you do this, you can’t do so again until you finish a short or long rest.

\subsection*{Aether Heart Origin Quirks}

The following are some optional quirks for a player of this Origin.

\begin{minipage}{0.48\textwidth}
\begin{tabularx}{\textwidth}\toprule
{}XXXXX}
\midrule
d6 & \multicolumn{5}{c}{Quirk} \\
\midrule
1 & \multicolumn{5}{c}{Standing in a high magic area feels like a pleasant sunbathing.} \\
\midrule
2 & \multicolumn{5}{c}{Your eyes turn pale and glowing for several hours after using magic.} \\
\midrule
3 & \multicolumn{5}{c}{You are addicted to using magic.} \\
\midrule
4 & \multicolumn{5}{c}{You have difficulty using magic when sad.} \\
\midrule
5 & \multicolumn{5}{c}{You sometimes see things as they appear to the weave.} \\
\midrule
6 & \multicolumn{5}{c}{You idly weave strands of magic around your hands while waiting.} \\
\midrule
\end{tabularx}
\end{minipage}\hfill
\begin{minipage}{0.48\textwidth}

\end{minipage}

\begin{itemize}
  \item Sorcerer: Aether Heart Origin
\end{itemize}

\section*{Planetouched}

Connected to planes beyond their own, these sorcerers find the walls between the planes thinner, they pull power from beyond them or step through them with greater ease.

Perhaps they were born in the ethereal plane or became connected to an outer plane, they now find themselves with one foot in the material and one foot beyond.

\subsection*{Riftborn}

Starting at 1st level, when you draw on your magic, the planar walls begin to weaken for you. After casting a spell of 1st level or higher, you can use Phase Rift as a bonus action.

You can do this a number of times equal to your proficiency bonus, regaining all uses on a long rest.

\subsection*{Phase Rift}

Psionic power

Casting Time: 1 action
 Range: 10 feet
 Components: S
 Duration: Instantaneous

You step through space traveling up to 10 feet in a straight line leaving a spatial tear behind. You can pass through creatures but can’t pass through objects, buildings or terrain more than 4 inches thick. Any creature in the path of this tear must make a Dexterity saving throw or take 1d8 force damage.

\begin{minipage}{0.48\textwidth}
Variant: Origin Spells

If you allow Sorcerers to take additional spells based on their subclass, the following are the recommended Origin Spells for the Planetouched Origin.
\end{minipage}\hfill
\begin{minipage}{0.48\textwidth}
\begin{tabularx}{\textwidth}\toprule
{}XX}
\midrule
Sorcerer Level & \multicolumn{2}{c}{Spell} \\
\midrule
1st & \multicolumn{2}{c}{flicker*} \\
\midrule
3rd & \multicolumn{2}{c}{misty step} \\
\midrule
5th & \multicolumn{2}{c}{blink} \\
\midrule
7th & \multicolumn{2}{c}{dimension door} \\
\midrule
9th & \multicolumn{2}{c}{teleportation circle} \\
\midrule
\end{tabularx}
\end{minipage}

\subsection*{Phase Out}

Starting at 6th level, whenever you teleport or cross a planar boundary, you can choose to become invisible until the start of your next turn.

\begin{minipage}{0.48\textwidth}
Crossing Planar Boundaries

This would include rolling a 4 on flickerK (entirely phasing out of the material plane) or returning to the material plane after rolling 11 or higher on blink.
\end{minipage}\hfill
\begin{minipage}{0.48\textwidth}

\end{minipage}

\subsection*{Sorcerous Rifts}

Starting at 14th level, you can cast Phase Rift as an action at will. In addition, when you use Phase Rift, you can expend a sorcery point to make it both Long and Disruptive, causing you to move an additional 10 feet and deal an extra 1d8 damage to creatures that fail their save against it.

Disruptive: Each target that fails their saving throw takes an extra 1d8 force damage.

Long: You can travel an additional 10 feet.

\subsection*{Ethereal Control}

Additionally at 14th level, you can expend 1 sorcery point to reroll the value of a die rolled for blink or flicker* . Once you do this, you can’t do so again until you finish a short or long rest.

\subsection*{Planar Collision}

At 18th level you can draw planes together causing them to smash together with devastating results. Select a plane to intersect the one you are on, you can’t select the plane you are currently on. As an action, you can spend 5 sorcery points to cause the planar intersection to occur at a point you can see within 120 feet, causing effects to occur based on the table below in a 30-foot-radius sphere around the chosen point. The effect lasts for a number of rounds equal to your Charisma modifier.

\begin{tabularx}{\textwidth}\toprule
{}XXXXXXX}
\midrule
Plane & \multicolumn{7}{c}{Effect} \\
\midrule
Shadow & \multicolumn{7}{c}{The area is plunged into darkness and can’t be illuminated by natural or magical means.} \\
\midrule
Fey & \multicolumn{7}{c}{Magic in the area goes haywire. A creature that casts a spell in the area or that targets a point within the area must succeed a Charisma saving throw when casting a spell, or a random spell from their spell list of that level or lower is cast (at the level of the spell they were trying to cast). Damage and healing dice are maximized for spells cast in the area.} \\
\midrule
Fire & \multicolumn{7}{c}{The area is incinerated by fire. Creatures that end their turn within the area take 10d10 fire damage. This damage ignores resistance.} \\
\midrule
Ice & \multicolumn{7}{c}{The area freezes over. Creatures that end their turn within the area must make a Constitution saving throw or take 3d12 cold damage and become stunned frozen in ice until the end of their next turn.} \\
\midrule
Earth & \multicolumn{7}{c}{The area becomes filled with stone and earth, shoving all creatures out of it outward from the center. If this would trap them into a wall they take 3d12 bludgeoning damage and are shoved to the closest free space.} \\
\midrule
Radiant & \multicolumn{7}{c}{All creatures that end their turn in the area take 6d6 radiant damage. All living creatures that end their turn in the area are healed for 3d6 hit points.} \\
\midrule
Beyond & \multicolumn{7}{c}{All creatures that end their turn within the area must make a Wisdom saving throw or take 3d12 psychic damage and be affected by the effects of the confusion spell.} \\
\midrule
\end{tabularx}

Once you use this ability, you can’t use it again until you finish a long rest.

\subsection*{Planetouched Quirks}

The following are some optional quirks for a player of the Gadgeteer You can optionally roll or select a quirk that suits your character.

\begin{minipage}{0.48\textwidth}
\begin{tabularx}{\textwidth}\toprule
{}XXXXX}
\midrule
d6 & \multicolumn{5}{c}{Quirk} \\
\midrule
1 & \multicolumn{5}{c}{You occasionally appear in the dreams of others, wandering into the plane of dreams when you sleep.} \\
\midrule
2 & \multicolumn{5}{c}{You see fleeting mirage like images of other planes intersecting your own.} \\
\midrule
3 & \multicolumn{5}{c}{You have a deep fear of portals.} \\
\midrule
4 & \multicolumn{5}{c}{You often toss a pebble through a doorway first before passing through it. Just to check.} \\
\midrule
5 & \multicolumn{5}{c}{You are from a lost folded dimension.} \\
\midrule
6 & \multicolumn{5}{c}{Your hair and eyes sometimes assume otherworldly aspects as other planes brush unusually close to the material.} \\
\midrule
\end{tabularx}
\end{minipage}\hfill
\begin{minipage}{0.48\textwidth}

\end{minipage}

\begin{itemize}
  \item Sorcerer: Planetouched Origin
\end{itemize}

\section*{The Ancient Intelligence}

% [Image Inserted Manually]

You have made a pact of sorts with an intelligence that defies normal understanding; its aims are mysterious and often incomprehensible. This being could be an artificial intelligence of some lost race of ancients with fabulous technology, perhaps a high ranking mechanical overmind. They could even be an ancient lost rogue one, perhaps an Inventor of another era whose last act was to upload themselves in an incorporeal form, or perhaps it something more alien, the ancient navigation system of an otherworldly ship that has gained sentience.

The possibilities are endless, but whatever its nature, it can grant you sufficiently advanced knowledge that for all intents and purposes is fabulous magical power.

\subsection*{Expanded Spell List}

At 1st level, the Ancient Intelligence lets you choose from an expanded list of spells when you learn a warlock spell. The Ancient Intelligence Spells table shows the spells that are added to the warlock spell list for you:

The Ancient Intelligence Expanded Spells

\begin{tabularx}{\textwidth}\toprule
{}XXXX}
\midrule
Spell Level & \multicolumn{4}{c}{Ancient Intelligence Spells} \\
\midrule
1st & \multicolumn{4}{c}{identify, seeking projectile*} \\
\midrule
2nd & \multicolumn{4}{c}{animate object*, locate object} \\
\midrule
3rd & \multicolumn{4}{c}{clairvoyance, crushing singularity*} \\
\midrule
4th & \multicolumn{4}{c}{arcane eye, locate creature} \\
\midrule
5th & \multicolumn{4}{c}{arcane hand, commune} \\
\midrule
\end{tabularx}

\subsection*{The Contraption}

At 1st level, you gain access to a device granted to by your patron that grants some access to their vast knowledge, systems, and power. Your contraption is a Tiny object, and you can use it as a spellcasting focus for your warlock spells.

If the contraption is destroyed or you lose it, you can perform a 1-hour ceremony to receive a replacement from your patron.

This ceremony can be performed during a short or long rest, and the previous contraption is destroyed if it still exists. The contraption may vanish in a flash of light when you die.

\begin{minipage}{0.48\textwidth}
Form of the Contraption

The form of your contraption can vary based on your selection of pact, perhaps even adapting with its alien technology to suit your pact choice, taking the form of an exotic alien weapon, data storage device, or even unique familiar for Pact of the Chain (a GM can allow you to select "Construct" as the creature type of your familiar, for example, or consider if the contraption is perhaps the source of your eldritch blast or other Warlock magical effects.
\end{minipage}\hfill
\begin{minipage}{0.48\textwidth}

\end{minipage}

While you are possession of the contraption, you gain the following benefits:

\begin{minipage}{0.48\textwidth}
Autocaster

Your contraption serves as a repository for additional spells. At the end of a long rest, you can store spells from your Expanded Spell List, or any spell of the divination school of magic that is of a level you could cast. You can have a combined prepared level of spells equal to your proficiency bonus (for example, at 5th level when your proficiency bonus is +3, you can have it store three 1st-level spells, or one 3rd-level spell).

Each spell stored in the Autocaster can be cast without expending a warlock spell slot to cast the spell once, after which you can’t cast it without expending a spell slot again until you finish a long rest. You can cast the spell stored again before finishing a long rest by expending a warlock spell slot of equal or greater level to the stored spell.
\end{minipage}\hfill
\begin{minipage}{0.48\textwidth}
Firing Solution

Your contraption begins to develop the best way to hit a target. While you possess your contraption, you can use it to cast true strike. You can accelerate the calculations and cast true strike as a bonus action a number of times equal to your Spellcasting modifier. When you cast true strike using your contraption, it doesn’t require concentration.

You can do this a number of times equal to your spellcasting ability modifier before you must finish a long rest to do so again.
\end{minipage}

\begin{minipage}{0.48\textwidth}

\end{minipage}\hfill
\begin{minipage}{0.48\textwidth}
Calculation Time

Note that true strike grants you advantage on your next turn, even when cast as a bonus action.
\end{minipage}

\subsection*{Ancient Invocations}

Starting at 6th level, your patron grants you another fragment of their vast power in the form of a gadget or tool. Your Invocations Known increases by 1, but one of your invocations must be from the options described in this feature.

Battle Drones (Prerequisite: Pact of the Chain feature, drone swarm familiar) 
Your Drone swarm can attack without you using your action to direct it attack. Its attack modifier becomes equal to your spell attack modifier.

Data Bank 
While in possession of your contraption, you gain proficiency in two of Arcana, History, Medicine, or Nature which are selected when you gain this Invocation.

Dispatch Relay 
You can use your Autocaster to cast sending once without expending a spell slot. Once you cast it this way, you can’t do so again until you finish a long rest.

Gliding Cloak 
You make a cloak that allows you to glide when falling. When you fall more than 10 feet and aren’t incapacitated, you can spread this cloak to reduce your falling speed to 30 feet a round take no falling damage. While falling in this manner toward the ground under normal gravity, you can move horizontally 2 feet for every 1 foot you descend.

Mechanical Arm 
You create a mechanical arm, giving you an extra hand. This mechanical arm only functions while it is mounted on gear you are wearing, but can be operated mentally without the need for your hands. This mechanical arm can serve any function a normal hand could, such as holding things, making attacks, interacting with the environment, etc, but doesn’t give you any additional actions.

Personal Shield 
You gain the Armor of Shadows invocation. It doesn’t count against your Invocations known.

Neural Link 
When you calculate a firing solution, it now applies to your current turn. Additionally, when you benefit from true strike, you negate the disadvantage of being unable to see a target (such as invisible or obscured target).

\subsection*{Save State}

Starting at 10th level, as a bonus action you can cause your Contraption to record your existence. Record your hit points, location, and any conditions you are affected by. At the start of your next turn, your current hit points, location, and conditions are returned to the saved state in a flash of light. If you die while having your save state, you can make a DC 15 Charisma saving throw (making the check as if you were alive) when the start of your next turn would have been. On success, you return as normal. On failure, you remain dead.

Once you use this, you can’t use this again until you finish a long rest.

\subsection*{Synchronization}

Starting at 14th level, at the start of your turn (no action required) you can synchronize yourself with your Contraption, entering a cold and logical mental state of absolute focus until the start of your next turn. While in this state of synchronization, you gain the following benefits:

\begin{itemize}
  \item You are immune to the charmed and frightened conditions.
  \item You can cast spells with the casting time of 1 action as 1 bonus action.
  \item You gain advantage on all Intelligence and Wisdom checks and saving throws, as well as Constitution saving throws to maintain concentration on a spell.
\end{itemize}

You can be synchronized for a number of turns equal to your spellcasting ability modifier (minimum 1), after which you can’t use this feature again until you finish a long rest.

\begin{minipage}{0.48\textwidth}
Variant Warlock: Intelligence

The designers of 5e noted that it was originally their intention to make the Warlock an Intelligence caster, and that making Intelligence the Warlock spellcasting ability doesn’t break anything.

With the approval of your GM, this subclass may be a thematic fit for an Intelligence-based Warlock who studies their Contraption and tries to understand its knowledge and secrets. If you do, use Intelligence in place of Charisma for any Warlock feature that mentions Charisma.
\end{minipage}\hfill
\begin{minipage}{0.48\textwidth}

\end{minipage}

\subsection*{The Ancient Intelligence Quirks}

The following are some optional quirks for a player who has formed a pact with an Ancient Intelligence You can optionally roll or select a quirk that suits your character.

\begin{minipage}{0.48\textwidth}
\begin{tabularx}{\textwidth}\toprule
{}XXXXX}
\midrule
d6 & \multicolumn{5}{c}{Quirk} \\
\midrule
1 & \multicolumn{5}{c}{You can solve complicated mathematical problems. You have no idea how.} \\
\midrule
2 & \multicolumn{5}{c}{You use highly technical jargon when explaining basic things.} \\
\midrule
3 & \multicolumn{5}{c}{Your catalogue and count strange things in great detail for your patron.} \\
\midrule
4 & \multicolumn{5}{c}{You can speak a dead language no one seems to recognize.} \\
\midrule
5 & \multicolumn{5}{c}{You preform your magic from ornate gadgets that may or may not do anything.} \\
\midrule
6 & \multicolumn{5}{c}{You often disassemble and reassemble mechanical objects to learn their workings.} \\
\midrule
\end{tabularx}
\end{minipage}\hfill
\begin{minipage}{0.48\textwidth}

\end{minipage}

\begin{itemize}
  \item Warlock: The Ancient Intelligence
\end{itemize}

\subsection*{The Ancient Intelligence Familiar}

The following is an expanded Pact of the Chain familiar option for a Warlock of this Otherworldly Patron.

\begin{minipage}{0.48\textwidth}
\subsubsection*{Drone Swarm}

\begin{tabularx}{\textwidth}\toprule
{}XXXXX}
\midrule
\multicolumn{6}{c}{Tiny swarm of Tiny constructs, unaligned} \\
\midrule
\multicolumn{6}{c}{Armor Class 14 (natural armor)
 Hit Points 13 (3d4 + 6)
 Speed 0 ft., fly 30 ft. (hover)} \\
\midrule
STR & DEX & CON & INT & WIS & CHA \\
\midrule
2 (−4) & 12 (+1) & 14 (+2) & 14 (+2) & 16 (+3) & 1 (−5) \\
\midrule
\multicolumn{6}{c}{Saving Throws Int +4
 Skills History +4, Medicine +5, Perception +5
 Damage Resistances bludgeoning, cold, fire, piercing, slashing
 Damage Immunities poison, psychic
 Condition Immunities charmed, exhaustion, frightened, grappled, paralyzed, petrified, poisoned, prone, restrained
 Senses darkvision 60 ft., passive Perception 15
 Languages Common} \\
\midrule
\multicolumn{6}{c}{Challenge 1 (200 XP)
 Proficiency Bonus +2} \\
\midrule
\multicolumn{6}{c}{Traits} \\
\midrule
\multicolumn{6}{c}{Innate Psionics. The drone swarm’s spellcasting ability is Intelligence (spell save DC 12, +4 to hit with spell attacks). It can innately cast the following spells, requiring no material components:
 At will: mending , minor illusion} \\
\midrule
\multicolumn{6}{c}{Swarm. The swarm can occupy another creature’s space and vice versa, and the swarm can move through any opening large enough for a Tiny drone.} \\
\midrule
\multicolumn{6}{c}{Actions} \\
\midrule
\multicolumn{6}{c}{Zap. Melee Spell Attack: +4 to hit, reach 10 ft., one target. Hit: 7 (2d4 + 3) lightning damage.} \\
\midrule
\end{tabularx}

\begin{itemize}
  \item NPC: Drone Swarm
\end{itemize}
\end{minipage}\hfill
\begin{minipage}{0.48\textwidth}

\end{minipage}

\section*{That Which Is Beyond}

Your patron is defined by its sheer incomprehensibility to the mortal mind, an existence that is the very anathema of sanity itself, where any attempt to truly describe its nature is the gibbering of a mad man.

The most tame examples of these may have names or euphemisms that mortals know them as, while others may be concepts that most mortal minds remain blissfully incapable of knowing the existence of.

The patron need not be aware of your existence or invested in it for their power to have affected you, fundamentally warping your mind and granting you powers. Most that would have this opportunity have their mind shattered, left tattered and insane by the experience, but you’ve managed to hang onto some semblance of sanity and wield the power grafted into your mind.

\subsection*{Expanded Spell List}

At 1st level, That Which Is Beyond lets you choose from an expanded list of spells when you learn a warlock spell. That Which Is Beyond Spells table shows the spells that are added to the warlock spell list for you:

That Which Is Beyond Expanded Spells

\begin{tabularx}{\textwidth}\toprule
{}XXXXXXX}
\midrule
Spell Level & \multicolumn{7}{c}{That Which is Beyond Spells} \\
\midrule
1st & \multicolumn{7}{c}{terrifying visions*, hideous laughter} \\
\midrule
2nd & \multicolumn{7}{c}{detect thoughts, psychic drain*} \\
\midrule
3rd & \multicolumn{7}{c}{delve mind*, sending} \\
\midrule
4th & \multicolumn{7}{c}{black tentacles, summon horror} \\
\midrule
5th & \multicolumn{7}{c}{dominate person, dream} \\
\midrule
\end{tabularx}

\subsection*{Opened Mind}

Starting at 1st level, your mind has interacted with something that is incompatible to a mortal understanding, forcing it to be opened and adapted to new ideas and powers. You gain the ability to communicate telepathically with any creature you can see within 30 feet of you. You don’t need to share a language with the creature for it to understand your telepathic utterances, but the creature must be able to understand at least one language. That creature can reply in kind.

\subsection*{Psychic Onslaught}

Additionally at first level, when targeting a creature with a Warlock spell or attack roll, you can use your Telepathic Intrusion on them as a bonus action, assaulting them on both a physical and psychic level.

You can do this a number of times equal to your proficiency bonus, regaining all uses on a long rest. You can only use Telepathic Intrusion through this feature under the conditions it sets, unless you obtain it from another source.

\subsection*{Telepathic Intrusion}

Psionic power

Casting Time: 1 action
 Range: 60 feet
 Components: S Duration: Instantaneous

You assault the mind of a creature you can see directly. The target must succeed on a Wisdom saving throw, or take 1d8 psychic damage. If the target fails the saving throw, it has disadvantage on attacks made against you until the start of your next turn. You can choose to deal no damage to the creature when it fails its saving throw.

\subsection*{Gibbering Terror}

Starting at 6th level, when you use the Telepathic Intrusion power, you always add the Terrifying modifier, causing the target to be frightened of you until the end of your next turn if it fails its saving throw.

When they fail the saving throw by 5 or more, they additionally lose the ability to speak while frightened in this way, gibbering in fear in confusion.

\subsection*{Alien Mind}

Additionally at 6th level, you have advantage on saving throws against being magically charmed or frightened.

\subsection*{Rebound Intrusion}

Starting at 10th level, if a creature attempts to read your mind, you can make a Wisdom saving throw against the effect even if it would normally not allow a save. If it would normally allow a save, you have advantage on the save. If you succeed the save, you can instead read their mind, as if by the detect thoughts spell (this effect doesn’t require concentration, but lasts only until the end of your next turn, and targets only the creature that attempted to read your mind).

Additionally, you gain resistance to psychic damage, and when a creature deals psychic damage to you, it takes an equal amount of psychic damage.

\subsection*{Unleashed Psyche}

Starting at 14th level, you gain enough mental control to form the twisted nightmares that dwell within your mind into the world. As an action, targeting a point within 60 feet you spawn a malignant otherworldly nightmare, taking the form of a twisted aberration of terror.

The first time a creature other than you is within 20 feet of it during their turn (including starting their turn there), they must make a Wisdom saving throw. On failure they take 4d8 psychic damage and become frightened of it.

The spawned nightmare lasts until the start of your turn, when it fades away unless at least one creature has failed their saving throw against its effect, in which case it persists for another round (indefinitely until no creatures fail their saving throw against it, after which it fades away at the start of your next turn).

Once you use this ability, you can’t do so until you finish a short or long rest.

\subsection*{The That Which Is Beyond Invocations}

Psionic Intrusion (Prerequisite: 7th level) 
You gain 2 psi points you can use to apply Telepathy Discipline modifiers to your Telepathic Intrusion. You regain these psi points on a short or long rest.

\subsection*{The That Which Is Beyond Quirks}

The following are some optional quirks for a player who has formed a pact with That Which Is Beyond. You can optionally roll or select a quirk that suits your character.

\begin{minipage}{0.48\textwidth}
\begin{tabularx}{\textwidth}\toprule
{}XXXXXXX}
\midrule
d6 & \multicolumn{7}{c}{Quirk} \\
\midrule
1 & \multicolumn{7}{c}{You reply to your familiar or patron’s telepathic messages out loud.} \\
\midrule
2 & \multicolumn{7}{c}{You have a deep and abiding fear of stars.} \\
\midrule
3 & \multicolumn{7}{c}{You expect to find eldritch horrors under every rock.} \\
\midrule
4 & \multicolumn{7}{c}{You have elaborate theories about the impending end of the material plane.} \\
\midrule
5 & \multicolumn{7}{c}{You refer to the gods as "the new ones"} \\
\midrule
6 & \multicolumn{7}{c}{Sometimes even you aren’t sure you’re talking to when you’re talking to yourself.} \\
\midrule
\end{tabularx}
\end{minipage}\hfill
\begin{minipage}{0.48\textwidth}

\end{minipage}

\begin{itemize}
  \item Warlock: That Which Is Beyond
\end{itemize}

\subsection*{That Which Is Beyond Familiar}

The following is an expanded Pact of the Chain familiar option for a Warlock of this Otherworldly Patron.

\begin{minipage}{0.48\textwidth}
\subsubsection*{Flickering Madness}

\begin{tabularx}{\textwidth}\toprule
{}XXXXX}
\midrule
\multicolumn{6}{c}{Tiny aberration, unaligned} \\
\midrule
\multicolumn{6}{c}{Armor Class 16 (natural armor)
 Hit Points 1
 Speed 0 ft., fly 30 ft.} \\
\midrule
STR & DEX & CON & INT & WIS & CHA \\
\midrule
1 (−5) & 1 (−5) & 1 (−5) & 15 (+2) & 15 (+2) & 15 (+2) \\
\midrule
\multicolumn{6}{c}{Saving Throws Int +4
 Skills Deception +4, Insight +4
 Damage Immunities cold, necrotic, poison, psychic; bludgeoning, piercing, and slashing from nonmagical attacks
 Condition Immunities charmed, frightened, grappled, paralyzed, petrified, prone, restrained, stunned
 Senses darkvision 60 ft., passive Perception 12
 Languages telepathy 60 ft.} \\
\midrule
\multicolumn{6}{c}{Challenge 1 (200 XP)
 Proficiency Bonus +2} \\
\midrule
\multicolumn{6}{c}{Traits} \\
\midrule
\multicolumn{6}{c}{Abstract Existence. If a creature attempts to make an attack roll against the flickering madness, that creature must make a DC 12 Wisdom saving throw. On a failure, the attack fails and the creature becomes frightened of the flickering madness until the end of its turn. Additionally, the flickering madness can attempt to hide using a Charisma (Deception) check in place of a Dexterity (Stealth) check.} \\
\midrule
\multicolumn{6}{c}{Arcane Vulnerability. If the flickering madness casts a spell, it loses its damage immunities until the start of its next turn.} \\
\midrule
\multicolumn{6}{c}{Folding Space. The flickering madness can pass through a narrow gap in a surface, as if becoming two-dimensional along the surface.} \\
\midrule
\multicolumn{6}{c}{Innate Psionics. The flickering madness’s spellcasting ability is Intelligence (spell save DC 12, +4 to hit with spell attacks). It can innately cast the following spells, requiring no components:
 At will: message, vicious mockery} \\
\midrule
\multicolumn{6}{c}{Spider Climb. The flickering madness can climb difficult surfaces, including upside down on ceilings, without needing to make an ability check.} \\
\midrule
\end{tabularx}

\begin{itemize}
  \item NPC: Flickering Madness
\end{itemize}
\end{minipage}\hfill
\begin{minipage}{0.48\textwidth}

\end{minipage}

\section*{Order of Creation}

Wizards that belong to this order of thought seek to advance the application of magic through the merging of mechanical and magical pieces. They share much in common with Inventors in mind set, but delve deeper into the knowledge of magic, relying more heavily on it to power their mechanical inventions, and in turn creating mechanical inventions that more primarily serve to further their magic.

They often seek to look beyond tradition and common knowledge, to seek out new spells and new ways of doing things. The value of new research and new technology is often more than dusty old tomes unless those dusty old tomes contain secrets of those who had reached the heights they seek in ages gone by.

\subsection*{Crafting Fundamentals}

Beginning when you select this school at 2nd level, you gain proficiency in tinker’s tools.

\subsection*{Arcane Automaton}

Starting at 2nd level, you apply your skills of magic and creation to forge a small mechanical helper. The find familiar spell is added to your spellbook. (If its there already add another 1st level Wizard spell of your choosing) Its material component becomes tinker’s tools (they are not consumed on casting), and when you summon it gains the traits for the creature selected, as well as the following additional traits:

\begin{itemize}
  \item Its type becomes construct.
  \item It acts on your turn.
  \item It gains temporary hit points when summoned or at the end of a long rest equal to your Intelligence modifier + your wizard level.
  \item As long as it is within 30 feet of you, you can cast spells as if you were in its position.
  \item Select three wizard cantrips you can cast. Your familiar can cast these cantrips, casting them (in total) a number of times equal to your proficiency bonus (casting them as a 1st level caster). It uses your spellcasting ability modifier when casting these spells. It regains all uses when you finish a long rest.
\end{itemize}

\begin{minipage}{0.48\textwidth}
Automaton Appearance

Most arcane automatons look like a mechanical version of the creature selected for find familiar , but it can have any appearance as long as that appearance maintains the necessary configuration to have the traits it has (for example, some method of flying for flying familiars).
\end{minipage}\hfill
\begin{minipage}{0.48\textwidth}

\end{minipage}

\subsection*{Arcane Gadget}

Additionally at 2nd level, when you prepare spells for the day, you can turn them into arcane gadgets. Select spells with a casting time of 1 action, 1 bonus action, or 1 reaction. These spells do not count against your spells prepared, but you build them into tiny gadgets. You expend the spell slot as normal for casting the spell, and can choose if you expend any material component with a gold piece cost to incorporate it into the gadget (if the material component is not consumed, it can be reclaimed after the gadget is used). Subsequently a creature holding the gadget can use the gadget to cast the spell; they do not require verbal or somatic components, but do require material components if they are not included in the gadget. The gadget loses its power if you use it to cast the spell, if you finish a long rest, or if you regain the spell slot expended to create it (such as with Arcane Recovery).

You can prepare a number spell slot levels this way equal to your Intelligence modifier (for example, if your Intelligence modifier is +3, you could prepare three 1st-level spells or one 3rd-level spell in this way). Creatures other than you can use your arcane gadgets.

Creatures that do so are treated as if they cast the spell though it uses your spellcasting modifiers and proficiency bonus. If the spell requires concentration, they must pass an Intelligence (Arcana) check with a DC of 10 + the spell’s level to cast the spell using the gadget. On a failure, the spell fizzles and is lost.

Any spell cast by the gadget ends when the wizard recovers the spell slot spent to create the arcane gadget.

\begin{minipage}{0.48\textwidth}
\subsubsection*{Imbued Gear}

Starting at 6th level, select a 1st- or 2nd-level wizard spell, and forge it into a magical item. You can cast this spell once without expending a spell slot, regaining the ability to cast it again when you finish a long rest.

If the spell targets only you, has a duration of more than 10 minutes, and doesn’t require concentration, it lasts until you finish a long rest instead of its normal duration.
\end{minipage}\hfill
\begin{minipage}{0.48\textwidth}
Extended Duration Spells

The spells that fit this criteria for expanding their duration include include:

\begin{itemize}
  \item 1st level: false life, longstrider, mage armor
  \item 2nd level: continual flame, darkvision
\end{itemize}
\end{minipage}

\subsection*{Reverse Engineer}

Additionally at 6th level, you gain the ability to reverse engineer the arcane magic of magic items. You can learn a spell from a magic item that can cast a Wizard spell as you would from a scroll, though the magic item is not destroyed in the process.

\subsection*{Clever Creation}

Starting at 10th level, your Arcane Automaton gains proficiency in two skills of your choice and can activate spells imbued in Arcane Gadgets without making an Intelligence (Arcana) check as long as the spell doesn’t require concentration. You can change what skills your familiar is proficient in when you summon it with find familiar again.

\subsection*{Creator’s Vision}

Starting at 14th level, you can forge your Arcane Automation into new forms limited only by your creative vision. Select two of the following additional properties for it:

\begin{itemize}
  \item It becomes Medium or Large (your choice).
  \item It gains a flying speed equal to its walking speed (if it doesn’t already have one).
  \item It gains additional temporary hit points at the end of a long rest or when summoned equal to Wizard level (giving it combined total hit points equal to twice your Wizard level).
  \item It can cast its cantrips as a 5th level caster, and regains half the uses of its cantrip casting when you use your Arcane Recovery feature.
  \item It can cast a spell from an Arcane Gadget without making an Intelligence (Arcana) check even if the spell requires concentration, as long as the spell is 2nd level or lower.
\end{itemize}

You can select the same properties or new properties to replace them when you summon your familiar again with find familiar.

\subsection*{Order of Creation Quirks}

The following are some optional quirks for a player that selects Order of Creation. You can optionally roll or select a quirk that suits your character.

\begin{minipage}{0.48\textwidth}
\begin{tabularx}{\textwidth}\toprule
{}XXXXX}
\midrule
d6 & \multicolumn{5}{c}{Quirk} \\
\midrule
1 & \multicolumn{5}{c}{You look for optimizations of every day processes.} \\
\midrule
2 & \multicolumn{5}{c}{You are always sketching out a new blueprint.} \\
\midrule
3 & \multicolumn{5}{c}{You can sometimes get lost in the technical details of dangerous situations.} \\
\midrule
4 & \multicolumn{5}{c}{You like to test your creations in the field. Stressful situations provide the best data.} \\
\midrule
5 & \multicolumn{5}{c}{Everything is a problem to be solved. Logic can solve every problem.} \\
\midrule
6 & \multicolumn{5}{c}{You prefer to number your plans. Trying to use the alphabet for your alternate plans limits them.} \\
\midrule
\end{tabularx}
\end{minipage}\hfill
\begin{minipage}{0.48\textwidth}

\end{minipage}

\begin{itemize}
  \item Wizard: Order of Creation
\end{itemize}

\section*{Multiclassing}

As should be apparent, many of these subclasses are built around the themes introduced by the Inventor and Psion class in Chapter 1. They are intended to be fully standalone, but as with all things, it’s likely that players will want to mix and match. I’m not going to try to define every interaction, simply offer the general guide that, where possible, things will probably be generally fine if they are allowed to interact as one would hope.

\section*{Upgrades and Upgrade-Like Things}

Many of the Inventor themed classes get things that are similar to upgrades, but gained through their features. Generally these features can be treated as a similar upgrade in most cases where it would matter.

For example, a Techknight that wants a quick way to get a second Ramming Gauntlet might opt to take a level of Inventor and pick up a Power Fist from Warsmith. This level of mixing and matching will result in some situations where the power curve will be lightly modified, but rarely result in any long term imbalances.

Likewise, if they had Arcane Retrofit from either class, I’d probably let them use that Arcane Retrofit on the similar item from the other class as well, even if they didn’t have the sufficient level of the other class.

The main exception to this general rule is that I would generally recommend anything that stacks a flat bonus to be, at very least, treated with caution.

\section*{Psi Point Interchangeability}

All classes that use Psi Points and Ki assume that these points are interchangeable in terms of balance and multiclassing. This might not be the case in your setting if Ki is a significantly different power than Psionics, but it should be fine either way.

Any time a subclass gets Psi Points, I would allow those to be used on Psion class features like normal. There are exceptions to this under the standard disclaimer of specifics. Psionic Mastery still only applies to psionic powers, and wouldn’t apply to features gained through other subclasses by default, unless they also have the psionic power tag.

% [Image Inserted Manually]

% [Image Inserted Manually]

\chapter{Chapter 3: Backgrounds, Races, Feats}

\section*{Backgrounds}

The following are additional background options for characters with a professional or artisan background. These integrate nicely with the crafting system providing useful proficiencies and features.

\section*{Apothecary}

You were a mixer of potions, a master of the subtle art of producing potent magical effects from the right distillations. Hailing from the largest cities or the smallest villages, an apothecary’s work is always in demand, though many that would demand it do not fully appreciate the care that goes into making a potion safe and effective.

\begin{itemize}
  \item Skill Proficiencies: Medicine, Nature
  \item Tool Proficiencies: Alchemist’s supplies, herbalism kit
  \item Equipment: A potion of healing , a set of alchemist’s supplies, an herbalism kit, and a set of common clothes
\end{itemize}

\subsection*{Feature: Herblore}

You know what the potions people drink are actually made of. Whenever traveling at half speed or less through wilderness or camping in wilderness, you can acquire 15 gp a day worth of potion reagents that can be used as ingredients for the next potion you craft.

\begin{minipage}{0.48\textwidth}
Ideals

\begin{tabularx}{\textwidth}\toprule
{}XXXXX}
\midrule
d6 & \multicolumn{5}{c}{Ideals} \\
\midrule
1 & \multicolumn{5}{c}{Practice. I work to become the greatest at my craft. (Neutral)} \\
\midrule
2 & \multicolumn{5}{c}{Discovery. Every combination of reagents deserves to be tried. (Chaotic)} \\
\midrule
3 & \multicolumn{5}{c}{Communication. I make sure my patients know exactly what they are about to consume. (Lawful)} \\
\midrule
4 & \multicolumn{5}{c}{Fairness. Those divine folks only treat their followers, everyone deserves my services. (Good)} \\
\midrule
5 & \multicolumn{5}{c}{Greed. Never cure a disease, it’s more profitable to treat the symptoms. (Evil)} \\
\midrule
6 & \multicolumn{5}{c}{Precision. There’s a tonic for any ailment. (Any)} \\
\midrule
\end{tabularx}

\begin{itemize}
  \item Table: Apothecary Ideal
\end{itemize}
\end{minipage}\hfill
\begin{minipage}{0.48\textwidth}
Bonds

\begin{tabularx}{\textwidth}\toprule
{}XXXXX}
\midrule
d6 & \multicolumn{5}{c}{Bond} \\
\midrule
1 & \multicolumn{5}{c}{A book of passed down recipes was lost or destroyed, I aim to recreate it.} \\
\midrule
2 & \multicolumn{5}{c}{A magical malady still plagues my hometown, I adventure in hopes of a cure.} \\
\midrule
3 & \multicolumn{5}{c}{I recite or make up rhymes to remember alchemical properties.} \\
\midrule
4 & \multicolumn{5}{c}{I was unable to cure a sick child, I will not fail anyone like that again.} \\
\midrule
5 & \multicolumn{5}{c}{I come from a long line of apothecaries and my surname has renown or infamy.} \\
\midrule
6 & \multicolumn{5}{c}{I have a different scented incense for every occasion and mood.} \\
\midrule
\end{tabularx}

\begin{itemize}
  \item Table: Apothecary Bond
\end{itemize}
\end{minipage}

\begin{minipage}{0.48\textwidth}
Flaws

\begin{tabularx}{\textwidth}\toprule
{}XXXXX}
\midrule
d6 & \multicolumn{5}{c}{Flaw} \\
\midrule
1 & \multicolumn{5}{c}{You diagnose the illnesses in those around you... whether they have any or not.} \\
\midrule
2 & \multicolumn{5}{c}{You are always on the hunt for the newest alternative medicine remedies.} \\
\midrule
3 & \multicolumn{5}{c}{The best way to know if something is poisonous is to put it in your mouth, builds strong immune systems.} \\
\midrule
4 & \multicolumn{5}{c}{You don’t trust these “divine” powers in the sky and their healing powers with no in basis proper herblore.} \\
\midrule
5 & \multicolumn{5}{c}{An apothecary is paid for their services.} \\
\midrule
6 & \multicolumn{5}{c}{Toxins in the blood can be removed by removing the blood. Leeches work well, but in a pinch...} \\
\midrule
\end{tabularx}

\begin{itemize}
  \item Table: Apothecary Flaw
\end{itemize}
\end{minipage}\hfill
\begin{minipage}{0.48\textwidth}
\begin{itemize}
  \item Background: Apothecary
\end{itemize}
\end{minipage}

\section*{Engineer}

Scholars and academics are great, but many a kingdom has needed those of the more intellectual bent to put their ability to more practical uses: designing everything from siege equipment to bridges.

While construction may or may not have used magic here and there, at the end of the day something built without an engineer’s oversight is much more likely to fall back down when it is needed most.

\begin{itemize}
  \item Skill Proficiencies: Investigation, Nature
  \item Tool Proficiencies: Carpenter’s tools, mason’s tools
  \item Equipment: A bottle of black ink, a quill, a small knife, the blue prints to the last project you were working on, a set of common clothes, and a belt pouch containing 10 gp
\end{itemize}

\begin{minipage}{0.48\textwidth}
\subsubsection*{Feature: Blueprints}

Given one day to plan, you can produce blueprints or schematics for bridges, siege equipment, buildings, dams, or any other common feat of engineering that anyone proficient with the necessary artisan tools could follow to craft the outlined construction given the time and resources. Buildings more fantastical in nature may still be within your grasp to plan with skill-appropriate ability checks at the GM’s discretion.
\end{minipage}\hfill
\begin{minipage}{0.48\textwidth}
Engineering with Blueprints

If you are using crafting, these blue prints can reduce the time and DC for engineering projects, reducing the crafting DC by your proficiency bonus for the construction project.
\end{minipage}

\begin{minipage}{0.48\textwidth}
Ideals

\begin{tabularx}{\textwidth}\toprule
{}XXXXX}
\midrule
d6 & \multicolumn{5}{c}{Ideals} \\
\midrule
1 & \multicolumn{5}{c}{Accomplishment. I love nothing more than the feeling of completing a project. (Neutral)} \\
\midrule
2 & \multicolumn{5}{c}{Innovation. Rules and assumptions get in the way of true innovation. (Chaotic)} \\
\midrule
3 & \multicolumn{5}{c}{Documentation. All knowledge should be codified so it can be applied to future projects. (Lawful)} \\
\midrule
4 & \multicolumn{5}{c}{Equality. Anything I make needs to benefit everyone, from the mightiest king to the lowliest peasant.} \\
\midrule
5 & \multicolumn{5}{c}{Glory. Lesser structures and people need to be relocated for my great works. (Evil)} \\
\midrule
6 & \multicolumn{5}{c}{Discovery. I travel to see the great works created by others and to inspire my own work. (Any)} \\
\midrule
\end{tabularx}

\begin{itemize}
  \item Table: Engineer Ideal
\end{itemize}
\end{minipage}\hfill
\begin{minipage}{0.48\textwidth}
Bonds

\begin{tabularx}{\textwidth}\toprule
{}XXXXX}
\midrule
d6 & \multicolumn{5}{c}{Bond} \\
\midrule
1 & \multicolumn{5}{c}{I found part of an extremely complicated blueprint I could not decipher, I seek the other pages of this design.} \\
\midrule
2 & \multicolumn{5}{c}{I carry an antiquated set of tools to remind me of where I came from.} \\
\midrule
3 & \multicolumn{5}{c}{One day I want a building or structure named after me.} \\
\midrule
4 & \multicolumn{5}{c}{I studied at a prestigious institution.} \\
\midrule
5 & \multicolumn{5}{c}{Someone destroyed one of my prized creations and I will have my revenge.} \\
\midrule
6 & \multicolumn{5}{c}{I failed an important exam and lost my spot in the academy where I studied.} \\
\midrule
\end{tabularx}

\begin{itemize}
  \item Table: Engineer Bond
\end{itemize}
\end{minipage}

\begin{minipage}{0.48\textwidth}
Flaws

\begin{tabularx}{\textwidth}\toprule
{}XXXXX}
\midrule
d6 & \multicolumn{5}{c}{Flaw} \\
\midrule
1 & \multicolumn{5}{c}{IHAFE (I have acronyms for everything).} \\
\midrule
2 & \multicolumn{5}{c}{I will spend hours automating a task that would have taken minutes.} \\
\midrule
3 & \multicolumn{5}{c}{I scoff at and belittle what I consider less practical professions.} \\
\midrule
4 & \multicolumn{5}{c}{I won’t stop talking about the prestigious institution where I studied.} \\
\midrule
5 & \multicolumn{5}{c}{I require invigorating hot bean juice in order to get my best work done.} \\
\midrule
6 & \multicolumn{5}{c}{I explain everything in great detail at anyone who happens to be nearby.} \\
\midrule
\end{tabularx}

\begin{itemize}
  \item Table: Engineer Flaw
\end{itemize}
\end{minipage}\hfill
\begin{minipage}{0.48\textwidth}
\begin{itemize}
  \item Background: Engineer
\end{itemize}
\end{minipage}

\section*{Tinker}

Falling somewhere between a merchant, a sage, and—if you ask some villagers—a vagabond, a Tinker wanders from town to town bringing a cart (or sometimes just a donkey) of knickknacks and knowhow. While not always welcomed with open arms, they can be a lifeline to the smallest and most widely flung towns, as no route is too odd and winding for them to wander. Tinkers may be wanderers by nature, but what opens doors is their useful knack for being able to fix problems with a dash of knowhow and ingenuity.

\begin{itemize}
  \item Skill Proficiencies: Insight, Nature
  \item Tool Proficiencies: Tinker’s tools
  \item Languages: One of your choice
  \item Equipment: A set of tinker’s tools, a set of traveler’s clothes, a set of saddle bags, and a pack horse
\end{itemize}

\subsection*{Feature: Know-How}

You’ve been around the block, and fixed more than one problem you had no business fixing with some application of ingenuity. When you need to construct or fix an item, you can often tinker up a replacement part from an alternate source. Any item that you could craft with Tinker’s Tools for less then 50gp pieces, you can craft for free with misce

\begin{minipage}{0.48\textwidth}
Ideals

\begin{tabularx}{\textwidth}\toprule
{}XXXXX}
\midrule
d6 & \multicolumn{5}{c}{Ideals} \\
\midrule
1 & \multicolumn{5}{c}{Study. I will someday understand the intricate motions of the world itself. (Neutral)} \\
\midrule
2 & \multicolumn{5}{c}{Overhaul. Everything in the world is made up of parts, I can use old parts to make new things. (Chaotic)} \\
\midrule
3 & \multicolumn{5}{c}{Maintenance. Everything needs a tune up now and then to keep it functioning as intended. (Lawful)} \\
\midrule
4 & \multicolumn{5}{c}{Upgrade. Just because something isn’t broken doesn’t mean it can’t be improved. (Good)} \\
\midrule
5 & \multicolumn{5}{c}{Greed. If I wait until a problem worsens, people will pay me more to fix it. (Evil)} \\
\midrule
6 & \multicolumn{5}{c}{Synergy. I do my job so others can keep doing theirs, like the gears of a clock. (Any)} \\
\midrule
\end{tabularx}

\begin{itemize}
  \item Table: Tinker Ideal
\end{itemize}
\end{minipage}\hfill
\begin{minipage}{0.48\textwidth}
Bonds

\begin{tabularx}{\textwidth}\toprule
{}XXXXX}
\midrule
d6 & \multicolumn{5}{c}{Bond} \\
\midrule
1 & \multicolumn{5}{c}{One of my creations backfired in front of a noble leaving me humiliated and disgraced.} \\
\midrule
2 & \multicolumn{5}{c}{A mentor trained me in tinkering to keep me out of trouble.} \\
\midrule
3 & \multicolumn{5}{c}{I once accidentally removed the screws... to a lock... to the town jail. The guards still view me as an accomplice.} \\
\midrule
4 & \multicolumn{5}{c}{The people of a distant town view me as family, and give me the warmest of welcomes.} \\
\midrule
5 & \multicolumn{5}{c}{I made an intricate mechanical creation... that wandered off in the night.} \\
\midrule
6 & \multicolumn{5}{c}{A wizard started offering magical solutions to mechanical problems, driving me out of business.} \\
\midrule
\end{tabularx}

\begin{itemize}
  \item Table: Tinker Bond
\end{itemize}
\end{minipage}

\begin{minipage}{0.48\textwidth}
Flaws

\begin{tabularx}{\textwidth}\toprule
{}XXXXX}
\midrule
d6 & \multicolumn{5}{c}{Flaw} \\
\midrule
1 & \multicolumn{5}{c}{I can’t resist taking things apart to see how they work, I can mostly put them back together.} \\
\midrule
2 & \multicolumn{5}{c}{Stress testing is mandatory. Even things I don’t own. Even without permission.} \\
\midrule
3 & \multicolumn{5}{c}{Aesthetics are irrelevant to function.} \\
\midrule
4 & \multicolumn{5}{c}{I solve problems with a depth first methodology, often creating new problems.} \\
\midrule
5 & \multicolumn{5}{c}{I can never remember which pocket or pouch the item I am looking for is stored.} \\
\midrule
6 & \multicolumn{5}{c}{I’ll start attaching new “features” to anything around me that isn’t nailed down.} \\
\midrule
\end{tabularx}

\begin{itemize}
  \item Table: Tinker Flaw
\end{itemize}
\end{minipage}\hfill
\begin{minipage}{0.48\textwidth}
\begin{itemize}
  \item Background: Tinker
\end{itemize}
\end{minipage}

\section*{Races}

The following are new race options that you can select for your character. This options are subject to the approval of your GM, and due to the unique nature may not be present in all settings, or may need to be adapted to fit within your GM’s settings.

\begin{minipage}{0.48\textwidth}
\subsubsection*{Adapting Ironwrought}

The Ironwrought are easy to adapt to many settings, though the nature of their origin may vary widely. Perhaps they are ancient relics of a magical empire recently unearthed or perhaps they were mass produced for a terrible war the world still remembers. Perhaps they are entirely unique hurled across time and space by forces unknown, and no one is more confused as to how they got there than the Ironwrought themselves.

While their lore is easy to adapt, their thematic presence may not fit all games. At the end of the day they are a concept more akin to sci-fi than the most classical of fantasy, and while the lines blur, setting up a world and what is in it is always the domain of the GM.

\begin{itemize}
  \item Ironwrought
\end{itemize}
\end{minipage}\hfill
\begin{minipage}{0.48\textwidth}
\subsubsection*{Adapting Farling}

Farlings will general find a home anywhere psionics and strange outer planes beyond can be found. Due to the chaotic warping nature of existence beyond the orderly planes, the possibilities of what they spawn is near endless, and consequently slotting one of these strange alien wanderers in tends to be an easy fit.

They represent a more benign interaction with these planes than most creatures that hail from such twisting madness, and can make excellent vehicles of exposition about them... or perhaps not. They may only know first hand that such things are unknowable.

\begin{itemize}
  \item Farling
\end{itemize}
\end{minipage}

\section*{Ironwrought}

% [Image Inserted Manually]

The origins of Ironwrought are many and varied. They can be the artisanal masterpiece of a Golemsmith, or perhaps a relic of a mass production of an ancient lost civilization... or perhaps not-so-lost a civilization. Their source and purpose will vary greatly. Were they made with sapience in mind, or did they gain it later? Were they made for labor or war?

Some Ironwrought will dwell on these questions and what it means for them to exist outside of the boundary of their original purpose, but many simply move on, acquiring new parameters to live their life from those they interact with, finding their purpose in life as any of their more fleshy counterparts might.

How they are treated in society will vary based on how common they are. In settings where Ironwrought created for labor are common, they will generally be treated quite neutrally, while they may be treated with greater fear or concern if Ironwrought were primarily created as war machines, though in both cases they might be mistaken for their last sapient cousins.

\subsection*{Self Aware}

\begin{minipage}{0.48\textwidth}
While most constructs are not truly self aware, an Ironwrought is a special case of a cosntruct that has attained a full degree of self-awareness... sometimes in a more rudimentary way, sometimes as self-aware as any human. How that occurred is often dependent on your origin. Here is a table of some origins of your self-awareness. You can roll on the following table, pick one that suits your character, or select a different reason entirely based on your game’s setting and character’s backstory.
\end{minipage}\hfill
\begin{minipage}{0.48\textwidth}
Origin of Sapience

\begin{tabularx}{\textwidth}\toprule
{}XXXXX}
\midrule
d8 & \multicolumn{5}{c}{Origin} \\
\midrule
1 & \multicolumn{5}{c}{You existed without self awareness for years before a magical accident rendered you self aware.} \\
\midrule
2 & \multicolumn{5}{c}{You were a new experiment, designed for more complicated tasks, and ended up more sapient than expected} \\
\midrule
3 & \multicolumn{5}{c}{Left to a mindless task for hundreds of years in a long abandoned location, you slowly developed a sense of self until you abandoned your programmed task.} \\
\midrule
4 & \multicolumn{5}{c}{Your sapience is the result of the influence of a supernatural being (celestial, fey, or even fiend) granting you self-awareness for their purposes (or entertainment).} \\
\midrule
5 & \multicolumn{5}{c}{You awoke from a long hibernation changed, with greater self awareness. You don’t remember the past.} \\
\midrule
6 & \multicolumn{5}{c}{The Inventor that created you intentionally granted you full sapience, trying to build a child, friend, or adventuring companion.} \\
\midrule
7 & \multicolumn{5}{c}{You are the consciousness of a mortal imbued into a metal shell, transferred to a Construct to save your life.} \\
\midrule
8 & \multicolumn{5}{c}{You are the inventor that created yourself, transferring your consciousness to the Construct for reasons of your own.} \\
\midrule
\end{tabularx}
\end{minipage}

\subsection*{Personality}

\begin{minipage}{0.48\textwidth}
The personality of Ironwrought can vary greatly as their origins can vary greatly. In some cases they are humanoids that have been bound to an Ironwrought, melancholy about their new metalbound existence. Other times they have a pragmatic nature unconcerned with the unknowable trivialities, focused on the here and now. Ironwrought most frequently have pragmatic personalities that are inclined to logical outcomes. While as a sapient creature they are capable of understanding emotion, they may not fully experience it, or not let it affect them to the degree of the more easily influenced fleshy fellows.
\end{minipage}\hfill
\begin{minipage}{0.48\textwidth}
Quirks

\begin{tabularx}{\textwidth}\toprule
{}XXXXX}
\midrule
d6 & \multicolumn{5}{c}{Quirks} \\
\midrule
1 & \multicolumn{5}{c}{You provide one detail too many in any conversation.} \\
\midrule
2 & \multicolumn{5}{c}{You feel pity for fleshy creatures with their fleeting life spans.} \\
\midrule
3 & \multicolumn{5}{c}{You are inherently rational, and can be persuaded of anything with appropriate logic.} \\
\midrule
4 & \multicolumn{5}{c}{You have a programmed code of conduct that you will never violate. You may or may not be aware of it.} \\
\midrule
5 & \multicolumn{5}{c}{A vestige of a previously set task occasionally surfaces, compelling you to perform a strange action, such as sweeping a floor clean.} \\
\midrule
6 & \multicolumn{5}{c}{You have some fragments of the memories of another creature, perhaps a past version of yourself, in your mind.} \\
\midrule
\end{tabularx}
\end{minipage}

\subsection*{Ironwrought Names}

Your name can be virtually anything. Many Ironwrought have no real name, often having either acquired a nickname before or after gaining sapience. Some take on human names, though more often they adopt simple names, placing little importance on it:

Ironwrought Names: Aegis, Chomper, Champ, Chump, Bolts, Driver, Finder, Gears, Kaz, Mium, Rusty, Sparky, Stomper, Tackler, Thing, Titan, Vel.

\subsection*{Ironwrought Traits}

Your Ironwrought character has the following traits.

Ability Score Increase. Your Constitution score increases by 2, and one other ability score of your choice increases by 1.

Age. Constructs do not age physically, though their mental abilities may change greatly through their life as they process new information and experiences.

Alignment. Ironwrought tend to be inclined toward Lawful Neutral, following an internal logic and purpose, but player Ironwrought have gained full sapiance, and can make their own decisions. Some opt to rebel entirely against their nature, acting in unpredictable chaotic patterns like any mortals.

Speed. Your base walking speed is 30 feet.

Living Construct. You are immune to disease. You do not need to eat, drink, or breathe. You do not need sleep, and magic can’t put you to sleep.

Languages. You can speak, read, and write Common and one other language of your choice (usually the language of your creator if they spoke a language other than common).

Power Down. You do not need to sleep. When you take a long rest, you must spend at least 4 hours during a long rest in a motionless lower power state. While in this state, you have disadvantage on Wisdom (Perception) checks, but remain aware of your surroundings.

Construct. Your type is construct.

Arcane Life. You can absorb certain types of healing magic, allowing you to recover hit points from magical effects that do not otherwise affect constructs (such as the cure wounds spell).

Modular Design. You can select two of the following specialized traits representing the functionality built into your chassis.

\begin{itemize}
  \item Integrated Armor. You gain a +1 bonus to Armor Class.
  \item Integrated Weapon. You have a simple or martial weapon of your choice built into your frame. You gain proficiency with this weapon. This doesn’t remove the need for ammunition for weapons that have the ammunition property. This weapon is typically integrated into your limbs, and requires hands to use as normal.
  \item Language Module. You learn one additional language, and can cast comprehend language s without consuming a spell slot. Once you cast this spell, you can’t cast it again until you finish a long rest. You can cast this spell using spell slots.
  \item Natural Armor. You have built in armor, giving you a base AC of 16 (your Dexterity modifier doesn’t affect this number). You can use a shield and apply the shield’s bonus as normal.
  \item Night Mode. You can see in dim light within 60 feet of you as if it were bright light, and in darkness as if it were dim light. You can’t discern color in darkness, only shades of gray.
  \item Rapid Movement. Your base walking speed increases by 5 feet.
  \item Redundant Logic. You process all inputs twice to ensure they are logical. You have advantage on saving throws against becoming frightened or charmed.
  \item Specialized Skill. You gain proficiency in one skill of your choice.
  \item Specialized Tool. You gain one tool built into your frame. You gain proficiency with this tool.
  \item Terrain Adaptation . You gain a climbing or swimming speed equal to your walking speed.
\end{itemize}

\subsection*{Chassis Configuration}

Ironwrought come in many shapes and sizes. Select one of the following configurations of your chassis.

Humanoid

Your shape is roughly that of a bipedal humanoid. You gain the following traits.

Adaptable Design. You can make one additional selection from your modular design trait.

Size. Your size is Medium.

Quadrupedal

Your shape has four legs and no hands. You gain the following traits.

Quadrupedal. You do not have hands. You can hold one object or grapple one thing using your mouth, but can’t speak when you do so, and any weapons held in this way are wielded with disadvantage. You can perform the somatic components of spells with one of your legs, however doing so requires 15 feet of your movement.

Natural Weapons. You gain a selection of natural weapons. You are considered to be holding natural weapons for the purposes of abilities that require you to to hold a weapon (this doesn’t confer disadvantage to them). Select two of the following:

\begin{itemize}
  \item Metal Chompers. You have metal fangs suitable for chomping. On hit, they deal 1d10 piercing damage.
  \item Razor Claws. You have two sets of razor claws integrated into your front limbs. On hit, they deal 1d6 slashing damage and have the light property.
  \item Bladed Tail. You have a lashing bladed tail. On hit, it deals 1d4 slashing damage. It has the reach property.
\end{itemize}

Barding. You can only wear armor specially adapted to your unique shape. This custom armor has the same price as normal armor, but may be harder to find or more expensive when custom ordered.

Size. Your size is Medium.

Variant Large Constructs

At the discretion of your GM, a quadrupedal Ironwrought construct can be Large sized. It gains the following traits:

Large Natural Weapons. The damage dice of your natural weapons increases by one size (from a d10 to a d12 or from a d6 to a d8). Your natural weapons without the light property gain the heavy property.

Bulwark. You count as three-quarters cover for allied creatures when you would normally count as half cover.

Barding. You can’t wear armor designed for normal-sized creatures. Your armor has to be designed for Large-sized creatures of your form, and generally costs four times the normal cost of armor.

Size. Your size is Large

Large Variant

A variant large Ironwrought is often designed in a configuration that can serve as a mount for other creatures. Large humanoid Constructs may exist, but are not recommended for Player Creatures without dedicated large PC rules.

\begin{itemize}
  \item Race: Ironwrought
\end{itemize}

Fantasy Grounds Conversion Note

The three Chassis Configuration options for the Ironwrought have been coded as subraces of the Ironwrought race. When the race is selected at character creation the player will be prompted to select one of the three subraces, and the appropriate traits will be added to the character sheet.

Each of the Modular Design options has been coded as a "spell" with Ironwrought Modular Design noted as the source, and the chosen options can simply be dragged to the actions tab on the character sheet for reference.

\section*{Farling}

% [Image Inserted Manually]

Farlings are a race that comes from beyond the material plane, though few are certain where. While humanoid in shape, they are immediately distinguished from mortal races. They have a slender build, large eyes, and tendrils in place of hair. Their strange nature and inherent psionic abilities make them easily feared, but they tend to be far more friendly to mortal races than most of that which comes from the beyond.

Farlings tend to be blue to purple, but can be more exotic colors, or even have strange and exotic colorations and patterns in some cases. The nature and location of their creation results in little consistency of exact appearance, though they will often have some traits of their progenitor (such as sharing an eye or skin coloration).

\subsection*{Adaptable Mind}

Farlings are exceedingly adept at picking up social cues and cultural norms, and despite their strange appearance can easily fit into a tolerant society, but often maintain exceedingly strange senses of humor and behavior, fully aware of its oddity, sometimes even playing up their alien nature. They enjoy seeing the reactions of other creatures to strange circumstances, even if they have to cause those situations.

\subsection*{Memory and Aging}

Farlings age by gaining memories. They do not biologically age, but have the ability to make permanent crystalized memories, and age roughly one day each time they do. Young Farlings make these memories a dozen times a day, aging rapidly while absorbing new information at an incredible rate, while aging Farlings become more selective in what they choose to remember. If they do not commit some memory in this way, their memory tends to be still somewhat better than a human, but they view these memories as transient things that will be eventually forgotten.

\subsection*{Nature and Biology}

Farlings do not have a gender, though those that travel among other races may associate with one. They have a generally androgynous and alien apparence. They are capable of forming strong emotional bonds, though some (particularly younger ones) will find the idea of gender and the associated customs hilarious (potentially inappropriately so based on the culture). Farlings that exist for hundreds of years and absorb thousands of memories can create a new Farling by traveling to realms of mind and thought where reality is more flexible, and psionically willing a new one into existence.

A newly created Farling is often left to experience the world on their own shortly after the creation, left with associates or friends of its creator (that they have hopefully made arrangements with) who will then move on, seeking to let the new Farling develop on its own, often seeking to not influence their experiences more than necessary.

Taking care of a young Farling can be exhausting due to the rampant curiosity and inquisitive nature rapidly consuming new experiences and memories, but their inquisitive nature causes them to age exceedingly quickly. A young Farling can fully absorb many memories a day, aging exceedingly fast. It is usually not long before they seek to wander broader environments, and set out into the world.

Farlings are not aquatic, but are unusually suited to it, often finding water quite comfortable and are natural swimmers.

Dark Reflection 
Some Farlings become addicted to acquiring memories, and particularly savor those from other creatures. While a player character can only absorb memories from a willing creature, some stories tell of Farlings that have fallen into evil ways, ripping and devouring the memories from mortal creatures compulsively devouring their memories until nothing is left, savoring the exotic memories they discover.

\begin{minipage}{0.48\textwidth}
\subsubsection*{Farling Quirks}

\begin{tabularx}{\textwidth}\toprule
{}XXXXX}
\midrule
d6 & \multicolumn{5}{c}{Quirks} \\
\midrule
1 & \multicolumn{5}{c}{You ask blunt personal questions of strangers.} \\
\midrule
2 & \multicolumn{5}{c}{You go to great lengths to try strange food. Your definition of food is unusually broad.} \\
\midrule
3 & \multicolumn{5}{c}{You don’t like to sleep in the same place twice.} \\
\midrule
4 & \multicolumn{5}{c}{You place as little value on other’s names as your own, calling them whatever springs into your mind at the moment.} \\
\midrule
5 & \multicolumn{5}{c}{You like to occasionally sing. Telepathically.} \\
\midrule
6 & \multicolumn{5}{c}{You love to absorb particularly dangerous memories.} \\
\midrule
\end{tabularx}
\end{minipage}\hfill
\begin{minipage}{0.48\textwidth}

\end{minipage}

\begin{minipage}{0.48\textwidth}
\subsubsection*{Farling Names}

Farling don’t have traditional names, identifying each other through psionic projections of certain thought patterns. Those that interact with members of other races will adopt an assumed name, running the gambit from common and boring to outlandish and absurd. Their name often reflects the culture they most interacted with when finding they needed a name that could be made with sounds.

They rarely place great importance in their name, occasionally changing it or accepting a new nickname without much thought, as they don’t particularly associate the name with themselves; to them, that’s merely what people call them.

Farling Names: Blue, Dustin, Jonny, Lulu, Newbie, Pattern, Purple, Rey, Sala, Stripes, Squid, Thomas, Uanqor, Quentin, Yara, Yldel, Wis.
\end{minipage}\hfill
\begin{minipage}{0.48\textwidth}
Completely Random

If it seems like these names are random, it is because they are. There is no pattern to Farling names beyond the nature of creatures naming them, sometimes being simply descriptive.
\end{minipage}

\subsection*{Traits}

As a farling, your character has the following traits.

Ability Score Increase. Your Intelligence score, Wisdom score, and Charisma score all increase by 1.

Age. A Farling ages by absorbing memories. Each time they permanently absorb a memory, they age roughly one day. A young Farling can absorb dozens of memories a day, aging rapidly. Once reaching adulthood (a process that often only takes a decade), the rate at which they can absorb memories begins to slow, they become more selective, aging far slower.

\begin{minipage}{0.48\textwidth}
Old Age

If a Farling can die of old age is unknown. They become far more selective about absorbing memories as they grow older, often losing their curious nature and becoming withdrawn from society after absorbing thousands of memories, ruminating on what they have experienced.
\end{minipage}\hfill
\begin{minipage}{0.48\textwidth}

\end{minipage}

Alignment. Farlings show little consistency in alignment, and are shaped by their circumstances and background. Their curious and impulsive nature often leads younger members of the race to exhibit chaotic tendencies.

Size. Most Farlings stand between 5 and 6 feet tall, with a slighter build than humans. Your size is Medium.

Aberration. Your creature type is Aberration, rather than humanoid.

Speed. Your base walking speed is 30 feet. You have a swimming speed of 20 feet.

Languages. You can speak, read, and write Common, Deep Speech, and one extra language of your choice.

Adaptive Intuition. You gain proficiency in Insight, and have advantage on Wisdom (Insight) checks to determine social cues or cultural customs.

Psionic Mind. You gain the Telepathy Discipline of the psion, though you can’t use the Alternate Effects unless you gain this Discipline from another source. If you gain this Discipline from another source, you gain 1 additional psi point instead.

Crystalized Memory. When you learn or experience something, you can fix that memory, making it a permanent memory you can always remember with crystal clarity. This memory can be anything that occurs within a 1 hour span (it can be as short as a fleeting sensation, or as long as 1 hour of events).

Once you do this, you can’t do so again until you finish a long rest. Absorbing these memories is how a Farling ages, and you age one day for each memory you absorb in this way. You can absorb a memory from a willing creature via your telepathy in this way.

Broad Horizons. You have resistance to psychic damage. Once per day, if take psychic damage from a critical hit, you can choose to absorb that damage as a memory, taking no damage.

Skill Absorption. Your rapid ability to learn new skills grants you proficiency in one artisan tool and one skill of your choice. At your GM’s discretion, if your game allows for the training in tools and skills, you may gain proficiency in them twice as fast using your Crystalized Memory feature daily.

\subsection*{Variant Farling}

At the discrection of your GM, the following is an intended feature of Farlings that can be added in place of Skill Absorption. This is a recommended feature to allow additional options, but may not be suited for all games due to interacterations with other custom rules at some tables:

Alien Mind. When you gain the Spellcasting or Psionics ability from any class or feature, you can change the associated ability score to your choice of Intelligence, Wisdom, or Charisma. Any time that class references the previous spell casting ability, you can replace it with your selected score (for example, if you are playing a Psion and change its associated ability score to Charisma, Empowered Psiosnics would add Charisma instead of Intelligence).

Alien Possibilities

This feature is very unique. It means that Farlings can play classes in new ways, and multiclass in ways that would not normally work, opening unique doors and possibilities that other races cannot share.

If combining these rules with ones that allow for the relocation of ability scores, one should be careful to not unlock an option that may be unexpectedly potent. Work with your GM when combining homebrew and variant rules.

\begin{itemize}
  \item Race: Farling
\end{itemize}

Fantasy Grounds Conversion Note

The base Farling and the Variant Farling are coded into FG as subraces of the Farling. When this race is selected at character creation the player will be prompted to select either 'Farling' or 'Variant Farling' as their subrace. Selecting Farling gives the character the Skill Absorption trait, while selecting Variant Farling gives the character the Alien Possibilities trait. This is the only difference between the subraces.

\section*{Templates}

The following are templates that can be placed over any race. These are not races, rather members of other races that have been defined by modifications that have been made to them. These templates can be treated as a modifier of a racial option. For example, one might refer to an elf using this template as an “augmented elf.” Though they superficially inherent the appearance of another race, all mechanical benefits of that race are replaced by the following properties.

Consequently, you qualify for any feat or other requirement that applies to the base selected race, but all other bonuses are replaced by the listed bonuses for the class.

\begin{minipage}{0.48\textwidth}
\subsubsection*{Augmented}

The augmented are creatures that have replaced parts of the body with mechanical and magical parts—some sort of an arcane cyborg. The origin of these parts is many and varied, as is their nature: anything from living magic to mechanical means is possible, based on the setting and character. These provide bonuses that replace or augment the default features of your race selection.

\begin{itemize}
  \item Augmented
\end{itemize}
\end{minipage}\hfill
\begin{minipage}{0.48\textwidth}
\subsubsection*{Warped}

The warped are members of races that have been deeply influenced by the powers from beyond, twisting their mind and body with mysterious power that has left them fundamentally changed.

\begin{itemize}
  \item Warped
\end{itemize}
\end{minipage}

Fantasy Grounds Conversion Note

Augmented and Warped function as full races as far as FG is concerned. Short of creating an individual subrace for every potential combination of Augmented or Warped, there is no way to apply one of these "templates" over another race. The simplest way to make use of these races is to apply them as the character's race upon creation. Then drag the size, languages, and any other needed Traits from the 'Other' tab of the original race.

\section*{Augmented}

% [Image Inserted Manually]

The augmented are creatures that have replaced parts of the body with mechanical and magical parts—some sort of an arcane cyborg. The origin of these parts is many and varied, as is their nature: anything from living magic to mechanical means is possible, based on the setting and character. These provide bonuses that replace or augment the default features of your race selection.

\subsection*{Intentionally Created}

Almost all augmented are intentionally created—not always by the intentions of the individual, but almost always by the intention of someone. This often leaves open the question of why, and for what purpose.

Many of them are created for war, some as bizare experiments, some to restore missing pieces, sometimes they had a choice in the process, and some will insist that they never asked for it.

\begin{minipage}{0.48\textwidth}
\subsubsection*{Augmented Quirks}

\begin{tabularx}{\textwidth}\toprule
{}XXXXX}
\midrule
d4 & \multicolumn{5}{c}{Quirks} \\
\midrule
1 & \multicolumn{5}{c}{You tinker with yourself and collect spare parts.} \\
\midrule
2 & \multicolumn{5}{c}{When your allies fail at things, you suggest what augments may help them succeed in the future.} \\
\midrule
3 & \multicolumn{5}{c}{You decorate your augmented parts with guady decorations.} \\
\midrule
4 & \multicolumn{5}{c}{You are afraid of getting wet, and terrified of large bodies of water.} \\
\midrule
\end{tabularx}
\end{minipage}\hfill
\begin{minipage}{0.48\textwidth}

\end{minipage}

\subsection*{Traits}

As an augmented, the following traits replace any traits that your race would normally grant you:

Modified Origin. Select an existing race option. You gain the creature type, size, and languages of that race.

Ability Score Increase. Your Constitution score increases by 1, and one ability score of your choice increases by 1.

Alignment. An augmented creature can be any alignment, but have a inclination toward lawful alignments.

Speed. Your base walking speed is 30 feet.

Inherited Abilities. You can keep the following elements of that race: any skill proficiencies you gained from it and any climbing, flying, or swimming speed you gained from it. Alternatively, you can forgo this to select two Augmented Abilities (below).

\subsection*{Augmented Abilities}

On top of your inherited abilities, you have been augmented with one of the following, granting you special new abilities:

\begin{itemize}
  \item Arcane Arm. Your arm is replaced by a mechanical or magical replacement. Your Strength score increases by 1, and it can serve as a natural weapon. It deals 1d6 bludgeoning damage on a hit, and counts as a simple weapon with the light property.
  \item Arcane Eye. Your eye (one or both) are replaced by mechanical or magical replacements. You can see in dim light within 120 feet of you as if it were bright light and in darkness as if it were dim light. You discern colors in that darkness as shades of gray, and gain proficiency in the Perception skill. You can cast the detect magic spell, and once you reach 5th level, the see invisible spell with this eye. Casting them in this way requires no material components. Once you cast either of these spells with this trait, you can’t cast that spell with it again until you finish a long rest. You can also cast these spells using any spell slots you have.
  \item Arcane Heart. Your heart (or equivalent organ) is replaced by a mechanical or magical replacement. Your Constitution score increases by 1, and whenever you roll Hit Dice, you gain temporary hit points equal to the number of Hit Dice expended.
  \item Arcane Mind. Your mind is enhanced by mechanical or magical minds. Your Intelligence score increases by 1, and you gain advantage on any Intelligence check to recall information you’ve seen before, solve mathematical equations, detect patterns, or break ciphers and codes.
\end{itemize}

\begin{itemize}
  \item Race: Augmented
\end{itemize}

Fantasy Grounds Conversion Note

Augmented presented unique coding challenges to make this dynamic racial template work within the confines of FG. Each of the four Augmented Abilities above is coded as a subrace, and one will be selected at character creation. If the Augmented character chooses to forgo their Inheritied Abilities Trait then simply drag the Traits from the additional two Augmented Abilities subraces to the character sheet.

\section*{Warped}

The warped are members of races that have been deeply influenced by the powers from beyond, twisting their mind and body with mysterious power that has left them fundamentally changed.

\subsection*{Twisted Origin}

Most often this comes from long exposure, sometimes only manifesting in creatures born under its influence, but there are those that have been warped later in life, generally by more direct interactions with the creatures that lurk beyond the veil of sanity.

Many of the warped fall to madness, unable to escape the whispering voices and strange visions, but ones that become adventurers have managed to learn to live with their condition, and consequently have minds of steel.

\begin{minipage}{0.48\textwidth}
\subsubsection*{Warped Quirks}

\begin{tabularx}{\textwidth}\toprule
{}XXXXX}
\midrule
d4 & \multicolumn{5}{c}{Quirks} \\
\midrule
1 & \multicolumn{5}{c}{You hide your warped nature under wrappings.} \\
\midrule
2 & \multicolumn{5}{c}{You like to show off your enhanced abilities.} \\
\midrule
3 & \multicolumn{5}{c}{Your eyes and hair are unnatural colors.} \\
\midrule
4 & \multicolumn{5}{c}{You mutter to yourself frequently.} \\
\midrule
\end{tabularx}
\end{minipage}\hfill
\begin{minipage}{0.48\textwidth}

\end{minipage}

\subsection*{Traits}

Warped Origin. Select an existing race option. You gain the creature type, size, and languages of that race.

Whispers of Madness. You have learned to understanding the whispering voices you hear. You learn Deep Speech.

Ability Score Increase. Your Constitution score increases by 1, and your Intelligence score increases by 2.

Aligmment. A warped creature can be any alignment, but have a inclination toward chaotic alignments.

Speed. Your base walking speed is 30 feet.

Inured of Madness. Things that cause madness in others seem relatively benign to you. You have resistance to psychic damage. You automatically pass any saves against madness.

Inner Voice. You have an inner voice that doesn’t belong to you, frequently muttering nonsense, but occasionally offering useful advice, letting you add 1d6 to any ability check you make. You can do this a number of times equal to your proficiency bonus, regaining all uses at the end of a long rest.

\subsection*{Warped Gift.}

Your condition grants you one special trait. Some may view it as a curse or burden, but you’ve adapted to using it to your advantage. Select one of the following options:

\begin{itemize}
  \item Warped Eyes. You see the world in a different way. You have advantage on Wisdom (Perception) checks against living creatures, and disadvantage on Wisdom (Perception) checks against non-living things (such as traps or details in your environment).
  \item Warped Limb. One of your limbs is abnormally long and flexible. Any weapon wielded in that hand gains the reach property.
  \item Warped Mind. Your mind is a foreign and alien place compared to other humanoids. You are immune to magic that allows other creatures to read your thoughts, determine whether you are lying, know your alignment, or know your creature type. Creatures can telepathically communicate with you only if you allow it.
  \item Warped Skin. Your skin is a rough mottled hide. You have resistance to acid damage.
  \item Warped Form. Your body is more malleable and flexible than a normal humanoid. You can move through spaces for a smaller creatures without squeezing and when you aren’t wearing armor, your AC is 12 + your Constitution modifier. You can use your natural armor to determine your AC if the armor you wear would leave you with a lower AC. A shield’s benefits apply as normal while you use your natural armor.
\end{itemize}

\begin{itemize}
  \item Race: Warped
\end{itemize}

Fantasy Grounds Conversion Note

Each of the Warped Gifts has been coded as a Warped subrace. The player will be promted to select a gift at character creation, and the appropirate traits will be added to the character sheet.

\section*{Feats}

\subsection*{Adept Poisoner}

You specialize in developing poisons, granting you proficiency with poisoner’s kits and the following benefits:

\begin{itemize}
  \item When you create a poison, you can specify a creature type that poison is specifically made for. That poison ignores resistance and immunity to poison damage for that creature.
  \item You can apply poison to a weapon or piece of ammunition as a bonus action, instead of an action.
  \item During a long rest, you can perform 1 hour of work using a poisoner’s kit to create a number of doses of potent poison equal to your proficiency bonus, which last until you apply them to a weapon or finish your next long rest. Once applied to a weapon or piece of ammunition, the poison retains its potency for 1 minute or until you hit with the weapon or ammunition. When a creature takes damage from the coated weapon or ammunition, that creature must succeed on a Constitution saving throw (DC equal to 8 + twice your proficiency bonus) or take 1d8 poison damage and become poisoned until the end of your next turn. On success, they take half as much damage and aren’t poisoned.
\end{itemize}

When you use this feature to create poison, you can expend 15 gp per dose of poison to create a more deadly poison. When you do so, creatures that fail their Constitution saving throw against it take an extra 1d8 poison damage.

\begin{itemize}
  \item Adept Poisoner
\end{itemize}

\subsection*{Expert Alchemist}

Prerequisite: Proficiency with alchemist’s supplies

Your long practice of alchemy has granted you the following benefits:

\begin{itemize}
  \item Increase your Intelligence or Wisdom score by 1, to a maximum of 20.
  \item When you make an Alchemy crafting check, you can take 10 as the result of the d20 (instead of rolling) without spending additional time.
  \item During a long rest, you can produce one Potion of Healing without material components. This is a temporary potion that loses its potency after 24 hours.
\end{itemize}

\begin{itemize}
  \item Expert Alchemist
\end{itemize}

\subsection*{Expert Blacksmith}

Prerequisite: Proficiency with smith’s tools

Your long practice of blacksmithing has granted you the following benefits:

\begin{itemize}
  \item Increase your Strength score by 1, to a maximum of 20.
  \item When you make an Blacksmithing crafting check, you can take 10 as the result of the d20 (instead of rolling) without spending additional time.
  \item The Maintenance \& Modifications options of Blacksmithing can be completed without any gold piece cost.
\end{itemize}

\begin{itemize}
  \item Expert Blacksmith
\end{itemize}

\subsection*{Expert Cook}

Prerequisite: Proficiency with cook’s utensils

Your long experience with cooking has granted you the following benefits:

\begin{itemize}
  \item Increase your Wisdom score by 1, to a maximum of 20.
  \item When you make a Cooking crafting check, you can take 10 as the result of the d20 (instead of rolling) without spending additional time.
  \item When you cook a uncommon or rarer cooking option, you can make a single uncommon snack on the side without additional ingredients or time. This snack lasts 24 hours before expiring if unconsumed.
\end{itemize}

\begin{itemize}
  \item Expert Cook
\end{itemize}

\subsection*{Expert Tinkerer}

Prerequisite: Proficiency with tinker's tools

Your long experience with tinkering has granted you the following benefits:

\begin{itemize}
  \item Increase your Intelligence score by 1, to a maximum of 20.
  \item When you make a Tinkering crafting check, you can take 10 as the result of the d20 (instead of rolling) without spending additional time.
  \item During a long rest, you can produce two pieces of Advanced Ammunition worth less than a total of 50 gp without material components beyond pieces of ammunition. This piece of ammunition is of temporary construction, and becomes a normal piece of ammunition after 24 hours.
\end{itemize}

\begin{itemize}
  \item Expert Tinkerer
\end{itemize}

\subsection*{Magical Researcher}

Prerequisite: Proficiency in the Arcana skill

You delve the secrets of magic. You can make scrolls of spells you don’t know, but each crafting check takes 8 hours (this can be completed incrementally). When you craft a scroll or make a magic item that reproduces a spell that deals damage, you can change the damage type of that spell it reproduces. Doing so raises the difficulty of crafting it by +2.

Additionally, you can cast a spell from a scroll even if it is not on your spell list. If it is higher level than you can normally cast, you have advantage on the spell casting check to cast it.

\begin{itemize}
  \item Magical Researcher
\end{itemize}

\begin{minipage}{0.48\textwidth}
\subsubsection*{Old Hand}

You have mastered the way of your craft in a way the young whippersnappers can only dream of. You can leverage your vast knowledge of the craft to use your Intelligence or Wisdom modifier in place of Strength or Dexterity in a branch of crafting that you have proficiency in the related tool of.

\begin{itemize}
  \item Old Hand
\end{itemize}
\end{minipage}\hfill
\begin{minipage}{0.48\textwidth}
Minor Feat

This feat primarily exists to be given for free to allow additional character concepts to excel at craftsmanship, and is provided by some backgrounds. It is an option that can be taken, but provides little mechanical power on its own.
\end{minipage}

\subsection*{Resourceful}

You’ve mastered every play that’ll give you an edge when it comes to using common items, and gain the following benefits:

\begin{itemize}
  \item You are proficient with improvised weapons.
  \item You can add your proficiency bonus to the saving throw DC against for common items used by you.
  \item When taking time to recover reusable items such as ball bearings or caltrops, no roll is required.
  \item Additionally, whenever you deal damage with alchemical acid, alchemical fire, caltrops, holy water, or oil, that damage increases by 1d4 at 5th level, and again at 11th and 17th levels.
\end{itemize}

\begin{itemize}
  \item Resourceful
\end{itemize}

\subsection*{School Specialist}

Prerequisite: The ability to cast at least one spell

When you gain this feat, choose one of the following schools of magic: Abjuration, Divination, Conjuration, Enchantment, Evocation, Illusion, Necromancy, or Transmutation.

You learn a spell from the selected school. It must either come from the spell list of a class you have the spell casting feature of, or be a 1st- or 2nd-level spell. You can only select a spell of a level which you can cast. You can change your selected spell when you gain a level, selecting a new spell that fits the criteria.

Additionally, you can cast cantrips of the selected school as a bonus action. When you cast them as a bonus action, the cantrip is cast as if you were a 1st level caster. You can do this a number of times equal to your proficiency bonus, regaining all uses on a long rest.

\begin{itemize}
  \item School Specialist
\end{itemize}

% [Image Inserted Manually]

\subsection*{Soul Investor}

You have the special gift to invest your soul into magic items. This becomes a new way to enchant permanent magic items for you, allowing you to replace an essence in a crafting check with part of your soul. When you do so, you can replace one essence of a varying rarity (depending on your level, as per the table below) with your soul, and gain advantage on crafting checks to make that item, the item becomes attunement if it was not already, and becomes attuned to you on completion. If you end your attunement to the item, the item becomes mundane, losing all magical properties. This item doesn’t count against your maximum number of attuned items.

You can have one item infused this way at a time. You can replace two essences of a tier of quality lower than the maximum.

\begin{tabularx}{\textwidth}\toprule
{}XX}
\midrule
Character Level & \multicolumn{2}{c}{Essence Rarity} \\
\midrule
1–4 & \multicolumn{2}{c}{Common} \\
\midrule
5–9 & \multicolumn{2}{c}{Uncommon} \\
\midrule
10–14 & \multicolumn{2}{c}{Rare} \\
\midrule
15–19 & \multicolumn{2}{c}{Very Rare} \\
\midrule
20 & \multicolumn{2}{c}{Legendary} \\
\midrule
\end{tabularx}

\begin{itemize}
  \item Soul Investor
\end{itemize}

\subsection*{Tool Expert}

You have honed your proficiency with particular tools, granting you the following benefits:

\begin{itemize}
  \item Increase the chosen ability score by 1, to a maximum of 20.
  \item You gain proficiency with one tool of your choice.
  \item Choose one tool in which you have proficiency. You gain expertise with that tool, which means your proficiency bonus is doubled for any ability check you make with it.
\end{itemize}

\begin{itemize}
  \item Tool Expert
\end{itemize}

\subsection*{Ultimate Improviser}

You find new ways to do anything. You gain the following benefits:

\begin{itemize}
  \item Increase the chosen ability score by 1, to a maximum of 20.
  \item You can improvise a replacement for any material that costs a number of gold pieces less than or equal to your character level. This ability doesn’t create matter out of nothing, it simply assumes you can create the item with materials at hand or adapt a simple components to your needs. If no materials at all are available, the item can’t be created. A consumable item created with an improvised ingredient lasts only 24 hours before breaking.
  \item You can apply half your proficiency bonus (rounded up) to any check you make with tools.
\end{itemize}

\begin{itemize}
  \item Ultimate Improviser
\end{itemize}

\begin{minipage}{0.48\textwidth}
\subsubsection*{Wand Slinger}

If you have a spell save DC higher than a wand you are attuned to, you can use your spell save DC instead. If you expend the last charge of a wand, you do not need to make a roll to see if it is destroyed.

When you cast a spell with an item, you can make a single weapon attack with a weapon held in your other hand as a bonus action.
\end{minipage}\hfill
\begin{minipage}{0.48\textwidth}
Interactions with Other Features

This would interact with the Blast Stick common wand as that counts as a simple weapon. This would partially interact with a Blasting Rod: while you could not use a Blasting Rod as bonus action with this (as it is not a weapon), you could use a Blast Stick as a bonus action after using the Blasting Rod to cast a spell.
\end{minipage}

\section*{Class Feats}

\section*{Inventor Feats}

\subsection*{Innovator's Upgrade}

Prerequisite: Inventor

You’ve honed your mind into a nonstop analytical machine. You can select an additional upgrade from your subclass list. The upgrade must be from the Unrestricted list. This upgrade doesn’t count against your class upgrade total.

If a feature allows you to exchange upgrades, you can exchange upgrades selected with this feat, but they must be selected from the same (or lower) level requirement category as the originally selected upgrade.

\begin{itemize}
  \item Innovator's Upgrade
\end{itemize}

\subsection*{Mental Adaptability}

Prerequisite: Inventor

You’ve found ways to think outside the box, making connections that other—lesser—minds can never seem to understand.

When you have to make an Intelligence, Wisdom or Charisma saving throw, you can roll all three and pick the highest result. Once you do this, you can’t do it again until you finish a long rest.

You can select an additional upgrade that is not from your subclass list, so long as you can apply it to something in your possession. This upgrade can’t be a level-restricted upgrade.

\begin{itemize}
  \item Mental Adaptability
\end{itemize}

\subsection*{Rune Expert}

Prerequisite: Inventor

You can select a single Runic Mark from the Inventor Runesmith subclass. You learn and can mark this rune as per the Runic Mark feature of the Inventor class.

\begin{itemize}
  \item Rune Expert
\end{itemize}

\section*{Psion Feats}

\subsection*{Inner Power}

Prerequisite: At least 1 psi or ki point

Your body is powered by the inner powers that course through it. Your maximum psi or ki points (if you have both, your choice of which) increases by 1. At the end of your turn after expending one or more psi or ki points during your turn (not counting temporary or free points, such as those from Psionic Mastery), you heal for a number of hit points equal to the psi or ki points spent.

\begin{itemize}
  \item Inner Power
\end{itemize}

\subsection*{Psionic Adept}

You develop a minor grasp of psionic power, either uncovering an innate potential within yourself, through contact with a psionic source, or through training. You gain one of the psionic disciplines of the Psion class, gaining the attached psionic feature and psionic power, but you do not gain use of any associated spells when you gain a Discipline from this feat. You gain 1 psi point that you can use to empower the Discipline. You regain use of this Psi Point when you finish a short or long rest.

\begin{itemize}
  \item Psionic Adept
\end{itemize}

\subsection*{Psionic Mind}

Prerequisite: The ability to use at least one Psionic Discipline

You tap deeper into your psionic potential drawing out a new talent. You can select a psionic talent from the Psion class psionic talent list. You can’t select a talent you already know, or one that requires a level restriction, even if you are already of that level.

\begin{itemize}
  \item Psionic Mind
\end{itemize}

\subsection*{War Psion}

Prerequisite: The ability to use at least one Psionic Discipline

You have mastered utilizing psionics in the midst of combat, learning techniques that grant you the following benefits:

\begin{itemize}
  \item You have advantage on Constitution saving throws that you make to maintain your concentration on a psionic ability when you take damage.
  \item You can perform the somatic components of psionic abilities even when you have weapons or a shield in one or both hands.
  \item When a hostile creature’s movement provokes an opportunity attack from you, you can use your reaction to use a psionic disciplines power or spell targeting the creature, rather than making an opportunity attack. The spell must have a casting time of 1 action and must target only that creature.
\end{itemize}

\begin{itemize}
  \item War Psion
\end{itemize}

\section*{Racial Feats}

\section*{Ironwrought Feats}

\subsection*{Upgraded}

Prerequisite: Ironwrought

You can apply one unrestricted Golemsmith or Warsmith upgrade to yourself (as if you were the golem or the warplate). This can’t be an upgrade that increases your ability scores. Any ability the does the same thing as a feature from Modular Design doesn’t stack. (For example, you can’t take an upgrade that increases your AC by +1 if you already have Integrated Armor.)

\begin{itemize}
  \item Upgraded
\end{itemize}

\subsection*{Specialized}

Prerequisite: Ironwrought

You have additional specialized systems integrated into your form. You can selected two additional options from the Modular Design feature of your race.

\begin{itemize}
  \item Specialized
\end{itemize}

\subsection*{Subsystem}

Prerequisite: Ironwrought

You gain the ability to divide your attention into a subroutine that can operate somewhat independently from your primary mind. You gain the following benefits from it:

\begin{itemize}
  \item You can use this subsystem to maintain concentration on a cantrip. Effects that would cause you to make a Constitution saving throw to maintain your concentration apply to it, and it makes its save with a +0 bonus. It can maintain concentration on only one cantrip at a time.
  \item If the subsystem is not concentrating on a spell, you can use it to make a Intelligence or Wisdom check (no action required) once per turn on your turn.
  \item Each time you fail at the same specific task during your turn (saving against the same spell effect, attacking the same target, searching for a specific hidden creature), if you attempt the exact same task on your next turn, you can roll an additional die, taking the higher die. All additional dice are reset when you succeed the task.
\end{itemize}

\begin{itemize}
  \item Subsystem
\end{itemize}

\section*{Farling Feats}

\begin{minipage}{0.48\textwidth}
\subsubsection*{Alien Weapon Training}

You gain the ability to use a weapon, and the ability to use it in unique ways. You gain proficiency with one weapon of your choice, and when you gain proficiency with it you gain the ability to use it an unorthodox manner other creatures may find strange. You can add one of the following properties to that weapon that it doesn’t already have when you use it: heavy (if it doesn’t have the light property), light (if it does have heavy property), finesse, versatile (if it doesn’t have the two-handed property, increasing the size of the damage die by one step), or thrown (10/30). The weapon and property added are selected when you take the feat, and can’t be changed.

\begin{itemize}
  \item Alien Weapon Training
\end{itemize}
\end{minipage}\hfill
\begin{minipage}{0.48\textwidth}
Alien Weapon Retraining

The inability to change the property and weapon are to limit potential problems, but if a Farling wants to change weapons or properties for a reasonable reason (such as boredom— Farlings can be fickle) I would generally allow it, probably as a downtime activity.
\end{minipage}

\subsection*{Aquatic Adaptation}

Prerequisite: Farling

Your aquatic adaptions are more extensive. You gain the following traits:

\begin{itemize}
  \item Increase your Constitution score by 1, to a maximum of 20.
  \item You can breathe under water.
  \item You gain a swimming speed of 40 feet.
  \item You gain resistance to your choice of acid or cold damage.
\end{itemize}

\begin{itemize}
  \item Aquatic Adaptation
\end{itemize}

\subsection*{Flexible Form}

Prerequisite: Farling

You find the your material body isn’t quite as set in stone as other mortals, being more of a whim of expression. You can minorly change your appearance (coloration, hair length, superficial features) or adapt yourself to new environments, gaining resistance to one type of damage of your choice from acid, cold, fire, lightning, or poison during a long rest. You lose any previous resistance from this feature when you adapt yourself to a new environment.

Additionally, you can adapt your self to a greater extent, gaining the ability to cast alter self without expending a spell slot. Once you do so, you must expend a spell slot to cast it again until you finish a short or long rest.

\begin{itemize}
  \item Flexible Form
\end{itemize}

\subsection*{Spell Library}

Prerequisite: Farling, any class with the Spellcasting or Pact Magic feature

Your ability to perfectly memorize that which you experience enhances your ability to grasp and memorize spells. The number of spells you know or can prepare for one class you have levels in (your choice if you have multiple) increases by your proficiency bonus.

\begin{itemize}
  \item Spell Library
\end{itemize}

\chapter{Chapter 4: Spells}

\section*{Spells - A}

\begin{minipage}{0.48\textwidth}
\subsubsection*{Acid Rain}

5th-level conjuration

Classes: Druid, Occultist, Wizard 
Casting Time: 1 action 
Range: 300 feet 
Components: V, S 
Duration: Concentration, up to 1 minute

Acid rain begins falling within a 40-foot-radius, 60-foot-high cylinder centered on a point you choose within range. When a creature moves into the spell's area for the first time on a turn or starts its turn there, the creature must succeed on a Dexterity saving throw or take 6d4 acid damage, and become covered in acid. On a successful save, a creature takes half the initial damage and is not covered in acid.

A creature takes 3d4 acid damage if it ends its turn while covered with acid. The target or a creature within 5 feet of it can end this damage by using its action to clear away the acid.

\begin{itemize}
  \item Spell: Acid Rain
\end{itemize}

\subsubsection*{Aero Barrage}

4th-level transmutation

Classes: Sorcerer, Wizard 
Casting Time: 1 action 
Range: 120 feet 
Components: V, S 
Duration: Instantaneous

You create four lances of rapidly spinning condensed wind and hurl them at targets within range. You can hurl them at one target or several.

Make a ranged spell attack for each lance. On a hit, the target takes 2d8 slashing damage and is knocked 10 feet backwards.

At Higher Levels. When you cast this spell using a spell slot of 5th level or higher, you create one additional lance for each slot level above 4th.

\begin{itemize}
  \item Spell: Aero Barrage
\end{itemize}

\subsubsection*{Aether Lance}

3rd-level evocation

Classes: Sorcerer, Wizard 
Casting Time: 1 action 
Range: Self (30-foot line) 
Components: V, S 
Duration: Instantaneous

You gather raw aether in your hand and expel it in a lance of power forming a line 30 foot long and 5 foot wide. Each creature in a line takes 8d4 force damage.

At Higher Levels. When you cast this spell using a spell slot of 4th level or higher, the damage increases by 1d4 for each slot level above 3rd.

\begin{itemize}
  \item Spell: Aether Lance
\end{itemize}

\subsubsection*{Aether Storm}

5th-level evocation

Classes: Sorcerer, Wizard 
Casting Time: 1 action 
Range: 120 feet 
Components: V, S 
Duration: Concentration, up to 1 minute

You conjure a storm of aether erupting from a point of your choice within range. The aether storm fills a 10-foot-radius, 40-foot-high cylinder. When the storm appears, each creature within its area takes 8d4 force damage.

A creature takes 1d4 force damage for each 5 feet they move through the storm, and if a creature ends their turn within the aether storm, they take 8d4 force damage. On your turn, you can move the storm 10 feet in any direction as a bonus action.

At Higher Levels. When you cast this spell using a spell slot of 6th level or higher, the damage a creature takes from the storm appearing by ending their turn in it increases by 1d4 for each slot level above 5th.

\begin{itemize}
  \item Spell: Aether Storm
\end{itemize}

\subsubsection*{Alacrity}

2nd-level transmutation

Classes: Bard, Occultist, Ranger, Sorcerer, Wizard 
Casting Time: 1 bonus action 
Range: Self 
Components: V, S 
Duration: 1 round

Until the start of your next turn, your speed is doubled, you gain a +2 bonus to AC, you have advantage on Dexterity saving throws, and you gain an additional action. That action can be used only to take the Attack (one weapon attack only), Dash, Disengage, Hide, or Use an Object action.

If you are under the effect of haste, you gain no benefit from this spell.

\begin{itemize}
  \item Spell: Alacrity
\end{itemize}

\subsubsection*{Animate Object}

2nd-level transmutation

Classes: Bard, Inventor, Occultist, Sorcerer, Wizard 
Casting Time: 1 action 
Range: 60 feet 
Components: V, S 
Duration: Concentration, up to 1 minute

You bring a Tiny object to life. Its Constitution is 10 and its Intelligence and Wisdom are 3, and its Charisma is 1. Its speed is 30 feet; if the object lacks legs or other appendages it can use for locomotion, it instead has a flying speed of 30 feet and can hover. The object has the following stats: HP: 20, AC: 18, Str: 4, Dex: 18. The object has an attack modifier equal to your spell attack modifier. If the object is not a weapon, it deals 1d4 + your Spellcasting modifier damage on hit. Select from bludgeoning, piercing, or slashing damage based on the nature of the item. If the object is a weapon, it deals the weapon's damage dice + your Spellcasting modifier of the weapon's damage type. The spell can only animate one handed weapons without the special modifier this way.

As a bonus action, you can mentally command the animated object as long as it is within 60 feet of you. You decide what action the creature will take and where it will move during its next turn, or you can issue a general command, such as to guard a particular chamber or corridor. If you issue no commands, the creature only defends itself against hostile creatures. Once given an order, the creature continues to follow it until its task is complete.

If the object is securely attached to a surface or a larger object, such as a chain bolted to a wall, its speed is 0. It has blindsight with a radius of 30 feet and is blind beyond that distance. When the animated object drops to 0 hit points, it reverts to its original object form, and any remaining damage carries over to its original object form.

\begin{itemize}
  \item Spell: Animate Object
\end{itemize}
\end{minipage}\hfill
\begin{minipage}{0.48\textwidth}
\subsubsection*{Arcane Ablation}

1st-level transmutation

Classes: Inventor 
Casting Time: 1 action 
Range: Touch 
Components: V, S 
Duration: 1 hour

You touch a piece of worn armor or clothing and imbue it with magic. The creature wearing this imbued item gains 4 temporary hit points. When these temporary hit points are exhausted, at the start of the creature’s next turn it will gain hit points equal to 1 hit point less than the previous amount gained from this spell (for example, from 4 to 3), until no temporary hit points would be gained and the spell ends. Temporary hit points from this spell are lost when this spell ends.

\begin{itemize}
  \item Spell: Arcane Ablation
\end{itemize}

At Higher Levels. The initial temporary hit points increases by 1 for each slot level above 1st.

\subsubsection*{Arcane Infusion}

2nd-level transmutation

Classes: Inventor 
Casting Time: 1 minute 
Range: Self 
Components: V, S, M (spare parts that could form the upgrade selected worth at least 1 sp) 
Duration: 1 hour

You use arcane power to briefly bring to power or modify your inventions. For the duration, you gain the effects of one unrestricted upgrade. All normal prerequisite apply (including subclass and level requirements). The creation is magical, held together and formed of magic and spare parts, taking the form of the upgrade or empowering an existing upgrade with temporary new features. Casting this spell again ends the effects of any previous castings of this spell.

At Higher Levels. When you cast this spell with a 3rd-level spell slot or higher, you can infuse the effects of an upgrade that requires 5th level or higher. When you cast this spell using a spell slot of 4th level or higher, you can infuse the effects of an upgrade that requires 9th level or higher. When you cast this spell with a 5th-level spell slot or higher, you can infuse the effects of an upgrade that requires 11th level or higher.

\begin{itemize}
  \item Spell: Arcane Infusion
\end{itemize}

\subsubsection*{Arcane Weapon}

1st-level transmutation

Classes: Inventor 
Casting Time: 1 bonus action 
Range: Touch 
Components: V, S 
Duration: 1 hour

You touch a weapon and imbue it with magic. For the duration the weapon counts as a magical weapon and any damage dealt by it is Force damage. When casting this on a weapon with the ammunition property, it no longer consumes ammunition when fired, and doesn’t need to be reloaded.

At Higher Levels. When you cast this spell using a spell slot of 3rd or 4th level, the duration becomes 8 hours. When you use a spell slot of 5th level or higher, the duration becomes 24 hours.

\begin{itemize}
  \item Spell: Arcane Weapon
\end{itemize}

\subsubsection*{Arctic Breath}

1st-level conjuration

Classes: Druid, Sorcerer, Wizard 
Casting Time: 1 action 
Range: Self (30-foot line) 
Components: V, S 
Duration: Instantaneous

A line of freezing arctic wind 30 feet long and 5 feet wide blasts out from you in a direction you choose. Each creature in the line must make a Dexterity saving throw. On a failed save, a creature takes 2d8 cold damage and their speed is reduced by 10 feet until the end of their next turn. On a successful save, a creature takes half as much damage and isn’t slowed.

At Higher Levels. When you cast this spell using a spell slot of 2nd level or higher, the damage increases by 1d8 for each slot level above 1st.

\begin{itemize}
  \item Spell: Arctic Breath
\end{itemize}

\subsubsection*{Awaken Rope}

1st-level transmutation

Classes: Bard, Inventor, Occultist, Ranger, Wizard 
Casting Time: 1 action 
Range: Touch 
Components: V, S, M (10 to 60 feet of cord or rope, worth at least 1 cp) 
Duration: Instantaneous

As an action, you can touch a rope 10 to 60 feet long and issue a single command to it, selecting from the following options:

\begin{itemize}
  \item Bind. The rope attempts to bind a creature of your choice within 20 feet of you. The creature must make a Dexterity saving throw or become restrained until it is freed. A creature can use its action to make a DC 10 Strength check, freeing itself or another creature within its reach on a success. Dealing 5 slashing damage to the rope (AC 10) also frees the creature without harming it, ending the effect and destroying the rope.
  \item Fasten. The rope flies up 60 feet and ties one end to an object or surface that a rope could be tied to, before becoming inanimate again, hanging from the object.
  \item Grab. The rope lashes out grabs one Small or smaller object that is not being worn by a creature within a range equal to the length of the rope and pulls that object back to your hand. If that object is being carried by a creature, it must make a Strength saving throw. On success, it retains the object, and on failure the object is pulled from the creature.
\end{itemize}

At Higher Levels. When you cast this spell using a spell slot of 2nd level or higher, you can target a chain instead of a rope. It has the same available actions, but it has a DC 15, an AC of 15, and resistance to slashing damage when taking the Bind action. When cast with a spell slot of 3rd level or higher targeting a rope, that rope is magically imbued for 1 minute, gaining an DC of 15, an AC 20, and 20 hit points.

\begin{itemize}
  \item Spell: Awaken Rope
\end{itemize}
\end{minipage}

\section*{Spells - B}

\begin{minipage}{0.48\textwidth}
\subsubsection*{Bad Blood}

1st-level necromancy

Classes: Druid, Occultist, Warlock, Wizard 
Casting Time: 1 action 
Range: 60 feet 
Components: V, S, M (a piece of rotten meat) 
Duration: Concentration, up to 1 minute

Targeting a creature you can see within range, you attempt to corrupt its blood. Creatures without blood are immune to this effect. The target must make a Constitution saving throw. On failure, they take 1d12 poison damage and become poisoned for the duration.

At the end of each of its turns, the target can make another Constitution saving throw. On a success, the spell ends on the target, on failure; they take an extra 1d4 poison as the poison continues to ravage them.

At Higher Levels. When you cast this spell using a spell slot of 2nd level or higher, you can target one additional target for each slot level above 2nd. The targets must be within 30 feet of each other when you target them.

\begin{itemize}
  \item Spell: Bad Blood
\end{itemize}

\subsubsection*{Beam of Annihilation}

6th-level evocation

Classes: Sorcerer, Wizard 
Casting Time: 1 action 
Range: Self (60-foot line) 
Components: S 
Duration: Concentration, up to 3 rounds

You unleash a beam of pure energy, selecting cold, fire, force, or lightning energy when you cast this spell and blasting it outward in a line that is 60 feet long and 10 feet wide that persists until the start of your next turn. Any creature that starts their turn in this beam must make a Dexterity saving throw. On a failed save they take 8d8 damage of the beam's energy type, or taking half as much on a successful save.

While you are concentrating on this spell, your speed is 0. At the start of each of your turns, you can use your action to maintain the beam or redirect it, rotating it up to 90 degrees in any direction. Any creature the beam passes through while rotating (if the beam passes completely through them and they will not start their turn inside of it) must make a Dexterity saving throw or, take 4d8 damage of the beams energy type on a failed save, and taking no damage on a successful save.

If you do not use your action maintain or redirect it, the spell ends early.

\begin{itemize}
  \item Spell: Beam of Annihilation
\end{itemize}

\subsubsection*{Befuddling Curse}

1st-level enchantment

Classes: Occultist 
Casting Time: 1 action 
Range: 60 feet 
Components: V, S, M (something from the target creature (such as blood, hair, or scales) which the spell consumes) 
Duration: Concentration, up to 1 minute

You befuddle a creature's mind, swapping the position of two things it can see that are of the same size and category (for example, two Medium creatures or two Gargantuan buildings). The target creature must make a Wisdom saving throw. On failure, it is unaware the two things have been swapped.

Each time the creature interacts with, attacks, or is attacked by a swapped targets, it can repeat its saving throw against the effect.

\begin{itemize}
  \item Spell: Befuddling Curse
\end{itemize}
\end{minipage}\hfill
\begin{minipage}{0.48\textwidth}
\subsubsection*{Binding Curse}

1st-level enchantment

Classes: Occultist 
Casting Time: 1 action 
Range: 60 feet 
Components: V, S, M (something from the target creature (such as blood, hair, or scales) which the spell consumes) 
Duration: Concentration, up to 1 minute

You bind a creature to a point within 5 feet of it, causing a glowing chains of light to connect it to that point. For the duration of the spell, if the creature attempts to move away from that point, the must make a Wisdom saving throw, or be unable to move more than 5 feet away from from that point until the start of their next turn.

If a creature starts its turn more than 10 feet from the binding point, they must make a Strength saving throw or be dragged 5 feet toward the binding point.

\begin{itemize}
  \item Spell: Binding Curse
\end{itemize}

\subsubsection*{Bond Item}

1st-level conjuration

Classes: Inventor 
Casting Time: 1 minute 
Range: Touch 
Components: V, S 
Duration: 8 hours

You touch an item weighing no more than 100 pounds and form a link between you and it. Until the spell ends, you can recall it to your hand as a bonus action.

If another creature is holding or wearing the item when you try to recall it, they must make a Charisma saving throw to retain possession of the item, and if they succeed, the spell fails. They make this save with advantage if they have had possession of the item for more than 1 minute.

\begin{itemize}
  \item Spell: Bond Item
\end{itemize}

\subsubsection*{Burn}

Transmutation cantrip

Classes: Druid, Occultist, Sorcerer, Warlock 
Casting Time: 1 action 
Range: Touch 
Components: V, S 
Duration: Instantaneous

You ignite a brilliant flame around your hand that sears anything you touch. Make a melee spell attack against the a creature or object within range. On hit, the target takes 1d12 fire damage.

The spell's damage increases by 1d12 when you reach 5th level (2d12), 11th level (3d12), and 17th level (4d12).

\begin{itemize}
  \item Spell: Burn
\end{itemize}
\end{minipage}

\section*{Spells - C}

\begin{minipage}{0.48\textwidth}
\subsubsection*{Cold Snap}

2nd-level evocation

Classes: Sorcerer, Wizard 
Casting Time: 1 action 
Range: 90 ft (5 ft radius) 
Components: S 
Duration: Instantaneous

With a snap of your fingers a swirling burst of freezing wind erupts at a point you choose within range. Each creature in a 5-foot-radius sphere centered on that point must make a Constitution saving throw. On a failed save, a creature takes 3d8 cold damage and becomes stuck in the ice, reducing their speed by 10 feet until the start of your next turn. On a success, the target takes half as much damage and is not stuck in ice.

The ground in the area is covered with slick ice and snow, making it difficult terrain until the start of your next turn.

At Higher Levels. When you cast this spell using a spell slot of 3rd level or higher, the damage increases by 1d8 for each slot level above 2nd.

\begin{itemize}
  \item Spell: Cold Snap
\end{itemize}

\subsubsection*{Compelled Query}

1st-level psionic

Casting Time: 1 action 
Range: 60 feet 
Components: S 
Duration: Instantaneous

You focus your telepathic powers on a creature and ask it a simple question. It must make an Intelligence saving throw, or it conjures a short mental thought or image that answers your question to the best of its ability that you can perceive telepathically. A creature gains a +5 to the saving throw against this spell for each time it has been used on them in the past 24 hours.

\begin{itemize}
  \item Spell: Compelled Query
\end{itemize}

\subsubsection*{Crackle}

2nd-level evocation

Classes: Druid, Occultist, Sorcerer, Warlock, Wizard 
Casting Time: 1 action 
Range: 60 feet Components: V, S 
Duration: Instantaneous

You create three arcs of lightning striking targets in range. You can direct them at one target or several. Make a ranged spell attack for each arc. On a hit, the target takes 1d12 lightning damage. If three or more arcs hit a single target, they must make a Constitution saving throw or become shocked, stunning them until the start of their next turn.

At Higher Levels. When you cast this spell using a spell slot of 3rd level or higher, you create one additional arc for each slot level above 2nd.

\begin{itemize}
  \item Spell: Crackle
\end{itemize}

\subsubsection*{Crashing Wave}

1st-level conjuration

Classes: Druid, Sorcerer, Wizard 
Casting Time: 1 action 
Range: Self (15-foot cone) 
Components: V, S 
Duration: Instantaneous

A wave of water sweeps out from you. Each creature in a 15-foot cone must make a Strength saving throw. On a failed save, a creature takes 2d6 bludgeoning damage and is knocked 10 feet away from you. If a creature is knocked into a wall, another creature, or fails by 5 or more, it is additionally knocked prone. On a successful save, the creature takes half as much damage and is not knocked back. If there is a source of water of at least 5 cubic feet within 5 feet of you when you cast the spell, you can displace that water, increasing the range of the spell to a 25-foot cone.

At Higher Levels. When you cast this spell using a spell lot of 2nd level or higher, the damage increases by 1d6 for each level above 1st.

\begin{itemize}
  \item Spell: Crashing Wave
\end{itemize}
\end{minipage}\hfill
\begin{minipage}{0.48\textwidth}
\subsubsection*{Crippling Agony}

1st-level necromancy

Classes: Occultist 
Casting Time: 1 action 
Range: 60 feet 
Components: V, S, M (a joint bone) 
Duration: Concentration, up to 1 minute

You can inflict crippling agony on a foe. Choose one creature that you can see within range to make a Constitution saving throw. If the target fails, it becomes crippled with horrific pain. Whenever the creation moves more than half of its speed or takes an action, the crippling pain causes it to take 1d6 necrotic damage.

It can repeat the saving throw at the end of each of its turns, the target can make a Constitution saving throw. On a success, the spell ends.

\begin{itemize}
  \item Spell: Crippling Agony
\end{itemize}

\subsubsection*{Cruel Puppetry}

3rd-level necromancy (ritual)

Classes: Occultist 
Casting Time: 1 action 
Range: 120 feet 
Components: V, S, M (a small humanoid doll worth at least 5 gp and something from the target creature (such as blood, hair, or scales) both of which the spell consumes) 
Duration: Concentration, up to 1 minute

You attempt to bind a creatures soul to a doll, linking the creature to the doll in a sympathetic link. The target must make a Charisma saving throw. On failure, the creature becomes bound to the doll. On a successful save, the creature is not bound and the spell ends.

As part of casting the spell when the creature fails the save, and on subsequent turns using your action until the spell ends, you can perform one of the following actions:

\begin{itemize}
  \item Hold the doll still, causing the creature to be Restrained until start of your next turn.
  \item Force the doll to move, causing the creature to move 15 feet in a direction of your choice that it can move.
  \item Stab the doll, causing the creature take 4d6 piercing damage.
  \item Rip the doll in half, ending the spell, destroying the doll, and dealing 4d12 necrotic damage to the creature.
  \item Each time after the first you use an action to manipulate the doll, after the effect takes place, the creature can repeat the Charisma with disadvantage, ending the effect on a successful save.
\end{itemize}

Once a creature has been targeted by this spell, they cannot be targeted again for 24 hours.

At Higher Levels. When cast with a 5th level spell slot or above, the range of the spell becomes unlimited, as long as the target is on the same plane as the caster.

\begin{itemize}
  \item Spell: Cruel Puppetry
\end{itemize}

\subsubsection*{Crushing Singularity}

3rd-level transmutation

Classes: Wizard 
Casting Time: 1 action 
Range: 60 feet 
Components: V, S 
Duration: 1 round

You create an overwhelming gravitational singularity at a point within range that lasts until the start of your next turn. When you cast this spell, any creature within 15 feet of the point must make a Strength saving throw. Creatures that fail their saving throw are moved to the closest available space adjacent to the singularity and take 3d6 bludgeoning damage, and an extra 1d6 bludgeoning damage for each other creature that fails the saving throw, up to a maximum of 6d6 bludgeoning damage.

While within 15 feet of the singularity, moving away from the singularity requires twice as much movement. If a creature ends its turn within 15 feet of the singularity, it must make a Strength saving throw. On failure, they take 2d6 bludgeoning damage are dragged back to the closest available spot to the center of the singularity.

\begin{itemize}
  \item Spell: Crushing Singularity
\end{itemize}
\end{minipage}

\section*{Spells - D}

\begin{minipage}{0.48\textwidth}
\subsubsection*{Dancing Wave}

2nd-level conjuration

Classes: Druid, Occultist, Sorcerer, Wizard 
Casting Time: 1 Action 
Range: 30 feet 
Components: V, S 
Duration: Concentration, up to 1 minute

You summon a surging mass of water into existence at a point on the ground within range. The mass of water remains cohesive filling a 5-foot radius, though only rises 3 feet from the ground. The area is difficult terrain for any creature without a swimming speed.

For the duration of the spell, as a bonus action you can move the wave of water up to 30 feet along a surface in any direction. The first time the wave enters any creature's space during a your turn, they must make a Strength saving throw or take 1d6 bludgeoning damage and be knocked prone. A creature automatically fails the saving throw against this spell if they are prone.

\begin{itemize}
  \item Spell: Dancing Wave
\end{itemize}

\subsubsection*{Delve Mind}

3rd-level psionic

Casting Time: 1 action 
Range: 60 feet 
Components: S 
Duration: Concentration, up to 1 minute

You delve into a creature's mind, forcing it to make an Intelligence saving throw. On a failure, for the duration or until you end the spell you gain access to its memories from the past 12 hours, and are able to recall things it remembers as if they are your own memories, but these memories contain only things the target creature remembers.

At Higher Levels. When you cast this spell using a spell slot of 4th level or higher, you can delve an additional 12 hours further back in the creature's memories for each slot level above 3rd.

\begin{itemize}
  \item Spell: Delve Mind
\end{itemize}

\subsubsection*{Devouring Darkness}

5th-level necromancy

Classes: Occultist, Warlock, Wizard 
Casting Time: 1 action 
Range: Self (20-foot radius) 
Components: V, S 
Duration: Instantaneous

Dark tendrils burst out from you in all directions. Creatures of your choice within must make a Constitution saving throw. On failure, they take 6d8 necrotic damage, and you can move them in a straight line to within 5 feet of you if there is an empty space they can be pulled to. On success, they take half as much damage and are not moved.

You regain hit points equal to one quarter (rounded down) of the necrotic damage taken by all targets effected by the spell.

At Higher Levels. When you cast this spell using a spell slot of 6th level or higher, the damage increases by 1d8 for each slot level above 5th.

\begin{itemize}
  \item Spell: Devouring Darkness
\end{itemize}

\subsubsection*{Dimension Cutter}

4th-level conjuration

Classes: Ranger 
Casting Time: 1 action 
Range: Self (15-foot cone) 
Components: V, M (a melee weapon you are proficient with worth at least 1 cp) 
Duration: Instantaneous

You flourish a weapon weapon you are proficient with used in the casting and sweep through the air, slashing apart the dimensional space. Each creature in a 15-foot cone takes 6d6 force damage.

\begin{itemize}
  \item Spell: Dimension Cutter
\end{itemize}
\end{minipage}\hfill
\begin{minipage}{0.48\textwidth}
\subsubsection*{Disorient}

2nd-level illusion

Classes: Bard, Occultist, Wizard 
Casting Time: 1 action 
Range: 60 feet 
Components: V, S, M (a mobius strip) 
Duration: 1 minute

Targeting a creature with you can see, you flip their perception of reality. The target creature must pass a Wisdom saving throw or become disoriented. A disoriented creature has disadvantage on all attack rolls and at the start of their turn moves 10 feet (up to its speed) in a random direction before their speed becomes zero until the start of their next turn.

At the end of each of its turns, the target can make another Wisdom saving throw. On a success, the spell ends, but if the target fails by 5 or more, it fails prone.

\begin{itemize}
  \item Spell: Disorient
\end{itemize}

\subsubsection*{Dispel Construct}

3rd-level abjuration

Classes: Inventor 
Casting Time: 1 action 
Range: 60 feet 
Components: V, S 
Duration: Instantaneous

You can attempt to purge the magic animating a construct within range, rendering it inert. The target takes 4d10 force damage and must succeed on a Constitution saving throw or become stunned for 1 minute. At the end of each of its turns, the target can make another Constitution saving throw. On a success, the spell ends on the target. If the target has less than 50 hit points remaining when it fails, it is reduced to zero hit points.

\begin{itemize}
  \item Spell: Dispel Construct
\end{itemize}

\subsubsection*{Divide Self}

5th-level illusion

Classes: Occultist, Sorcerer, Wizard 
Casting Time: 1 bonus action 
Range: Self 
Components: S 
Duration: Concentration, up to 1 minute

You create an exact duplicate of yourself in an empty space you can see within 30 feet of you. When you cast this spell and at the start of each of your turns for the duration, you can switch places with your duplicate.

The duplicate has all of your stats, abilities, and equipment (including magic items). It acts on your initiative, and has its own actions, though it shares its concentration on this spell, and if either of you lose concentration, the spell ends.

Your current hit points are divided between you and the duplicate and it shares all other resources and abilities with you (including limited use magic items), with any usage by either you or the duplicate depleting the resource for both of you.

Your duplicate can take any action you can take, but it can deal a maximum of 15 damage on its turn (any additional damage dealt deals no further damage, when dealing area of effect damage, damage is split between all targets equally up to the maximum).

If either you or the duplicate is reduced to zero hit points, the spell ends and you become the copy that was not reduced to zero hit points. When the spell ends, if both you and the duplicate are still present, decide which is you, and the other vanishes. Anything that was copied during the spell has the copied version vanish.

At Higher Levels. When you cast this spell using a spell slot of 6th level or higher, the starting hit points of you and the duplicate both increase by 15 (up to a maximum of you and the duplicate starting with your current hit points) and the maximum damage the duplicate can do during its turn increases by 10 for each slot level above 5th

\begin{itemize}
  \item Spell: Divide Self
\end{itemize}
\end{minipage}

\section*{Spells - E}

\begin{minipage}{0.48\textwidth}
\subsubsection*{Earth Ripple}

2nd-level transmutation

Classes: Druid, Occultist, Sorcerer, Wizard 
Casting Time: 1 action 
Range: 60 feet 
Components: V, S 
Duration: Instantaneous

You cause the earth to deform and ripple, a target creature must make a Dexterity saving throw or suffer one of the following effects (your choice):

\begin{itemize}
  \item It is pulled into the earth, taking 1d8 bludgeoning damage and reducing its speed to 0 until a creature spends an action to dig it free.
  \item It is slammed 5 feet in a direction of your choice by a wave of earth, taking 2d8 bludgeoning damage and being knocked prone.
  \item It is impaled by a spike of earth, taking 4d8 piercing damage.
\end{itemize}

\begin{itemize}
  \item Spell: Earth Ripple
\end{itemize}

\subsubsection*{Echoing Lance}

4th-level evocation

Classes: Bard, Occultist, Sorcerer, Wizard 
Casting Time: 1 action 
Range: 60 feet 
Components: V, S 
Duration: Concentration, up to 1 minute

You emit a targeted burst of intense sonic energy at a creature within range. The target must make a Constitution saving throw. On a failure, they take 3d8 thunder damage and become stunned for the duration by the intense sound. On a successful save, the target takes half as much damage and isn't stunned.

At the end of each of its turns, the target can make another Constitution saving throw. On a success, the spell ends, on failure, they take an extra 1d8 thunder damage from the echoes within their mind.

At Higher Levels. When you cast this spell using a spell slot of 5th level or higher, the damage increases by 1d8 for each slot level above 4th.

\begin{itemize}
  \item Spell: Echoing Lance
\end{itemize}

\subsubsection*{Electrify}

1st-level evocation

Classes: Occultist, Sorcerer, Wizard 
Casting Time: 1 bonus action 
Range: Self 
Components: V, S 
Duration: 1 round

You channel lightning into your hands. The next time you hit a creature with a melee attack (including a melee spell attack) before the start of your next turn, the target takes 1d10 lightning damage and must make a Constitution saving throw. On a failed save, the target becomes stunned until the start of their next turn.

The spell ends after dealing damage, or at the start of your next turn, whichever occurs first. For the duration of the spell, you can cast the spell shocking grasp .

\begin{itemize}
  \item Spell: Electrify
\end{itemize}
\end{minipage}\hfill
\begin{minipage}{0.48\textwidth}
\subsubsection*{Electrocute}

3rd-level evocation

Classes: Sorcerer, Wizard 
Casting Time: 1 action 
Range: 60 feet 
Components: V, S 
Duration: Instantaneous

A massive arc of lightning leaps from your hand to a target you can see within range. The target must make a Constitution saving throw. On a failed save, the target takes 4d12 lightning damage and is stunned until the start of your next turn. On a successful save, the target takes half as much damage and isn't stunned.

At Higher Levels. When you cast this spell using a spell slot of 4th level or higher, the damage increases by 1d12 for each slot level above 3rd.

\begin{itemize}
  \item Spell: Electrocute
\end{itemize}

\subsubsection*{Entomb}

1st-level transmutation

Classes: Wizard 
Casting Time: 1 action 
Range: 60 feet 
Components: V, S 
Duration: Concentration, up to 1 minute

You attempt to encase a Medium or smaller creature you can see within ice. The creature must make a Strength saving throw or become restrained by ice for the duration. At the end of each of its turns, the target takes 1d8 cold damage and can make another Strength saving throw. On success, the spell ends on the target.

If the creature takes more than 5 fire or bludgeoning damage from a single damage roll while restrained, the ice breaks and the target is freed, ending the spell for the target.

At Higher Levels. When you cast this spell using a spell slot of 2nd level or higher, the damage increases by 1d8 for each slot level above 1st.

\begin{itemize}
  \item Spell: Entomb
\end{itemize}

\subsubsection*{Erode}

3rd-level conjuration

Classes: Occultist, Wizard 
Casting Time: 1 action 
Range: 20 feet 
Components: V, S 
Duration: Instantaneous

You blast a target with a glob of acid. The target must make a Dexterity saving throw. On failure, the target takes 8d4 acid damage immediately and becomes covered in acid. On a success, the target takes half as much damage and is not covered in acid. While covered in acid, the target takes 2d4 acid damage at the end of each of its turns. The target or a creature within 5 feet of it can end this damage by using its action to clear away the acid.

At Higher Levels. When you cast this spell using a spell slot of 4th level or higher, the damage (both initial and later) increases by 1d4 for each slot level above 3rd.

\begin{itemize}
  \item Spell: Erode
\end{itemize}
\end{minipage}

\section*{Spells - F}

\begin{minipage}{0.48\textwidth}
\subsubsection*{Fall}

1st-level transmutation

Classes: Inventor, Sorcerer, Wizard 
Casting Time: 1 action 
Range: Self 
Components: V, S 
Duration: Instantaneous

You alter gravity for yourself, causing you to reorient which way is down for you until the end of your turn. You can pick any direction to fall as if under the effect of gravity, falling up to 500 feet before the spell ends.

If you collide with something during this time, you take falling damage as normal, but you can control your fall as you could under normal conditions by holding onto objects or move along a surface according to your new orientation as normal until your turn ends and gravity returns to normal.

If you collide with something during this time, you take falling damage as normal, but you can control your fall as you could under normal conditions by holding onto objects or move along a surface according to your new orientation as normal until your turn ends and gravity returns to normal.

\begin{itemize}
  \item Spell: Fall
\end{itemize}

\subsubsection*{Fire Cyclone}

3rd-level conjuration

Classes: Druid, Sorcerer, Wizard 
Casting Time: 1 action 
Range: 60 feet Components: V, S, M (a pinch of ashes from a forest fire) 
Duration: Concentration, up to a 1 minute

Targeting a point you can see, you cause a cyclone made of whipping flames with a radius of 5 feet and height of 30 feet to form.

When a creature starts its turn inside the cyclone's radius or enters it for the first time during a turn, it must make a Strength saving throw. On a failed saving throw, it takes 3d6 fire damage and, if it is entirely inside the cyclone's area, it's also flung 15 feet upwards and lands 15 feet in a randomly determined horizontal direction. On a successful save, the creature takes half as much damage and is not flung.

When a creature is not entirely inside the cyclone's radius but within 30 feet of its center at the start of its turn, it still feels the intense draw of the raging cyclone, and must spend 2 feet (or 3 feet if it is flying) of movement for every 1 foot it moves away from the cyclone. If a creature starts its turn outside of the cyclone's radius but within 10 feet of its center, it must make a Strength saving throw or be pulled 5 feet toward the center of it.

At Higher Levels. When you cast this spell using a spell slot of 4th level or higher, the damage increases by 1d6.

\begin{itemize}
  \item Spell: Fire Cyclone
\end{itemize}

\subsubsection*{Fireburst Mine}

3rd-level abjuration

Classes: Inventor 
Casting Time: 1 minute 
Range: Touch 
Components: V, S, M (any Tiny nonmagical item, which is destroyed by the activation of the spell) 
Duration: 8 hours

You can set a magical trap by infusing explosive magic into an item. You can set this item to detonate when someone comes within 5 feet of it, or by a verbal command using your reaction (one or more mines can be detonated).

When the magic trap detonates, each creature in a 20-footradius sphere centered on the item must make a Dexterity saving throw. A creature takes 5d8 fire damage on a failed save, or half as much damage on a successful one. If a creature is in the area of effect of more than one fireburst mineK during a turn, they take half damage from any mines beyond the first.

A magical mine must be set 5 feet or more from another mine, and can't be moved once placed; any attempt to move it results in it detonating unless the caster that set it disarms it with an action.

\begin{itemize}
  \item Spell: Fireburst Mine
\end{itemize}

\subsubsection*{Fissure}

5th-level transmutation

Classes: Druid, Sorcerer, Wizard 
Casting Time: 1 Action 
Range: Self (60-foot line) 
Components: V, S 
Duration: Instantaneous

You rend asunder the earth in a 60-foot-long, 5-foot-wide line, targeting an area of dirt, sand, or rock at least 10 feet deep.

Creatures in that line must make a Dexterity saving throw. On a failure, a creature falls into a suddenly opened crevice in the ground, falling into it before it snaps shut, crushing them. Creatures that fail the saving throw take 6d10 bludgeoning damage from the fall and crushing. The creature is buried in 10 feet of rubble, and creatures without a burrowing speed require 25 feet of movement to extract themselves from the loose rubble to return to where they failed the saving throw. If they end their turn while buried, they take an extra 1d10 bludgeoning damage.

\begin{itemize}
  \item Spell: Fissure
\end{itemize}

\subsubsection*{Flash Freeze}

3rd-level evocation

Classes: Sorcerer, Wizard 
Casting Time: 1 action 
Range: Self (30-foot cone) 
Components: V, S 
Duration: Instantaneous

A freezing wind ripples outward. Each creature in a 30-foot cone must make a Constitution saving throw. On a failed save, a creature takes 4d8 cold damage and is restrained by ice until the start of your next turn. On a successful save, the target takes half as much damage and isn't restrained.

At Higher Levels. When you cast this spell using a spell slot of 4th level or higher, the damage increases by 1d8 for each slot level above 3rd.

\begin{itemize}
  \item Spell: Flash Freeze
\end{itemize}

\subsubsection*{Flicker}

1st-level psionic

Casting Time: 1 reaction, when you would take damage 
Range: Self 
Components: S 
Duration: 1 round You flicker between the material and ethereal planes.

Until the start of your next turn, each time you would take damage, including the triggering attack, roll a d4. On a 2, you gain resistance to that instance of damage. On a 4, you don't take any damage.

\begin{itemize}
  \item Spell: Flicker
\end{itemize}
\end{minipage}\hfill
\begin{minipage}{0.48\textwidth}
\subsubsection*{Flickering Strikes}

5th-level conjuration

Classes: Ranger, Wizard 
Casting Time: 1 action 
Range: Self (30-foot radius) 
Components: V, S, M (a melee weapon you are proficient with worth at least 1 sp) 
Duration: Instantaneous

You flourish a weapon weapon you are proficient with used in the casting and then vanish, instantly teleporting to and striking up to 5 targets within range. Make a weapon attack against each target. On hit, a target takes the weapon damage from the attack + 6d6 force damage.

You can then teleport to an unoccupied space you can see within 5 feet of one of the targets you hit or missed.

\begin{itemize}
  \item Spell: Flickering Strikes
\end{itemize}

\subsubsection*{Fling}

2nd-level transmutation

Classes: Sorcerer, Wizard 
Casting Time: 1 action 
Range: 30 feet 
Components: V, S 
Duration: Instantaneous

You manipulate gravity around one Large or smaller creature. The target creature makes a Strength saving throw. On failure, you can fling them 40 feet straight up or 20 feet in any direction. If you fling them straight up they immediately fall, taking 4d6 damage falling damage, and fall prone. If you fling them any other direction, they take 2d6 damage and fall prone. If their movement would be stopped early by a creature or object, both the target and creature or object takes 3d6 bludgeoning damage.

\begin{itemize}
  \item Spell: Fling
\end{itemize}

\subsubsection*{Force Blade}

4th level evocation

Classes: Sorcerer, Wizard 
Casting Time: 1 bonus action 
Range: Self (5 feet) 
Components: V, S 
Duration: Concentration, up to 1 minute

You create an oversized blade of pure scintillating force energy in your hand. For the duration of the spell, as an action, you can sweep the blade through one creature within reach, dealing 2d12 force damage.

At Higher Levels. When you cast it using a 5th- or 6th-level spell slot, the damage increases to 3d12. When you cast it using a spell slot of 7th level or higher, the damage increases to 4d12.

\begin{itemize}
  \item Spell: Force Blade
\end{itemize}

\subsubsection*{Force Bolt}

Evocation cantrip

Classes: Sorcerer 
Casting Time: 1 action 
Range: 120 ft. 
Components: V, S 
Duration: Instantaneous

You hurl a mote of arcane energy at a creature or object within range. Make a ranged spell attack against the target. On a hit, the target takes 2d4 force damage.

This spell's damage increases by 2d4 when you reach 5th level (4d4), 11th level (6d4), and 17th level (8d4).

\begin{itemize}
  \item Spell: Force Bolt
\end{itemize}

\subsubsection*{Freezing Shell}

1st-level abjuration

Classes: Warlock 
Casting Time: 1 action 
Range: Self 
Components: V, S 
Duration: 1 hour

A freezing shell shrouds you, covering you and your gear. You gain 5 temporary hit points for the duration. If a creature hits you with a melee attack while you have these hit points, the creature takes 5 cold damage.

At Higher Levels. When you cast this spell using a spell slot of 2nd level or higher, both the temporary hit points and the cold damage increase by 5 for each slot level above 1st.

\begin{itemize}
  \item Spell: Freezing Shell
\end{itemize}

\subsubsection*{Frighten}

1st-level necromancy

Classes: Occultist, Warlock, Wizard 
Casting Time: 1 action 
Range: 60 feet 
Duration: Concentration, up to 1 minute

You invoke a sudden fear within a creature you can see within range. The target creature must succeed a Wisdom saving throw, or become frightened for the duration. The frightened target can repeat the saving throw at the end of each of its turns, ending the effect on itself on a success.

At Higher Levels. When you cast this spell using a 2nd level or higher, you can target one additional creature for each slot above 1st.

\begin{itemize}
  \item Spell: Frighten
\end{itemize}

\subsubsection*{Future Insight}

1st-level psionic

Casting Time: 1 action 
Range: Self 
Components: S 
Duration: 10 minutes

You roll 3d4 or 1d12 (your choice) and record the results. During the duration, you can expend one of these dice to add or subtract them from any attack roll, saving throw, or ability check made by a creature within 60 feet of you until the dice are exhausted or the spell ends. You must expend the die after the roll is made, but before you know the outcome of the roll.

\begin{itemize}
  \item Spell: Future Insight
\end{itemize}
\end{minipage}

\section*{Spells - G}

\begin{minipage}{0.48\textwidth}
\subsubsection*{Gale Bolt}

1st-level evocation

Casting Time: 1 action 
Range: 120 feet 
Components: V, S 
Duration: Instantaneous

A blast of concentrated wind streaks toward a creature of your choice within range. Make a ranged spell attack against the target. On a hit, the target takes 2d8 bludgeoning damage and if it is Large or smaller is knocked 10 feet away from you.

At Higher Levels. When you cast this spell using a spell lot of 2nd level or higher, the damage increases by 1d8 for each level above 1st.

\begin{itemize}
  \item Spell: Gale Bolt
\end{itemize}

\subsubsection*{Geyser}

4th-level conjuration

Classes: Druid, Sorcerer, Wizard 
Casting Time: 1 action 
Range: 120 feet 
Components: V, S 
Duration: Instantaneous

You cause a massive eruption of water to blast upwards from the ground at a point within range. Creatures within 10 feet of the point must make a Dexterity saving throw or take 4d6 bludgeoning damage and be knocked 60 feet into the air. On a successful save, creatures take half as much damage, and are instead knocked their choice of 10 feet away from the point or 10 feet upward.

\begin{itemize}
  \item Spell: Geyser
\end{itemize}

\subsubsection*{Glimpse the Future}

2nd-level psionic

Casting Time: 1 action 
Range: 60 feet 
Components: S 
Duration: 10 minutes

You give a creature within range a glimpse of their future. Roll a d4 to determine outcome:

\begin{tabularx}{\textwidth}\toprule
{}XXX}
\midrule
d4 & \multicolumn{3}{c}{Effect} \\
\midrule
1 & \multicolumn{3}{c}{The target foresees an action to come. Roll a d20 and record the value. Until the duration of the spell ends, they can replace one of their d20 rolls with the value rolled.} \\
\midrule
2 & \multicolumn{3}{c}{The target sees their own death. If they are reduced to zero hit points by an attack or failing a save throw during the duration, they instead evade the attack or pass the saving throw if they are not otherwise incapacitated prior to being reduced to zero.} \\
\midrule
3 & \multicolumn{3}{c}{They see a future victory, growing confident. They gain 10 temporary hit points and are immune to the frightened condition for the duration of the spell.} \\
\midrule
4 & \multicolumn{3}{c}{The target sees an ambush or surprise, the first time they would be surprised they are not, or the first time an attack would be made against them with advantage, it is instead made with disadvantage.} \\
\midrule
\end{tabularx}

Once any of the events foreseen occur, the spell ends.

At Higher Levels. When cast using 3 or more psi points, you can select the effect instead of rolling a d4.

\begin{tabularx}{\textwidth}\toprule
{}}
\midrule
The Unavoidable Death 
In the case of number 2, passing the save will not always be enough to make them not be reduced to zero hit points. Sometimes they are reduced to zero by something like falling damage that has no save or attack. Sometimes life's a bummer that way. \\
\midrule
\end{tabularx}

\begin{itemize}
  \item Spell: Glimpse the Future
\end{itemize}

\subsubsection*{Glyph of Absorption*}

1st-level abjuration (glyph)

Casting Time: 1 bonus action 
Range: 5 feet
 Components: S 
Duration: 1 minute

You draw an ethereal mark in the air at a point within range. Creatures within 10 feet of this glyph have resistance to all damage, but when any creature in the rune's radius takes damage, the rune takes an equal amount. The glyph has 10 hit points. If the glyph is destroyed, it fades without detonating and the spell ends.

As an action, you can detonate the glyph. Any creature within 15 feet of this glyph must make a Dexterity saving throw, or take force damage equal to half the amount of damage the glyph has absorbed. After using this action, the spell ends.

You can end the spell early by deactivating it with a bonus action.

At Higher Levels. When you cast this spell using a spell slot of 2nd level or higher, the glyph's hit points increase by 5 for each slot level above 1st.

\begin{itemize}
  \item Spell: Glyph of Absorbtion
\end{itemize}

\subsubsection*{Glyph of Fire*}

1st-level evocation (glyph)

Casting Time: 1 bonus action 
Range: 5 feet 
Components: S 
Duration: 1 minute

You draw an ethereal mark in the air at a point within range. Any creature that starts their turn within 5 feet of this glyph takes 1 fire damage.

As an action, you can force all creatures within 15 feet of this glyph to make a Dexterity saving throw, taking 3d6 fire damage on failure, or half as much on success. After using this action, the spell ends.

You can end the spell early by deactivating it with a bonus action.

\begin{itemize}
  \item Spell: Glyph of Fire
\end{itemize}
\end{minipage}\hfill
\begin{minipage}{0.48\textwidth}
\subsubsection*{Glyph of Frost*}

1st-level evocation (glyph)

Casting Time: 1 bonus action 
Range: 5 feet 
Components: S 
Duration: 1 minute

You draw an ethereal mark in the air at a point within range. The area within 5 feet of this glyph becomes difficult terrain.

As an action, you can force all creatures within 15 feet of this glyph to pass a Constitution saving throw, or take 1d8 cold damage and become restrained, or half as much damage on success. After using this action, the spell ends.

You can end the spell early by deactivating it with a bonus action.

\begin{itemize}
  \item Spell: Glyph of Frost
\end{itemize}

\subsubsection*{Glyph of Gravity*}

4th-level transmutation (glyph)

Casting Time: 1 bonus action 
Range: 5 feet 
Components: S 
Duration: 1 minute

You draw an ethereal mark in the air at a point within range. Within 30 feet of this Glyph, moving away from it requires twice as much movement, and any creature that ends its turn within 30 feet of it is pulled 10 feet closer to it if there is space available.

As an action, you can force all creatures within 30 feet of this glyph to make a Strength saving throw or be flung 40 feet away from this glyph, taking 4d6 bludgeoning damage and falling prone. After using this action, the spell ends.

You can end the spell early by deactivating it with a bonus action.

\begin{itemize}
  \item Spell: Glyph of Gravity
\end{itemize}

\subsubsection*{Glyph of Nullification*}

3rd-level abjuration (glyph)

Casting Time: 1 bonus action 
Range: 5 feet 
Components: S 
Duration: 1 minute

You draw an ethereal mark in the air at a point within range. Creatures within 5 feet of this glyph have resistance to force damage.

As an action for the duration of the spell, you can cause the glyph to form a globe of invulnerability centered on the glyph, which lasts until the start of your next turn, at which point the spell ends.

You can end the spell early by deactivating it with a bonus action.

\begin{itemize}
  \item Spell: Glyph of Nullification
\end{itemize}

\subsubsection*{Glyph of Translocation*}

2nd-level conjuration (glyph)

Casting Time: 1 bonus action 
Range: 5 feet 
Components: S 
Duration: 1 minute

You draw an ethereal mark in the air at a point within range.

As an action for the duration of the spell, you instantly teleport a willing creature of your choice within 120 feet of you to the closest free space next to this glyph (your choice if multiple free spaces are available). If you expend a higher-level spell slot, the number of creatures you can teleport increases by 1 for each level above 2nd, at which point the spell ends.

You can end the spell early by deactivating it with a bonus action.

At Higher Levels. If you expend a higher level spell slot, the number of creatures you can teleport increases by 1 for each level above 2nd, at which point the spell ends.

\begin{itemize}
  \item Spell: Glyph of Translocation
\end{itemize}

\subsubsection*{Grip of the Dead}

1st-level necromancy

Classes: Occultist 
Casting Time: 1 action 
Range: Touch 
Components: V, S 
Duration: Concentration, up to 1 minute

You channel unholy strength into you hand, and reach out to grab a creature. The creature must make a Strength saving throw. On a failed save the creature is restrained by your deathly iron grasp. As an action on its turn, the creature can attempt to escape using a Strength (Athletics) or Dexterity (Acrobatics) check against your Spell Save DC.

While you maintain the spell and grip, the creature takes 1d8 necrotic damage at the start of each of its turns as you drain the life from it, and you regain hit points equal to half the damage dealt.

At Higher Levels. When you cast this spell using a spell slot of 2nd level or higher, the damage increases by 1d8 for each slot level above 1st.

\begin{itemize}
  \item Spell: Grip of the Dead
\end{itemize}
\end{minipage}

\section*{Spells - H}

\begin{minipage}{0.48\textwidth}
\subsubsection*{Hurricane Slash}

2nd-level evocation

Classes: Druid, Occultist, Ranger, Sorcerer, Wizard 
Casting Time: 1 action 
Range: Self (30-foot line) 
Components: V, S 
Duration: Instantaneous

You condense wind into a razor sharp blast that shreds a 30-foot-long, 5-foot-wide line. Creatures in the area must make a Dexterity saving throw. A creature takes 3d8 slashing damage on a failed save or half as much damage on a successful one.

At Higher Levels. When you cast this spell using a spell slot of 3rd level or higher, you can create an additional line of effect. A creature in the area of more than one slash is affected only once.

\begin{itemize}
  \item Spell: Hurricane Slash
\end{itemize}
\end{minipage}\hfill
\begin{minipage}{0.48\textwidth}

\end{minipage}

\section*{Spells - I}

\begin{minipage}{0.48\textwidth}
\subsubsection*{Ice Spike}

4th-level evocation

Classes: Sorcerer, Wizard 
Casting Time: 1 action 
Range: 60 feet 
Components: V, S 
Duration: Instantaneous

You create a lance of ice that shoots up from the ground to impale a creature within range. The target must make a Dexterity saving throw. The target takes 4d8 piercing damage and 4d8 cold damage on a failed save. The target takes only the 4d8 cold damage on a successful save.

At Higher Levels. When you cast this spell using a spell slot of 6th or 7th level, you can create a second spike. When you cast this spell using a spell slot of 8th or 9th level, you can create a third spike. Additional spikes can target the same or different creatures.

\begin{itemize}
  \item Spell: Ice Spike
\end{itemize}

\subsubsection*{Ignite Fire}

Conjuration cantrip

Classes: Druid, Occultist, Sorcerer, Warlock, Wizard 
Casting Time: 1 action 
Range: 60 feet Components: V, S 
Duration: Concentration, up to 1 minute

You ignite a magical fire that fills a 5-foot cube in a space you can see on the ground. A creature in the fire's space when you cast the spell must suceeed a Dexterity saving throw or take 1d8 fire damage. A creature that enters the fire's space for the first time or ends their turn there must repeat the saving throw against the effect. Flamable objects in the area that aren't being worn or carried catch fire.

The spell's damage increases by 1d8 when you reach 5th level (2d8), 11th level (3d8), and 17th level (4d8).

\begin{itemize}
  \item Spell: Ignite Fire
\end{itemize}

\subsubsection*{Inner World}

8th-level psionic

Casting Time: 1 action 
Range: Self 
Components: S 
Duration: Concentration, up to 1 minute

As an action, you create and enter an imaginary world. All other creatures within 120 feet are pulled this world with you. This world is centered on you, and extends in 120 feet in all directions.

A creature that reaches the edge of this world can make a Charisma saving throw to attempt to exit, spending 5 feet of movement to return where they were before being pulled into the world on success, and being unable to move out the world until the start of their next turn on failure. A creature outside the world can attempt to enter it by moving to where you cast the spell (which is marked by a glowing psionic rift) and making a Charisma saving throw to enter the world. You can allow a creature to automatically pass their save to enter or exit the world.

When you create this world, you can create obstacles and terrain of your choice, creating walls, pillars, and other obstacles that take up to twenty 5 by 5 square foot areas (stylistically, these can appear however you choose). These can be placed consecutively or spread out in any method of your choosing, but any area with a creature must contain a path that creature can fit through to both you and the edge of the of the world.

You can additionally create up to five hazardous spaces on the ground that are 5 foot squares. These can be fires, spikes, biting mouths, or whatever you choose, but regardless of its form if a creature takes 4d4 + 4 psychic damage when it enters the effect's area for the first time on a turn or starts its turn there. Each of these hazards must be at least 20 feet from another hazard.

While in this inner world, if you fail saving throw, you choose to succeed instead. You can do up to 3 times during the duration of the spell. All spells and powers have their psi point cost reduced by one.

During the spell, as an action, you can attempt to destroy a creature within the world. The target must make an Intelligence saving throw. On failure, it takes 8d8 + 8 psychic damage and is removed from the imaginary world, returning to where they were before being pulled into it.

The world can be brightly or dimly lit, and you control the weather within it.

When the spell ends, you and any creature that remains in the world exit the world returning to space you entered the world from.

\begin{tabularx}{\textwidth}\toprule
{}}
\midrule
Quickly Constructed Worlds 
The casting time of this spell is 1 action, which is 6 seconds in game. It is recommended that that you come up with a general layout and any resources needed for this spell in collaboration with your GM when you select the spell, and make at most minor changes to the world when casting it. \\
\midrule
\end{tabularx}

\begin{itemize}
  \item Spell: Inner World
\end{itemize}
\end{minipage}\hfill
\begin{minipage}{0.48\textwidth}
\subsubsection*{Imbue Luck}

2nd-level abjuration

Classes: Inventor 
Casting Time: 1 action 
Range: Touch 
Components: V, S, M (a four leaf clover) 
Duration: 1 hour

You touch a weapon or worn item and imbue luck into it. If imbued on a weapon, for the duration, on an attack roll, the wielder can roll an additional d20 (they can choose to do this after they roll, but before the outcome is determined). The creature can choose which of the d20s is used for the attack roll.

If imbued into a worn item, they can roll a d20 when attacked, then choose whether the attack uses the attacker’s roll or theirs.

With either use, the spell immediately ends upon rolling the extra d20.

\begin{itemize}
  \item Spell: Imbue Luck
\end{itemize}

\subsubsection*{Invest Life}

3rd-level psionic

Casting Time: 1 action 
Range: Touch 
Components: S 
Duration: Instantaneous

You sacrifice some of your health to mend another creature's injuries. You take 4d8 necrotic damage, which can't be reduced in any way, and one creature of you choice that you can see within range regains a number of hit points equal to twice the necrotic damage you take.

At Higher Levels. When you cast this spell using a spell slot of 4th level or higher, the damage increases by 1d8 for each slot level above 3rd.

\begin{itemize}
  \item Spell: Invest Life
\end{itemize}

\subsubsection*{Invested Competency}

5th-level Psionic

Casting Time: 1 action 
Range: Touch 
Components: S 
Duration: Concentration, up to 1 hour

You touch a willing creature imbuing psionic competency into them. Until the spell ends, they gain expertise in on skill of your choice, adding double their proficiency bonus to that skill.

\begin{itemize}
  \item Spell: Invested Competency
\end{itemize}
\end{minipage}

\section*{Spells - J}

\begin{minipage}{0.48\textwidth}
\subsubsection*{Jumping Jolt}

4th-level evocation

Classes: Sorcerer, Wizard 
Casting Time: 1 action 
Range: 60 feet 
Components: V, S 
Duration: Instantaneous

You release an arc of lighting at a creature within range. Make a ranged spell attack roll against the target. On hit, the target takes 4d12 lightning damage, and you can cause the spell to jump to another target within 20 feet of the first target making a new attack roll for each target. The spell can't hit the same target twice, or jump to a target out of the spells range. The spell can jump a maximum of five times.

On a miss, the target takes half as much damage and the spell doesn't jump to a new target.

At Higher Levels. When you cast this spell using a spell slot of 5th level or higher, the starting damage increases by 1d12 for each slot level above 4th.

\begin{itemize}
  \item Spell: Jumping Jolt
\end{itemize}
\end{minipage}\hfill
\begin{minipage}{0.48\textwidth}

\end{minipage}

\section*{Spells - K}

\begin{minipage}{0.48\textwidth}
\subsubsection*{Killing Curse}

5th-level necromancy

Classes: Occultist 
Casting Time: 1 action 
Range: 60 feet 
Components: V, S, M (something from the target creature (such as blood, hair, or scales) which the spell consumes) 
Duration: Concentration, up to 1 minute

You curse a target to die. The targets current and maximum hit points are reduced by 3d10 + 10. If this causes a creatures to have zero hit points, the creature dies.

For the duration of the spell, the target cannot regain hit points unless from a spell cast using a spell slot of higher level than the spell slot this curse was cast with, and any death saving throw they roll is automatically considered a 1.

At the start of a creatures turn while they are under the effect of this spell, they make a Charisma saving throw. On failure, their current and maximum hit points is reduced by 1d10 + 10. On a successful save, the spell ends. A creature's maximum hit points are restored when it takes a long rest.

\begin{itemize}
  \item Spell: Killing Curse
\end{itemize}
\end{minipage}\hfill
\begin{minipage}{0.48\textwidth}

\end{minipage}

\section*{Spells - L}

\begin{minipage}{0.48\textwidth}
\subsubsection*{Launch Object}

1st-level transmutation

Classes: Inventor 
Casting Time: 1 action 
Range: 60 ft. 
Components: S 
Duration: Instantaneous

Choose one object weighing 1 to 5 pounds within range that isn't being worn or carried. The object flies in a straight line up to 90 feet in a direction you choose before falling to the ground, stopping early if it impacts against a solid surface. If the object would strike a creature, that creature must make a Dexterity saving throw. On a failed save, the object strikes the target and stops moving. When the object strikes something, the object and what it strikes each take 3d8 bludgeoning damage.

At Higher Levels. When you cast this spell using a spell slot of 2nd level or higher, the maximum weight of objects that you can target with this spell increases by 5 pounds, and the damage increases by 1d8, for each slot level above 1st.

\begin{itemize}
  \item Spell: Launch Object
\end{itemize}

\subsubsection*{Lightning Charged}

2nd-level evocation

Classes: Inventor 
Casting Time: 1 action 
Range: Touch 
Components: V, S, M (a piece of metal once used in a lightning rod) 
Duration: 10 minutes

You channel lightning energy into a creature. The energy is harmless to the creature, but escapes in dangerous bursts to other nearby creatures.

Every time that creature strikes another creature with a melee attack, a spell with a range of touch, is struck by another creature with a melee attack, or ends their turn while grappling or being grappled by another creature, they deal 1d6 lightning damage to that creature.

Once this spell has discharged 6 times (dealing up to 6d6 damage), the spell ends.

At Higher Levels. The spell can discharge damage 2 additional times (dealing 2d6 more total damage) before the spell ends for each slot level above 2nd.140 Chapter 4 | Spells

\begin{itemize}
  \item Spell: Lightning Charged
\end{itemize}
\end{minipage}\hfill
\begin{minipage}{0.48\textwidth}
\subsubsection*{Lightning Tendril}

1st-level evocation

Classes: Druid, Occultist, Sorcerer, Warlock, Wizard 
Casting Time: 1 bonus action 
Range: Self (20 feet) 
Components: V, S, M (a twig from a tree that has been struck by lightning) 
Duration: Concentration, up to 1 minute

Crackling beams of blue energy leap from your hands. For the duration of the spell, as an action, you can direct them toward a creature within range, dealing 1d12 lightning damage to that creature.

At Higher Levels. When you cast this spell using a 3rd- or 4th-level spell slot, the damage increases to 2d12 and the range increases to 30 feet. When you cast it using a 5th- or 6th-level spell slot, the damage increases to 3d12 and the range increases to 60 feet. When you cast it using a spell slot of 7th level or higher, the damage increases to 4d12 and the range increases to 120 feet.

\begin{itemize}
  \item Spell: Lightning Tendril
\end{itemize}
\end{minipage}

\section*{Spells - M}

\begin{minipage}{0.48\textwidth}
\subsubsection*{Martial Transformation}

6th-level transmutation

Classes: Wizard 
Casting Time: 1 action 
Range: Self 
Components: V, S, M (a few hairs from a bull) 
Duration: 10 minutes

You endow yourself with endurance and martial prowess fueled by magic. Until the spell ends, you can't cast spells or concentrate them, and you gain the following benefits:

\begin{itemize}
  \item You gain 50 temporary hit points. If any of these remain when the spell ends, they are lost. • You have advantage on attack rolls that you make with simple and martial weapons.
  \item When you hit a target with a weapon attack, that target takes an extra 2d12 force damage.
  \item You have proficiency with all armor, shields, simple weapons, and martial weapons. • You have proficiency in Strength and Constitution saving throws.
  \item You can attack twice, instead of once, when you take the Attack action on your turn. You ignore this benefit if you already have a feature, like Extra Attack, that gives you extra attacks.
  \item You can conjure and equip (as part of the action used to cast the spell) and set of heavy or medium armor and any simple or martial weapon of your choice. These items have no strength requirements and are magical in nature though have the same properties as their nonmagical counterparts, vanishing when the spell ends.
\end{itemize}

Immediately after the spell ends, you must succeed on a DC 15 Constitution saving throw or suffer one level of exhaustion.

\begin{itemize}
  \item Spell: Martial Transformation
\end{itemize}

\subsubsection*{Mind Blast}

6th-level psionic

Casting Time: 1 action 
Range: Self (60-foot cone) 
Components: S 
Duration: Instantaneous

You emit a blast of psychic energy. Each creature in a 60-foot cone must make an Intelligence saving throw. A creature takes 6d8 psychic damage and is stunned until the end of their next turn on a failed save. A creature takes half as much damage and is not stunned on a successful save.

\begin{itemize}
  \item Spell: Mind Blast
\end{itemize}
\end{minipage}\hfill
\begin{minipage}{0.48\textwidth}
\subsubsection*{Mutate}

3rd-level transmutation

Classes: Druid, Occutlist, Sorcerer, Warlock, Wizard 
Casting Time: 1 action 
Range: Self
 Components: V, S, M (something from an extinct animal) 
Duration: Concentration, up to 10 minutes

You manipulate the nature of your body with magic temporarily giving it new properties. You can select three of the following properties:

\begin{itemize}
  \item Your body becomes malleable and amorphous. You have advantage on saves and checks against grapples and the restrained condition, you do not suffer disadvantage from squeezing into smaller spaces, and you can squeeze through openings two sizes smaller than you.
  \item You grow one additional appendage. This appendage serves as an arm and a hand, though it can take the shape of an arm, tentacle, or similar appendage.
  \item You extend the length of your limbs, increasing the reach on melee attacks, touch spells, and object interactions by 5 feet.
  \item Your flesh hardens, your base AC becomes 14 + your dexterity modifier if it is not already higher.
  \item You grow more resilient, adapting against one external threat. You gain advantage on one type of saving throw of your choice.
  \item You adapt your body to an aquatic environment, sprouting gills and growing webbing between your fingers. You can breathe underwater and gain a swimming speed equal to your walking speed.
  \item Your body grows ablative armor. You gain temporary hit points equal to your spellcasting ability modifier at the start of each of your turns. • You can grow one size larger or smaller.
  \item You sprout wings. You gain a flying speed of 30 feet.
  \item You grow a natural weapon; this weapon can have the statistics of any martial melee weapon without the thrown property, and takes on a form vaguely reminiscent of it. You have proficiency with this weapon, and are considered to be holding it. You can use your spellcasting modifier in place of your Strength or Dexterity modifier for attack and damage rolls with this natural weapon. The natural weapon is magic and you have a +1 bonus to the attack and damage rolls you make using it. For the duration of the spell, you can use an action to change one or all of the properties, losing the benefits of your previously selected properties and gaining the benefits of the new selected properties.
\end{itemize}

At Higher Levels. When you cast this spell using a spell slot of 4th or higher, you can select one additional property from the list of options, with one additional property per spell level above 3rd.

\begin{itemize}
  \item Spell: Mutate
\end{itemize}
\end{minipage}

\section*{Spells - N}

\begin{minipage}{0.48\textwidth}
\subsubsection*{Nauseating Poison}

1st-level necromancy

Classes: Druid, Occultist, Warlock 
Casting Time: 1 bonus action 
Range: Self 
Components: V, S 
Duration: 1 round

You shroud your hand, a weapon you are holding, or a natural weapon in dark ichorous miasma. The next time you hit a creature with a melee attack (including a melee spell attack) before the start of your next turn, the attack deals an extra 1d12 poison damage and the target must succeed on a Constitution saving throw or be poisoned until the end of your next turn.

The spell ends after dealing damage, or at the start of your next turn, whichever occurs first.

\begin{itemize}
  \item Spell: Nauseating Poison
\end{itemize}
\end{minipage}\hfill
\begin{minipage}{0.48\textwidth}
\subsubsection*{Nullify Effect}

2nd-level psionic

Casting Time: 1 reaction, which you take when you are forced to make an Intelligence, Wisdom, or Charisma saving throw 
Range: Self 
Components: S 
Duration: 1 round

You gain advantage on Intelligence, Wisdom, and Charisma saving throws (Including the triggering save) until the start of your next turn. You also gain resistance to Psychic and Force damage until the start of your next turn.

At Higher Levels. When you cast this spell using a spell slot of 4th level or higher, you can roll an additional d20 as part of your advantage roll. If you cast this at the 5th level or higher, it grants immunity to Psychic and Force damage for the duration.

\begin{itemize}
  \item Spell: Nullify Effect
\end{itemize}
\end{minipage}

\section*{Spells - O}

\begin{minipage}{0.48\textwidth}
\subsubsection*{Orbital Stones}

4th-level transmutation

Classes: Druid, Sorcerer, Wizard 
Casting Time: 1 action 
Range: Self 
Components: V, S 
Duration: Concentration, up to 1 minute

You lift three inanimate Small- or Medium-sized rocks or similar objects from within 10 feet of you, causing them to defy gravity and slowly circle you. With all three stones orbiting, you have three quarters cover. With at least one stone remaining, you have half cover.

As a bonus action while at least one stone remains in orbit, you can magically fling a stone at target within 60 feet. Make a ranged spell attack roll. On hit, the target takes 3d10 bludgeoning damage and is knocked backward 5 feet.

\begin{itemize}
  \item Spell: Orbital Stones
\end{itemize}
\end{minipage}\hfill
\begin{minipage}{0.48\textwidth}

\end{minipage}

\section*{Spells - P}

\begin{minipage}{0.48\textwidth}
\subsubsection*{Poison Dart}

2nd-level transmutation

Classes: Occultist, Sorcerer, Warlock, Wizard 
Casting Time: 1 action 
Range: 60 ft 
Components: V, S 
Duration: Instantaneous

You conjure a dart of pure poison and hurl it at a creature you can see within range. Make a ranged spell attack. On a hit, the target takes 3d12 poison damage and must succeed a Constitution saving throw or become poisoned until the start of your next turn.

At Higher Levels. When you cast this spell using a spell slot of 3rd level or higher, the damage increases by 1d12 for each slot level above 2nd.

\begin{itemize}
  \item Spell: Poison Dart
\end{itemize}

\subsubsection*{Poison Puff}

4th-level transmutation

Classes: Druid, Occultist, Warlock, Wizard 
Casting Time: 1 action 
Range: Self (30-foot cone) 
Components: V, S 
Duration: 1 round

You exhale a cloud of poison that magically expands to fill a 30-foot cone. Creatures in that area must make a Constitution saving throw. On a failure, they take 4d12 poison damage and become poisoned until the start of their next turn. On a success, the target takes half as much damage and is not poisoned.

The area is lightly obscured until the start of your turn, and any creature that ends their turn within the area takes 2d4 poison damage.

\begin{itemize}
  \item Spell: Poison Puff
\end{itemize}

\subsubsection*{Pressure Cutter}

5th-level conjuration

Classes: Sorcerer, Wizard 
Casting Time: 1 action 
Range: Self (60-foot line) 
Components: V, S 
Duration: Instantaneous

You unleash a blast of highly pressurized water in a 60-footlong, 15-foot-wide line, slashing through everything in its path. Each creature in the line must make a Dexterity saving throw, taking 10d6 slashing damage on a failure. On a successful save, a creature takes half as much damage.

At Higher Levels. When you cast this spell using a spell lot of 6th level or higher, the damage increases by 1d6 for each level above 5th.

\begin{itemize}
  \item Spell: Pressure Cutter
\end{itemize}
\end{minipage}\hfill
\begin{minipage}{0.48\textwidth}
\subsubsection*{Prismatic Weapon}

3rd-level transmutation

Classes: Inventor, Sorcerer, Wizard 
Casting Time: 1 action 
Range: Touch 
Components: V, S 
Duration: Concentration, up to 1 hour

A weapon you touch is infused with elemental power, becoming a magical weapon. Choose of the following damage types: acid, cold, fire, lightning, poison, or thunder. The weapon deals 1d6 damage of the chosen weapon type, and if it doesn’t already have a bonus to attack and damage, it gains a +1 bonus to attack and damage rolls.

At Higher Levels. When you cast this spell with a 4th- or 5th-level spell slot, the damage increases by 1d6 (to 2d6). When you use a spell slot of 6th level or higher, the damage increases by 2d6 (to 3d6).

\begin{itemize}
  \item Spell: Prismatic Weapon
\end{itemize}

\subsubsection*{Psychic Drain}

2nd-level psionic

Casting Time: 1 action 
Range: 60 feet 
Components: S 
Duration: Instantaneous

You draw on the psychic energy of another creature you can see to sustain yourself. The target must make a Charisma saving throw. On failure it takes 3d8 psychic damage and you gain temporary hit points equal to half the amount of psychic damage dealt and regain one expended psi point.

At Higher Levels. When you cast this spell using a spell slot of 3rd level or higher, the spell deals an extra 1d8 psychic damage.

\begin{itemize}
  \item Spell: Psychic Drain
\end{itemize}
\end{minipage}

\section*{Spells - R}

\begin{minipage}{0.48\textwidth}
\subsubsection*{Rain of Spiders}

3rd-level conjuration

Classes: Druid, Occultist, Sorcerer, Warlock, Wizard 
Casting Time: 1 bonus action 
Range: 60 feet 
Components: V, S, M (a spider leg) 
Duration: Concentration, up to 1 minute

A vertical column of spiders begins to rain down in 20 foot radius, 40 foot high cylinder, centered on a location you specify. A Swarm of Spiders (Basic Rules , pg. 391) descends onto each creature within the cylinder when the spell is cast.

This swarm is considered to be climbing on the target creature and moves with it, even if they leave the affected area, and takes its turn immediately after that creature's turn. A creature can make use its action to attempt to remove the spiders, making a Strength (Athletics) or Dexterity (Acrobatics) check against the spell save DC of the caster.

The swarm uses the caster's spell attack modifier when attacking (if it is higher than their attack modifier). A swarm will attack the creature it fell on if it can, or move to chase the creature if it has been knocked off of them. Any spiders that remain when the spell ends disappear.

\begin{itemize}
  \item Spell: Rain of Spiders
\end{itemize}

\subsubsection*{Repair}

4th-level transmutation

Classes: Inventor 
Casting Time: 1 action 
Range: Touch 
Components: V, S 
Duration: Instantaneous

You touch a construct or inanimate object, causing it to regain 10d6 hit points. This causes any parts or material that has broken away from the construct or object to reattach, repairing it to the condition it was in before losing those hit points.

If the construct or object’s damaged state is the result of age, you can instead repair to the condition it was in 10d6 years ago, if it was previously in a better condition during that time (the condition can only improve or not change).

At Higher Levels. The hit points restored increases by 2d6 (or the years restored) for each slot above 4th.

\begin{itemize}
  \item Spell: Repair
\end{itemize}
\end{minipage}\hfill
\begin{minipage}{0.48\textwidth}
\subsubsection*{Returning Weapon}

1st-level transmutation

Classes: Inventor 
Casting Time: 1 action 
Range: Self 
Components: V, S 
Duration: 24 hours

You touch a weapon granting it the thrown 20/60 property. If it already has the thrown property, its range increases by 20/60. It also gains the "returning" property. After being thrown it automatically reappears in the thrower’s hand.

\begin{itemize}
  \item Spell: Returning Weapon
\end{itemize}

\subsubsection*{Rotting Curse}

1st-level necromancy

Classes: Occultist 
Casting Time: 1 action 
Range: 60 feet 
Components: V, S, M (something from the target creature (such as blood, hair, or scales) which the spell consumes) 
Duration: Concentration, up to 1 minute

You inflict a rotting decay on a creature, causing it to to begin to rot. For the duration of the spell, every time the creature takes damage, it takes an extra 1d4 necrotic damage, and the effect of all healing on the creature is reduced by half. The target creature has disadvantage on any Charisma checks for social interaction during the effect of the spell.

\begin{itemize}
  \item Spell: Rotting Curse
\end{itemize}
\end{minipage}

\section*{Spells - S}

\begin{minipage}{0.48\textwidth}
\subsubsection*{Seeking Orb}

2nd-level evocation

Classes: Sorcerer, Wizard 
Casting Time: 1 action 
Range: 5 feet 
Components: V, S 
Duration: Concentration, up to 1 minute

You create a Tiny orb of pure arcane energy that hovers within range, and designate a target creature within 120 feet. For the duration of the spell, at the end of each of your turns, the orb grows larger and moves 30 feet directly toward the creature. If the orb reaches the target, it will detonate dealing 6d4 force damage and an extra 1d4 damage for each round since you cast the spell to the target. The spell ends after it deals damage. If the orb doesn't reach the target before the spell ends, it fades away without dealing damage.

\begin{itemize}
  \item Spell: Seeking Orb
\end{itemize}

\subsubsection*{Seeking Projectile}

1st-level transmutation

Classes: Inventor, Ranger 
Casting Time: 1 action 
Range: Touch 
Components: V, S, M (a piece of ammunition or weapon with the thrown property worth at least 1 cp) 
Duration: Concentration, up to 10 minutes

You touch a piece of ammunition or weapon with the thrown property imbuing it with the property of seeking its target. When a ranged attack roll is made with that weapon, the attack roll can add your spell casting modifier to the value on the dice. If that makes the value on the die a 20 or more, the attack is a critical hit as if a 20 was rolled. After making the attack roll, the spell ends.

\begin{itemize}
  \item Spell: Seeking Projectile
\end{itemize}

\subsubsection*{Shockwave}

5th-level psionic

Casting Time: 1 action 
Range: Self (30-foot radius) 
Components: S 
Duration: Instantaneous

You unleash a burst of telekinetic force in all directions. Each creature of your choice within 30 feet of you is knocked 5 feet back and must succeed on a Strength saving throw or take 8d6 damage bludgeoning damage and be knocked prone. A creature that succeeds on its saving throw takes half as much damage and isn't knocked prone.

\begin{itemize}
  \item Spell: Shockwave
\end{itemize}

\subsubsection*{Sky Burst}

5th-level evocation

Classes: Druid, Sorcerer, Wizard 
Casting Time: 1 action 
Range: 120 feet 
Components: V, S 
Duration: Instantaneous

Five bolts of lightning strike five points of your choice that you can see within range. Each creature within 5 feet of the chosen points must make a Dexterity saving throw. A creature takes 4d12 lightning damage on a failed save, or half as much on a successful one. A creature in the area of more than one lightning burst is affected only once.

At Higher Levels. When you cast this spell using a spell slot of 6th level or higher, you can call down an additional bolt of lightning targeting another point within range for each slot level above 5th.

\begin{itemize}
  \item Spell: Sky Burst
\end{itemize}

\subsubsection*{Sonic Shriek}

5th-level evocation

Classes: Bard, Occultist, Sorcerer, Wizard 
Casting Time: 1 action 
Range: Self (120-foot cone) 
Components: V, S 
Duration: Instantaneous

You emit a sonic blast covering a massive area. Each creature in a 120-foot cone must make a Constitution saving throw. On a failed save, a creature takes 6d8 thunder damage. On a successful save, a creature takes half as much damage. A creature automatically succeed on its saving throw if it is more than 60 feet from you.

At Higher Levels. When you cast this spell using a spell slot of 6th level or higher, the damage increases by 1d8 for each slot level above 5th.

\begin{itemize}
  \item Spell: Sonic Shriek
\end{itemize}

\subsubsection*{Spatial Manipulation}

5th-level conjuration

Classes: Sorcerer, Wizard 
Casting Time: 1 action 
Range: Self (120-ft radius) 
Components: V, S 
Duration: Instantaneous

You can swap the position two creatures you can see within range. An unwilling creature can make a Charisma saving throw, preventing the swap on success.

At Higher Levels. When you cast this spell using a spell slot of 6th level or higher, you can swap an additional set of creatures of each level about 5th.

\begin{itemize}
  \item Spell: Spatial Manipulation
\end{itemize}

\subsubsection*{Spider Bite}

3rd-level transmutation

Classes: Druid, Occultist, Warlock, Wizard 
Casting Time: 1 action 
Range: Touch 
Components: V, S 
Duration: Instantaneous

You prick a target with a tiny magical fang of venom. Make a melee spell attack against a creature within reach. On a hit, the target takes 4d12 poison damage and must succeed on a Constitution saving throw or becoming poisoned for 1 minute. At the end of each of its turns, the target can make another Constitution saving throw. On a success, the target is no longer poisoned.

If you miss your melee attack roll, you can concentrate (as if concentrating on a spell) to maintain the attack for another attempt until the end of your next turn. (You may make subsequent attempts until you hit or lose concentration)

At Higher Levels. When you cast this spell using a spell slot of 4th level or higher, the damage increases by 1d12 for each slot level above 3rd.

\begin{itemize}
  \item Spell: Spider Bite
\end{itemize}
\end{minipage}\hfill
\begin{minipage}{0.48\textwidth}
\subsubsection*{Star Dust}

2nd-level evocation

Classes: Sorcerer, Wizard 
Casting Time: 1 action 
Range: Self (30-foot cone) 
Components: V, S 
Duration: Instantaneous

You evoke a burst of brilliant particles of force energy sweeping out in a 30-cone originating from you. Creatures in the radius take 3d4 force damage and the next attack roll made against them before the start of your next turn has advantage.

\begin{itemize}
  \item Spell: Star Dust
\end{itemize}

\subsubsection*{Stinging Swarm}

4th-level conjuration

Classes: Druid, Occultist, Warlock, Wizard 
Casting Time: 1 action 
Range: 60 ft 
Components: V, S 
Duration: Concentration, up to 1 minute

You conjure a magical swarm of flying insects that fill a 5 foot cube within range. For the duration of the spell, the swarm is magically replenished and can't be destroyed. As a bonus action, you can direct the swarm to move up to 30 feet. If the swarm enters another creature's space, it stops and swarms them, stinging repeatedly, and can't be moved until the start of your next turn. The creature takes 2d4 piercing damage and must make a Constitution saving throw, taking 2d12 poison damage on failure.

\begin{itemize}
  \item Spell: Stinging Swarm
\end{itemize}

\subsubsection*{Stone Fist}

1st-level transmutation

Classes: Druid, Occultist, Sorcerer, Warlock, Wizard 
Casting Time: 1 action
 Range: Self 
Components: V, S 
Duration: 1 round

You turn your hand and forearm (or similar appendage) to stone until the start of your next turn. As part of casting the spell, you can make a melee spell attack against one creature you can reach. On a hit, the target takes 2d10 bludgeoning damage.

Until the start of your next turn, you can use your reaction when you take slashing or piercing damage from an attack to gain resistance to damage from that attack.

At Higher Levels. When you cast this spell using a spell slot of 2nd level or higher, the damage increases by 1d10 for each slot level above 1st.

\begin{itemize}
  \item Spell: Stone Fist
\end{itemize}

\subsubsection*{Suffocate}

4th-level transmutation

Classes: Occultist, Sorcerer, Wizard 
Casting Time: 1 action 
Range: 60 feet 
Components: V, S 
Duration: Concentration, up to 1 minute

You create a whirling sphere of air around a creature that causes them to struggle to breathe. The target must make a Constitution saving throw. On a failure, the target loses 5d8 hit points due to lack of air, has disadvantage on all ability checks, and can't speak. On a success, the target takes half as much damage and suffers no other effects. For the duration, as an action, you can force the creation to make a saving throw against the ability again.

If a target fails their saving throw against this spell 3 times in a row, they become incapacitated until they succeed on a save or the spell ends. If you don't use your action to force the target to make a save, it counts as a success.

A creature that doesn't need to breathe is unaffected by this spell.

\begin{itemize}
  \item Spell: Suffocate
\end{itemize}

\subsubsection*{Summon Horror}

4th-level conjuration

Classes: Occultist, Warlock, Wizard 
Casting Time: 1 action 
Range: 90 feet 
Components: V, S, M (a book with an ornate cover filled with the records of madmen worth at least 400 gp) 
Duration: Concentration, up to 1 hour

You call forth a twisted horror. It manifests in an unoccupied space that you can see within Range. This manifested form uses the Horror Spirit stat block. When you cast the spell, choose Star, Sea, or Void. The creature takes on elements of the selected type, which determine certain traits in its stat block. The creature disappears when it drops to 0 hit points or the spell ends.

The creature is an ally to you and your companions. In combat, the creature shares your initiative count, but it takes its turn immediately after yours. It obeys your verbal commands (no action required by you). If you don't issue any, it takes the Dodge action and uses its move to avoid danger.

At Higher Levels. When you cast this spell using a spell of 5th level or higher, use the higher level whenever the spell's level appears in the stat block.

\begin{itemize}
  \item Spell: Summon Horror
\end{itemize}

\subsubsection*{Summon Ooze}

1st-level conjuration

Classes: Occultist, Wizard 
Casting Time: 1 action 
Range: 30 feet 
Components: V, S, M (a gold vial worth at least 100 gp) 
Duration: Concentration, up to 1 hour

You call forth a magical ooze. It manifests in an unoccupied space that you can see within range. This corporeal form uses the Ooze Spirit stat block. When you cast the spell, choose from Green, Red, or Yellow. The creature resembles the creature of your choice, which determines certain traits in its stat block. The creature disappears when it drops to 0 hit points or when the spell ends.

The creature is an ally to you and your companions. In combat, the creature shares your initiative count, but it takes its turn immediately after yours. It obeys your verbal commands (no action required by you). If you don't issue any, it takes the Dodge action and uses its move to avoid danger.

At Higher Levels. When you cast this spell using a spell of 2nd level or higher, use the higher level whenever the spell's level appears in the stat block.

\begin{itemize}
  \item Spell: Summon Ooze
\end{itemize}
\end{minipage}

\section*{Spells - T}

\begin{minipage}{0.48\textwidth}
\subsubsection*{Terrifying Visions}

1st-level enchantment

Classes: Bard 
Casting Time: 1 action 
Range: 60 feet 
Components: V 
Duration: Instantaneous

You instill a vision of terrifying hallucinations into the mind of a target you can see. The target must make a Wisdom saving throw. On failure, it takes 3d6 psychic damage and must immediately use its reaction to move to move it's movement speed directly away from you. This movement does not force the creature to move into any hazard or take movements that cause it to take damage (such as jumping off a cliff or moving into a spell effect). On a successful save, the target takes half as much damage and isn't forced to move.

At Higher Levels. When you cast this spell using a spell slot of 2nd level or higher, the damage increases by 1d6 for each slot level above 1st.

\begin{itemize}
  \item Spell: Terrifying Visions
\end{itemize}

\subsubsection*{Thunder Note}

Evocation cantrip

Classes: Bard 
Casting Time: 1 action 
Range: 60 feet 
Components: V, S 
Duration: Instantaneous

You emit a crashing bang with a localized point of intensity targeting a creature within range. The target must succeed on a Constitution saving throw or take 1d8 thunder damage and become deafened until the start of their next turn. Constitution saving throws to maintain concentration on spells triggered by this damage are made with disadvantage.

This spell's damage increases by 1d8 when you reach 5th level (2d8), 11th level (3d8), and 17th level (4d8).

\begin{itemize}
  \item Spell: Thunder Note
\end{itemize}

\subsubsection*{Thunder Pulse}

3rd-level evocation

Classes: Bard, Sorcerer, Wizard 
Casting Time: 1 action 
Range: Self (15-foot cone) 
Components: V, S 
Duration: Concentration, up to 1 minute

You gather sonic energy and can expel as a shockwave in a 15-foot cone. Each creature in that area must make a Constitution saving throw. On a failed save, a creature takes 3d8 thunder damage is knocked 10 feet away. On a successful save, the creature takes half as much damage and not being knocked away. You can create a new shockwave as your action on subsequent turn until the spell ends.

At Higher Levels. When you cast this spell using a spell slot of 3rd level or higher, the damage increases by 1d8 for each slot level above 2nd.

\begin{itemize}
  \item Spell: Thunder Pulse
\end{itemize}

\subsubsection*{Thunder Punch}

1st-level evocation

Classes: Sorcerer, Wizard 
Casting Time: 1 action 
Range: Touch 
Components: V, S 
Duration: Instantaneous

You charge your hand (or similar appendage) with thunder power. Make a melee spell attack against the target. On a hit, there is a thunderous crash audible from up to 300 feet of you and the target takes 3d8 thunder damage, and is knocked 10 feet away from you.

At Higher Levels. When you cast this spell using a spell slot of 2nd level or higher, the damage increases by 1d8 for each slot level above 1st.

\begin{itemize}
  \item Spell: Thunder Punch
\end{itemize}

\subsubsection*{Thunderburst Mine}

2nd-level abjuration

Classes: Inventor 
Casting Time: 1 minute 
Range: Touch 
Components: V, S, M (any Tiny nonmagical item, which is destroyed by the activation of the spell) 
Duration: 8 hours

You can set a magical trap by infusing explosive magic into an item. You can set this item to detonate when someone comes within 5 feet of it, or by a verbal command using your reaction (one or more mines can be detonated).

When the magic trap detonates, each creature in a 10-footradius sphere centered on the item must make a Constitution saving throw. A creature takes 3d8 thunder damage on a failed save, or half as much damage on a successful one. If a creature is in the area of effect of more than one thunderburst mine during a turn, they take half damage from any mines beyond the first.

A magical mine must be set 5 feet or more from another mine, and can't be moved once placed; any attempt to move it results in it detonating unless the casterer that set it disarms it with an action.

\begin{itemize}
  \item Spell: Thunderburst Mine
\end{itemize}
\end{minipage}\hfill
\begin{minipage}{0.48\textwidth}
\subsubsection*{Tornado}

5th-level transmutation

Classes: Druid, Sorcerer, Wizard 
Casting Time: 1 action 
Range: 120 feet 
Components: V, S 
Duration: Concentration, up to 1 minute

A whirling tornado erupts, filling a 20-foot-radius, 40-foot-high cylinder centered on a point within range.

Any creature that starts its turn within the tornado must make a Strength saving throw. On a failed save, the creature takes 4d8 bludgeoning damage and is pushed 10 feet away and 40 feet up. On a successful save, the creature takes half as much damage and isn't pushed.

As a bonus action, you can move the tornado up to 30 feet in any direction. Any ranged weapon attack against a target within 20 feet of the tornado has disadvantage, and any ranged attack that passes through it automatically misses.

\begin{itemize}
  \item Spell: Tornado
\end{itemize}

\subsubsection*{Translocating Shot}

4th-level conjuration

Classes: Inventor, Ranger, Wizard 
Casting Time: 1 bonus action 
Range: 5 feet 
Components: V, S, M (a piece of ammunition worth at least 1 cp) 
Duration: Concentration, up to 1 minute

You magically bind a willing creature within range into a piece of ammunition. When the piece of ammunition is fired, the creature bound to the piece of ammunition is teleported to the target destination. You can fire the ammunition at a creature, object, or point within the normal range of the weapon. When attacking a creature or object, the target is teleported to within 5 feet of the target hit or miss.

When you cast this spell, if you cast it a Large or larger piece of ammunition, you can bind up to 4 creatures to the piece of ammunition.

At Higher Levels. When you using 6th-level slot or higher, you can cast it on a Huge piece of ammunition, binding up to nine creatures to the piece of ammunition.

\begin{itemize}
  \item Spell: Translocating Shot
\end{itemize}

\subsubsection*{Trary's Terrific Transposition}

3rd-level conjuration (ritual)

Classes: Inventor, Occultist, Wizard 
Casting Time: 10 minutes 
Range: Self (10-foot-radius hemisphere) 
Components: V, S, M (an ornate brass key worth at least 100 gp, and a satchel or bag worth at least 1 sp) 
Duration: Instantaneous

You weave an enchantment that conjures compresses all objects of your choice within range into a the satchel or bag used in casting the spell. The contents become harmlessly compressed and stored in an magical state of miniaturized suspension within the container. The weight of miniaturized stored items is the weight of the item divided by one hundred. The bag can store all items that fit in the radius when the spell is cast, but can't store any individual item larger than Medium. Items can't be individually removed from the bag, but the process can be reversed by casting the spell again, at which point all items are deposited from the bag in the arrangement they were before being stored.

If the bag is destroyed or placed into an interdimensional space, the contents of the bag are instantly emptied onto the ground in a chaotic manner, each item taking 4d4 force damage, but dealing no damage to anything else. If the bag is broken, the key used as a material in casting the spell breaks.

\begin{itemize}
  \item Spell: Trary's Terrific Transposition
\end{itemize}

\subsubsection*{Turbulent Warp}

3rd-level psionic

Casting Time: 1 action 
Range: 90 ft. 
Components: V 
Duration: Instantaneous

You teleport yourself to an unoccupied space you can see within range, leaving behind a spatial distortion. Each creature within 10 feet of the space you left must make a Charisma saving throw. On a failure, they take 5d4 force damage and are teleported to an empty space of your choice within 5 feet of where they were. On success they take half as much damage and are not teleported.

You can also teleport one willing creature of your size or smaller who is carrying gear up to its carrying capacity. The creature must be within 5 feet of you when you cast this spell, and there must be an unoccupied space within 5 feet of your destination space for the creature to appear in; otherwise, the creature is left behind.

At Higher Levels. When you cast this spell using a spell slot of 4th level or higher, the damage increases by 2d4 for each slot level above 3rd.

\begin{itemize}
  \item Spell: Turbulent Warp
\end{itemize}
\end{minipage}

\section*{Spells - U}

\begin{minipage}{0.48\textwidth}
\subsubsection*{Unburden}

1st-level transmutation

Classes: Inventor 
Casting Time: 1 action 
Range: Touch 
Components: V, S 
Duration: 1 hour

A creature you touch no longer suffers the penalties to its speed or to its Dexterity (Stealth) checks from wearing medium or heavy armor, and is no longer encumbered from carry weight unless it is carrying more than twice the weight that would encumber it.

\begin{itemize}
  \item Spell: Unburden
\end{itemize}

\subsubsection*{Unlocked Potential}

1st-level psionic

Casting Time: 1 action 
Range: 60 feet 
Components: S 
Duration: Concentration, up to 1 minute

You unlock the potential of a creature’s mind, allowing it to fully reach its limits. For the duration, once per turn the creature can add 1d4 to any attack roll, damage roll, or saving throw it makes. Each time it adds the extra 1d4, it takes 1 psychic damage as it pushes beyond its natural limitations.

\begin{itemize}
  \item Spell: Unlocked Potential
\end{itemize}
\end{minipage}\hfill
\begin{minipage}{0.48\textwidth}
\subsubsection*{Unstable Explosion}

2nd-level evocation

Classes: Sorcerer, Warlock, Wizard 
Casting Time: 1 action 
Range: 60 ft (10 ft radius) 
Components: V, S 
Duration: Instantaneous

You cause an unstable explosion to erupt at a point of your choice within range, rolling 3d6. For each die that rolls a 6, roll an additional d6 and the radius of the spell expands by 5 feet. Each creature within the final range of the spell must make a Dexterity saving throw. On a failed save, they take fire damage equal to the total value of the rolled dice. On a success the target, the target takes half as much fire damage.

At Higher Levels. When you cast this spell using a spell slot of 3rd level or higher, the damage increases by 1d6 for each slot level above 2nd.

\begin{itemize}
  \item Spell: Unstable Explosion
\end{itemize}
\end{minipage}

\section*{Spells - V}

\begin{minipage}{0.48\textwidth}
\subsubsection*{Vicious Vapors}

2nd-level transmutation

Classes: Druid, Occultist, Warlock, Wizard 
Casting Time: 1 action 
Range: 60 ft 
Components: V, S 
Duration: Concentration, up to 1 minute

You fill the air with poisonous vapors in a cube 5 feet on each side. A creature must make a Constitution saving throw when it enters the spell's area for the first time on their turn or starts its turn there. On a failed save, they take 1d12 poison damage and become poisoned until the end of their next turn. On a successful save, they take half as much damage and do not become poisoned.

You can move the cloud of vapors up to 20 feet as a bonus action during your turn.

\begin{itemize}
  \item Spell: Vicious Vapors
\end{itemize}

\subsubsection*{Vorpal Weapon}

5th-level transmutation

Classes: Inventor 
Casting Time: 1 action 
Range: Touch 
Components: V, S, M (a weapon worth at least 1 cp) 
Duration: Concentration, up to 1 hour

You touch a weapon and imbue it with power. Until the spell ends, the weapon becomes indescribably sharp, ignoring resistance to slashing damage, and gains the Siege property, dealing double damage to inanimate objects such as structures. If a weapon has a modifier of less than +3 to attack and damage rolls, its modifier becomes +3 to attack and damage rolls for the duration of the spell.

Additionally, if a critical strike of this weapon would leave a creature with less than 50 hit points, the target creature is decapitated, killing it.

\begin{itemize}
  \item Spell: Vorpal Weapon
\end{itemize}
\end{minipage}\hfill
\begin{minipage}{0.48\textwidth}
\subsubsection*{Vortex Blast}

3rd-level evocation

Classes: Druid, Occultist, Sorcerer, Wizard 
Casting Time: 1 Action 
Range: Self (30-foot cone) 
Components: V, S 
Duration: Instantaneous

You create a sudden violent vortex that blasts outwards in a 30-foot cone, tossing characters and objects within the area. Creatures in the area take 2d6 bludgeoning damage and must succeed a Strength saving throw or be knocked 20 feet backward and 40 feet upward.

At Higher Levels. When you cast this spell using a spell slot of 4th level or higher, the damage increases by 1d6 for each slot level above 3rd.

\begin{itemize}
  \item Spell: Vortex Blast
\end{itemize}
\end{minipage}

\section*{Spells - W}

\begin{minipage}{0.48\textwidth}
\subsubsection*{Water Blast}

1st-level conjuration

Classes: Druid, Occultist, Sorcerer, Warlock, Wizard 
Casting Time: 1 action 
Range: 30 feet 
Components: V, S 
Duration: Instantaneous

You conjure a ball of water before hurling it at a target. Make a ranged spell attack against the target. On a hit, the target takes 3d6 bludgeoning damage and if it is Large or smaller must make a Strength saving throw or be knocked prone.

At Higher Levels. When you cast this spell using a spell lot of 2nd level or higher, the damage increases by 1d6 for each level above 1st.

\begin{itemize}
  \item Spell: Water Blast
\end{itemize}

\subsubsection*{Water Cannon}

3rd-level evocation

Classes: Druid, Sorcerer, Wizard 
Casting Time: 1 action 
Range: Self (30-foot line) 
Components: V, S 
Duration: Instantaneous

You unleash a spout of water that blasts out in a 30-foot line that is 5 feet wide. Creatures in the area must make a Strength saving throw, or take 6d6 bludgeoning damage and be pushed to an open space at the end of the line away from you. If there is no open space to move to (for example they would move into a wall or another creature), they are pushed to the closest space and take an extra 2d6 bludgeoning damage and are knocked prone. On a successful save, they take half as much damage and are not pushed.

\begin{itemize}
  \item Spell: Water Cannon
\end{itemize}
\end{minipage}\hfill
\begin{minipage}{0.48\textwidth}
\subsubsection*{Windborne Weapon}

Transmutation cantrip

Classes: Druid, Sorcerer, Wizard 
Casting Time: 1 action 
Range: 150 feet 
Components: V, S, M (a piece of ammunition or weapon with the thrown property worth at least 1 cp) 
Duration: Instantaneous

Make an attack using this spell’s material component as part of the action used to cast this spell, turning it into a ranged spell attack. This attack ignores cover. On hit, it deals 1d8 damage of the weapon’s damage type.

This spell’s damage increases by 1d8 when you reach 5th level (2d8), 11th level (3d8), and 17th level (4d8).

\begin{itemize}
  \item Spell: Windborne Weapon
\end{itemize}
\end{minipage}

\chapter{Chapter 5: Monsters}

\section*{NPCs}

The following are templates for psion and inventor NPCs (Non-Player Characters). You can build NPCs using the Player Character rules, or you can use these templates for a quick and combat-ready version. Their abilities as presented here are streamlined to generally only the most relevant features, but you can add any abilities from the class mechanics that you see fit for that character or their usage in your story.

The versions presented here tend to present a psion with only one psionic discipline for simplicity. It is likely many NPC psions simply only have one discipline, or you can add abilities from other templates to represent a mix-and-match.

Monster Statblock vs PC Character

Sheet Some GMs always use simplified monster statblocks for NPCs, and that is definitely an option. Personally as a GM, I use monster statblocks for enemies and temporary allies, eventually converting them to PC-style stat blocks if they will be present for a long time or controlled by players.

\begin{minipage}{0.48\textwidth}
\begin{itemize}
  \item NPC: Animator
  \item NPC: Ascendant Master
  \item NPC: Cryokinetic
  \item NPC: Doctor
  \item NPC: Elemental Adept
  \item NPC: Flesh Abomination
  \item NPC: Gadgeteer
  \item NPC: Gunslinger
\end{itemize}
\end{minipage}\hfill
\begin{minipage}{0.48\textwidth}
\begin{itemize}
  \item NPC: Golem Mechanic
  \item NPC: Ironwrought Golem
  \item NPC: Pyrokinetic
  \item NPC: Shaper
  \item NPC: Telepath
  \item NPC: Thunderer
  \item NPC: Warsmith
\end{itemize}
\end{minipage}

\section*{Monster Section}

The following is a list of new monsters and NPCs. These are designed to fit a variety of roles and provide a diverse cast of new threats to challenge your players, but also to tie into the content and themes of the book.

Some of these monsters are designed to be solo encounters. These monsters present a significant and multifaceted fight for their CR.

\subsection*{Solo Threats (by CR)}

\begin{itemize}
  \item Adaptive Slime (CR 5)
  \item Lurking Maw (CR 9)
  \item Unearthed Destroyer (CR 13)
  \item Ascendant Master (CR 16)
  \item Clockwork Drake (CR 16)
  \item Ancient Sentinel (CR 19)
\end{itemize}

The rest of the monsters are intended to be encounter building tools. Some have easy combinations: for example, a Silver Guardian (CR 9), a handful of Silver Soldiers (CR 6), and a few Silver Sentries (CR 1) will build a challenging, dynamic, and thematic encounter, and can each stand on their own as a minor threat—just probably not a campaign “BBEG” encounter. Likewise, the Uncaged monsters or the whole writhing mass of eldritch horrors are designed to naturally work together in fun (and challenging) ways.

Other instances provide simple, flexible tools that can be dropped into encounters. I particularly wanted to flesh out the roster of easy-to-use NPCs.

\subsection*{Design Philosophy}

These monsters are built to the CR specifications of the base game, but should generally be expected to punch slightly above their CR. This is because each monster tends to be designed with some degree of synergy inside its own skills that is not well represented by CR.

While the solo monsters (listed above) endeavor to provide a handful of unique challenges across the fight, all the monsters aim to bring at least one memorable encounter to bear.

\subsection*{Class Abilities \& Psionics}

The class abilities and psionics that appear in the monsters are simplified for convenience. Monsters use recharge mechanics rather than psi points, for example. This is merely to make running monsters easier. If you prefer to use psi points and more integrated class mechanics, you can generally treat a monster with psionics as a psion of a level equal to its CR, and with psi points to match. This will make monsters stronger than intended, but not by a large margin.

\subsection*{Monsters by CR}

\begin{minipage}{0.48\textwidth}
\subsubsection*{CR 1/4}

\begin{itemize}
  \item Golem Mechanic
\end{itemize}

\subsubsection*{CR 1/2}

\begin{itemize}
  \item Rampaging Imprint
  \item Ironwrought Worker
\end{itemize}

\subsubsection*{CR 1}

\begin{itemize}
  \item Elemental Adept
  \item Silver Sentry
\end{itemize}

\subsubsection*{CR 2}

\begin{itemize}
  \item Doctor
  \item Telepath
  \item Thunderer
\end{itemize}

\subsubsection*{CR 3}

\begin{itemize}
  \item Fear Eater
\end{itemize}

\subsubsection*{CR 4}

\begin{itemize}
  \item Nightmare Stalker
  \item Ironwrought Soldier
\end{itemize}

\subsubsection*{CR 5}

\begin{itemize}
  \item Aberrant Abstraction
  \item Adaptive Slime
  \item Animator
  \item Cryokinetic
  \item Gadgeteer
  \item Gunslinger
  \item Flesh Abomination
  \item Madness Regurgitator
  \item Pyrokinetic
  \item Shaper
  \item Silver Stalker
  \item Uncaged
  \item Ironwrought Golem
\end{itemize}
\end{minipage}\hfill
\begin{minipage}{0.48\textwidth}
\subsubsection*{CR 6}

\begin{itemize}
  \item Screaming Runner
  \item Silver Soldier
  \item Uncaged (Winged)
  \item Warsmith
\end{itemize}

\subsubsection*{CR 7}

\begin{itemize}
  \item Mind Lurker
\end{itemize}

\subsubsection*{CR 9}

\begin{itemize}
  \item Lurking Maw
  \item Uncaged Warden
\end{itemize}

\subsubsection*{CR 10}

\begin{itemize}
  \item Silver Guardian
\end{itemize}

\subsubsection*{CR 13}

\begin{itemize}
  \item Unearthed Destroyer
\end{itemize}

\subsubsection*{CR 16}

\begin{itemize}
  \item Clockwork Drake
  \item Ascendant Master
\end{itemize}

\subsubsection*{CR 19}

\begin{itemize}
  \item Ancient Sentinel
\end{itemize}
\end{minipage}

% [Image Inserted Manually]

\section*{Ancient Sentinel}

Legend speaks of ancient constructs built to hold back a terrible calamity of untold horrors from beyond. Posted as eternal guards over ancient lost portals, these construct sentinels were thought to be immune to whispers of madness that seeped through—a theory that has proved tragically misguided. Now twisted by the forces beyond, the sentinels have warped into eldritch engines of destruction, completely mad and not truly aware of their own actions.

Awakening from ancient, forgotten sites, often long-lost vaults and shrines, these can appear without warning and wreak terrible destruction in a maddened rampage. These sentinels are massive creatures from an ancient era, towering 30 feet tall, and crushing anything built to a human scale with ease.

\subsection*{Deployment}

The ancient sentinel is intended primarily as a solo encounter, and particularly excels as an event. A rampaging, gigantic mechanical elephant is most dramatic when it has plenty of buildings—and maybe some villagers—to trample, and for the players to save. It is a siege monster that is exceptionally good at engaging many targets simultaneously. It can even enter a campaign a little earlier than its CR might suggest as a set piece encounter where the PCs are rallying a defense against it.

If you do need some more chaos, you can have the ancient sentinel be the harbinger to a wave of psionic monsters that have broken through in its wake. Its immunity to psychic damage means it is largely unfazed by many of their effects, making this a devastating pairing.

\subsection*{Simple Script}

Here’s a simple script that you can follow to get an effective use from an ancient sentinel:

\begin{itemize}
  \item The sentinel should always move when able, ideally trampling any smaller creatures in its path and smashing buildings if possible. Move for maximum destruction unless it has a specific objective.
  \item Move towards something that can be smashed.
  \item During the advance, unleash a Bladed Trunk attack, moving the target into or out of its path.
  \item Move through a target using Trampling Advance.
  \item If it knocks a target prone or is moving through its last target, use its stomp attack.
  \item Lastly, it should lash out with its tentacles against any target that is attacking it, either grabbing the attackers or flinging things at them.
  \item If a target is bothering it with ranged attacks, it can fling creatures or objects it is grappling at the target.
\end{itemize}

\subsection*{Considerations}

While the sentinel lacks ranged abilities, it can use Fling as a ranged offensive tool quite effectively, if needed, simply by picking up objects and throwing them. While its speed isn’t particularly high, its Unstoppable Advance trait and Unstoppable Trample legendary action mean its effective movement can be quite substantial.

The sentinel may not care about the party at all. If they are insufficiently powerful to stop it, it may simply rampage through a city before wandering off to rampage elsewhere. Only if players prove a danger to it will it focus on them. While a very challenging monster, the dynamic nature of the fight will come from what the sentinel is destroying.

\subsection*{Ancient Sentinel}

\begin{StatblockMinipage}
\MonsterName{Ancient Sentinel}{Gargantuan construct, chaotic evil}

\MonsterHeader
	{18 (natural armor)}
	{297 (17d20 + 119) [Max: 459]}
	{40 ft.}
	{26 (+8)}{6 (-2)}{24 (+7)}{7 (-2)}{16 (+3)}{13 (+1)}

\StatEntry{Saving Throws}{Str +14, Con +13, Wis +9, Cha +7}
\StatEntry{Skills}{Athletics +14, Perception +9}
\StatEntry{Damage Resistances}{bludgeoning, piercing, and slashing from nonmagical attacks that aren't adamantine}
\StatEntry{Damage Immunities}{poison, psychic}
\StatEntry{Condition Immunities}{charmed, exhaustion, frightened, paralyzed, petrified, poisoned, prone}
\StatEntry{Senses}{blindsight 60 ft., darkvision 120 ft., passive Perception 19}
\StatEntry{Languages}{understands Deep Speech but can't speak}
\StatEntry{Challenge}{19 (22,000 XP)}
\StatEntry{Proficiency Bonus}{+6}
\RedThickMidrule

\MonsterSection{Traits}
\MonsterTrait{Immutable Form.}{The sentinel is immune to any spell or effect that would alter its form.}
\MonsterTrait{Legendary Resistance (3/Day).}{If the sentinel fails a saving throw, it can choose to succeed instead.}
\MonsterTrait{Magic Resistance.}{The sentinel has advantage on saving throws against spells and other magical effects.}
\MonsterTrait{Scale Immunity.}{The sentinel has resistance to the damage of any spell that doesn't target an area of effect, and it suffers no other conditions or effects from them.}
\MonsterTrait{Siege Monster.}{The sentinel deals double damage to objects and structures.}
\MonsterTrait{Trampling Advance.}{Any Large or smaller creature in a space the sentinel passes through must succeed on a DC 16 Dexterity saving throw or take 13 (2d12) bludgeoning damage and be knocked prone.}
\MonsterTrait{Unstoppable Advance.}{The sentinel's advance cannot be stopped. If the sentinel's speed would be reduced to zero, it is reduced to half its speed instead. Any obstacle or building that would stop its advance takes 39 (6d12) bludgeoning damage. The sentinel treats magical barriers that would stop its movement as difficult terrain.}
\MonsterTrait{Whispering Madness.}{The sentinel is steeped in eldritch madness. If a creature starts its turn within 5 feet of the sentinel or while grappled by it, the creature must succeed on a DC 16 Wisdom saving throw or take 10 (3d6) psychic damage and roll a d4, suffering an additional condition until the start of its next turn, as noted in the following table.
	\vspace{4pt}
	\begin{tabular}{c l}
		\toprule
		d4 & Condition \\
		\midrule
		1-2 & Frightened \\
		3 & Blinded \\
		4 & Paralyzed \\
		\bottomrule
	\end{tabular}
\MonsterTrait{Eruption of Corruption.}{If the sentinel is bloodied (reduced to less than half of its maximum hit points), it unleashes a psionic blast immediately (no action required), violently discharging the eldritch corruption within it. All creatures within 60 feet of it must make a DC 16 Intelligence saving throw. On failed save, a creature takes 21 (6d6) psychic damage and becomes stunned until the end of its next turn. On successful save, the creature takes half as much damage and isn't stunned. The range of Whispering Madness then increases to 60 feet until the sentinel dies or recovers hit points above half its maximum.}

\MonsterSection{Actions}
\MonsterAction{Multiattack.}{The sentinel makes a bladed trunk attack, a stomp attack, and two tentacle attacks. It can replace each tentacle attack with one use of Fling.}
\MonsterAction{Bladed Trunk.}{Melee Weapon Attack: +14 to hit, reach 10 ft., one target. Hit: 27 (3d12 + 8) slashing damage, and the target must succeed on a DC 22 Strength saving throw or be pushed up to 10 feet in a direction of the sentinel's choice.}
\MonsterAction{Stomp.}{Melee Weapon Attack: +14 to hit, reach 5 ft., one target. Hit: 27 (3d12 + 8) bludgeoning damage.}
\MonsterAction{Tentacle.}{Melee Weapon Attack: +14 to hit, reach 20 ft., one target. Hit: 18 (3d6 + 8) bludgeoning damage, and the target is grappled (escape DC 16). Until this grapple ends, the target is restrained. The sentinel has four tentacles, each of which can grapple one target.}
\MonsterAction{Fling.}{One Large or smaller object held or creature grappled by the sentinel is thrown up to 60 feet in a direction of the sentinel's choice and knocked prone. If a thrown target strikes a solid surface, the target takes 3 (1d6) bludgeoning damage for every 10 feet it was thrown. If the target is thrown at another creature, that creature must succeed on a DC 16 Dexterity saving throw or take the same damage and be knocked prone.}

\MonsterSection{Legendary Actions}
\MonsterAction{The sentinel can take 3 legendary actions, choosing from the options below.}{Only one legendary action option can be used at a time and only at the end of another creature's turn. The sentinel regains spent legendary actions at the start of its turn.}
\MonsterAction{Tentacle Attack or Fling.}{The sentinel makes one tentacle attack or uses its Fling.}
\MonsterAction{Unstoppable Trample.}{The sentinel moves up to half its speed. This movement doesn't provoke opportunity attacks.}
\end{StatblockMinipage}

\begin{itemize}
  \item NPC: Ancient Sentinel
\end{itemize}

% [Image Inserted Manually]

\section*{Clockwork Drake}

Clockwork drakes, sometimes improperly called clockwork dragons (or informally called “metal danger birbs”), are flying, metal constructs of intricate machinery and magic. Most often created by wizards or inventors, they have a distrust of other magic users, and the ability to nullify magic through blasts of their own power. Typically serving as guardians, seekers, or shock troops, many that exist now have long since outlived the creators that developed their esoteric behavior.

\subsection*{Simple Script}

This is a simple script to run a clockwork drake on:

\begin{itemize}
  \item If the drake can target 2 or more creatures with its breath weapon, it will use its breath weapon and then fly up 40 feet from the ground, if possible.
  \item Between its turns, the drake will use its Arc legendary action if more than one creature is within range, and then its Wing Attack to position closer to a target that has drawn its ire.
  \item If the drake’s breath weapon doesn’t recharge at the start of its turn, it will use its movement to descend, use its Crushing Stomp to grapple a target, and lash out with its tail.
  \item The drake will use its Wing Attack to take to the air, its tail to attack the grappled target, and then drop that target when it perceives that such a fall would take the target out of the fight.
  \item If the drake can use no other action, it will attempt its Arcane Nullification against any harmful magical effect.
\end{itemize}

\subsection*{Tactics}

The clockwork drake is in many ways a variant of a dragon fight. It has high mobility, dangerous area of effect capabilities, and relies on its flight to get into the air and move around. The clockwork drake is an enemy of moderate intelligence and will generally attack creatures based on their perceived or demonstrated threat to it if it doesn’t have an objective target.

Clockwork drakes favor open areas and will generally avoid a fight in enclosed spaces without room to fly around unless forced to fight in such conditions by their objective or purpose.

\subsection*{Considerations}

A clockwork drake will pose a very large kill risk to players when deployed in such a fashion. The party’s main goal will be to kill it as fast as possible; otherwise, they will likely succumb to its brutal skill set. Don’t try to keep it alive longer than its shelf life, or it will likely succeed in defeating the players.

To increase the drake’s effectiveness, target ranged characters with its smash, grab, and drop combo. Not only will this eliminate the greatest threats, but the targets will also be most likely to fail the save and least able to operate effectively in the clockwork drake’s grasp.

\subsection*{Deployment}

A clockwork drake can serve as a solo encounter—providing a dangerous, high-stakes fight—but it will wither under heavy, direct fire. It can benefit from some supporting, lower-damage threats to absorb some incoming damage and distract its foes, but it’s extremely deadly when paired with other high-damage combatants. These drakes are best paired with a wizard or golemsmith that can give the fight some dynamic elements.

\subsection*{Clockwork Drake}

\begin{minipage}{0.48\textwidth}
Huge construct, unaligned

Armor Class 18 (natural armor) 
Hit Points: 225 (18d12 +108) [Max: 324] 
Speed 40 ft., fly 80 ft.

\begin{tabularx}{\textwidth}\toprule
{}XXXXX}
\midrule
STR & DEX & CON & INT & WIS & CHA \\
\midrule
25 (+7) & 8 (-1) & 23 (+6) & 12 (+1) & 15 (+2) & 8 (-1) \\
\midrule
\end{tabularx}

Saving Throws Dex +4, Con +11, Cha +4 
Skills Arcana +6, Athletics +12, Perception +7 
Damage Resistances fire, lightning, bludgeoning, piercing, and slashing from nonmagical attacks that aren't adamantine 
Damage Immunities poison, psychic 
Condition Immunities charmed, exhaustion, frightened, paralyzed, petrified, poisoned 
Senses blindsight 60 ft., darkvision 120 ft., passive Perception 17 
Languages understands Deep Speech but can't speak 
Challenge 16 (15,000 XP) Proficiency Bonus +5

\subsubsection*{Traits}

Adamantine Weapons. The drake's weapon attacks are adamantine.

Crushing Descent. If the drake flies at least 20 feet downward and then hits a target with a Crushing Stomp attack, the target takes an extra 6 (1d12) bludgeoning damage, and it must succeed on a DC 19 Strength saving throw or be knocked prone and grappled by the drake (escape DC 17). Until the grapple ends, the target is restrained and has disadvantage on Strength checks and Strength saving throws, and the drake can't use the same claw on another target.

Immutable Form. The drake is immune to any spell or effect that would alter its form.

Legendary Resistance (3/Day). If the drake fails a saving throw, it can choose to succeed instead.

Lightning Powered. If the drake takes at least 10 lightning damage from any source, it regains 1 spent legendary action.

\subsubsection*{Bloodied}

Overcharged. If the drake is bloodied (reduced to less than half of its hit points), it starts to crackle with arcs of lightning. Its Flaming Tar Breath automatically recharges, and any creature that hits it with a melee attack before the start of its next turn (including the attack that bloodied it, if applicable) takes 6 (1d12) lightning damage.
\end{minipage}\hfill
\begin{minipage}{0.48\textwidth}
\subsubsection*{Actions}

Multiattack. The drake makes two attacks: one with its bladed tail and one with its crushing stomp.

Bladed Tail. Melee Weapon Attack: +12 to hit, reach 15 ft., one target. Hit: 16 (2d8 + 7) slashing damage.

Crushing Stomp. Melee Weapon Attack: +12 to hit, reach 5 ft., one Medium or smaller target. Hit: 13 (1d12 + 7) bludgeoning damage.

Arcane Nullification. The drake ends one spell effect within 30 feet of it. If the spell is of 6th level or higher, the drake must succeed on an Intelligence (Arcana) check (DC 10 + the level of the spell) to end it.

Flaming Tar Breath (Recharge 5–6). The drake unleashes a blast of flaming tar in a 30-foot cone. Each creature in the area must make a DC 19 Dexterity saving throw. On a failed save, a creature takes 45 (13d6) fire damage and becomes covered in a thick layer of flaming tar. On a successful save, the creature takes half as much damage and isn’t covered in tar. While a creature is covered in tar, its speed is halved and it takes 10 (3d6) fire damage at the start of each of its turns. A creature can spend an action clearing the flaming tar from itself or another creature within 5 feet of it.

\subsubsection*{Reactions}

Windy Flapping. The drake adds 5 to its AC against one ranged weapon attack that would hit it. To do so, the dragon must see the attacker and be flying.

\subsubsection*{Legendary Actions}

The drake can take 3 legendary actions, choosing from the options below. Only one legendary action option can be used at a time and only at the end of another creature’s turn. The drake regains spent legendary actions at the start of its turn.

Arc. All creatures within 60 feet of the drake take 2 (1d4) lightning damage. Once the drake uses Arc, it can’t use it again until the start of its next turn.

Tail Attack. The drake makes a bladed tail attack.

Wind Up. The drakeconserves energy to unleash in a burst. It gains 10 temporary hit points, and it can take the Dash action as a bonus action on its next turn.

Wing Attack (Costs 2 Actions). The drake beats its wings. Each creature within 10 feet of the drake must succeed on a DC 19 Dexterity saving throw or take 14 (2d6 + 7) bludgeoning damage and be knocked prone. The drake can then fly up to half its flying speed.
\end{minipage}

\begin{itemize}
  \item NPC: Clockwork Drake
\end{itemize}

% [Image Inserted Manually]

\section*{Unearthed Destroyer}

Few know the origin or purpose of these engines of destruction, only the path of carnage left behind whenever one is unearthed from its ancient resting place. Were these forged as terrifying siege weapons, or are they some form of guardian bereft of purpose and understanding? Whatever their origin, these constructs now serve as a potent warning for anyone who would dare delve too deeply into ancient, ruined history. The journey of many an adventuring group has ended when they unearth one of these ancient destroyers.

\subsection*{Deployment}

These destroyers most often serve as a dangerous solo encounter, found deep in an ancient ruin best left undisturbed—though perhaps an adventuring group encounters one previously unearthed mid-rampage.

They are best fit as a potent campaign foe for a Tier 2 group as they approach 10th level, or as a dangerous obstacle for a group in early Tier 3.

They can also serve as a potent lieutenant (or pet) of a higher-level villain in Tier 4, though their damage and disruption mean they will generally remain too potent to be a minion-level threat.

\subsection*{Ancient and Decaying}

These constructs are terrible engines of destruction but are remnants of another era that are barely intact. They are unstable and prone to explosions and devastating ends, wreaking one last burst of carnage on their way out.

\subsection*{Simple Script}

Here’s a simple script that you can follow to get an effective use from an unearthed destroyer.

\begin{itemize}
  \item Use Gravity Surge to increase gravity.
  \item Move toward the closest target.
  \item Cast lightning bolt if it can hit multiple targets or electrocute if it can’t.
  \item Bite a target, if possible.
  \item Use tail attacks to batter targets in range or cast crackle if no targets are in range. Use lightning charged if no good targets are available.
  \item Use Gravity Surge to lower gravity. Cycle raising and lowering gravity, unless there is a flying target, in which case increase the gravity to ground it. The destroyer does not like flying creatures.
  \item Only use Annihilation Blast if the destroyer is trapped by magic or its target is defended by magic that would otherwise impede chomping it.
  \item Repeat steps.
\end{itemize}

The destroyer is not stupid, but it is simple. It destroys, consuming creatures and turning them to ash. It doesn’t fear destruction, and it particularly dislikes things that attempt to control it or impede it.

\subsection*{Considerations}

While a relatively straightforward monster, these creatures are significantly more fatal than they might seem at first glance. Allow a PCs to gather some information about them: an Arcana check of DC 10 or higher might tell the players that the power core looks unstable, and a result of 15 or higher might tell them that any creature that is devoured by the destroyer would be turned to ash. At DC 20 the irreversible nature its Meltdown may be revealed.

\subsection*{Unearthed Destroyer}

\begin{minipage}{0.48\textwidth}
Huge construct, unaligned

Armor Class 18 (natural armor) 
Hit Points 187 (15d12 + 90) [Max: 270] 
Speed 40 ft.

\begin{tabularx}{\textwidth}\toprule
{}XXXXX}
\midrule
STR & DEX & CON & INT & WIS & CHA \\
\midrule
26 (+8) & 10 (+0) & 22 (+6) & 14 (+2) & 14 (+2) & 8 (-1) \\
\midrule
\end{tabularx}

Saving Throws Str +13, Con +11, Wis +7, Cha +4 
Skills Athletics +13, Perception +7 
Damage Resistances lightning; bludgeoning, piercing, and slashing from nonmagical attacks that aren't adamantine 
Damage Immunities poison, psychic 
Condition Immunities charmed, exhaustion, frightened, petrified, poisoned 
Senses blindsight 60 ft., darkvision 120 ft., passive Perception 17 
Languages understands the languages of its creator but can't speak 
Challenge 13 (10,000 XP) Proficiency Bonus +5

\subsubsection*{Traits}

Death Burst. When the destroyer is reduced to 0 hit points, its power containment fails, unleashing massive destructive power. All creatures within 120 of it must make a DC 16 Constitution saving throw, taking 36 (8d8) lightning damage on a failed save, or half as much damage on a successful one. This damage is halved for creatures that are more than 60 feet away from the destroyer.

Gravity Surge. As a bonus action at the start of its turn, the destroyer can raise or lower gravity. If it raises gravity, all other creatures within 60 feet of it must succeed on a DC 19 Strength saving throw or fall prone. A flying creature that fails this save falls to the ground even if it has the ability to hover or is being held aloft by magic. If the destroyer lowers gravity, all other creatures in the area must succeed on a DC 19 Strength saving or float 10 feet upwards (as if by the levitate spell) until the start of the destroyer's next turn.

Innate Spellcasting. The destroyer's innate spellcasting ability is Constitution (spell save DC 19). It can innately cast the following spells, requiring no components:

At will: crackle*, fling*, lightning charged* 
3/day each: electrocute*, jumping jolt*, lightning bolt

Unstable Trajectories. If the destroyer has used its Gravity Surge since the start of its last turn, all ranged weapon attacks originating more than 30 feet away from the destroyer that pass within 30 feet of the destroyer are made with disadvantage.

Weak Point. When a creature within 10 feet of the destroyer makes an attack against it, the creature can target its power core. On a hit, the attack deals an extra 4 (1d8) damage to the destroyer, and the attacker takes 4 (1d8) lightning damage from the backlash.

\subsubsection*{Bloodied}

Meltdown. When the destroyer becomes bloodied (reduced to less than half of its maximum hit points), it becomes too damaged to contain its massive power core. At the start of each of its turns, it and all creatures within 60 feet of it must make a DC 19 Constitution saving throw, taking 18 (4d8) lightning damage on a failed save, or half as much damage on a successful one. This is a runaway process that can't be stopped once started. The third time this would occur, the destroyer is instead reduced to 0 hit points, triggering its Death Burst.
\end{minipage}\hfill
\begin{minipage}{0.48\textwidth}
\subsubsection*{Actions}

Multiattack. The destroyer makes two attacks: one with its bite and one with its tail. It can use its Swallow instead of its bite, and it can cast a spell in place of either attack.

Bite. Melee Weapon Attack: +13 to hit, reach 10 ft., one target. Hit: 27 (3d12 + 8) piercing damage. If the target is a creature, it is grappled (escape DC 18). Until this grapple ends, the target is restrained, and the destroyer can't bite another target.

Tail. Melee Weapon Attack: +10 to hit, reach 10 ft., one target. Hit: 21 (3d8 + 8) bludgeoning damage. It can't target a creature grappled by its bite with this attack.

Annihilation Blast. The destroyer unleashes a blast of power in a 30-feet cone. Any spell of 7th level or lower in the area ends, and each creature in the area must make a DC 19 Dexterity saving throw, taking 27 (6d8) lightning damage on a failed save, or half as much damage on a successful one. Structures automatically take maximum damage from this effect.

Swallow. The destroyer makes one bite attack against a Medium or smaller creature it is grappling. If the attack hits, the target takes the bite's damage, the target is swallowed, and the grapple ends. While swallowed, the creature is blinded and restrained, it has total cover against attacks and other effects outside the destroyer, and it takes 22 (5d8) lightning damage at the start of each of the destroyer's turns. If this damage kills the target, it is reduced to fine ash.

If the destroyer takes 30 damage or more on a single turn from a creature inside it, the destroyer must succeed on a DC 21 Constitution saving throw at the end of that turn or regurgitate all swallowed creatures, which fall prone in a space within 10 feet of the destroyer. If the destroyer dies, a swallowed creature is no longer restrained by it and can escape from the corpse by using 30 feet of movement, exiting prone.

\subsubsection*{Legendary Actions}

The destroyer can take 2 legendary actions, choosing from the options below. Only one legendary action option can be used at a time and only at the end of another creature's turn. The destroyer regains spent legendary actions at the start of its turn.

Lightning Power. The destroyer casts one of its at will spells.

Move. The destroyer moves up to half its speed.

Tail Attack. The destroyer makes one tail attack.
\end{minipage}

\begin{itemize}
  \item NPC: Unearthed Destroyer
\end{itemize}

\section*{Silver Relics}

These monsters are a set of minor foes that are relics of a long-lost time... perhaps simply misplaced by a distant civilization. They are advanced constructs with a relatively basic purpose and abilities, but they are highly effective at those simple roles.

\subsection*{Silver Stalker}

\begin{minipage}{0.48\textwidth}
Sleek, seek-and-destroy units, silver stalkers are a piece of lost technology that adventurers are more likely to encounter. Single minded and dogmatic with an eternal drive, they have a relatively simple and specialized intelligence that entirely revolves around their purpose. They can be intelligent in their ability to track and navigate, but they will rarely apply advanced reasoning.

\subsubsection*{Tactics}

In combat, silver stalkers will generally employ their heat ray (if they aren't currently frying a target with it), and then move to attack a target that looks like it will be a threat to them. If they are defending a location, they will focus on the first person to enter; if they are hunting a target, they will lock onto it with unreasonable dedication. If they detect a magical or adamantine weapon, they will frequently target that with their heat ray first, trying to eliminate it as a threat. They never consider self-preservation, unless it helps them accomplish their objective.

\subsubsection*{Usage}

Stalkers can be deployed in a wide range of circumstances and are often the first hint that there is something strange afoot, as they can range further abroad. Created by mages, advanced societies, or parts of ancient tech, these silver relics are an uncommon foe, but one that poses a significant threat early on, and remain a relevant threat throughout the game as a minion due to their high hit chance and the guaranteed damage of their heat ray. Their excessive Stealth, Survival, and Perception make them a useful tool throughout the game, as they are difficult to evade, even with magic.

\begin{itemize}
  \item At levels 1-4, a stalker is a solid solo threat that will pose a difficult and dangerous challenge to a party.
  \item At levels 5-10, it is no longer a solo threat, but can be a relevant and dangerous facet of an encounter. It is capable of significant damage but will likely die rapidly and should be deployed in numbers.
  \item At levels 11-17+, a stalker's impact is relevant but minor. Due to its heat ray and high hit chance, it will remain a threat that has to be dealt with, even if its damage and hit points render it more of a nuisance than a true threat. Its high Dexterity means it has a better chance of surviving area of effect abilities than many lower-level threats.
\end{itemize}

\begin{itemize}
  \item NPC: Silver Stalker
\end{itemize}
\end{minipage}\hfill
\begin{minipage}{0.48\textwidth}
Medium construct, unaligned

Armor Class 18 (natural armor) 
Hit Points 67 (9d8 + 27) [Max: 99] 
Speed 40 ft., climb 40 ft.

\begin{tabularx}{\textwidth}\toprule
{}XXXXX}
\midrule
STR & DEX & CON & INT & WIS & CHA \\
\midrule
16 (+3) & 16 (+3) & 16 (+3) & 10 (+0) & 16 (+3) & 1 (-5) \\
\midrule
\end{tabularx}

Saving Throws Dex +9 
Skills Perception +9, Stealth +9, Survival +9 
Damage Resistances bludgeoning, piercing, and slashing from nonmagical attacks that aren't adamantine 
Damage Immunities poison, psychic 
Condition Immunities charmed, exhaustion, frightened, petrified, poisoned  Senses darkvision 120 ft., passive Perception 19 
Languages understands one language known by its creator but can't speak 
Challenge 5 (1,800 XP) Proficiency Bonus +3

\subsubsection*{Traits}

Advanced Technology. The stalker applies double its proficiency bonus to proficient ability checks, attack rolls, and saving throws (included in this stat block).

Short Circuit. Whenever the stalker takes 10 or more lightning damage, it takes an extra 6 (1d12) lightning damage.

\subsubsection*{Actions}

Multiattack. The stalker makes two attacks. It can use its Heat Ray in place of one attack.

Arc. Ranged Spell Attack: +6 to hit, range 30 ft., one target. Hit: 5 (2d4) lightning damage.

Claw. Melee Weapon Attack: +9 to hit, reach 5 ft., one target. Hit: 7 (1d8 + 3) slashing damage.

Heat Ray. The stalker casts the heat metal spell, requiring no components. When cast in this way, the spell ends early if the target ends its turn out of sight of the stalker or more than 60 feet away from it.
\end{minipage}

\subsection*{Silver Sentry}

\begin{minipage}{0.48\textwidth}
Tiny construct, unaligned

Armor Class 16 (natural armor) 
Hit Points 22 (5d4 + 10) [Max: 30] 
Speed 0 ft., fly 30 ft. (hover)

\begin{tabularx}{\textwidth}\toprule
{}XXXXX}
\midrule
STR & DEX & CON & INT & WIS & CHA \\
\midrule
6 (−2) & 16 (+3) & 14 (+2) & 14 (+2) & 16 (+3) & 1 (−5) \\
\midrule
\end{tabularx}

Saving Throws Int +6, Wis +7 
Skills Perception +7 
Damage Resistances bludgeoning, piercing, and slashing from nonmagical attacks that aren’t adamantine 
Damage Immunities poison, psychic 
Condition Immunities charmed, exhaustion, frightened, petrified, poisoned 
Senses blindsight 60 ft., darkvision 120 ft., passive Perception 17 
Languages understands one language known by its creator but can’t speak 
Challenge 1 (200 XP) Proficiency Bonus +2

\subsubsection*{Traits}

Advanced Technology. The sentry applies double its proficiency bonus to proficient ability checks, attack rolls, and saving throws (included in this stat block).

Short Circuit. Whenever the sentry takes 10 or more lightning damage, it takes an extra 6 (1d12) lightning damage.

\subsubsection*{Actions}

Multiattack. The sentry makes two Laser Beam attacks.

Laser Beam. Ranged Spell Attack: +6 to hit, range 60/180 ft., one target. Hit: 5 (2d4) fire damage.

Targeting Lock. The sentry locks onto a creature or object it can see, holographically illuminating it. Until the start of the sentry’s next turn, any attack roll against the affected creature or object has advantage if the attacker can see it, and the affected creature or object can’t benefit from being invisible.
\end{minipage}\hfill
\begin{minipage}{0.48\textwidth}
Small, floating contraptions, an unwary observer might deem these tiny, insect-like things a minor threat, but they can prove to be a vexing foe. Most notable for their excellent perceptive capabilities and ability to highlight targets for their allies, they rarely pose a significant threat by themselves, but a wise adventurer is cautious of these sentries in the presence of harder-hitting foes.

Most often deployed as sentinels or force multipliers of more powerful foes, these constructs are capable of limited adaptation in their plans, but are dogmatic in their devotion to a purpose, not generally being truly sentient despite their high capacity for reasoning.

\subsubsection*{Tactics}

Silver sentries often have a primary directive, and when engaging in violent response will do so as efficiently as possible to execute their directive. They are capable of adaptive planning, but they won’t deviate from their task, even in the face of certain destruction.

\subsubsection*{Usage}

Silver sentries can be deployed as a threat against a low-level group to guard a location of importance, but they are most often deployed as potent support minions for a higher-level, advanced construct threat.

\begin{itemize}
  \item At levels 1–4, these are a minor solo threat. They can pose a small challenge to a party due to their all-around hit statistics (high AC, high attack bonus, and damage resistance), but their inability to do large amounts of damage will generally make them not a deadly threat.
  \item At levels 5–17+, they remain a significant minion due to the Targeting Lock ability. They will always be an enemy that players will be encouraged to focus down and eliminate first, and they are a thorn in the side of stealthy operations.
\end{itemize}

\begin{itemize}
  \item NPC: Silver Sentry
\end{itemize}
\end{minipage}

\subsection*{Silver Soldier}

\begin{minipage}{0.48\textwidth}
Golems that few would recognize as such—somewhere between a human and a skeleton, slender and sleek—silver soldiers exhibit swift, smooth movements.

They were likely once common foot soldiers in a time and place long forgotten. Any remaining units find themselves deadly weapons in an age unprepared for them.

\subsubsection*{Tactics}

Silver soldiers are most effective at a distance and will attempt to engage from range. Even at close range they will often use their ranged attack, relying on their excessively high attack bonus. They can determine the AC of creatures and will use this knowledge to pick their targets efficiently.

\subsubsection*{Usage}

With a high damage output, they make excellent ranged threats at higher levels, or can be a challenge to overcome in specialized situations where they have cover and positioning.

\begin{itemize}
  \item At levels 1–4, silver soldiers are exceedingly dangerous. Their damage is very high, and they have few weaknesses to exploit.
  \item At levels 5–10, they are a significant, high-damage threat. They serve as artillery that need to be dealt with but will be somewhat fragile as a solo encounter.
  \item At levels 11+, they still make excellent artillery threats that must be respected. Their long range and high attack bonus mean they will remain a significant threat.
\end{itemize}

\begin{itemize}
  \item NPC: Silver Soldier
\end{itemize}
\end{minipage}\hfill
\begin{minipage}{0.48\textwidth}
Medium construct, unaligned

Armor Class 18 (natural armor) 
Hit Points 75 (10d8 + 30) [Max: 110] 
Speed 30 ft.

\begin{tabularx}{\textwidth}\toprule
{}XXXXX}
\midrule
STR & DEX & CON & INT & WIS & CHA \\
\midrule
14 (+2) & 16 (+3) & 16 (+3) & 16 (+3) & 14 (+2) & 1(−5) \\
\midrule
\end{tabularx}

Saving Throws Dex +9, Wis +8 
Skills Perception +8 
Damage Resistances bludgeoning, piercing, and slashing from nonmagical attacks that aren’t adamantine 
Damage Immunities poison, psychic 
Condition Immunities charmed, exhaustion, frightened, petrified, poisoned 
Senses darkvision 120 ft., passive Perception 18 
Languages understands one language known by its creator but can’t speak 
Challenge 6 (2,300 XP) Proficiency Bonus +3

\subsubsection*{Traits}

Advanced Technology. The soldier applies double its proficiency bonus to proficient ability checks, attack rolls, and saving throws (included in this stat block).

Short Circuit. Whenever the soldier takes 10 or more lightning damage, it takes an extra 6 (1d12) lightning damage.

\subsubsection*{Actions}

Multiattack. The soldier makes three Laser Beam attacks.

Calculated Shot. Ranged Spell Attack: +18 to hit, range 150/600 ft., one target. Hit: 10 (4d4) fire damage.

Laser Beam. Ranged Spell Attack: +9 to hit, range 150/600 ft., one target. Hit: 10 (4d4) fire damage.

\subsubsection*{Reactions}

Overwatch. When a creature moves within 600 feet of the soldier and isn’t benefiting from cover, the soldier makes one Laser Beam attack against the creature
\end{minipage}

\subsection*{Silver Guardian}

% [Image Inserted Manually]

Larger, more advanced, and more dangerous than other silver relics, these appear as hulking brutes, but have a keen intellect and the ability to manipulate gravity. They may either lead other silver relic golems or serve as a standalone threat.

\subsection*{Tactics}

Silver guardians excel in durability and control and will engage and contain the most dangerous threats. While highly intelligent, their intelligence is a robotic logic that can be fooled by player characters behaving in irregular ways.

\subsection*{Usage}

They tend to serve as mid- to high-level threats, rarely taking a minion role. They can, however, make an excellent companion for a squishier threat.

\begin{itemize}
  \item At levels 1–4, this would be an insurmountable solo threat.
  \item At levels 5–10, they serve as a moderate solo threat, or an excellent duologist or leader of a tough encounter.
  \item At level 11+, they will no longer be a substantial solo threat, but can make a great bruiser or controller.
\end{itemize}

\begin{itemize}
  \item NPC: Silver Guardian
\end{itemize}

\begin{minipage}{0.48\textwidth}
Large construct, lawful neutral

Armor Class 19 (natural armor) 
Hit Points 95 (10d10 + 40) [Max: 140] 
Speed 30 ft., fly 30 ft. (hover)

\begin{tabularx}{\textwidth}\toprule
{}XXXXX}
\midrule
STR & DEX & CON & INT & WIS & CHA \\
\midrule
20 (+5) & 10 (+0) & 18 (+4) & 22 (+6) & 18 (+4) & 7 (−2) \\
\midrule
\end{tabularx}

Saving Throws Con +12, Wis +12 
Skills Athletics +13, History +14, Insight +12, Investigation +14, Nature +14, Perception +12 
Damage Resistances bludgeoning, piercing, and slashing from nonmagical attacks that aren’t adamantine 
Damage Immunities poison, psychic 
Condition Immunities charmed, exhaustion, frightened, petrified, poisoned 
Senses blindsight 60 ft., darkvision 120 ft., passive Perception 22 
Languages all 
Challenge 10 (5,900 XP) Proficiency Bonus +4

\subsubsection*{Traits}

Advanced Technology. The guardian applies double its proficiency bonus to proficient ability checks, attack rolls, and saving throws (included in this stat block).

Forcefield Projector. Ranged weapon attacks against the guardian or any creature of its choice within 10 feet of it that originate more than 10 feet away from the guardian are made with disadvantage.

Sufficiently Advanced Technology (Innate Spellcasting). The guardians innate spellcasting ability is Intelligence (spell save DC 22). It can innately cast the following spells, requiring no components:

At will: fling*, floating disk, levitate, thunderwave, warding bond 
1/day each: dimension door, orbital stones*, telekinesis, vortex blast*
\end{minipage}\hfill
\begin{minipage}{0.48\textwidth}
\subsubsection*{Bloodied}

Destabilized Gravity Generator. When the guardian is bloodied (reduced to less than half its maximum hit points), it immediately unleashes a blast of gravity. All creatures within 30 feet of it must make a DC 22 Strength saving throw. On a failed save, a creature is pushed 30 feet away from the guardian. If the saving throw fails by 5 more, the creature is also thrown 30 feet into the air. Until the guardian finishes a long rest, it loses its Forcefield Projector trait, but it can target up to 3 creatures with its Manipulate Gravity.

\subsubsection*{Actions}

Multiattack. The guardian uses Manipulate Gravity. It then casts a spell or makes two attacks.

Grab. Melee Weapon Attack: +13 to hit, reach 5 ft., one target. Hit: 10 (2d4 + 5) bludgeoning damage, and the target is grappled (escape DC 15). Until this grapple ends, the target is restrained.

Slam. Melee Weapon Attack: +13 to hit, reach 5 ft., one target. Hit: 14 (2d8 + 5) bludgeoning damage.

Manipulate Gravity. The guardian forces one creature within 60 feet of it to make a DC 22 Strength saving throw. On a failed save, the guardian can move the target up to 30 feet in any direction. The target lands prone and takes 3 (1d6) bludgeoning damage, unless it was thrown straight up, in which case it takes falling damage as normal, if applicable.

\subsubsection*{Legendary Actions}

The guardian can take 2 legendary actions, choosing from the options below. Only one legendary action option can be used at a time and only at the end of another creature's turn. The guardian regains spent legendary actions at the start of its turn.

Calculate. The guardian chooses a number between 1 and 20. Once before the end of its next turn, when a creature it can see would make an ability check, attack roll, or saving throw, the guardian can make the creature use the chosen number, instead of rolling.

Slam Attack. The guardian makes one slam attack
\end{minipage}

\section*{Ironwrought}

Collectively known as the Ironwrought, these are a series of adaptable humanoid constructs. Their sources and abilities are many and varied. There are even reports of fully sapient members of the categorization, with some adapting to the adventuring lifestyle (see Races section).

The ones presented here are simple versions that might be encountered. Ironwrought constructs tend to be viewed as the creations of societies more magically included and less technologically advanced than Silver Relics, though some evidence suggests of great wars fought between them.

\subsection*{Ironwrought Soldier}

A simple and common variety, these Ironwrought were often produced in large numbers in ancient, forgotten conflicts.

\begin{itemize}
  \item NPC: Ironwrought Soldier
\end{itemize}

% [Image Inserted Manually]

Medium construct, unaligned

Armor Class 16 (natural armor) 
Hit Points 45 (6d8 + 18) [Max: 66] 
Speed 30 ft.

\begin{tabularx}{\textwidth}\toprule
{}XXXXX}
\midrule
STR & DEX & CON & INT & WIS & CHA \\
\midrule
16 (+3) & 12 (+1) & 16 (+3) & 14 (+2) & 14 (+2) & 1(-5) \\
\midrule
\end{tabularx}

Saving Throws Str +5, Wis +4 
Skills Perception +4 
Damage Resistances bludgeoning, piercing, and slashing from nonmagical attacks that aren't adamantine 
Damage Immunities poison, psychic 
Condition Immunities charmed, exhaustion, frightened, petrified, poisoned 
Senses darkvision 60 ft., passive Perception 14 
Languages Common, one language known by its creator 
Challenge 4 (1,100 XP) Proficiency Bonus +2

\subsection*{Traits}

Ironwrought Fortitude. If damage reduces the soldier to 0 hit points, it must make a Constitution saving throw with a DC of 5 + the damage taken, unless the damage is lightning or from a critical hit. On a success, the soldier drops to 1 hit point instead.

\subsection*{Actions}

Multiattack. The soldier makes two attacks.

Bladed Warstaff. Melee Weapon Attack: +5 to hit, reach 10 ft., one target. Hit: 10 (3d4 + 3) slashing damage.

Energy Staff. Ranged Weapon Attack: +4 to hit, range 60/120 ft., one target. Hit: 5 (1d6 + 2) force damage.

\subsection*{Reactions}

Vigilant Sentinel. When the soldier is holding its bladed warstaff, other creatures provoke an opportunity attack from it when they move 5 feet or more while within 10 feet of it, or when they hit it with an attack while within its reach.

\subsection*{Ironwrought Worker}

\begin{minipage}{0.48\textwidth}
The most basic form, these are created for specialized labor.

\begin{itemize}
  \item NPC: Ironwrought Worker
\end{itemize}
\end{minipage}\hfill
\begin{minipage}{0.48\textwidth}
Medium construct, unaligned

Armor Class 16 (natural armor) 
Hit Points 30 (4d8 + 12) [Max: 44] 
Speed 30 ft.

\begin{tabularx}{\textwidth}\toprule
{}XXXXX}
\midrule
STR & DEX & CON & INT & WIS & CHA \\
\midrule
16 (+3) & 12 (+1) & 16 (+3) & 14 (+2) & 14 (+2) & 1(−5) \\
\midrule
\end{tabularx}

Saving Throws Str +5, Wis +4 
Skills Athletics +5, Perception +4 
Damage Resistances bludgeoning, piercing, and slashing from nonmagical attacks that aren’t adamantine 
Damage Immunities poison, psychic 
Condition Immunities charmed, exhaustion, frightened, petrified, poisoned 
Senses darkvision 60 ft., passive Perception 14 
Languages Common, one language known by its creator 
Challenge 1/2 (50 XP) Proficiency Bonus +2

\subsubsection*{Traits}

Ironwrought Fortitude. If damage reduces the worker to 0 hit points, it must make a Constitution saving throw with a DC of 5 + the damage taken, unless the damage is lightning or from a critical hit. On a success, the worker drops to 1 hit point instead.

\subsubsection*{Actions}

Multiattack. The worker makes two attacks.

Bladed Warstaff. Melee Weapon Attack: +5 to hit, reach 10 ft., one target. Hit: 5 (1d4 + 3) slashing damage.
\end{minipage}

% [Image Inserted Manually]

\section*{Lurking Maw}

A lurking maw is no natural creature, having slithered out of twisted realms of nightmare beyond the veil of mortal comprehension. They do not need to eat to survive and can live indefinitely lurking in darkness, but possess great hunger, devouring all they encounter.

They despise light and life and will seek to destroy either that enters the darkness of their lair. They cannot be drawn out into sunlight or from their lair by almost any means.

\subsection*{Deployment}

A lurking maw makes a great solo boss encounter for the bottom of a dungeon or forbidden area. It can be an eldritch evil to guard what the heroes must retrieve, the ancient scar of some ill-advised magic or portal, or the lurking nightmare of a forgotten lab or wizard's abode.

It can serve as the capstone for a 5th-level party as a known factor they will have to confront, or as a nasty surprise for a higher-level party, becoming a lesser threat beyond 11th level.

\subsection*{Tactics}

There are two important factors when combating a lurking maw: light and mobility. Parties without a way to keep the battlefield illuminated will find themselves in great danger, as even darkvision begins to fail with the multiple instances of magical darkness the lurking maw can produce.

The second factor is trying to avoid a creature being consumed, leaving no room for picking them back up. The lurking maw is most deadly when emphasizing these hurdles-by targeting light sources and devouring those that try to make light in its presence, particularly loathsome sources of sunlight.

\subsection*{Simple Script}

Here's a simple script to get the most from a lurking maw. You can use this as a guide when running one:

\begin{itemize}
  \item The lurking maw will attempt to ambush an unaware party, though it is unlikely to succeed against a cautious party.
  \item It will open with Expunge Light, removing the brightest light source it can or a light source that will plunge the most of its targets into darkness.
  \item With many targets available, it will attempt to grapple three of them: one in each tentacle, and one in its maw.
  \item Note that while each tentacle can only grapple one target, the lurking maw can still attack a single creature that is isolated against it with multiple tentacles or its bite.
  \item As a lair action, it will use Inky Pool to create darkness if there isn't already suitable darkness, or it will use Grasping Darkness to spawn a new maw tendril if suitable darkness is already present.
  \item It will use its legendary action as soon as possible to use Mind Terrors, targeting any creature it doesn't have grappled first in priority.
  \item While it is waiting for that to recharge, it will use Crush if any target is grappled, or Tentacle Attack if it has no target grappled.
\end{itemize}

\subsection*{Considerations}

The first time a party encounters a lurking maw, be generous with the hints they are going to need. A party with weak light sources will very likely meet their end if they are unprepared, and the lurking maw can even punch above its weight class. Conversely, it lacks durability when heavily exposed to light, losing many of its features and defensive layers.

\begin{itemize}
  \item NPC: Lurking Maw
\end{itemize}

\subsection*{Lurking Maw}

\begin{minipage}{0.48\textwidth}
Huge aberration, chaotic evil

Armor Class 15 (natural armor) 
Hit Points 115 (11d12 + 44) [Max: 176] 
Speed 30 ft., climb 30 ft., swim 30 ft.

\begin{tabularx}{\textwidth}\toprule
{}XXXXX}
\midrule
STR & DEX & CON & INT & WIS & CHA \\
\midrule
20 (+5) & 12 (+1) & 18 (+4) & 8 (−1) & 16 (+3) & 9 (−1) \\
\midrule
\end{tabularx}

Saving Throws Con +8, Wis +7 
Skills Athletics +9, Perception +7, Stealth +5 
Damage Resistances poison, psychic 
Condition Immunities charmed, frightened, paralyzed 
Senses blindsight 120 ft. (blind beyond this radius), passive Perception 17 
Languages — 
Challenge 9 (5,000 XP) Proficiency Bonus +4

\subsubsection*{Traits}

Consuming Hunger. If a creature swallowed by the lurking maw dies, the lurking maw is psionically nourished, regaining 13 (2d12) hit points and either one legendary resistance or one legendary action (its choice).

Expunge Light. At the start of its turn, the lurking maw can extinguish one light source within 120 feet of it (no action required). A lightgenerating spell cast with a spell slot of 3rd level or higher can’t be extinguished in this way.

Legendary Resistance (2/Day). If the lurking maw fails a saving throw, it can choose to succeed instead.

Lurk. While in darkness, the lurking maw can take the Hide action as a bonus action to hide one of its tentacles.

Radiant Weakness. Whenever the lurking maw takes fire or radiant damage, or any damage while it’s in direct sunlight, it takes an extra 3 (1d6) damage of that type.

\subsubsection*{Bloodied}

Hungering Abyss. The lurking maw’s wounds bleed an inky black, magical smoke that boils up into a haze of magical darkness when it is bloodied (reduced to less than half its maximum hit points), filling the area within a 60-foot radius of it. This haze lasts for 2 (1d4) rounds. This magical darkness is treated as a magical effect created by a 4thlevel spell slot. It also immediately gains 1 additional legendary action. It doesn’t regain this additional legendary action at the start of its turn.
\end{minipage}\hfill
\begin{minipage}{0.48\textwidth}
\subsubsection*{Actions}

Multiattack. The lurking maw makes three attacks: one with its bite and two with its tentacles.

Bite. Melee Weapon Attack: +9 to hit, reach 20 ft., one target. Hit: 9 (1d8 + 5) bludgeoning damage, and the target is grappled (escape DC 15). Until this grapple ends, the target is restrained. A creature grappled in this way takes 7 (3d4) piercing damage at the start of each of its turns. If a creature the lurking maw is grappling in this way has 0 hit points at the start of the lurking maw’s turn, that creature is swallowed (no action required), and the grapple ends. While swallowed, the creature is blinded and restrained, it has total cover against attacks and other effects outside the lurking maw, and it takes 14 (4d6) acid damage at the start of each of the lurking maw’s turns.

If the lurking maw takes 20 damage or more on a single turn from a creature inside it, the lurking maw must succeed on a DC 18 Constitution saving throw at the end of that turn or regurgitate all swallowed creatures, which fall prone in a space within 10 feet of the lurking maw. If the lurking maw dies, a swallowed creature is no longer restrained by it and can escape from the corpse using 15 feet of movement, exiting prone. Such a creature can also escape by being cut out of the corpse by a creature dealing 10 slashing damage to it.

Tentacle. Melee Weapon Attack: +9 to hit, reach 20 ft., one target. Hit: 9 (1d8 + 5) bludgeoning damage, and the target is grappled (escape DC 13). Until this grapple ends, the target is restrained. The lurking maw has four tentacles, each of which can grapple one target.

\subsubsection*{Legendary Actions}

The lurking maw can take 1 legendary action, choosing from the options below. Only one legendary action option can be used at a time and only at the end of another creature’s turn. The lurking maw regains spent legendary actions at the start of its turn.

Crush. The lurking maw crushes a target it is grappling, dealing 7 (2d6) bludgeoning damage to it. If the target is grappled by the lurking maw’s bite, it takes an extra 3 (1d6) piercing damage.

Tentacle Attack. The lurking maw makes a single tentacle attack.

Mind Terrors (Recharge 5–6). Up to five creatures within 60 feet of the lurking maw must make a DC 12 Wisdom saving throw, with disadvantage if the creature is in darkness. On a failed save, the creature becomes frightened for 1 minute. If the saving throw fails by 5 or more, the creature is stunned until the end of its next turn.

\subsubsection*{Lair Actions}

When fighting inside its lair, a lurking maw can invoke the ambient magic and terrain around it to take lair actions. On initiative count 20 (losing initiative ties), the lurking maw can take one lair action to cause one of the following effects:

Grasping Darkness. A new maw tendril appears in a patch of darkness that is a 5-foot cube or larger. It has AC 15, 15 hit points, immunity to poison and psychic damage, and uses a single tentacle attack as if it is the lurking maw. Roll initiative for the tendril. It makes one tentacle attack immediately when it appears and again on each of its turns until it is killed, or until its space is fully illuminated by bright light, destroying the tendril.

Inky Pool. A 5-foot-radius sphere of magical darkness (as if by the darkness spell) appears at a point of the lurking maw’s choice within 120 feet of it. This magical darkness grows by 5 feet every round on initiative count 20 until it reaches a 20-foot-radius sphere.
\end{minipage}

% [Image Inserted Manually]

\section*{The Uncaged}

The Uncaged can appear anywhere, operating with unknowable purpose. Some claim they are the souls of ancient warriors, caught up in a planar war through spaces of thought and time until they became the warped forms they now possess. Others speculate they are the fault of some wizard, as so many strange horrors are.

They are a seemingly inexhaustible force, appearing in groups lead by their enigmatic wardens. Though extremely dangerous foes, the Uncaged can only exist in the presence of their planar portals or the strange lanterns carried by their wardens. They collapse—crumbling to dust as if aging millenia—when deprived of their connection, making some scholars suspect they have become unmoored from time itself.

\subsection*{Doors of the Unknown}

The Uncaged come through mysterious portals, pouring out when they open. Due to their vital connection to these portals, the Uncaged can only travel 1 mile from their portal, and they disappear back through the portal when their work is done. Where these portals go, no one knows. Any attempt to cross such a portal results in the seemingly complete destruction of the object or creature that passes through.

\subsection*{Lanterns of the Wardens}

The exception to this restriction is the wardens, who carry a lantern-like object that projects the same power. An Uncaged can exist as long as it remains within 300 feet of the lantern. If a warden is slain, its lantern is destroyed. Any living creatures within 30 feet of a warden’s lantern find it quickly drains away their life force (as per the warden’s stat block).

\begin{minipage}{0.48\textwidth}
\subsubsection*{Purpose and Use}

The Uncaged make a set piece encounter that can be freely deployed in many odd situations. They are a force of collectors, enforcers, and investigators of fundamentally unknowable purpose, that can show up to interfere whenever things that manipulate plans, time, or other powerful magics are involved... sometimes connected in ways only they can understand.

They rarely speak and never negotiate. They show no hesitation in inflicting terrible casualties to achieve their goals, and they do not fear death or pain. They usually come and go quickly, but sometimes set up longer-term residence within the small sanctuaries of their lanterns.

\subsubsection*{Tactics}

The Uncaged provide an encounter that largely builds itself. The warden serves as the leader and support of a group of basic Uncaged soldiers, adding and mixing in winged Uncaged as needed. As they are dependent on their wardens, the other Uncaged will work to defend them. These soldiers are simple combatants, but highly effective and deadly. The warden provides the more complicated mechanics of the fight, and it will avoid a direct confrontation, if possible.
\end{minipage}\hfill
\begin{minipage}{0.48\textwidth}
True Nature

The original nature of the Uncaged are wardens of a planar prison that was destroyed and pulled into a realm of negative energy and madness during the escape of its most dangerous prisoner. They were trapped in their own jail, twisted and warped, bound in eternal duty by their obsession with punishing those responsible. They now scour the planes seeking clues to finally exact their revenge and mete out draconian punishment to whomever they see fit.
\end{minipage}

\subsection*{Uncaged}

\begin{minipage}{0.48\textwidth}
\begin{itemize}
  \item NPC: Uncaged
\end{itemize}
\end{minipage}\hfill
\begin{minipage}{0.48\textwidth}
Medium undead, neutral evil

Armor Class 17 (natural armor) 
Hit Points 68 (8d8 + 32) [Max: 96] 
Speed 30 ft.

\begin{tabularx}{\textwidth}\toprule
{}XXXXX}
\midrule
STR & DEX & CON & INT & WIS & CHA \\
\midrule
16 (+3) & 16 (+3) & 18 (+4) & 10 (+0) & 14 (+2) & 8 (-1) \\
\midrule
\end{tabularx}

Saving Throws Con +7, Wis +5 
Skills Athletics +6, Perception +5 
Damage Resistances cold, force; bludgeoning, piercing, and slashing from nonmagical attacks 
Damage Immunities necrotic 
Condition Immunities charmed, exhaustion, frightened, poisoned 
Senses darkvision 120 ft., passive Perception 15 
Languages Common, Deep Speech 
Challenge 5 (1,800 XP) Proficiency Bonus +3

\subsubsection*{Traits}

Turn Resistance. The Uncaged has advantage on saving throws against any effect that turns undead.

\subsubsection*{Actions}

Multiattack. The Uncaged makes two attacks.

Shortsword. Melee Weapon Attack: +6 to hit, reach 5 ft., one target. Hit: 6 (1d6 + 3) piercing damage plus 4 (1d8) force damage.

Shortbow. Ranged Weapon Attack: +6 to hit, range 80/320 ft., one target. Hit: 6 (1d6 + 3) piercing damage plus 4 (1d8) force damage.
\end{minipage}

\subsection*{Uncaged Warden}

\begin{minipage}{0.48\textwidth}
Medium undead, neutral evil

Armor Class 17 (natural armor) 
Hit Points 102 (12d8 + 48) [Max: 144] 
Speed 30 ft.

\begin{tabularx}{\textwidth}\toprule
{}XXXXX}
\midrule
STR & DEX & CON & INT & WIS & CHA \\
\midrule
16 (+3) & 14 (+2) & 18 (+4) & 18 (+4) & 20 (+5) & 14 (+2) \\
\midrule
\end{tabularx}

Saving Throws Dex +6, Con +8, Wis +9, Cha +6 
Skills Arcana +8, History +8, Perception +13, Psionics +8, Insight +9 
Damage Resistances cold, force; bludgeoning, piercing, and slashing from nonmagical attacks 
Damage Immunities necrotic 
Condition Immunities charmed, exhaustion, frightened, poisoned 
Senses truesight 300 ft., passive Perception 23 
Languages all, telepathy 120 ft. 
Challenge 9 (5,000 XP) Proficiency Bonus +4

\subsubsection*{Traits}

Innate Spellcasting. The warden’s innate spellcasting ability is Wisdom (spell save DC 17). It can innately cast the following spells, requiring no material components:

At will: binding curse*, command, compelled query*, detect thoughts, hold person, inflict wounds, misty step 
3/day each: delve mind*, blight, plane shift, psychic drain*, spatial manipulation* 
1/day each: banishment, circle of death , fear, hold monster, mind blast*

Lantern. The warden’s lantern sheds bright light in a 150-foot radius and dim light for an additional 150 feet, which is visible only to the Uncaged. This light travels through walls and obstacles and can’t be blocked or extinguished. If an Uncaged starts its turn in this light, it gains 4 (1d8) temporary hit points. The warden can see any creature inside the light of its lantern with its truesight.

Magic Resistance. The warden has advantage on saving throws against spells and other magical effects.

Unmaking Aura. The unnatural energy of the warden twists the world around it, draining the life from all nearby living creatures. When a living creature starts its turn within 60 feet of the warden, it must make a DC 17 Charisma saving throw. On failed save, it takes 7 (2d6) nectrotic damage.

\subsubsection*{Actions}

Multiattack. The warden uses its Battlefield Presence or Teleport. It then casts a spell.

Battlefield Presence. The warden commands an Uncaged to make a single weapon attack. The attack roll is made with advantage.

Teleport. The warden magically teleports itself and up to five willing creatures it can see within 60 feet of it, along with any equipment they are wearing or carrying, up to 300 feet to an unoccupied space it can see.

Uncaged Gate. The warden opens a gate to its unknowable realm. The gate is 10 feet wide and 10 feet tall. Any creature other than an Uncaged that passes through it dies, and can only be resurrected by a wish spell. The gate closes after 1 minute, or when all Uncaged in the area have passed through it.
\end{minipage}\hfill
\begin{minipage}{0.48\textwidth}
\begin{itemize}
  \item NPC: Uncaged Warden
\end{itemize}
\end{minipage}

\subsection*{Uncaged (Winged)}

\begin{minipage}{0.48\textwidth}
\begin{itemize}
  \item NPC: Uncaged (Winged)
\end{itemize}
\end{minipage}\hfill
\begin{minipage}{0.48\textwidth}
Medium undead, neutral evil

Armor Class 17 (natural armor) 
Hit Points 68 (8d8 + 32) [Max: 96] 
Speed 30 ft., fly 40 ft.

\begin{tabularx}{\textwidth}\toprule
{}XXXXX}
\midrule
STR & DEX & CON & INT & WIS & CHA \\
\midrule
16 (+3) & 16 (+3) & 18 (+4) & 10 (+0) & 14 (+2) & 8 (−1) \\
\midrule
\end{tabularx}

Saving Throws Dex +6, Wis +5 
Skills Athletics +6, Perception +5 
Damage Resistances cold, force; bludgeoning, piercing, and slashing from nonmagical attacks 
Damage Immunities necrotic 
Condition Immunities charmed, exhaustion, frightened, poisoned 
Senses darkvision 120 ft., passive Perception 15 
Languages Common, Deep Speech 
Challenge 6 (2,300 XP) Proficiency Bonus +3

\subsubsection*{Traits}

Turn Resistance. The Uncaged has advantage on saving throws against any effect that turns undead.

\subsubsection*{Actions}

Multiattack. The Uncaged makes two attacks.

Claws. Melee Weapon Attack: +6 to hit, reach 5 ft., one target. Hit: 5 (1d4 + 3) slashing damage plus 4 (1d8) force damage.

Necrotic Bolt. Ranged Spell Attack: +7 to hit, range 120 ft., one target. Hit: 8 (1d10 + 3) necrotic damage plus 4 (1d8) force damage.
\end{minipage}

\section*{Aberrant Abstraction}

Few know the true appearance of an aberrant abstraction as their very existence is antithetical to reality and the sane minds that inhabit it. To most, they appear as flashing, flickering, shadowy things of writhing and ever-changing features—the most often reported of which include tentacles, eyes, and mouths. Their appearance seems to be an illusory construct of mortal minds trying to cope with the intersection of reality and abstract unreality, and piercing that veil with truesight drives most creatures instantly mad.

\subsection*{Tactics}

An aberrant abstraction is most often completely chaotic and unpredictable. It doesn’t care about “winning” in a classical sense, and doesn’t comprehend losing. It merely acts in a way that will inflict disorder and chaos, usually entirely randomly.

It will ignore other aberrations, but otherwise treats all creatures with equal hostility, unless it is under some sort of psionic compulsion. The true nature of these monsters is incomprehensible, and they can more accurately be considered environmental obstacles than creatures.

\subsection*{Usage}

Aberrant abstractions can be deployed in almost any circumstance in which you want to highlight the unworldly weirdness of an environment: near a breaking down portal, in a psionic scar ripped across the land, or in an ancient temple to an outsider god.

They can be added to other encounters with aberrations at higher levels as nuisances that disrupt and wear down players. They also can be used as a solo encounter against low-level parties, or as agents of chaos in fights against other nonaberrations to add an element of randomness and disorder to the fight.

\begin{itemize}
  \item At levels 1–4, aberrant aberrations would be a deadly and difficult encounter.
  \item At levels 5–10, they are a unique threat, though not a challenging solo encounter, and can play a significant role in encounter design.
  \item At levels 11–16, these aberrations remain impactful and provide complexity as minor threats in an encounter.
  \item At level 17+, they pose a small, unique challenge and annoyance as minor minions to aberration threats.
\end{itemize}

\begin{itemize}
  \item NPC: Aberrant Abstraction
\end{itemize}

\section*{Adaptive Slime}

A particularly vexing foe, aberrant abstractions are strange slime creatures—likely artificial in nature—that are capable of psionically adapting to not only better suit their environment, but to completely mimic the actions and abilities of those around them. They generally take on the form of other creatures they see, leaving their true form something of a mystery, though they always retain the same slate grey color and have a somewhat melted appearence.

\subsection*{Tactics}

You have several options for the intelligence of these creatures: you can play them as truly cunning, selecting the most devastating ability to copy; you can play them in the middle of just copying whatever did the most damage to them recently (if it did more damage than their Shred Reality would do); or you can play them lacking intelligence, randomly rolling for what they mimic.

\subsection*{Usage}

It is often best that the first time a group of players encounters these it is as a solo set piece. While it will not be an exceedingly difficult encounter in that environment for an appropriately leveled party, it will likely be memorable and they will be forewarned of its abilties if they encounter it in a more difficult, mixed encounter later, upping the tactical nature of that encounter.

\begin{itemize}
  \item At levels 1–4, aberrant abstractions are a challenging but not insurmountable threat.
  \item At levels 5–10, they remain a challenging encounter, even as a solo threat. While they are unlikely to be deadly, their ability to mimic others can provide something of a scaling threat, particularly in an early encounter. They can be quite deadly at these levels as part of a group encounter.
  \item At levels 11–16, they can be a dangerous threat as part of a group due to the difficulty of killing them and their mimic abilities. They should still be considered a significant threat as part of an encounter design at these levels.
  \item At level 17+, aberrant abstractions will remain an unusual, dangerous, minion-level threat that will require strategic handling, but they will only provide a challenge when paired with more dangerous enemies.
\end{itemize}

\begin{itemize}
  \item NPC: Adaptive Slime
\end{itemize}

\section*{Rampaging Imprint}

Rampaging imprints are spawned into existence when a powerful psion starts to lose control of its abilities. They are particularly easily created should a psion find themselves in the Astral Plane. It is said that when a particularly powerful psion dies, sometimes hundreds of these undead can be spawned and flung out across the multiverse as the powers explode out of control in the psion’s dying moments.

Appearing as a vaguely person-shaped distortion of swirling, purple energy, they are beings of manifested rage, rampaging eternally until destroyed.

\subsection*{Tactics}

A rampaging imprint is most often devoid of tactics, and will attack whatever it sees. When multiple targets are presented, it will most often attack randomly or whatever attacked it last. Some retain a vestige of purpose or intent that will drive them to attack a specific thing.

\subsection*{Usage}

Rampaging imprints can appear anywhere, cast out into the world through psionic events and echoes. However, they rarely coexist with living creatures as they tend to attack things on sight and thus are best off deployed in isolated or otherwise undisturbed areas.

If deployed as part of a bigger encounter, they can often serve as summons or minions to similar creatures that direct their otherwise directionless rage through psionic power.

\begin{itemize}
  \item At levels 1–4, rampaging imprints pose a moderate threat.
  \item At levels 5–10, they are a minor nuisance that can pose a threat to low-Strength creatures, knocking the player characters down to give bigger threats advantage.
  \item At level 11+, they are unlikely to be a notable threat.
\end{itemize}

\begin{itemize}
  \item NPC: Rampaging Imprint
\end{itemize}

\section*{Eldritch Spawn}

\subsection*{Mind Lurker}

% [Image Inserted Manually]

An eldritch foe that shows a surprising mimicry of true thought, these twisted lurkers seem to be capable of some degree of patience and planning. They slowly infiltrate the mind of their vicitims and twist them to madness, sewing chaos while lurking in the darkness. If any have learned their name and retained their sanity, records do not indicate.

\subsection*{Usage}

These make excellent solo threats at low levels, usually tied to a plot point or quest. Beyond that, they serve as mind-meddling minions.

\begin{minipage}{0.48\textwidth}
\begin{itemize}
  \item NPC: Mind Lurker
\end{itemize}
\end{minipage}\hfill
\begin{minipage}{0.48\textwidth}

\end{minipage}

\subsection*{Madness Regurgitator}

% [Image Inserted Manually]

Called “Sargurkthul” by ancient texts, madness regurgitators are a pure example of the twisted imitation of life spouted out by the intersection where the very idea of life is a mockery. These creatures devour (and violently regurgitate) the emotions and psionic energy of living creatures. They can be kept by other aberrations as something akin to horrifying pets.

\subsection*{Usage}

These can serve as an artillery unit for any aberration fight that needs more damage and debuffing. They are not particularly dangerous as a solo threat.

\begin{minipage}{0.48\textwidth}

\end{minipage}\hfill
\begin{minipage}{0.48\textwidth}
\begin{itemize}
  \item NPC: Madness Regurgitator
\end{itemize}
\end{minipage}

\subsection*{Nightmare Stalker}

% [Image Inserted Manually]

Also known as “Urktathlin” by the rare scholar that has studied beings of the beyond and not gone immediately mad, these creatures seem drawn to those descending into madness, stalking and devouring as their mind deteriorates, savoring that tenuous last grasp of sanity slipping away beneath their fangs.

\subsection*{Usage}

These are a low-level solo threat that can become an excellent pack minion at higher levels.

\begin{minipage}{0.48\textwidth}
\begin{itemize}
  \item NPC: Nightmare Stalker
\end{itemize}
\end{minipage}\hfill
\begin{minipage}{0.48\textwidth}

\end{minipage}

\subsection*{Fear Eater}

% [Image Inserted Manually]

Fear eaters look ponderous and slow, but they can burst into horrifying, lopsided bursts of momentum, bringing all the terror of a jumping spider as they close distance far too quickly for comfort. They possess potent psionic abilities that often reduce their targets to shattered husks before feasting on their flesh. These aberrations are said to gather where the concentrated fear of mortals pools, devouring them and savoring fear and flesh alike.

\subsection*{Usage}

The plague of many small communities, these make a good, early bounty quest to hunt down in a horror-inspired mission.

\begin{minipage}{0.48\textwidth}

\end{minipage}\hfill
\begin{minipage}{0.48\textwidth}
\begin{itemize}
  \item NPC: Fear Eater
\end{itemize}
\end{minipage}

\subsection*{Screaming Runner}

% [Image Inserted Manually]

Running and screaming is a common reaction to encountering one of these, but they are named for their habits rather than the response they instill in others. Swift predators, they revel in terror and chaos.

\subsection*{Usage}

These are a great reason to be afraid of the dark at lower levels, and they are a chaotic pack hunter at higher levels that will continue to be a hefty nuisance.

\begin{minipage}{0.48\textwidth}
\begin{itemize}
  \item NPC: Screaming Runner
\end{itemize}
\end{minipage}\hfill
\begin{minipage}{0.48\textwidth}

\end{minipage}

\SetPartTwo

\section{Introduction}

% [Image Inserted Manually]

\section*{Crafting}

Crafting has long been a major focus of the game, and something countless players have sought to engage with since the early days; after all, what adventurer wouldn’t want to be better outfitted, better equipped and more prepared... and what adventurer wouldn’t want the chance to put their personal mark on their gear?

The core principle of this crafting system is simplicity and specificity. Everything works with simple, easy to follow rules, but has enough depth to lift the burden from a GM having to fill in every blank by giving specific details for everything. While a GM is always the arbiter of what exists and if a crafting recipe works, by and large this system can be run with as much or as little input as a GM wants to add. It’s a complete package that provides all the necessarily details for your players to get started—and plenty of tables and guides for GMs as well.

\subsection*{Why Do You Need Crafting?}

Some people may wonder: why do you need a crafting system? Isn’t that what loot is for? The truth is, in some games, that’s true. Not every adventurer is going to want to pursue crafting. But with a crafting system, not only can you craft what you need without finding it in a dragon hoard, what you find in that dragon hoard can be so much more.

In a game with a robust crafting system, there is no junk, there are just more opportunities and fresh possibilities. A +1 shortsword that no one can use could be the valuable basis of a new spear. Gems, gold, relics, and recyclables... all valid entry points for the crafter’s creativity.

Adventurers are inherently innovative folks on a quest for creative solutions to difficult problems. Crafting gives them that toolbox.

So... why do you need crafting? You don’t. But you should probably want it.

\subsection*{Who Can Craft?}

Anyone! Who can succeed in making something useful? Perhaps a bit of a different story. Crafting is not inherently tied to class, though in some cases some classes may give benefits to it (like Inventor); crafting may come from your background in the form of a tool proficiency, or it may be something you learn during your adventures following the old adage of necessity being the mother of invention.

Crafting is mostly about the time, effort, knowledge and materials. As such, most of crafting is knowing the recipe and having the time and materials needed... but a skilled craftsman works quicker and is more successful, and in this business practice makes perfect, so there are various progression modifiers that apply. Your GM can determine if your background would merit starting your adventure with any, otherwise guidelines for how to gain them are included.

\subsection*{What Can I Craft?}

Anything! But this guide is made by a mere mortal, and is thus limited in scope. This guide will provide the principles of crafting for many fields—from alchemy to engineering to woodworking.

The basis of how crafting works is similar between each field, but the recipes, material, and most important results will be radically different... After all, a healing potion, a catapult, and a magic sword are all things you can craft, but the process for each varies quite a bit.

The goal of this document is to teach you how to get started, and provide the basics that will get you a long way into your adventure, but not make a complete codex of everything that could potentially be crafted. When you hit something doesn’t appear in this document, just reference the closest items and make a bit of a leap to what extra steps might be needed to realize your vision into your game world!

\subsection*{A Player-Driven System}

\begin{minipage}{0.48\textwidth}
One of the fundamental goals and inspirations of the crafting system is to make it a player-driven system.

It is a system where the player can say “I would like to harvest the monster for ingredients” and “I would like to forage as we go through the forest looking for alchemy reagents” and ultimately “I would like to make a healing potion” and all those rules can be exposed to the player allowing them to directly browse and consider what they want to make, as well as how to make it.

The GM still adjudicates many instances of them, but the ideal is to have a system in which the GM does not have to handcraft every instance of gathering materials and crafting.
\end{minipage}\hfill
\begin{minipage}{0.48\textwidth}
Working With Your GM

While the system is meant to enable player-driven choice, always remember to work with your GM. Check in with them when selecting an item you would like to make and confirm that item exists in their world, and any assumptions hold true.

Some GMs may rule that some items have higher rarity or are simply not available at their default assigned rarity. A common example of this may be something like Winged Boots, which are only rated as Uncommon, but some GMs may feel they are a better fit for rare and adjust their difficulty accordingly.
\end{minipage}

\subsection*{Hooking Your Players In}

On the other hand, if the GM wants to get the players into it, there are some tools they can use. By far the most effective tool is to give the players reagents as part of loot that don’t have an obvious place to sell them. If you give players 2 curative reagents, they are going to start looking into how they can use those, as they’d much rather have a healing potion.

If you want to go one further though, if you give them 5 curative reagents and they realize they will have a remainder of one... then they start looking into “Well, how do we get a 6th!”

\subsection*{Depth and Complexity}

This system has two goals: to be simple and easy to use, and to be deep and extensible. Naturally these are somewhat at odds, and accomplished by having a great deal of optional depth. To produce standard items with standard effects, the process for finding or buying the materials and using them to make what you want to make will be straightforward.

How much of the detail you want to engage with as a GM can be easily adjusted by how you hand out reagents. By sticking to the standard ingredients and using their generic names, the materials are no more complicated than handing out gold or other rewards (and can even be fully converted easily to a gold-based system if you want the most simplified version), but if you’d like to have specific ingredient names and exotic ingredients there is information in the appendix you can draw from for that.

\subsection*{How to Craft}

\begin{minipage}{0.48\textwidth}
Crafting under this system is very simple. You collect the ingredients for a particular thing you want to make, and use your skills with tools to make it into that item. The crafting time of this system is very compressed compared to the systems you might find in the base options because this system is not intended to use downtime to gate crafting.

Rather, this is built to work with an adventuring lifestyle. You might need a little time here and there, but it will be measured in hours rather than days and weeks. Consequently, the gates to crafted items are instead the materials and skill required rather than time and gold, though you can certainly use gold to buy those materials in some cases.

Following there is a quick summary, as well as more indepth break down of crafting time and checks, materials and where to find them, and finally the specifics of each branch.
\end{minipage}\hfill
\begin{minipage}{0.48\textwidth}
Generic Ingredients

Above and throughout the document, you will see that ingredients are referred to by generic tags like “common curative reagent” rather than specific natures. For example, you may harvest magical herbs, and find Kingsbane in the forest, a poisonous plant. For the purposes of crafting, this can be recorded simply as a “common poisonous reagent” and used as such in crafting.

This greatly simplifies the process of crafting and recording what your supplies are. Narratively speaking, a skilled alchemist can render down the ingredients they want to use in the form they need.

Each crafting profession will have some profession-wide materials that are used in their recipes: reagents for alchemy, metals for blacksmithing, etc.

Some very rare and legendary items will have specific ingredients; this is for flavor rather than balance, though it is up to your GM.
\end{minipage}

\section{Introduction}

\section*{Summary and References}

\section*{Quick Summary}

Each step will be outlined in more depth, but the following serves as a summary and reference for the process.

\begin{itemize}
  \item Select an Item. Find the item you would like to craft on a crafting table.
  \item Check with your GM. They will confirm if that item exists and has the default rarity in their game. This system is a tool, it does not supersede worldbuilding.
  \item Gather the materials. Materials can be looted from humanoids, harvested from more exotic creature types, purchased at stores, or gathered from the wilderness. The specifics of any material and where it can be found is covered in the materials section.
  \item Begin crafting! You can find the related tool and ability score on the table below. Use the related tool and skill to determine your crafting modifier using the following formula:
\end{itemize}

Crafting Modifier = your Related Tool proficiency bonus + your Related Ability Score modifier.

\begin{itemize}
  \item You can make progress in 2 hour increments. For each 2 hours spent, make a crafting roll using the formula above and compare the result to the DC listed for the item
  \item On failure, no progress is made. If you fail three times in a row, the crafting fails and all materials are lost.
  \item On success, 2 hours of progress is made, and you can mark your progress put it aside or continue to craft.
  \item Once your completed progress on an item is equal to the crafting time listed, the item is complete.
  \item You’re done! Enjoy your shiny new item!
\end{itemize}

Special Reminders:

\begin{itemize}
  \item You can “take 10” on a crafting recipe by doubling the crafting time.
  \item The final say on any item is up to your GM. If they disagree with the written recipe, they are correct!
  \item Don’t be afraid to ask where you can find materials—the GM wants you to engage with the world and find things!
  \item A crafting roll is a special kind of check. You can’t apply boosts other than expertise to your crafting check. In general, the only way other than expertise to boost a crafting roll is to get help from another skilled crafted for the entire duration of the craft. Remember you can use the take 10 option to get slower more certain success, and if a craft has a higher DC than you can achieve using that method, it likely should involve some degree of difficulty and risk.
\end{itemize}

\section*{Quick Reference}

\begin{tabularx}{\textwidth}\toprule
{}XX}
\midrule
Profession & Related Tool & Related Ability Score \\
\midrule
Alchemy & Alchemist’s Supplies & Wisdom or Intelligence \\
\midrule
Blacksmithing & Blacksmith’s Tools & Strength \\
\midrule
Cooking & Cook’s Utensils & Wisdom \\
\midrule
Enchanting & Arcana & Intelligence \\
\midrule
Engineering & Carpenter’s Tools & Intelligence \\
\midrule
Jewelcrafting & Jeweler’s Tools & Dexterity \\
\midrule
Leatherworking & Leatherworker’s Tools & Dexterity \\
\midrule
Poisoncraft & Poisoner’s Kit & Wisdom or Intelligence \\
\midrule
Runecarving & Varies & Wisdom or Intelligence \\
\midrule
Scrollscribing & Calligrapher’s Supplies & Intelligence \\
\midrule
Tinkering & Tinker’s Tools & Intelligence \\
\midrule
Wandwhittling & Woodcarver’s Tools & Dexterity \\
\midrule
Woodcarving & Woodcarver’s Tools & Dexterity \\
\midrule
\end{tabularx}

\section*{Reference Prices}

The prices of this system reference the default prices presented for magic items in 5e, as follows. You don’t need to use these prices if you use an alternate economy, but the prices presented are default for ease of use and conversion.

\subsection*{Default Pricing}

\begin{tabularx}{\textwidth}\toprule
{}XX}
\midrule
Rarity & Consumable Price & Price \\
\midrule
Common & 25–50 gp & 50–100 gp \\
\midrule
Uncommon & 50–250 gp & 101–500 gp \\
\midrule
Rare & 250–2,500 gp & 501–5,000 gp \\
\midrule
Very Rare & 2,500–25,000 gp & 5,001–50,000 gp \\
\midrule
Legendary & 25,000+ gp & 50,000+ gp \\
\midrule
\end{tabularx}

Converting Prices

There are alternative pricing models presented in the appendix. If you use a simple conversion, such as 1/10th pricing using some sort of silver-based pricing, you can simply divide based on that. If you have a more complicated pricing version, my recommendation is to pick a few core items and use them as a point of reference for a conversion formula (I’d recommend healing potions, +1/+2/+3 weapons, and plate armor).

\section*{Valuable Items}

In some cases, you may want to make something that is more valuable. While selling more valuable gear may be quite difficult, as is often said, magic has expensive tastes. Many magic items that an Enchanter might want require items of a certain value.

Here’s some examples of specific modifiers you can add to increase the value of an item:

\begin{tabularx}{\textwidth}\toprule
{}XXX}
\midrule
Modifier & Material & Difficulty & Value Added \\
\midrule
Bejeweled & Gemstones & +1 per gem & Value of Gem \\
\midrule
Gilded & 10 gold scraps & +2 & +20 \\
\midrule
\end{tabularx}

Additionally, you can generally increase the value of an item simply by making it more ornate and exercising greater craftsmanship. You can, when you start crafting an item, raise the DC and/or crafting time of that item artificially. When you do so, that items value is recalculated using the Labor Cost Table in Appendix A.

Here is a list of example break points of more expensive mundane items using different variations of material cost, DC, and number of checks. These are all simply the iterations of the formula presented in Appendix A.

\begin{tabularx}{\textwidth}\toprule
{}XXXXX}
\midrule
\multicolumn{2}{c}{Name} & Materials & Checks & DC & Price \\
\midrule
\multicolumn{2}{c}{Simple Item} & 1 sp & 1 & DC 8 & 2 sp \\
\midrule
\multicolumn{2}{c}{Complicated Item} & 1 sp & 2 & DC 15 & 36 gp \\
\midrule
\multicolumn{2}{c}{Difficult Item} & 1 sp & 8 & DC 15 & 150 gp \\
\midrule
\multicolumn{2}{c}{Fancy Item} & 100 gp & 8 & DC 15 & 250 gp \\
\midrule
\multicolumn{2}{c}{Lavish Item} & 250 gp & 7 & DC 15 & 500 gp \\
\midrule
\multicolumn{2}{c}{Masterwork Item} & 1 gp & 5 & DC 20 & 1000 gp \\
\midrule
\end{tabularx}

Example

To walk through these prices:

\begin{itemize}
  \item A simple item listed here has 1 sp of materials, and takes 1 check with a DC of 8. That is a labor cost of 1 sp, 1 sp of materials, and consequently, the item has a value of 2 sp.
  \item A fancy item here has 100 gp of materials, but also takes 8 checks at a DC of 15! That is a labor cost of 144 gp and a small risk multiplier (the chance you lose your materials) resulting in a value of 250 gp.
  \item The presented masterwork item has very little material cost with only 1 gp, but is extremely hard to make with 5 checks with a DC of 20, resulting in a massive labor cost of 995 gp, and a final value of 1,000 gp.
\end{itemize}

So, for example, if you needed a longsword that was worth 1,000 gp, you have a few options:

\begin{itemize}
  \item You can add a gem worth 985 gp to the materials when crafting it, resulting in 15 gp + 985 gp.
  \item You could raise the checks to 5 and the DC to 20.
  \item You could spend 250 gp of materials, raise the DC to 19, and spend 6 checks making it resulting a sword worth 1,103 gp.
  \item These cases are largely something you only need to consider when making items for enchanting, and it’s all quite a bit simpler than it might seem, as it’s just expressions of the formula presented in Appendix A.
\end{itemize}

\section*{Making Your Own Items}

This system is designed to be extensible. While it contains many things, it does not (and cannot!) contain all the things. You can find the rules for building new items in Appendix A: Creating New Items.

\section*{Item Key}

At some points on the tables that follow you will see some notification such as B . This represents an item that is not core to 5e and has been added by the creator of this Supplement (standing for Bakku).

\section{Introduction}

\section*{Crafting Time}

All items have a crafting time measured in hours. Crafting is completed in 2-hour chunks. Some branches of crafting must make all their checks consecutively (like Alchemy) while some do not (like Blacksmithing or Enchanting). A crafting roll is made every 2 hours of work on an item.

On success, you mark 2 hours of completed time. Once the completed time is equal to the crafting time, the item is complete. On failure, the crafting time is lost, and no progress has been made during the 2 hours.

If you fail three times in a row, the crafting is a failure, and all materials are lost.

\section*{Crafting During a Rest}

During a long rest, you can make up to 2 hours of progress during a crafting project if you do nothing else during the long rest besides craft and sleep. During this time, you have disadvantage on any Wisdom (Perception) checks and a −5 to all Passive Perception checks as you are focused on your craft.

You can make 2 hours of progress on any crafting project by taking this camp action. For most projects, this progress can be banked completing the project 2 hours at a time. For some types of crafting, like alchemy, you can’t make incremental progress, but you can continue for 2 additional hours of crafting before or after a long rest in which you took the camp action “craft” to get 4 hours (for example).

Camp Actions

A recommended system is the Camp Actions which can be found here and provide more formalized rules for how to make use of your time during a long rest.

\section*{Crafting Rolls}

\begin{minipage}{0.48\textwidth}
Each profession lists the related ability modifier and tool used to calculate your crafting roll modifier. In many cases, attempting to craft without the applicable tool is impossible. Your GM may allow improvised tools, and in that case the check is made with disadvantage.

Beyond the tool, most branches of crafting require a heat source, and some require more infastructure. These are generally approached logically and are at the discretion of the GM. For an adventurer looking for more portable workshop tools, investigate the Crafting Magic Spells and Crafting Utility Magic Items sections; these offer additional avenues for increased portability.
\end{minipage}\hfill
\begin{minipage}{0.48\textwidth}
\subsubsection*{Taking 10}

When you craft something, double the crafting period (from 2 hours to 4 hours per check) to “take a 10” on the crafting roll, meaning that your crafting checks are 10 + your related ability score + your related tool proficiency bonus. This provides a floor level that you can always succeed on if you choose to take your time.

If someone with proficiency is aiding you for the entire duration of the crafting, the crafting check doesn’t gain any bonus, but can be completed in the standard 2 hours.
\end{minipage}

\begin{minipage}{0.48\textwidth}
Here is a list of some of the additional requirements by branch:

\begin{itemize}
  \item Blacksmithing can perform minor crafts with a simple heat source but requires a forge and anvil to make new items.
  \item Alchemy and Cooking almost always requires a heat source.
  \item Tinkering, Jewelcrafting, and Poisoncraft sometimes require a heat source.
  \item Enchanting, Scroll Scribing, Woodworking, Wand Whittling, Poisoncraft, Leatherworking, Engineering, Runecarving rarely require anything besides the tools of the profession.
\end{itemize}
\end{minipage}\hfill
\begin{minipage}{0.48\textwidth}
NPC Crafting

It should always be assumed that when NPCs are crafting, they take this option, and thus craft more slowly but reliably. The slap-dash speed crafting is largely the domain of the players, though at the GM’s discretion the players may convince an NPC to craft faster assuming the risk of their materials.
\end{minipage}

\begin{minipage}{0.48\textwidth}
\subsubsection*{Expertise}

Features that grant tool expertise (the ability to add twice your proficiency bonus to ability checks made with a tool) are calculated into your crafting roll modifier, but do not add to the calculated DC the effects of that item have.

\subsubsection*{Bonuses}

A crafting roll is a special type of ability check, and you can’t add temporary bonuses to your roll (such as Bardic Inspiration or the spell guidance) as the crafting roll represents the entire time you spend working the item.

In general, the intention is that no external bonuses apply, unless they specifically state they do. This would include feats (such as ones that allow you to reroll certain rolls), class features (such as ones that allows you to add dice to certain rolls), etc. A GM can apply these bonuses, but they are not intended to work, and can result in checks becoming too easy.

You can gain advantage if another player with proficiency in the related tool helps for the whole crafting time.
\end{minipage}\hfill
\begin{minipage}{0.48\textwidth}
\subsubsection*{Workdays}

When a character is spending all day working, they make 8 hours of progress, and cannot exceed 8 hours working on a crafting project during a day. Players will occasionally want to work longer than 8 hours, but this fails to account for the limitations of mortals: player characters still need to eat, sleep, and will eventually lose their focus and ability to work. This is primarily implemented for balance reasons, but also helps model how much productive time a character can spend; they are not just working 8 hours straight and idling for 8 hours, but rather a model of the natural flow of a day, taking breaks, etc.

A GM can choose to overrule this and allow extreme crafting days in special circumstances, but in almost all cases working over 8 hours on a crafting project would leave a character with one or more levels of Exhaustion.
\end{minipage}

\section{Materials}

\section*{Materials}

Without materials, there is nothing to craft from. Crafting does not make things from thin air; it makes more useful things from less useful things. Gathering the materials will be the essential first step in any job. Materials are generally found in three ways.

\section*{Looting \& Salvaging}

Nothing is useless when you have a party of crafters. One of the main sources of materials will be the things you find. Nothing is useless when you have an expert craftsman in the party.

\section*{Gathering \& Harvesting}

How something is gathered varies depending on the profession; in many cases, it can only be gathered when the opportunity arises. Dragon scales, for example, are a lot easier to gather when there is a dead dragon nearby. Be it harvesting rare herbs, monster parts, or minerals, gathering is an opportunity you won’t want to pass up.

\section*{Purchasing}

Rarely will everything you need to craft what you want fall into your hands without the assistance of the oldest and most powerful tool of any craftsman: money. When you don’t have what you need, frequently you can buy it. For some professions, there will be a lot more materials that can be purchased, while others will rely more on the other routes.

Selling and Buying Materials

In general, the buying price of a material is its listed value, and its selling price is usually half of that to an interested property. Between negotiation, market fluctuation, and GM moods, you may get better or worse prices. Note that many materials are simply junk to a party that does not have a use for them and will only sell to interested parties that can use or resell them. Threatening to burn down a merchant’s shop because they will not offer you the listed price may result in an Intimidation check but does not change market realities and is typically not beneficial to your reputation; most merchants that have the gold to buy and sell expensive materials have dealt with adventurers before and are not easily intimidated.

The sources of materials are tied to the pillars of the game. Looting and Harvesting are tied into the combat pillar, Salvaging and Gathering are tied into the exploration pillar, and Purchasing and Rewards are tied into the social pillar. This provides a lot of routes to add these materials to your game based on what works best for your group.

\section*{Types of Materials}

\section*{Reagents}

Reagents are a huge range of things; most often they are plants that contain some magical essence, but almost as frequently they are harvested from various magically inclined monsters. The exact source of a reagent usually does not matter beyond defining its type, as the part of the reagent used is the fragment of magic contained within that is distilled out.

There are many different ways to make a potion. Consequently, the materials are sorted into categories. These categories include curative, reactive, and poisonous. These each come in the standard material rarities: common, uncommon, rare, very rare, and legendary.

Reagents can’t be salvaged once they have been combined into another form (such as potions, essences, or ink).

Reagants can be assumed to weigh 0.2 pounds each.

% [Image Inserted Manually]

Interchangeable Reagents

All curative, reactive, and poisonous ingredients are interchangeable. This is intentional to drastically simplify the crafting process and tracking thereof. Individual names are included only to deepen the immersion of the finding and buying ingredients, and can be treated as interchangeable by their label if preferred.

\section*{Magical Ink}

While ink has many uses, crafting is mostly concerned with magical ink which has the power to hold the arcane words of scrolls. This is synthesized by alchemists from the magical properties of reagents, as it is concerned with extracting their magical properties, the exact nature of the reagents used do not effect the final ink beyond its potency.

Magical ink is not typically found or harvested on its own, though it may be found as loot, and in some instances a GM could rule that some blood collected from a fiend, celestial or dragon could be counted as such. It is generally created from reagents or purchased from alchemists that create it from reagents.

Magical ink can’t be salvaged once they have been combined into another form (such as potions, essences, or ink).

Magical Ink can be assumed to wight 0.1 pounds each.

\section*{Essences}

While reagents are substances that contain a glimmer of magical power that can be harnessed through refinement, Essences are more purified forms of magical power. These come in three types: Arcane, Divine and Primal as well as in the five normal rarities (common, uncommon, rare, very rare, and legendary). These essences are the pure stuff of magic that makes things work.

You can get these by rendering down magical reagents, salvaging magic items, harvesting them from magical monsters, or through the hard work of spell casters. Or you can find them as loot from people that have already done one of those harder steps. The rules for rendering them down from materials are contained within each branch of crafting, while the rules for creating them yourself are under Enchanting, as it is their domain and skill set needed to do so.

While all branches occasionally use essences when extra magical power is needed, they are the primary material of Enchanters, and their pricing can be found in that section.

Essences can be found as loot during the courses of your adventures, but can also be harvested (from monsters), salvaged (from magical equipment), synthesized (from reagents), or created from the raw power of a spell caster, though the method is long and arduous.

Essences are flexible in their exact nature. There are many paths to each desired outcome, and this flexibility is represented in Essences. While the traditional way to make a belt of hill giant strength may call for a hill giant heart as its essence, an enchanter may substitute a dragon heart as their primal essences to make a belt of dragon strength that just has the same statistical effect.

Essences can be assumed to weigh 1 pound each.

\begin{minipage}{0.48\textwidth}
\subsubsection*{Salvaging Essences}

You also may be able to salvage magical essence from unwanted or broken magical items, though such a reclamation process can be difficult, and rarely results in more than a fraction of the essence infused into the original item. An item returns one essence equal to its rarity when harvested. The process takes 2 hours to complete, and doesn’t work if the item is currently attuned to any creature. An essence can only be salvaged from permanent magic items; a permanent magic item is one that recharges or doesn’t have a limitation on its charges or uses. A magic item with charges or uses can only be salvaged while it is at full charges or uses. The item becomes nonmagical after the essence is salvaged from it. If it required magic to function or exist, it is destroyed.
\end{minipage}\hfill
\begin{minipage}{0.48\textwidth}
\subsubsection*{Synthesizing Essences}

In addition to harvesting essences from magical monsters fully intact, a more approachable and incremental way is to combine several reagents to get an essence. You have to combine three reagents of the same rarity to gain one essence of that rarity. You can combine reagents in the following ways:

\begin{tabularx}{\textwidth}\toprule
{}XXX}
\midrule
Essence & \multicolumn{3}{c}{Component Reagents} \\
\midrule
Arcane & \multicolumn{3}{c}{1 curative, 1 poisonous, 1 reactive} \\
\midrule
Primal & \multicolumn{3}{c}{3 reactive} \\
\midrule
Divine & \multicolumn{3}{c}{2 curative, 1 reactive} \\
\midrule
\end{tabularx}

This process takes 4 hours, and requires alchemist’s supplies and a heat source.
\end{minipage}

\begin{minipage}{0.48\textwidth}
\subsubsection*{Making Essences}

Another potential source of an Essence is being created by a spellcaster. This process is long and arduous, and typically only suited to downtime. A creature with the spell casting feature can create 1 essence during 1 workweek (5 days, 8 hours a day); this process can’t be completed faster and for the duration they are considered to have spent all of their spell slots.

At 1st level or higher can make a common essence in this way, a caster 5th level or higher can make an uncommon essence this way, a caster at 11th level can make a rare essence in this way, and a caster at 17th level or higher can make a very rare essence this way. Legendary essences require special rituals, more casters, and take far longer; they are exceedingly hard to make.

The type of essence produced depends on the source of the spell casting levels as per the table below:
\end{minipage}\hfill
\begin{minipage}{0.48\textwidth}
\begin{tabularx}{\textwidth}\toprule
{}X}
\midrule
Caster & Essence Type \\
\midrule
InventorK & Arcane \\
\midrule
Bard & Arcane \\
\midrule
Cleric & Divine \\
\midrule
Druid & Primal \\
\midrule
Monk & Psionic \\
\midrule
OccultistK & Any* \\
\midrule
Paladin & Divine \\
\midrule
PsionK & Psionic \\
\midrule
Ranger & Primal \\
\midrule
Warlock & Varies* \\
\midrule
Sorcerer & Varies* \\
\midrule
Wizard & Arcane \\
\midrule
\end{tabularx}
\end{minipage}

% [Image Inserted Manually]

Special Cases Explained

\begin{itemize}
  \item Sorcerers produce a type based on their subclass; Dragon or Wild makes Primal, Divine Soul makes Divine, and Shadow makes Arcane.
  \item Warlocks likewise produce a type based on their subclass; Archfey makes Primal, Celestial makes Divine, and all others make Arcane.
  \item Occultist can produce any type, but takes 1.5x as long to produce an Essence in this manner. A GM can rule based on the special circumstances of a character their power source may be different than normal. This can stand in for Shaman, Witch, or Oracle classes if you use those instead of Occultist.
  \item Inventor can stand in for any half-arcane caster of a similar theme.
\end{itemize}

A half- or third-caster would generate essences at 1/2 or 1/3 their character level, respectively.

\section*{Ingots}

Ingots are chunks of metal that can be used to craft things. They are assumed to be relatively pure and weigh 2 pounds each. The default ingot listed in all the crafting tables is an ingot of Steel. These cost 2 gp per ingot. There are cheaper metals (such as Iron); pure Iron can’t be used to craft weapons and armor, but can be used for other items, resulting in a cheaper item. On the other end of the spectrum, more advanced metals such as Mithral and Adamantine can be used conferring special properties, but being far more difficult to work with and costing more.

Ingots can be assumed to weight 2 pounds each.

\subsection*{Salvaging Ingots}

% [Image Inserted Manually]

Metal items can be converted back to ingots quite efficiently, but require a forge to do so. With a forge and 2 hours per item, a metal item can be rendered down into its component ingots. Advanced metals may require special tools to smelt.

\subsection*{Smelting Ore}

Creating ingots from raw ore is largely out of scope for most adventurers, but you can create ingots from raw ore with a suitable facility. For more details see the Components and Materials table under Blacksmithing.

\section*{Hides \& Leather}

\begin{minipage}{0.48\textwidth}
Hides, scales, and carapaces all tend to be harvested from monsters. Leather is a product of hides that can be processed from what it is harvested from the monster.

The GM determines if a monster provides hide, scale, or carapace. Hides do not come in different sizes, rather larger creatures simply provide more hides, and monsters that are not large enough to produce one hide provide only hide scraps.

Scales are likewise abstracted: each increment is simply an arbitrary unit of scales that the unit of scales covers. Scales can be much larger or small from different-sized creatures.
\end{minipage}\hfill
\begin{minipage}{0.48\textwidth}
Processing Hides

The process of turning hide into leather takes quite awhile (as per the crafting table), and is often something adventurers can delegate to NPCs (delivering hides to be processed) or do during downtime. If you would like a more expedited system, there is no balance reason for this, and you can short the leather crafting process to taking 2 hours, it just won’t be exceedingly realistic.
\end{minipage}

Like other crafting branches, there are also named components for more iconic pieces of gear—the stock of a crossbow, for example, or other items. The cost for these items can be found on the common component table, and are generally minor.

Lastly, Tinkerers use essences when constructing things that push beyond the mundane principles of plausibility, crafting magical properties into their inventions.

\section*{Parts}

\begin{minipage}{0.48\textwidth}
The term “parts” is used to refer to gears, wires, springs, windy bits, screws, nails, and doodads. Parts can be either found or salvaged or forged from metal scraps (or even straight from ingots by a Blacksmith for those that really want to be industrial about it). The exact nature of each item making up this collection is left abstracted.

In addition, metal scraps are collections of salvaged material that generally fall into the category of things “too small to track” which can then be used for the creations of tinkerers. In addition to all of this, occasionally tinkers will use ingots... particularly ones of tin (which is their namesake, after all).
\end{minipage}\hfill
\begin{minipage}{0.48\textwidth}
Named Components

In almost all cases, named components (such as a “wooden stock” for a crossbow) can be simply abstracted out as a minor cost, but, as always, the level of abstraction is up to the GM.
\end{minipage}

\subsection*{Salvaging Parts}

The other main way to acquire parts is to salvage them. What can be salvaged is determined by the GM, but in general common items provide parts, uncommon or expensive items may provide fancy parts, and esoteric parts are found only from esoteric sources at your GM’s discretion. Tools, vehicles, and complex items generally return 1d4 metal scraps and 1d4 parts for a Small or smaller item, 2d6 metal scraps for a Medium-sized item, 3d8 metal scraps for a Large-sized item, and more for larger items, though they may return less of rare types of parts.

\section*{Wood}

Commonly available in its lowest quality (firewood), higher quality woods are often found in rather exotic locations. Wooden branches (including wood scraps) are assumed to be of a useful wood that can be worked, while firewood covers everything else, with more useful woods falling into categories such as “quality branches” or rarer options. Wood scraps are assumed to be scraps of common branch quality wood, and consequently can’t be salvaged from firewood. Wooden branches can be assumed to weigh 2 pounds each.

\subsection*{Salvaging}

\begin{minipage}{0.48\textwidth}
For the most part, wood can’t be easily salvaged. Wood carving is not necessarily a reversible process, and wood can’t be smelted down.

You can render wooden crafted product into wood scraps equal to 4 x the number of branches used to create it.
\end{minipage}\hfill
\begin{minipage}{0.48\textwidth}
Quality Branch

A quality branch refers to one that can be made into more precious objects, particularly bows. It is nonmagical in nature, but typically yew when dealing with bows, though ash, mulberry, elm, oak, hickory, hazel, and maple can be used under broader definitions.
\end{minipage}

\section{Materials}

\section*{Purchasing Tables by Type}

\subsection*{Leather \& Hide}

\begin{tabularx}{\textwidth}\toprule
{}XXXXXXXXX}
\midrule
\multicolumn{4}{c}{Materials} & Rarity & \multicolumn{4}{c}{Used For} & Price \\
\midrule
\multicolumn{4}{c}{Hide Scraps} & Trivial & \multicolumn{4}{c}{Leatherworking} & 1 sp \\
\midrule
\multicolumn{4}{c}{Leather Scraps} & Trivial & \multicolumn{4}{c}{Leatherworking} & 1 sp \\
\midrule
\multicolumn{4}{c}{Boiled Leather} & Common & \multicolumn{4}{c}{Leatherworking} & 3 gp \\
\midrule
\multicolumn{4}{c}{Hide} & Common & \multicolumn{4}{c}{Leatherworking} & 2 gp \\
\midrule
\multicolumn{4}{c}{Rawhide Leather} & Common & \multicolumn{4}{c}{Leatherworking} & 2 gp \\
\midrule
\multicolumn{4}{c}{Scales} & Common & \multicolumn{4}{c}{Leatherworking} & 1 gp \\
\midrule
\multicolumn{4}{c}{Tanned Leather} & Common & \multicolumn{4}{c}{Leatherworking} & 3 gp \\
\midrule
\multicolumn{4}{c}{Medium Carapace} & Common & \multicolumn{4}{c}{Leatherworking} & 4 gp \\
\midrule
\multicolumn{4}{c}{Large Carapace} & Common & \multicolumn{4}{c}{Leatherworking} & 30 gp \\
\midrule
\multicolumn{4}{c}{Tough Hide} & Uncommon & \multicolumn{4}{c}{Leatherworking} & 500 gp \\
\midrule
\multicolumn{4}{c}{Resistant Hide} & Uncommon & \multicolumn{4}{c}{Leatherworking} & 500 gp \\
\midrule
\multicolumn{4}{c}{Tough Leather} & Uncommon & \multicolumn{4}{c}{Leatherworking} & 600 gp \\
\midrule
\multicolumn{4}{c}{Resistant Leather} & Uncommon & \multicolumn{4}{c}{Leatherworking} & 600 gp \\
\midrule
\end{tabularx}

\subsection*{Cooking}

\begin{tabularx}{\textwidth}\toprule
{}XXXXXXXXX}
\midrule
\multicolumn{4}{c}{Materials} & Rarity & \multicolumn{4}{c}{Used For} & Price \\
\midrule
\multicolumn{4}{c}{Supplies (Salt, Staples, etc)} & Trivial & \multicolumn{4}{c}{Cooking} & 1 gp \\
\midrule
\multicolumn{4}{c}{Uncommon Supplies (Uncommon spices, oils, rare seeds, etc)} & Common & \multicolumn{4}{c}{Cooking} & 10 gp \\
\midrule
\multicolumn{4}{c}{Rare Supplies (Hard to find luxury goods)} & Uncommon & \multicolumn{4}{c}{Cooking} & 100 gp \\
\midrule
\end{tabularx}

\subsection*{Metals}

\begin{tabularx}{\textwidth}\toprule
{}XXXXXXXXX}
\midrule
\multicolumn{4}{c}{Materials} & Rarity & \multicolumn{4}{c}{Used For} & Price \\
\midrule
\multicolumn{4}{c}{Metal Scraps} & Trivial & \multicolumn{4}{c}{Tinkering, Blacksmithing} & 1 sp \\
\midrule
\multicolumn{4}{c}{Silver Scraps} & Trivial & \multicolumn{4}{c}{Jewelcrafting} & 1 sp \\
\midrule
\multicolumn{4}{c}{Gold Scraps} & Common & \multicolumn{4}{c}{Jewelcrafting} & 1 gp \\
\midrule
\multicolumn{4}{c}{Iron Ingot} & Common & \multicolumn{4}{c}{Blacksmthing} & 1 gp \\
\midrule
\multicolumn{4}{c}{Steel Chain (2 ft)} & Common & \multicolumn{4}{c}{Blacksmthing, Tinkering} & 1 gp \\
\midrule
\multicolumn{4}{c}{Steel Ingot} & Common & \multicolumn{4}{c}{Blacksmithing} & 2 gp \\
\midrule
\multicolumn{4}{c}{Mithril Ingot} & Uncommon & \multicolumn{4}{c}{Blacksmithing} & 30 gp \\
\midrule
\multicolumn{4}{c}{Adamant Ingot} & Uncommon & \multicolumn{4}{c}{Blacksmithing} & 40 gp \\
\midrule
\multicolumn{4}{c}{Adamantine Ingot} & Uncommon & \multicolumn{4}{c}{Blacksmithing} & 60 gp \\
\midrule
\end{tabularx}

\subsection*{Wood}

\begin{tabularx}{\textwidth}\toprule
{}XXXXXXXXX}
\midrule
\multicolumn{4}{c}{Materials} & Rarity & \multicolumn{4}{c}{Used For} & Price \\
\midrule
\multicolumn{4}{c}{Firewood} & Trivial & \multicolumn{4}{c}{Cooking, Wood Working} & 1 cp \\
\midrule
\multicolumn{4}{c}{Wood Scraps} & Trivial & \multicolumn{4}{c}{Tinkering, Wood Working} & 2 cp \\
\midrule
\multicolumn{4}{c}{Common Branch} & Common & \multicolumn{4}{c}{Wand Whittling, Wood Working} & 1 sp \\
\midrule
\multicolumn{4}{c}{Wooden Stock} & Common & \multicolumn{4}{c}{Tinkering} & 5 sp \\
\midrule
\multicolumn{4}{c}{Short Haft} & Common & \multicolumn{4}{c}{Blacksmithing} & 1 sp \\
\midrule
\multicolumn{4}{c}{Long Haft} & Common & \multicolumn{4}{c}{Blacksmithing} & 2 sp \\
\midrule
\multicolumn{4}{c}{Quality Branch} & Common & \multicolumn{4}{c}{Wand Whittling, Wood Working} & 2 gp \\
\midrule
\multicolumn{4}{c}{Uncommon Branch} & Uncommon & \multicolumn{4}{c}{Wand Whittling} & 25 gp \\
\midrule
\multicolumn{4}{c}{Rare Branch} & Rare & \multicolumn{4}{c}{Wand Whittling} & 80 gp \\
\midrule
\multicolumn{4}{c}{Very Rare Branch} & Very Rare & \multicolumn{4}{c}{Wand Whittling} & 800 gp \\
\midrule
\multicolumn{4}{c}{Legendary Branch} & Legendary & \multicolumn{4}{c}{Wand Whittling} & 2,000 gp \\
\midrule
\end{tabularx}

\subsection*{Magical Materials}

\begin{longtable}{p{2.5cm}\toprule
|p{2.5cm}|p{2.5cm}|p{2.5cm}|p{2.5cm}|p{2.5cm}|p{2.5cm}|p{2.5cm}|p{2.5cm}|p{2.5cm}|}
\midrule
\multicolumn{4}{c}{Materials} & Rarity & \multicolumn{4}{c}{Used For} & Price \\
\midrule
\multicolumn{4}{c}{Common Reagent} & Common & \multicolumn{4}{c}{Alchemy, Poisoncraft} & 15 gp \\
\midrule
\multicolumn{4}{c}{Glass Vial} & Common & \multicolumn{4}{c}{Alchemy, Poisoncraft} & 1 gp \\
\midrule
\multicolumn{4}{c}{Glass Flask} & Common & \multicolumn{4}{c}{Alchemy, Poisoncraft} & 1 gp \\
\midrule
\multicolumn{4}{c}{Crystal Vial} & Common & \multicolumn{4}{c}{Alchemy} & 10 gp \\
\midrule
\multicolumn{4}{c}{Normal Ink} & Common & \multicolumn{4}{c}{—} & 5 gp \\
\midrule
\multicolumn{4}{c}{Parchment} & Common & \multicolumn{4}{c}{Scroll Scribing} & 1 sp \\
\midrule
\multicolumn{4}{c}{Common Essence} & Common & \multicolumn{4}{c}{Alchemy, Enchanting, Scroll Scribing, Wand Whittling} & 45 gp \\
\midrule
\multicolumn{4}{c}{Common Magical Ink} & Common & \multicolumn{4}{c}{Scroll Scribing} & 15 gp \\
\midrule
\multicolumn{4}{c}{Uncommon Reagent} & Uncommon & \multicolumn{4}{c}{Alchemy, Poisoncraft} & 40 gp \\
\midrule
\multicolumn{4}{c}{Uncommon Essence} & Uncommon & \multicolumn{4}{c}{Alchemy, Enchanting, Scroll Scribing, Wand Whittling} & 150 gp \\
\midrule
\multicolumn{4}{c}{Uncommon Magical Ink} & Uncommon & \multicolumn{4}{c}{Scroll Scribing} & 40 gp \\
\midrule
\multicolumn{4}{c}{Uncommon Parchment} & Uncommon & \multicolumn{4}{c}{Scroll Scribing} & 40 gp \\
\midrule
\multicolumn{4}{c}{Rare Reagent} & Rare & \multicolumn{4}{c}{Alchemy, Poisoncraft} & 200 gp \\
\midrule
\multicolumn{4}{c}{Rare Essence} & Rare & \multicolumn{4}{c}{Alchemy, Enchanting, Scroll Scribing, Wand Whittling} & 700 gp \\
\midrule
\multicolumn{4}{c}{Rare Magical Ink} & Rare & \multicolumn{4}{c}{Scroll Scribing} & 200 gp \\
\midrule
\multicolumn{4}{c}{Rare Parchment} & Rare & \multicolumn{4}{c}{Scroll Scribing} & 200 gp \\
\midrule
\multicolumn{4}{c}{Very Rare Reagent} & Very Rare & \multicolumn{4}{c}{Alchemy, Poisoncraft} & 2,000 gp \\
\midrule
\multicolumn{4}{c}{Very Rare Essence} & Very Rare & \multicolumn{4}{c}{Alchemy, Enchanting, Scroll Scribing, Wand Whittling} & 7,000 gp \\
\midrule
\multicolumn{4}{c}{Very Rare Magical Ink} & Very Rare & \multicolumn{4}{c}{Scroll Scribing} & 2,000 gp \\
\midrule
\multicolumn{4}{c}{Very Rare Parchment} & Very Rare & \multicolumn{4}{c}{Scroll Scribing} & 2,000 gp \\
\midrule
\multicolumn{4}{c}{Legendary Reagent} & Legendary & \multicolumn{4}{c}{Alchemy, Poisoncraft} & 5,000 gp \\
\midrule
\multicolumn{4}{c}{Legendary Essence} & Legendary & \multicolumn{4}{c}{Alchemy, Enchanting, Scroll Scribing, Wand Whittling} & 25,000 gp \\
\midrule
\multicolumn{4}{c}{Legendary Magical Ink} & Legendary & \multicolumn{4}{c}{Scroll Scribing} & 5,000 gp \\
\midrule
\multicolumn{4}{c}{Legendary Parchment} & Legendary & \multicolumn{4}{c}{Scroll Scribing} & 5,000 gp \\
\midrule
\end{longtable}

\subsection*{Misc}

\begin{tabularx}{\textwidth}\toprule
{}XXXXXXXXX}
\midrule
\multicolumn{4}{c}{Materials} & Rarity & \multicolumn{4}{c}{Used For} & Price \\
\midrule
\multicolumn{4}{c}{Buckle} & Trivial & \multicolumn{4}{c}{Leatherworking} & 2 sp \\
\midrule
\multicolumn{4}{c}{Fletching} & Trivial & \multicolumn{4}{c}{Wood Working} & 5 cp \\
\midrule
\multicolumn{4}{c}{Length of String} & Trivial & \multicolumn{4}{c}{Wood Working} & 5 cp \\
\midrule
\multicolumn{4}{c}{Armor Padding} & Common & \multicolumn{4}{c}{Blacksmithing, Leatherworking} & 5 gp \\
\midrule
\multicolumn{4}{c}{Parts} & Common & \multicolumn{4}{c}{Tinkering} & 2 gp \\
\midrule
\multicolumn{4}{c}{Fancy Parts} & Common & \multicolumn{4}{c}{Tinkering} & 10 gp \\
\midrule
\multicolumn{4}{c}{Esoteric Parts} & Uncommon & \multicolumn{4}{c}{Tinkering} & 100 gp \\
\midrule
\end{tabularx}

\section{Materials}

\section*{Purchasing Tables by Rarity}

\subsection*{Trivial}

\begin{tabularx}{\textwidth}\toprule
{}XXXXXXXXX}
\midrule
\multicolumn{4}{c}{Materials} & Rarity & \multicolumn{4}{c}{Used For} & Price \\
\midrule
\multicolumn{4}{c}{Firewood} & Trivial & \multicolumn{4}{c}{Cooking, Wood Working} & 1 cp \\
\midrule
\multicolumn{4}{c}{Wood Scraps} & Trivial & \multicolumn{4}{c}{Tinkering, Wood Working} & 2 cp \\
\midrule
\multicolumn{4}{c}{Fletching} & Trivial & \multicolumn{4}{c}{Wood Working} & 5 cp \\
\midrule
\multicolumn{4}{c}{Length of String} & Trivial & \multicolumn{4}{c}{Wood Working} & 5 cp \\
\midrule
\multicolumn{4}{c}{Metal Scraps} & Trivial & \multicolumn{4}{c}{Tinkering, Blacksmithing} & 1 sp \\
\midrule
\multicolumn{4}{c}{Silver Scraps} & Trivial & \multicolumn{4}{c}{Jewelcrafting} & 1 sp \\
\midrule
\multicolumn{4}{c}{Hide Scraps} & Trivial & \multicolumn{4}{c}{Leatherworking} & 1 sp \\
\midrule
\multicolumn{4}{c}{Leather Scraps} & Trivial & \multicolumn{4}{c}{Leatherworking} & 1 sp \\
\midrule
\multicolumn{4}{c}{Buckle} & Trivial & \multicolumn{4}{c}{Leatherworking} & 2 sp \\
\midrule
\multicolumn{4}{c}{Supplies (Salt, Staples, etc)} & Trivial & \multicolumn{4}{c}{Cooking} & 1 gp \\
\midrule
\end{tabularx}

\subsection*{Common}

\begin{longtable}{p{2.5cm}\toprule
|p{2.5cm}|p{2.5cm}|p{2.5cm}|p{2.5cm}|p{2.5cm}|p{2.5cm}|p{2.5cm}|p{2.5cm}|p{2.5cm}|}
\midrule
\multicolumn{4}{c}{Materials} & Rarity & \multicolumn{4}{c}{Used For} & Price \\
\midrule
\multicolumn{4}{c}{Common Branch} & Common & \multicolumn{4}{c}{Wand Whittling, Wood Working} & 1 sp \\
\midrule
\multicolumn{4}{c}{Short Haft} & Common & \multicolumn{4}{c}{Blacksmithing} & 1 sp \\
\midrule
\multicolumn{4}{c}{Long Haft} & Common & \multicolumn{4}{c}{Blacksmithing} & 2 sp \\
\midrule
\multicolumn{4}{c}{Wooden Stock} & Common & \multicolumn{4}{c}{Tinkering} & 5 sp \\
\midrule
\multicolumn{4}{c}{Glass Vial} & Common & \multicolumn{4}{c}{Alchemy, Poisoncraft} & 1 gp \\
\midrule
\multicolumn{4}{c}{Glass Flask} & Common & \multicolumn{4}{c}{Alchemy, Poisoncraft} & 1 gp \\
\midrule
\multicolumn{4}{c}{Parchment} & Common & \multicolumn{4}{c}{Scroll Scribing} & 1 sp \\
\midrule
\multicolumn{4}{c}{Scales} & Common & \multicolumn{4}{c}{Leatherworking} & 1 gp \\
\midrule
\multicolumn{4}{c}{Parts} & Common & \multicolumn{4}{c}{Tinkering} & 2 gp \\
\midrule
\multicolumn{4}{c}{Quality Branch} & Common & \multicolumn{4}{c}{Wand Whittling, Wood Working} & 2 gp \\
\midrule
\multicolumn{4}{c}{Gold Scraps} & Common & \multicolumn{4}{c}{Jewelcrafting} & 1 gp \\
\midrule
\multicolumn{4}{c}{Iron Ingot} & Common & \multicolumn{4}{c}{Blacksmthing} & 1 gp \\
\midrule
\multicolumn{4}{c}{Steel Chain (2 ft)} & Common & \multicolumn{4}{c}{Blacksmthing, Tinkering} & 1 gp \\
\midrule
\multicolumn{4}{c}{Steel Ingot} & Common & \multicolumn{4}{c}{Blacksmithing} & 2 gp \\
\midrule
\multicolumn{4}{c}{Hide} & Common & \multicolumn{4}{c}{Leatherworking} & 2 gp \\
\midrule
\multicolumn{4}{c}{Rawhide Leather} & Common & \multicolumn{4}{c}{Leatherworking} & 2 gp \\
\midrule
\multicolumn{4}{c}{Tanned Leather} & Common & \multicolumn{4}{c}{Leatherworking} & 3 gp \\
\midrule
\multicolumn{4}{c}{Boiled Leather} & Common & \multicolumn{4}{c}{Leatherworking} & 3 gp \\
\midrule
\multicolumn{4}{c}{Medium Carapace} & Common & \multicolumn{4}{c}{Leatherworking} & 4 gp \\
\midrule
\multicolumn{4}{c}{Armor Pading} & Common & \multicolumn{4}{c}{Blacksmithing, Leatherworking} & 5 gp \\
\midrule
\multicolumn{4}{c}{Normal Ink} & Common & \multicolumn{4}{c}{—} & 5 gp \\
\midrule
\multicolumn{4}{c}{Uncommon Supplies (Uncommon spices, oils, rare seeds, etc)} & Common & \multicolumn{4}{c}{Cooking} & 10 gp \\
\midrule
\multicolumn{4}{c}{Fancy Parts} & Common & \multicolumn{4}{c}{Tinkering} & 10 gp \\
\midrule
\multicolumn{4}{c}{Crystal Vial} & Common & \multicolumn{4}{c}{Alchemy} & 10 gp \\
\midrule
\multicolumn{4}{c}{Common Reagent} & Common & \multicolumn{4}{c}{Alchemy, Poisoncraft} & 15 gp \\
\midrule
\multicolumn{4}{c}{Common Magical Ink} & Common & \multicolumn{4}{c}{Scroll Scribing} & 15 gp \\
\midrule
\multicolumn{4}{c}{Large Carapace} & Common & \multicolumn{4}{c}{Leatherworking} & 30 gp \\
\midrule
\multicolumn{4}{c}{Common Essence} & Common & \multicolumn{4}{c}{Alchemy, Enchanting, Scroll Scribing, Wand Whittling} & 45 gp \\
\midrule
\end{longtable}

\subsection*{Uncommon}

\begin{tabularx}{\textwidth}\toprule
{}XXXXXXXXX}
\midrule
\multicolumn{4}{c}{Materials} & Rarity & \multicolumn{4}{c}{Used For} & Price \\
\midrule
\multicolumn{4}{c}{Uncommon Branch} & Uncommon & \multicolumn{4}{c}{Wand Whittling} & 25 gp \\
\midrule
\multicolumn{4}{c}{Mithril Ingot} & Uncommon & \multicolumn{4}{c}{Blacksmithing} & 30 gp \\
\midrule
\multicolumn{4}{c}{Uncommon Reagent} & Uncommon & \multicolumn{4}{c}{Alchemy, Poisoncraft} & 40 gp \\
\midrule
\multicolumn{4}{c}{Uncommon Magical Ink} & Uncommon & \multicolumn{4}{c}{Scroll Scribing} & 40 gp \\
\midrule
\multicolumn{4}{c}{Uncommon Parchment} & Uncommon & \multicolumn{4}{c}{Scroll Scribing} & 40 gp \\
\midrule
\multicolumn{4}{c}{Adamant Ingot} & Uncommon & \multicolumn{4}{c}{Blacksmithing} & 40 gp \\
\midrule
\multicolumn{4}{c}{Adamantine Ingot} & Uncommon & \multicolumn{4}{c}{Blacksmithing} & 60 gp \\
\midrule
\multicolumn{4}{c}{Rare Supplies (Hard to luxury goods)} & Uncommon & \multicolumn{4}{c}{Cooking} & 100 gp \\
\midrule
\multicolumn{4}{c}{Esoteric Parts} & Uncommon & \multicolumn{4}{c}{Tinkering} & 100 gp \\
\midrule
\multicolumn{4}{c}{Uncommon Essence} & Uncommon & \multicolumn{4}{c}{Alchemy, Enchanting, Scroll Scribing, Wand Whittling} & 150 gp \\
\midrule
\multicolumn{4}{c}{Tough Leather} & Uncommon & \multicolumn{4}{c}{Leatherworking} & 600 gp \\
\midrule
\multicolumn{4}{c}{Resistant Hide} & Uncommon & \multicolumn{4}{c}{Leatherworking} & 600 gp \\
\midrule
\end{tabularx}

\subsection*{Rare}

\begin{tabularx}{\textwidth}\toprule
{}XXXXXXXXX}
\midrule
\multicolumn{4}{c}{Materials} & Rarity & \multicolumn{4}{c}{Used For} & Price \\
\midrule
\multicolumn{4}{c}{Rare Reagent} & Rare & \multicolumn{4}{c}{Alchemy, Poisoncraft} & 200 gp \\
\midrule
\multicolumn{4}{c}{Rare Branch} & Rare & \multicolumn{4}{c}{Wand Whittling} & 80 gp \\
\midrule
\multicolumn{4}{c}{Rare Magical Ink} & Rare & \multicolumn{4}{c}{Scroll Scribing} & 200 gp \\
\midrule
\multicolumn{4}{c}{Rare Parchment} & Rare & \multicolumn{4}{c}{Scroll Scribing} & 200 gp \\
\midrule
\multicolumn{4}{c}{Rare Essence} & Rare & \multicolumn{4}{c}{Alchemy, Enchanting, Scroll Scribing, Wand Whittling} & 700 gp \\
\midrule
\end{tabularx}

\subsection*{Very Rare}

\begin{tabularx}{\textwidth}\toprule
{}XXXXXXXXX}
\midrule
\multicolumn{4}{c}{Materials} & Rarity & \multicolumn{4}{c}{Used For} & Price \\
\midrule
\multicolumn{4}{c}{Very Rare Branch} & Very Rare & \multicolumn{4}{c}{Wand Whittling} & 800 gp \\
\midrule
\multicolumn{4}{c}{Very Rare Reagent} & Very Rare & \multicolumn{4}{c}{Alchemy, Poisoncraft} & 2,000 gp \\
\midrule
\multicolumn{4}{c}{Very Rare Magical Ink} & Very Rare & \multicolumn{4}{c}{Scroll Scribing} & 2,000 gp \\
\midrule
\multicolumn{4}{c}{Very Rare Parchment} & Very Rare & \multicolumn{4}{c}{Scroll Scribing} & 2,000 gp \\
\midrule
\multicolumn{4}{c}{Very Rare Essence} & Very Rare & \multicolumn{4}{c}{Alchemy, Enchanting, Scroll Scribing, Wand Whittling} & 7,000 gp \\
\midrule
\end{tabularx}

\subsection*{Legendary}

\begin{tabularx}{\textwidth}\toprule
{}XXXXXXXXX}
\midrule
\multicolumn{4}{c}{Materials} & Rarity & \multicolumn{4}{c}{Used For} & Price \\
\midrule
\multicolumn{4}{c}{Legendary Reagent} & Legendary & \multicolumn{4}{c}{Alchemy, Poisoncraft} & 5,000 gp \\
\midrule
\multicolumn{4}{c}{Legendary Magical Ink} & Legendary & \multicolumn{4}{c}{Scroll Scribing} & 5,000 gp \\
\midrule
\multicolumn{4}{c}{Legendary Parchment} & Legendary & \multicolumn{4}{c}{Scroll Scribing} & 5,000 gp \\
\midrule
\multicolumn{4}{c}{Legendary Branch} & Legendary & \multicolumn{4}{c}{Wand Whittling} & 2,000 gp \\
\midrule
\multicolumn{4}{c}{Legendary Essence} & Legendary & \multicolumn{4}{c}{Alchemy, Enchanting, Scroll Scribing, Wand Whittling} & 25,000 gp \\
\midrule
\end{tabularx}

\section{Harvesting and Looting}

\section*{Harvesting \& Looting}

Harvesting and looting are two paths to the same place, but generally depend on what kind of foe was vanquished and you are now collecting the “stuff” of. Typically humanoid creatures that carry stuff are candidates for the Individual Treasure tables, while Aberrations, Beasts, Dragons, Monstrosities, and Plants are harvesting candidates.

If you don’t normally provide loot equivalent to default treasure tables, you don’t need to start providing loot equivalent to them using these new tables, simple apply these tables as frequently as it makes sense for your game.

Remember that you can fully mix and match as it makes sense. You can replace coinage with gems or art pieces, you can replace crafting items that wouldn’t make sense with coinage, gems, or art pieces, etc. The tables are merely a guide and convenience for what sort of range of materials should come from what sort of creature.

\begin{minipage}{0.48\textwidth}
\subsubsection*{Harvesting}

The Harvesting tables replace the Individual Treasure for Aberration, Beast, Dragon, Monstrosity, and Plant type creatures.
\end{minipage}\hfill
\begin{minipage}{0.48\textwidth}
\subsubsection*{Remnants}

Remnants optionally replace the Individual Treasure table for creatures that leave behind no body on death, like Elementals, Celestials, or Fiends (ones that leave behind a body can use the Harvesting table).
\end{minipage}

\begin{minipage}{0.48\textwidth}
\subsubsection*{Loot}

The Loot tables optionally replace the Individual Treasure for humanoid type creatures. You can use this table in all cases or in some cases.

\subsubsection*{Hoards}

Rather than replacing the hoard tables, simply use the default hoard table and replace an amount of coins, gems, and art pieces with crafting materials. This ensures that players are still getting the sort of loot they expect, but also fills in new materials into things that would often fill little role beside being converted to coinage at a later date.
\end{minipage}\hfill
\begin{minipage}{0.48\textwidth}
Recommendation

I would recommend using the table for all humanoid enemies, but using the Equivalent Gold Value for roughly half of enemies to keep gold flowing into the PCs pockets while also providing abundant crafting supplies.
\end{minipage}

\section*{Basic Harvesting}

Beasts, Dragons, and Monstrosities can be harvested using Wisdom (Survival) for meat and hides. At a GM’s discretion, a Plant type creature can be harvested for food using the same DC and amount, but providing common fresh ingredients instead of meat. Basic Harvesting takes 10 minutes. At your GM’s discretion, it may take longer for larger creatures.

\section*{Exotic Creature Harvesting}

Applicable Targets: Aberration, Constructs, Dragons, Monstrosities, Plants, Some Undead

A random roll is performed to judge what can be harvested from the monster. For Dragons and Monstrosities, a Wisdom (Medicine) check is required to harvest the material without destroying it, for Constructs, an Intelligence (Arcana) check is required, and for Plants an Intelligence (Nature) check is required. Exotic Harvesting takes 10 minutes. At your GM’s discretion, it may take longer for larger creatures.

If a beast is sufficiently magical, poisonous, or venomous, a GM can opt to use the Dragon \& Monstrosity table for exotic harvesting, but this should be rare; even a poisonous beast is usually too mundane for the magical properties of harvested materials, and a beast should always be rolled on the 0–4 CR table regardess of its CR.

At a GM’s discretion, some Undead may be harvested as well if there is something that would make sense for them to provide in this manner, in which case they would use an Intelligence (Arcana) check. Undead are less likely to provide anything of use, simply having a rare chance of providing arcane essences, though some would consider the use of these essences evil.

Double Harvesting

If a monster is applicable for both Basic Harvesting and Exotic Harvesting, you can perform both, but the second check has disadvantage on the roll to successfully gather the materials.

\section*{Exotic Remnants}

Applicable Targets: Celestials, Elementals, Fiends, Some Undead

Some creatures typically do not leave behind corpses. While these most often disappear without a trace, sometimes they will leave behind a fragment of the magical forces that powered them as a remnant, in the form of a reagent or essence. These are less likely to result in a crafting item, but don’t require any check to gather it successfully. Gathering remnants is simple to do, and requires only 1 minute.

% [Image Inserted Manually]

Applying Material Tables

As a GM, never feel compelled to roll on a table if you feel it makes sense to do something else. The tables provide a baseline, but if you feel that it makes sense of a given monster to leave behind a given material, simply do so, requiring the check that seems most appropriate (using the tables as a guide if you wish).

\section{Harvesting and Looting}

\section*{Harvesting Tables}

\section*{Exotic Harvesting}

\subsection*{Exotic Harvesting (CR 0-4)}

\begin{tabularx}{\textwidth}\toprule
{}XXXXXXXXXXXXXXXXXXXXX}
\midrule
d100 & DC & \multicolumn{4}{c}{Dragons/Giants/ Monstrosities} & \multicolumn{4}{c}{Constructs} & \multicolumn{4}{c}{Aberrations} & \multicolumn{4}{c}{Undead} & \multicolumn{4}{c}{Plants} \\
\midrule
01–20 & 8 & \multicolumn{4}{c}{—} & \multicolumn{4}{c}{parts} & \multicolumn{4}{c}{—} & \multicolumn{4}{c}{—} & \multicolumn{4}{c}{—} \\
\midrule
21–50 & 8 & \multicolumn{4}{c}{common poisonous reagent} & \multicolumn{4}{c}{fancy parts} & \multicolumn{4}{c}{common reactive reagent} & \multicolumn{4}{c}{—} & \multicolumn{4}{c}{common poisonous reagent} \\
\midrule
51–70 & 8 & \multicolumn{4}{c}{common reactive reagent} & \multicolumn{4}{c}{fancy parts} & \multicolumn{4}{c}{common curative reagent} & \multicolumn{4}{c}{—} & \multicolumn{4}{c}{common curative reagent} \\
\midrule
71–80 & 8 & \multicolumn{4}{c}{common curative reagent} & \multicolumn{4}{c}{fancy parts} & \multicolumn{4}{c}{common poisonous reagent} & \multicolumn{4}{c}{—} & \multicolumn{4}{c}{common reactive reagent} \\
\midrule
81–00 & 8 & \multicolumn{4}{c}{common primal essence} & \multicolumn{4}{c}{common arcane essence} & \multicolumn{4}{c}{common psionic essence} & \multicolumn{4}{c}{common arcane essence} & \multicolumn{4}{c}{common primal essence} \\
\midrule
\end{tabularx}

\subsection*{Exotic Harvesting (CR 5-10)}

\begin{tabularx}{\textwidth}\toprule
{}XXXXXXXXXXXXXXXXXXXXX}
\midrule
d100 & DC & \multicolumn{4}{c}{Dragons/Giants/ Monstrosities} & \multicolumn{4}{c}{Constructs} & \multicolumn{4}{c}{Aberrations} & \multicolumn{4}{c}{Undead} & \multicolumn{4}{c}{Plants} \\
\midrule
01–30 & 10 & \multicolumn{4}{c}{uncommon reactive reagent} & \multicolumn{4}{c}{fancy parts} & \multicolumn{4}{c}{common reactive reagent} & \multicolumn{4}{c}{common arcane essence} & \multicolumn{4}{c}{common poisonous reagent} \\
\midrule
31–60 & 10 & \multicolumn{4}{c}{uncommon poisonous reagent} & \multicolumn{4}{c}{1d4 fancy parts} & \multicolumn{4}{c}{uncommon reactive reagent} & \multicolumn{4}{c}{1d4 common poisonous reagent} & \multicolumn{4}{c}{uncommon poisonous reagent} \\
\midrule
61–80 & 10 & \multicolumn{4}{c}{1d4 uncommon reactive reagent} & \multicolumn{4}{c}{1d6 fancy parts} & \multicolumn{4}{c}{uncommon curative reagent} & \multicolumn{4}{c}{1d4 uncommon curative reagent} & \multicolumn{4}{c}{1d4 uncommon poisonous reagents} \\
\midrule
81–90 & 10 & \multicolumn{4}{c}{uncommon primal essence} & \multicolumn{4}{c}{uncommon arcane essence} & \multicolumn{4}{c}{uncommon arcane essence} & \multicolumn{4}{c}{uncommon divine essence} & \multicolumn{4}{c}{uncommon primal essence} \\
\midrule
91–00 & 10 & \multicolumn{4}{c}{uncommon primal essence} & \multicolumn{4}{c}{uncommon arcane essence} & \multicolumn{4}{c}{uncommon psionic essence} & \multicolumn{4}{c}{uncommon arcane essence} & \multicolumn{4}{c}{uncommon primal essence} \\
\midrule
\end{tabularx}

\subsection*{Exotic Harvesting (CR 11-16)}

\begin{tabularx}{\textwidth}\toprule
{}XXXXXXXXXXXXXXXXXXXXX}
\midrule
d100 & DC & \multicolumn{4}{c}{Dragons/Giants/ Monstrosities} & \multicolumn{4}{c}{Constructs} & \multicolumn{4}{c}{Aberrations} & \multicolumn{4}{c}{Undead} & \multicolumn{4}{c}{Plants} \\
\midrule
01–30 & 12 & \multicolumn{4}{c}{uncommon reactive reagent} & \multicolumn{4}{c}{esoteric parts} & \multicolumn{4}{c}{uncommon reactive reagent} & \multicolumn{4}{c}{uncommon poisonous reagent} & \multicolumn{4}{c}{uncommon poisonous reagent} \\
\midrule
31–60 & 12 & \multicolumn{4}{c}{uncommon primal essence} & \multicolumn{4}{c}{1d4 esoteric parts} & \multicolumn{4}{c}{uncommon psionic essence} & \multicolumn{4}{c}{uncommon arcane essence} & \multicolumn{4}{c}{uncommon primal essence} \\
\midrule
61–70 & 12 & \multicolumn{4}{c}{rare reactive reagent} & \multicolumn{4}{c}{uncommon arcane essence} & \multicolumn{4}{c}{rare reactive reagent} & \multicolumn{4}{c}{rare poisonous reagent} & \multicolumn{4}{c}{rare curative reagent} \\
\midrule
71–80 & 12 & \multicolumn{4}{c}{rare poisonous reagent} & \multicolumn{4}{c}{uncommon arcane essence} & \multicolumn{4}{c}{rare poisonous reagent} & \multicolumn{4}{c}{uncommon arcane essence} & \multicolumn{4}{c}{rare poisonous reagent} \\
\midrule
81–90 & 12 & \multicolumn{4}{c}{rare primal essence} & \multicolumn{4}{c}{rare arcane essence} & \multicolumn{4}{c}{rare arcane essence} & \multicolumn{4}{c}{rare divine essence} & \multicolumn{4}{c}{rare primal essence} \\
\midrule
91–99 & 12 & \multicolumn{4}{c}{rare primal essence} & \multicolumn{4}{c}{rare arcane essence} & \multicolumn{4}{c}{rare psionic essence} & \multicolumn{4}{c}{rare arcane essence} & \multicolumn{4}{c}{rare primal essence} \\
\midrule
00 & 12 & \multicolumn{4}{c}{very rare primal essence} & \multicolumn{4}{c}{very rare arcane essence} & \multicolumn{4}{c}{very rare psionic essence} & \multicolumn{4}{c}{very rare arcane essence} & \multicolumn{4}{c}{very rare primal essence} \\
\midrule
\end{tabularx}

\subsection*{Exotic Harvesting (CR 17+)}

\begin{tabularx}{\textwidth}\toprule
{}XXXXXXXXXXXXXXXXXXXXX}
\midrule
d100 & DC & \multicolumn{4}{c}{Dragons/Giants/ Monstrosities} & \multicolumn{4}{c}{Constructs} & \multicolumn{4}{c}{Aberrations} & \multicolumn{4}{c}{Undead} & \multicolumn{4}{c}{Plants} \\
\midrule
01–30 & 15 & \multicolumn{4}{c}{1d4 rare reactive reagent} & \multicolumn{4}{c}{1d4 esoteric parts} & \multicolumn{4}{c}{1d4 rare reactive reagent} & \multicolumn{4}{c}{1d4 rare poisonous reagent} & \multicolumn{4}{c}{1d4 rare poisonous reagent} \\
\midrule
31–50 & 15 & \multicolumn{4}{c}{rare primal essence} & \multicolumn{4}{c}{rare arcane essence} & \multicolumn{4}{c}{rare psionic essence} & \multicolumn{4}{c}{rare arcane essence} & \multicolumn{4}{c}{rare primal essence} \\
\midrule
51–89 & 15 & \multicolumn{4}{c}{very rare primal essence} & \multicolumn{4}{c}{very rare arcane essence} & \multicolumn{4}{c}{very rare arcane essence} & \multicolumn{4}{c}{very rare arcane essence} & \multicolumn{4}{c}{very rare primal essence} \\
\midrule
90–94 & 15 & \multicolumn{4}{c}{legendary primal essence} & \multicolumn{4}{c}{legendary arcane essence} & \multicolumn{4}{c}{legendary arcane essence} & \multicolumn{4}{c}{legendary divine essence} & \multicolumn{4}{c}{legendary primal essence} \\
\midrule
95–00 & 15 & \multicolumn{4}{c}{legendary primal essence} & \multicolumn{4}{c}{legendary arcane essence} & \multicolumn{4}{c}{legendary psionic essence} & \multicolumn{4}{c}{legendary arcane essence} & \multicolumn{4}{c}{legendary primal essence} \\
\midrule
\end{tabularx}

\begin{minipage}{0.48\textwidth}
\subsubsection*{Dragons, Giants, \& Monstrosities}

\begin{itemize}
  \item Table: Exotic Harvesting (CR 0-4): Dragons, Giants, \& Monstrosities
  \item Table: Exotic Harvesting (CR 05-10): Dragon, Giants, \& Monstrosities
  \item Table: Exotic Harvesting (CR 11-16): Dragons, Giants, \& Monstrosities
  \item Table: Exotic Harvesting (CR 17+): Dragons, Giants, \& Monstrosities
\end{itemize}

\subsubsection*{Constructs}

\begin{itemize}
  \item Table: Exotic Harvesting (CR 0-4): Constructs
  \item Table: Exotic Harvesting (CR 05-10): Constructs
  \item Table: Exotic Harvesting (CR 11-16): Constructs
  \item Table: Exotic Harvesting (CR 17+): Constructs
\end{itemize}

\subsubsection*{Aberrations}

\begin{itemize}
  \item Table: Exotic Harvesting (CR 0-4): Aberrations
  \item Table: Exotic Harvesting (CR 05-10): Aberrations
  \item Table: Exotic Harvesting (CR 11-16): Aberrations
  \item Table: Exotic Harvesting (CR 17+): Aberrations
\end{itemize}
\end{minipage}\hfill
\begin{minipage}{0.48\textwidth}
\subsubsection*{Undead}

\begin{itemize}
  \item Table: Exotic Harvesting (CR 0-4): Undead
  \item Table: Exotic Harvesting (CR 05-10): Undead
  \item Table: Exotic Harvesting (CR 11-16): Undead
  \item Table: Exotic Harvesting (CR 17+): Undead
\end{itemize}

\subsubsection*{Plants}

\begin{itemize}
  \item Table: Exotic Harvesting (CR 0-4): Plants
  \item Table: Exotic Harvesting (CR 05-10): Plants
  \item Table: Exotic Harvesting (CR 11-16): Plants
  \item Table: Exotic Harvesting (CR 17+): Plants
\end{itemize}
\end{minipage}

\section*{Exotic Remnants}

\subsection*{Exotic Remnants (CR 0-4)}

\begin{tabularx}{\textwidth}\toprule
{}XXXXXXXXXXXXXXXX}
\midrule
d100 & \multicolumn{4}{c}{Celestials} & \multicolumn{4}{c}{Fiends} & \multicolumn{4}{c}{Elementals} & \multicolumn{4}{c}{Incorporeal Undead} \\
\midrule
01–50 & \multicolumn{4}{c}{—} & \multicolumn{4}{c}{—} & \multicolumn{4}{c}{—} & \multicolumn{4}{c}{—} \\
\midrule
51–70 & \multicolumn{4}{c}{—} & \multicolumn{4}{c}{—} & \multicolumn{4}{c}{common reactive reagent} & \multicolumn{4}{c}{—} \\
\midrule
71–80 & \multicolumn{4}{c}{common curative reagent} & \multicolumn{4}{c}{common reactive reagent} & \multicolumn{4}{c}{common reactive reagent} & \multicolumn{4}{c}{common poisonous reagent} \\
\midrule
81–95 & \multicolumn{4}{c}{common divine essence} & \multicolumn{4}{c}{common arcane essence} & \multicolumn{4}{c}{common primal essence} & \multicolumn{4}{c}{common divine essence} \\
\midrule
96–00 & \multicolumn{4}{c}{common divine essence} & \multicolumn{4}{c}{common divine essence} & \multicolumn{4}{c}{common primal essence} & \multicolumn{4}{c}{common arcane essence} \\
\midrule
\end{tabularx}

\subsection*{Exotic Remnants (CR 5-10)}

\begin{tabularx}{\textwidth}\toprule
{}XXXXXXXXXXXXXXXX}
\midrule
d100 & \multicolumn{4}{c}{Celestials} & \multicolumn{4}{c}{Fiends} & \multicolumn{4}{c}{Elementals} & \multicolumn{4}{c}{Incorporeal Undeads} \\
\midrule
01–20 & \multicolumn{4}{c}{—} & \multicolumn{4}{c}{—} & \multicolumn{4}{c}{—} & \multicolumn{4}{c}{—} \\
\midrule
21–50 & \multicolumn{4}{c}{common curative reagent} & \multicolumn{4}{c}{common reactive reagent} & \multicolumn{4}{c}{common reactive reagent} & \multicolumn{4}{c}{common poisonous reagent} \\
\midrule
51–80 & \multicolumn{4}{c}{uncommon curative reagent} & \multicolumn{4}{c}{uncommon reactive reagent} & \multicolumn{4}{c}{uncommon reactive reagent} & \multicolumn{4}{c}{uncommon poisonous reagent} \\
\midrule
81–90 & \multicolumn{4}{c}{common divine essence} & \multicolumn{4}{c}{common arcane essence} & \multicolumn{4}{c}{common primal essence} & \multicolumn{4}{c}{common arcane essence} \\
\midrule
91–00 & \multicolumn{4}{c}{uncommon divine essence} & \multicolumn{4}{c}{uncommon arcane essence} & \multicolumn{4}{c}{uncommon primal essence} & \multicolumn{4}{c}{uncommon arcane essence} \\
\midrule
\end{tabularx}

\subsection*{Exotic Remnants (CR 11-16)}

\begin{tabularx}{\textwidth}\toprule
{}XXXXXXXXXXXXXXXX}
\midrule
d100 & \multicolumn{4}{c}{Celestials} & \multicolumn{4}{c}{Fiends} & \multicolumn{4}{c}{Elementals} & \multicolumn{4}{c}{Incorporeal Undead} \\
\midrule
01–20 & \multicolumn{4}{c}{uncommon curative reagent} & \multicolumn{4}{c}{uncommon reactive reagent} & \multicolumn{4}{c}{uncommon reactive reagent} & \multicolumn{4}{c}{uncommon poisonous reagent} \\
\midrule
21–50 & \multicolumn{4}{c}{uncommon divine essence} & \multicolumn{4}{c}{uncommon arcane essence} & \multicolumn{4}{c}{uncommon primal essence} & \multicolumn{4}{c}{uncommon arcane essence} \\
\midrule
51–80 & \multicolumn{4}{c}{rare curative reagent} & \multicolumn{4}{c}{rare reactive reagent} & \multicolumn{4}{c}{rare reactive reagent} & \multicolumn{4}{c}{rare poisonous reagent} \\
\midrule
81–00 & \multicolumn{4}{c}{rare divine essence} & \multicolumn{4}{c}{rare arcane essence} & \multicolumn{4}{c}{rare primal essence} & \multicolumn{4}{c}{rare arcane essence} \\
\midrule
\end{tabularx}

\subsection*{Exotic Remnants (CR 17+)}

\begin{tabularx}{\textwidth}\toprule
{}XXXXXXXXXXXXXXXX}
\midrule
d100 & \multicolumn{4}{c}{Celestials} & \multicolumn{4}{c}{Fiends} & \multicolumn{4}{c}{Elementals} & \multicolumn{4}{c}{Incorporeal Undead} \\
\midrule
01–20 & \multicolumn{4}{c}{rare curative reagent} & \multicolumn{4}{c}{rare reactive reagent} & \multicolumn{4}{c}{rare reactive reagent} & \multicolumn{4}{c}{rare poisonous reagent} \\
\midrule
21–50 & \multicolumn{4}{c}{rare divine essence} & \multicolumn{4}{c}{rare arcane essence} & \multicolumn{4}{c}{rare primal essence} & \multicolumn{4}{c}{rare arcane essence} \\
\midrule
51–69 & \multicolumn{4}{c}{very rare curative reagent} & \multicolumn{4}{c}{very rare reactive reagent} & \multicolumn{4}{c}{very rare reactive reagent} & \multicolumn{4}{c}{very rare poisonous reagent} \\
\midrule
70–89 & \multicolumn{4}{c}{very rare divine essence} & \multicolumn{4}{c}{very rare arcane essence} & \multicolumn{4}{c}{very rare primal essence} & \multicolumn{4}{c}{very rare arcane essence} \\
\midrule
90–00 & \multicolumn{4}{c}{legendary divine essence} & \multicolumn{4}{c}{legendary arcane essence} & \multicolumn{4}{c}{legendary primal essence} & \multicolumn{4}{c}{legendary arcane essence} \\
\midrule
\end{tabularx}

\begin{minipage}{0.48\textwidth}
\subsubsection*{Celestial}

\begin{itemize}
  \item Table: Exotic Remnants (CR 0-4): Celestials
  \item Table: Exotic Remnants (CR 05-10): Celestials
  \item Table: Exotic Remnants (CR 11-16): Celestials
  \item Table: Exotic Remnants (CR 17+): Celestials
\end{itemize}

\subsubsection*{Fiends}

\begin{itemize}
  \item Table: Exotic Remnants (CR 0-4): Fiends
  \item Table: Exotic Remnants (CR 5-10): Fiends
  \item Table: Exotic Remnants (CR 11-16): Fiends
  \item Table: Exotic Remnants (CR 17+): Fiends
\end{itemize}
\end{minipage}\hfill
\begin{minipage}{0.48\textwidth}
\subsubsection*{Elementals}

\begin{itemize}
  \item Table: Exotic Remnants (CR 0-4): Elementals
  \item Table: Exotic Remnants (CR 5-10): Elementals
  \item Table: Exotic Remnants (CR 11-16): Elementals
  \item Table: Exotic Remnants (CR 17+): Elementals
\end{itemize}

\subsubsection*{Incorporeal Undead}

\begin{itemize}
  \item Table: Exotic Remnants (CR 0-4): Incorporeal Undead
  \item Table: Exotic Remnants (CR 05-10): Incorporal Undead
  \item Table: Exotic Remnants (CR 11-16): Incorporeal Undead
  \item Table: Exotic Remnants (CR 17+): Incorporeal Undead
\end{itemize}
\end{minipage}

\section*{Hide \& Meat Harvesting}

\begin{tabularx}{\textwidth}\toprule
{}XXXX}
\midrule
Creature Size & Difficulty & \multicolumn{2}{c}{Hide} & Meat \\
\midrule
Tiny & N/A & \multicolumn{2}{c}{—} & — \\
\midrule
Small & DC 12 & \multicolumn{2}{c}{1d4 hide scraps} & — \\
\midrule
Medium & DC 10 & \multicolumn{2}{c}{1 hide or 1 medium carapace or 2d6 scales} & 1 common meat \\
\midrule
Large & DC 12 & \multicolumn{2}{c}{5 hides or 1 large carapace or 3d6 scales} & 1d4 common meat \\
\midrule
Huge & DC 14 & \multicolumn{2}{c}{10 hides or 2 large carapaces or 6d6 scales} & 2d6 common meat \\
\midrule
Gargantuan & DC 14 & \multicolumn{2}{c}{15 hides or 3 large carapaces or 9d6 scales} & 3d8 common meat \\
\midrule
\end{tabularx}

\begin{itemize}
  \item Special materials can replace up to half of the materials harvested based on the qualifications of the monster.
\end{itemize}

\section*{Special Materials}

\begin{tabularx}{\textwidth}\toprule
{}XXXXXXX}
\midrule
Modifier & Minimum CR & Harvesting Difficulty & \multicolumn{3}{c}{Additional Requirements} & \multicolumn{2}{c}{Effect} \\
\midrule
tough hide/scales & 8 & +4 & \multicolumn{3}{c}{Harvested from a creature with AC 16 or higher} & \multicolumn{2}{c}{Armor crafted has +1 AC} \\
\midrule
resistant hide/scales & 8 & +5 & \multicolumn{3}{c}{Harvested from a creature with resistance to an elemental damage type} & \multicolumn{2}{c}{Armor crafted has related elemental resistance} \\
\midrule
dragon scales & 14 & +8 & \multicolumn{3}{c}{Harvested from a Dragon.} & \multicolumn{2}{c}{Armor crafted has +1 AC and Resistance to related element.} \\
\midrule
uncommon meat & 5 & +3 & \multicolumn{3}{c}{—} & \multicolumn{2}{c}{—} \\
\midrule
rare meat & 10 & +5 & \multicolumn{3}{c}{—} & \multicolumn{2}{c}{—} \\
\midrule
very rare meat & 17 & +7 & \multicolumn{3}{c}{—} & \multicolumn{2}{c}{—} \\
\midrule
legendary meat & 21 & +9 & \multicolumn{3}{c}{—} & \multicolumn{2}{c}{—} \\
\midrule
\end{tabularx}

\begin{itemize}
  \item If the difficulty modifier is not met, the material is harvested without the modifier, its special property ruined during harvesting.
\end{itemize}

% [Image Inserted Manually]

\section{Loot}

\section*{Looting Tables}

\section*{Individual Treasure}

\subsection*{Individual Treasure (CR 0-4)}

\begin{tabularx}{\textwidth}\toprule
{}XXXXXXX}
\midrule
d100 & \multicolumn{5}{c}{Materials Found} & Coinage & Equivalent Gold Value \\
\midrule
01–15 & \multicolumn{5}{c}{1d6 wood scraps, 1 length of string} & 2d4 cp & 18 cp \\
\midrule
16–30 & \multicolumn{5}{c}{1d4 metal scraps} & 1d4 sp & 6 sp \\
\midrule
31–40 & \multicolumn{5}{c}{1d4 leather scraps, 1 hide scrap} & 1d4 sp & 6 sp \\
\midrule
41–60 & \multicolumn{5}{c}{supplies} & 1d6 sp, 2d4 cp & 1 gp, 5 sp \\
\midrule
61–70 & \multicolumn{5}{c}{1d2 parts} & 1d6 gp, 2d4 sp & 6 gp, 5 sp \\
\midrule
71–75 & \multicolumn{5}{c}{uncommon supplies} & 1 gp, 1d10 sp & 10 gp \\
\midrule
76–80 & \multicolumn{5}{c}{common poisonous reagent} & 2d4 sp & 15 gp, 5 sp \\
\midrule
81–85 & \multicolumn{5}{c}{common curative reagent} & 2d4 sp & 15 gp \\
\midrule
86–90 & \multicolumn{5}{c}{common reactive reagent} & 2d4 sp & 15 gp \\
\midrule
91–94 & \multicolumn{5}{c}{common magical ink} & 1d6 gp, 1d10 sp & 18 gp, 5 sp \\
\midrule
95–96 & \multicolumn{5}{c}{common divine essence} & 1d6 sp, 1d10 cp & 46 gp \\
\midrule
97–98 & \multicolumn{5}{c}{common primal essence} & 1d6 sp, 1d10 cp & 46 gp \\
\midrule
99–00 & \multicolumn{5}{c}{common arcane essence} & 1d6 sp, 1d10 cp & 46 gp \\
\midrule
\end{tabularx}

\subsection*{Individual Treasure (CR 5-10)}

\begin{tabularx}{\textwidth}\toprule
{}XXXXXXX}
\midrule
d100 & \multicolumn{5}{c}{Materials Found} & Coinage & Equivalent Gold Value \\
\midrule
01–30 & \multicolumn{5}{c}{1d4 fancy parts} & 1d10 gp, 1d10 sp & 36 gp, 6 sp \\
\midrule
31–40 & \multicolumn{5}{c}{1d10 parts, 1d20 leather scraps, 1d20 metal scraps} & 1 pp, 1d10 gp, 2d10 sp & 30 gp \\
\midrule
41–50 & \multicolumn{5}{c}{uncommon poisonous reagent} & 1d6 x 10 gp & 75 gp \\
\midrule
51–60 & \multicolumn{5}{c}{uncommon curative reagent} & 1d6 x 10 gp & 75 gp \\
\midrule
61–80 & \multicolumn{5}{c}{uncommon reactive reagent} & 1d6 x 10 gp & 75 gp \\
\midrule
81–90 & \multicolumn{5}{c}{uncommon magical ink, uncommon parchment} & 1d6 x 10 gp & 125 gp \\
\midrule
91–94 & \multicolumn{5}{c}{esoteric parts} & 2d6 x 10 gp & 170 gp \\
\midrule
95–96 & \multicolumn{5}{c}{uncommon divine essence} & 1d6 x 10 gp & 185 gp \\
\midrule
97–98 & \multicolumn{5}{c}{uncommon primal essence} & 1d6 x 10 gp & 185 gp \\
\midrule
99–00 & \multicolumn{5}{c}{uncommon arcane essence} & 1d6 pp & 185 gp \\
\midrule
\end{tabularx}

\subsection*{Individual Treasure (CR 11-16)}

\begin{tabularx}{\textwidth}\toprule
{}XXXXXX}
\midrule
d100 & \multicolumn{5}{c}{Materials Found} & Equivalent Gold Value \\
\midrule
01–20 & \multicolumn{5}{c}{2 mithril ingots, 2 adamantine ingot, 2 fancy parts} & 300 gp \\
\midrule
21–30 & \multicolumn{5}{c}{rare branch, uncommon branch, rare poisonous reagent} & 305 gp \\
\midrule
31–40 & \multicolumn{5}{c}{5 dragon scales} \\
\midrule
41–50 & \multicolumn{5}{c}{rare magical ink, rare curative reagent} & 400 gp \\
\midrule
51–60 & \multicolumn{5}{c}{rare reactive reagent, 2 rare supplies} & 400 gp \\
\midrule
61–70 & \multicolumn{5}{c}{10x uncommon reagents} & 400 gp \\
\midrule
71–80 & \multicolumn{5}{c}{uncommon divine essence, uncommon primal essence} & 400 gp \\
\midrule
81–90 & \multicolumn{5}{c}{rare reactive reagent, rare poisonous reagent, rare curative reagent} & 600 gp \\
\midrule
91–94 & \multicolumn{5}{c}{tough leather} & 600 gp \\
\midrule
95–96 & \multicolumn{5}{c}{rare divine essence} & 700 gp \\
\midrule
97–98 & \multicolumn{5}{c}{rare primal essence} & 700 gp \\
\midrule
99–00 & \multicolumn{5}{c}{rare arcane essence} & 700 gp \\
\midrule
\end{tabularx}

\subsection*{Individual Treasure (CR 17+)}

\begin{tabularx}{\textwidth}\toprule
{}XXXXXX}
\midrule
d100 & \multicolumn{5}{c}{Materials Found} & Equivalent Gold Value \\
\midrule
01–15 & \multicolumn{5}{c}{very rare branch, very rare parchment, rare arcane essence} & 3,500 gp \\
\midrule
16–30 & \multicolumn{5}{c}{very rare curative reagent, very rare poisonous reagent, rare primal essence} & 5,000 gp \\
\midrule
31–45 & \multicolumn{5}{c}{very rare poisonous reagent, very rare reactive reagent, rare arcane essence} & 5,000 gp \\
\midrule
46–60 & \multicolumn{5}{c}{very rare curative reagent, very rare reactive reagent, rare divine essence} & 5,500 gp \\
\midrule
61–70 & \multicolumn{5}{c}{very rare divine essence} & 7,000 gp \\
\midrule
71–80 & \multicolumn{5}{c}{very rare primal essence} & 7,000 gp \\
\midrule
81–90 & \multicolumn{5}{c}{very rare arcane essence} & 7,000 gp \\
\midrule
95–96 & \multicolumn{5}{c}{legendary curative reagent, legendary poisonous reagent} & 10,000 gp \\
\midrule
97–98 & \multicolumn{5}{c}{legendary poisonous reagent, legendary reactive reagent} & 10,000 gp \\
\midrule
99–00 & \multicolumn{5}{c}{legendary curative reagent, legendary reactive reagent} & 10,000 gp \\
\midrule
\end{tabularx}

Replacing Hoards

Note that crafting materials found as part of a Hoard replace coinage, gems, or art objects of equal Equivalent Gold Value. You can use the same d100 roll for both the Treasure Hoard table and the crafting replacement.

\begin{minipage}{0.48\textwidth}
\begin{itemize}
  \item Table: Individual Treasure (CR 0-4)
  \item Table: Individual Treasure (CR 05-10)
\end{itemize}
\end{minipage}\hfill
\begin{minipage}{0.48\textwidth}
\begin{itemize}
  \item Table: Individual Treasure (CR 11-16)
  \item Table: Individual Treasure (CR 17+)
\end{itemize}
\end{minipage}

\section*{Treasure Crafting Substitutions}

\subsection*{Treasure Crafting Substitutions (CR 0-4)}

\begin{tabularx}{\textwidth}\toprule
{}XXXXXX}
\midrule
d100 & \multicolumn{5}{c}{Materials Found} & Equivalent Gold Value \\
\midrule
01–25 & \multicolumn{5}{c}{5 steel ingots} & 10 gp \\
\midrule
26–40 & \multicolumn{5}{c}{10 tanned leather} & 30 gp \\
\midrule
41–50 & \multicolumn{5}{c}{10 steel ingots, 50 scales, 10 rawhide leather} & 90 gp \\
\midrule
51–60 & \multicolumn{5}{c}{2 common curative reagents, 2 common reactive reagents, 2 common poisonous reagents} & 90 gp \\
\midrule
61–70 & \multicolumn{5}{c}{1 mithril ingot, 1 common arcane essence, 1 common divine essence} & 105 gp \\
\midrule
71–80 & \multicolumn{5}{c}{1 uncommon magical ink, 1 uncommon parchment, large carapace, 1 uncommon supplies} & 120 gp \\
\midrule
81–90 2 & \multicolumn{5}{c}{fancy parts, 2 mithril ingots, 1 rare poisonous reagent, 1 rare reactive reagent} & 140 gp \\
\midrule
91–99 & \multicolumn{5}{c}{1 esoteric part, 1 adamantine ingot, 1 rare curative reagent} & 200 gp \\
\midrule
00 & \multicolumn{5}{c}{1 uncommon arcane essence, 1 common divine essence, 1 common primal essence} & 235 gp \\
\midrule
\end{tabularx}

\subsection*{Treasure Crafting Substitutions (CR 5-10)}

\begin{tabularx}{\textwidth}\toprule
{}XXXXXX}
\midrule
d100 & \multicolumn{5}{c}{Materials Found} & Equivalent Gold Value \\
\midrule
01–25 & \multicolumn{5}{c}{20 steel ingots, 20 rawhide leather, 20 fancy parts, 20 scales, 10 quality branches} & 360 gp \\
\midrule
26–40 & \multicolumn{5}{c}{4 uncommon curative reagents, 4 uncommon poisonous reagents, 4 uncommon reactive reagents} & 480 gp \\
\midrule
41–50 & \multicolumn{5}{c}{1 rare magical ink, 1 rare parchment, 2 adamantine ingots} & 520 gp \\
\midrule
51–60 & \multicolumn{5}{c}{10 fancy parts, 10 mithril ingots, 2 rare branches} & 560 gp \\
\midrule
61–70 & \multicolumn{5}{c}{2 rare supplies, rare branch, 2 uncommon divine essences} & 580 gp \\
\midrule
71–80 & \multicolumn{5}{c}{1 uncommon arcane essence, 1 uncommone primal essence, 2 rare curative reagents, 2 rare branches, 1 adamantine ingot} & 600 gp \\
\midrule
81–90 & \multicolumn{5}{c}{3 esoteric part, 3 rare curative reagent, 3 rare poisonous reagents, 3 rare reactive reagents} & 680 gp \\
\midrule
91–99 & \multicolumn{5}{c}{1 tough leather, 1 uncomon arcane essence, 1 uncommon divine essence, 1 uncommon primal essence} & 950 gp \\
\midrule
00 & \multicolumn{5}{c}{1 rare arcane essence, 1 uncommon divine essence,
1 uncommon primal essence} & 1,000 gp \\
\midrule
\end{tabularx}

\subsection*{Treasure Crafting Substitutions (CR 11-16)}

\begin{tabularx}{\textwidth}\toprule
{}XXXXXX}
\midrule
d100 & \multicolumn{5}{c}{Materials Found} & Equivalent Gold Value \\
\midrule
01–25 & \multicolumn{5}{c}{10 adamantine ingots, 4 tough leather, 4 esoteric parts, 4 very rare branches, 10 mithril ingots} & 6,900 gp \\
\midrule
26–40 & \multicolumn{5}{c}{5 rare curative reagents, 5 rare poisonous reagents, 5 rare reactive reagents, 2 very rare parchment, 2 very rare magical ink} & 11,000 gp \\
\midrule
41–50 & \multicolumn{5}{c}{5 tough leather, 20 dragon scales, 4 rare primal essences, 4 rare divine essences, 5 rare reactive reagents, 5 rare curative reagents, 5 Adamantine Ingots} & 11,300 gp \\
\midrule
51–60 & \multicolumn{5}{c}{15 esotertic parts, 15 rare supplies, 1 very rare arcane essence, 3 rare poisonous reagents, 4 rare curative reagents} & 11,400 gp \\
\midrule
61–70 & \multicolumn{5}{c}{5 firesteel ingots, 1 very rare parchment, 2 very rare reactive reagents, 2 very rare poisonous reagents} & 12,250 gp \\
\midrule
71–80 & \multicolumn{5}{c}{1 very rare divine essence, 2 rare arcane essences, 2 very rare curative reagentS} & 12,400 gp \\
\midrule
81–90 & \multicolumn{5}{c}{1 very rare primal essence, 2 rare divine essences, 2 very rare poisonous reagentS} & 12,400 gp \\
\midrule
91–99 & \multicolumn{5}{c}{1 very rare arcane essence, 2 rare primal essences, 2 very rare reactive reagentS} & 12,400 gp \\
\midrule
00 & \multicolumn{5}{c}{1 very rare arcane essence, 1 very rare divine essence, 1 very rare primal essence.} & 21,000 gp \\
\midrule
\end{tabularx}

\subsection*{Treasure Crafting Substitutions (CR 17+)}

\begin{tabularx}{\textwidth}\toprule
{}XXXXXX}
\midrule
d100 & \multicolumn{5}{c}{Materials Found} & Equivalent Gold Value \\
\midrule
01–25 & \multicolumn{5}{c}{10 esoteric parts, 10 darksteel ingots, 10 firesteel ingots, 10 icesteel ingots, 5 very rare parchment, 20 admantine ingots, 20 mitril ingots} & 20,000 gp \\
\midrule
26–40 & \multicolumn{5}{c}{4 very rare curative reagents, 4 very rare reactive reagents, 4 very rare poisonous reagents} & 24,000 gp \\
\midrule
41–50 & \multicolumn{5}{c}{10 tough leather, 1 legendary magical ink, 1 legendary parchment, 4 very rare curative reagents} & 24,000 gp \\
\midrule
51–60 & \multicolumn{5}{c}{legendary curative reagent, legendary reactive reagent,
legendary poisonous reagent, 2 very rare primal essences} & 29,000 gp \\
\midrule
61–70 & \multicolumn{5}{c}{20 rare supplies, 20 esoteric parts, 20 rare reactive reagents, 20 rare curative reagents, 20 rare poisonous reagents, 10 rare divine essence, 10 rare arcane essences, 10 rare divine essences} & 29,000 gp \\
\midrule
71–80 & \multicolumn{5}{c}{legendary divine essence, 2 very rare primal essences,
2 legendary reactive reagents} & 49,000 gp \\
\midrule
81–90 & \multicolumn{5}{c}{legendary primal essence, 2 very rare arcane essences,
2 legendary poisonous reagents} & 49,000 gp \\
\midrule
91–99 & \multicolumn{5}{c}{legendary arcane essence, 2 very rare divine essences,
2 legendary curative reagents} & 49,000 gp \\
\midrule
00 & \multicolumn{5}{c}{legendary arcane essence, legendary primal essence,
legendary divine essence} & 75,000 gp \\
\midrule
\end{tabularx}

\begin{minipage}{0.48\textwidth}
\begin{itemize}
  \item Table: Treasure Crafting Substitutions (CR 0-4)
  \item Table: Treasure Crafting Substitutions (CR 05-10)
\end{itemize}
\end{minipage}\hfill
\begin{minipage}{0.48\textwidth}
\begin{itemize}
  \item Table: Treasure Crafting Substitutions (CR 11-16)
  \item Table: Treasure Crafting Substitutions (CR 17+)
\end{itemize}
\end{minipage}

\section{Gathering}

\section*{Gathering}

Many of the materials can simply be found growing in the wild, and can be gathered by someone that knows what to look for and spends the time doing just that. When traveling at a slow pace through wilderness for 8 hours or more (i.e. not urban land or farmland), you can make a gathering check, but have disadvantage on the check to harvest anything found.

If you dedicate 8 hours to gathering without traveling, you can make two checks (without disadvantage) or find one item other than an essence of your choice that is available within that biome’s table (making the ability check from the corresponding line of the table to harvest it) or 1d12 of any trivial item (making a DC 8 ability check to harvest it).

The found items then have to be gathered. You can choose to gather reagents, search for materials, or hunt wild game. Roll a d100 and consult the corresponding table below for the relevant biome to determine what is found.

\begin{minipage}{0.48\textwidth}
\subsubsection*{Gather Reagents}

Reagents are harvested by making a Wisdom check. If you have an Herbalism kit and are proficient with it, you can add your proficiency bonus to the roll.

\subsubsection*{Search for Materials}

Materials can be harvested with your choice of a Strength, Dexterity, or Wisdom check. If you have proficiency with the Survival skill, you can add your proficiency bonus to the roll.

\subsubsection*{Hunt Game}

Food can be gathered with your choice of a Dexterity or Wisdom check. If you have proficiency with the Survival akill, you can add your survival modifier to the roll.
\end{minipage}\hfill
\begin{minipage}{0.48\textwidth}
Intentionally Slow

Gathering is a time gated system. It is not intended to be the primary source of materials. Rather than being balanced against the loot tables, it’s balanced against the down time activities (and consequently not particularly efficient).
\end{minipage}

\section{Gathering}

\section*{Gathering Tables}

\section*{Gather Reagents}

\subsection*{Reagents}

\begin{tabularx}{\textwidth}\toprule
{}XXXXXXXXXXXXXXXX}
\midrule
d100 & DC & \multicolumn{3}{c}{Forests} & \multicolumn{3}{c}{Deserts} & \multicolumn{3}{c}{Grasslands} & \multicolumn{3}{c}{Marsh} & \multicolumn{3}{c}{Mountains} \\
\midrule
01–10 & — & \multicolumn{3}{c}{—} & \multicolumn{3}{c}{—} & \multicolumn{3}{c}{—} & \multicolumn{3}{c}{—} & \multicolumn{3}{c}{—} \\
\midrule
11–20 & 10 & \multicolumn{3}{c}{common curative
reagent} & \multicolumn{3}{c}{—} & \multicolumn{3}{c}{—} & \multicolumn{3}{c}{common poisonous
reagent} & \multicolumn{3}{c}{—} \\
\midrule
21–40 & 10 & \multicolumn{3}{c}{common curative
reagent} & \multicolumn{3}{c}{common reactive
reagent} & \multicolumn{3}{c}{common curative
reagent} & \multicolumn{3}{c}{common poisonous
reagent} & \multicolumn{3}{c}{common reactive
reagent} \\
\midrule
41–50 & 10 & \multicolumn{3}{c}{common poisonous
reagent} & \multicolumn{3}{c}{common curative
reagent} & \multicolumn{3}{c}{common reactive
reagent} & \multicolumn{3}{c}{common curative
reagent} & \multicolumn{3}{c}{common curative
reagent} \\
\midrule
51–60 & 10 & \multicolumn{3}{c}{common reactive
reagent} & \multicolumn{3}{c}{common poisonous
reagent} & \multicolumn{3}{c}{common poisonous
reagent} & \multicolumn{3}{c}{common reactive
reagent} & \multicolumn{3}{c}{common poisonous
reagent} \\
\midrule
61–70 & 10 & \multicolumn{3}{c}{1d4 common
poisonous reagents} & \multicolumn{3}{c}{1d2 common
reactive reagents} & \multicolumn{3}{c}{1d2 comon
poisonous reagents} & \multicolumn{3}{c}{1d4 common
poisonous reagents} & \multicolumn{3}{c}{1d2 common
curative reagents} \\
\midrule
71–80 & 10 & \multicolumn{3}{c}{1d4 common
curative reagents} & \multicolumn{3}{c}{1d2 common
reactive reagents} & \multicolumn{3}{c}{1d2 common
curative reagents} & \multicolumn{3}{c}{1d4 common
reactive reagents} & \multicolumn{3}{c}{1d2 common
reactive reagents} \\
\midrule
81–90 & 10 & \multicolumn{3}{c}{uncommon curative
reagent} & \multicolumn{3}{c}{uncommon reactive
reagent} & \multicolumn{3}{c}{uncommon curative
reagent} & \multicolumn{3}{c}{uncommon
poisonous reagent} & \multicolumn{3}{c}{uncommon reactive
reagent} \\
\midrule
91–95 & 10 & \multicolumn{3}{c}{uncommon
poisonous reagent} & \multicolumn{3}{c}{uncommon
poisonous reagent} & \multicolumn{3}{c}{uncommon reactive
reagent} & \multicolumn{3}{c}{uncommon reactive
reagent} & \multicolumn{3}{c}{common curative
reagent} \\
\midrule
96–00 & 10 & \multicolumn{3}{c}{common primal
essence} & \multicolumn{3}{c}{common arcane
essence} & \multicolumn{3}{c}{common divine
essence} & \multicolumn{3}{c}{common primal
essence} & \multicolumn{3}{c}{common primal
essence} \\
\midrule
\end{tabularx}

\subsection*{Reagents}

\begin{tabularx}{\textwidth}\toprule
{}XXXXXXXXXXXXXXXX}
\midrule
d100 & DC & \multicolumn{3}{c}{Caves} & \multicolumn{3}{c}{Underground} & \multicolumn{3}{c}{Jungles} & \multicolumn{3}{c}{Shore} & \multicolumn{3}{c}{Tundra} \\
\midrule
01–10 & — & \multicolumn{3}{c}{—} & \multicolumn{3}{c}{—} & \multicolumn{3}{c}{—} & \multicolumn{3}{c}{—} & \multicolumn{3}{c}{—} \\
\midrule
11–30 & 12 & \multicolumn{3}{c}{common reactive
reagent} & \multicolumn{3}{c}{common poisonous
reagent} & \multicolumn{3}{c}{common curative
reagent} & \multicolumn{3}{c}{common curative
reagent} & \multicolumn{3}{c}{common reactive
reagent} \\
\midrule
31–50 & 12 & \multicolumn{3}{c}{common poisonous
reagent} & \multicolumn{3}{c}{common reactive
reagent} & \multicolumn{3}{c}{common poisonous
reagent} & \multicolumn{3}{c}{common poisonous
reagent} & \multicolumn{3}{c}{common curative
reagent} \\
\midrule
51–60 & 12 & \multicolumn{3}{c}{1d4 common
reactive reagent} & \multicolumn{3}{c}{1d4 common
poisonous reagents} & \multicolumn{3}{c}{1d4 common
curative reagents} & \multicolumn{3}{c}{1d4 common
curative reagents} & \multicolumn{3}{c}{1d4 common
reactive reagents} \\
\midrule
61–70 & 12 & \multicolumn{3}{c}{uncommon reactive
reagent} & \multicolumn{3}{c}{uncommon
poisonous reagent} & \multicolumn{3}{c}{uncommon curative
reagent} & \multicolumn{3}{c}{uncommon reactive
reagent} & \multicolumn{3}{c}{uncommon curative
reagent} \\
\midrule
71–80 & 12 & \multicolumn{3}{c}{uncommon curative
reagent} & \multicolumn{3}{c}{uncommon curative
reagent} & \multicolumn{3}{c}{uncommon reactive
reagent} & \multicolumn{3}{c}{uncommon reactive
reagent} & \multicolumn{3}{c}{uncommon reactive
reagent} \\
\midrule
81–90 & 12 & \multicolumn{3}{c}{common divine
essence} & \multicolumn{3}{c}{common arcane
essence} & \multicolumn{3}{c}{common primal
essence} & \multicolumn{3}{c}{common primal
essence} & \multicolumn{3}{c}{common primal
essence} \\
\midrule
91–95 & 12 & \multicolumn{3}{c}{uncommon
poisonous reagent} & \multicolumn{3}{c}{uncommon
poisonous reagent} & \multicolumn{3}{c}{uncommon reactive
reagent} & \multicolumn{3}{c}{uncommon reactive
reagent} & \multicolumn{3}{c}{uncommmon
reactive reagent} \\
\midrule
96–00 & 12 & \multicolumn{3}{c}{uncommon divine
essence} & \multicolumn{3}{c}{uncommon arcane
essence} & \multicolumn{3}{c}{uncommon primal
essence} & \multicolumn{3}{c}{uncommon primal
essence} & \multicolumn{3}{c}{uncommon primal
essence} \\
\midrule
\end{tabularx}

\subsection*{Reagents}

\begin{tabularx}{\textwidth}\toprule
{}XXXXXXXXXXXXXXXXXXXX}
\midrule
\multicolumn{2}{c}{d100} & DC & \multicolumn{3}{c}{Feylands} & \multicolumn{3}{c}{Shadowlands} & \multicolumn{3}{c}{Elemental Planes} & \multicolumn{3}{c}{Lower Planes} & \multicolumn{3}{c}{Upper Planes} & \multicolumn{3}{c}{Outer Planes} \\
\midrule
\multicolumn{2}{c}{01–20} & 14 & \multicolumn{3}{c}{common
curative reagent} & \multicolumn{3}{c}{common
poisonous
reagent} & \multicolumn{3}{c}{common reactive
reagent} & \multicolumn{3}{c}{common
poisonous
reagent} & \multicolumn{3}{c}{common
curative reagent} & \multicolumn{3}{c}{common reactive
reagent} \\
\midrule
\multicolumn{2}{c}{21–40} & 14 & \multicolumn{3}{c}{common reactive
reagent} & \multicolumn{3}{c}{common reactive
reagent} & \multicolumn{3}{c}{common
curative reagent} & \multicolumn{3}{c}{common reactive
reagent} & \multicolumn{3}{c}{common reactive
reagent} & \multicolumn{3}{c}{common reactive
reagent} \\
\midrule
\multicolumn{2}{c}{41–60} & 14 & \multicolumn{3}{c}{1d4 common
curative reagent} & \multicolumn{3}{c}{1d4 common
poisonous
reagent} & \multicolumn{3}{c}{1d4 common
reactive reagent} & \multicolumn{3}{c}{1d4 common
poisonous
reagent} & \multicolumn{3}{c}{1d4 common
curative reagent} & \multicolumn{3}{c}{1d4 common
reactive reagent} \\
\midrule
\multicolumn{2}{c}{61–80} & 14 & \multicolumn{3}{c}{uncommon
curative reagent} & \multicolumn{3}{c}{uncommon
poisonous
reagent} & \multicolumn{3}{c}{uncommon
reactive reagent} & \multicolumn{3}{c}{uncommon
reactive reagent} & \multicolumn{3}{c}{uncommon
curative reagent} & \multicolumn{3}{c}{uncommon
reactive reagent} \\
\midrule
\multicolumn{2}{c}{81–99} & 14 & \multicolumn{3}{c}{uncommon
primal essence} & \multicolumn{3}{c}{uncommon
arcane essence} & \multicolumn{3}{c}{uncommon
primal essence} & \multicolumn{3}{c}{uncommon
arcane essence} & \multicolumn{3}{c}{uncommon
divine essence} & \multicolumn{3}{c}{uncommon
arcane essence} \\
\midrule
\multicolumn{2}{c}{00} & 14 & \multicolumn{3}{c}{rare primal
essence} & \multicolumn{3}{c}{rare arcane
essence} & \multicolumn{3}{c}{rare primal
essence} & \multicolumn{3}{c}{rare arcane
essence} & \multicolumn{3}{c}{rare divine
essence} & \multicolumn{3}{c}{rare arcane
essence} \\
\midrule
\end{tabularx}

\begin{minipage}{0.48\textwidth}
\begin{itemize}
  \item Reagents: Caves
  \item Reagents: Deserts
  \item Reagents: Elemental Planes
  \item Reagents: Feylands
  \item Reagents: Forests
  \item Reagents: Grasslands
  \item Reagents: Jungles
  \item Reagents: Lower Planes
\end{itemize}
\end{minipage}\hfill
\begin{minipage}{0.48\textwidth}
\begin{itemize}
  \item Reagents: Marshes
  \item Reagents: Mountains
  \item Reagents: Outer Planes
  \item Reagents: Shadowlands
  \item Reagents: Shore
  \item Reagents: Tundra
  \item Reagents: Underground
  \item Reagents: Upper Planes
\end{itemize}
\end{minipage}

\section*{Search for Materials}

\subsection*{Materials}

\begin{tabularx}{\textwidth}\toprule
{}XXXXXXXXXXXXXXXX}
\midrule
d100 & DC & \multicolumn{3}{c}{Forests} & \multicolumn{3}{c}{Deserts} & \multicolumn{3}{c}{Grasslands} & \multicolumn{3}{c}{Marsh} & \multicolumn{3}{c}{Mountains} \\
\midrule
01–20 & 10 & \multicolumn{3}{c}{1d4 x 10 firewood} & \multicolumn{3}{c}{—} & \multicolumn{3}{c}{1d4 firewood} & \multicolumn{3}{c}{1d4 firewood} & \multicolumn{3}{c}{1d4 firewood} \\
\midrule
21–40 & 10 & \multicolumn{3}{c}{1d12 common branches} & \multicolumn{3}{c}{1d12 scales} & \multicolumn{3}{c}{1d12 wood scraps} & \multicolumn{3}{c}{1d12 wood scraps} & \multicolumn{3}{c}{1d12 fletching} \\
\midrule
41–60 & 10 & \multicolumn{3}{c}{1d4 quality branches} & \multicolumn{3}{c}{1d4 medium carapace} & \multicolumn{3}{c}{uncommon supplies} & \multicolumn{3}{c}{1d4 quality branches} & \multicolumn{3}{c}{adamant ore} \\
\midrule
61–80 & 10 & \multicolumn{3}{c}{uncommon branch} & \multicolumn{3}{c}{large carapace} & \multicolumn{3}{c}{1d4 hides} & \multicolumn{3}{c}{supplies} & \multicolumn{3}{c}{mithral ore} \\
\midrule
80–95 & 10 & \multicolumn{3}{c}{1d2 uncommon branches} & \multicolumn{3}{c}{rare supplies} & \multicolumn{3}{c}{rare suppplies} & \multicolumn{3}{c}{uncommon branch} & \multicolumn{3}{c}{uncommon branch} \\
\midrule
96–00 & 10 & \multicolumn{3}{c}{common primal essence} & \multicolumn{3}{c}{common arcane essence} & \multicolumn{3}{c}{common divine essence} & \multicolumn{3}{c}{common primal essence} & \multicolumn{3}{c}{common primal essence} \\
\midrule
\end{tabularx}

\subsection*{Materials}

\begin{tabularx}{\textwidth}\toprule
{}XXXXXXXXXXXXXXXX}
\midrule
d100 & DC & \multicolumn{3}{c}{Caves} & \multicolumn{3}{c}{Underground} & \multicolumn{3}{c}{Jungles} & \multicolumn{3}{c}{Shore} & \multicolumn{3}{c}{Tundra} \\
\midrule
01–10 & 12 & \multicolumn{3}{c}{—} & \multicolumn{3}{c}{—} & \multicolumn{3}{c}{1d4 firewood} & \multicolumn{3}{c}{—} & \multicolumn{3}{c}{—} \\
\midrule
11–20 & 12 & \multicolumn{3}{c}{discarded armor padding} & \multicolumn{3}{c}{parts} & \multicolumn{3}{c}{1d4 common branches} & \multicolumn{3}{c}{shoft haft} & \multicolumn{3}{c}{wood scraps} \\
\midrule
21–40 & 12 & \multicolumn{3}{c}{1d12 scales} & \multicolumn{3}{c}{1d4 supplies} & \multicolumn{3}{c}{1d4 supplies} & \multicolumn{3}{c}{medium carapace} & \multicolumn{3}{c}{firewood} \\
\midrule
41–60 & 12 & \multicolumn{3}{c}{adamant ore} & \multicolumn{3}{c}{mithral ore} & \multicolumn{3}{c}{uncommon supplies} & \multicolumn{3}{c}{rare supplies} & \multicolumn{3}{c}{supplies} \\
\midrule
61–80 & 12 & \multicolumn{3}{c}{large carapace} & \multicolumn{3}{c}{uncommon branch} & \multicolumn{3}{c}{uncommon branch} & \multicolumn{3}{c}{1d4 common branches} & \multicolumn{3}{c}{uncommon supplies} \\
\midrule
80–95 & 12 & \multicolumn{3}{c}{1d4 mithral ore} & \multicolumn{3}{c}{large carapace} & \multicolumn{3}{c}{uncommon branch} & \multicolumn{3}{c}{slighty rusty fancy parts} & \multicolumn{3}{c}{1d4 icesteel ore} \\
\midrule
96–00 & 12 & \multicolumn{3}{c}{common primal essence} & \multicolumn{3}{c}{common arcane essence} & \multicolumn{3}{c}{common primal essence} & \multicolumn{3}{c}{common primal essence} & \multicolumn{3}{c}{common primal essence} \\
\midrule
\end{tabularx}

\begin{minipage}{0.48\textwidth}
\begin{itemize}
  \item Materials: Caves
  \item Materials: Deserts
  \item Materials: Forests
  \item Materials: Grasslands
  \item Materials: Jungles
\end{itemize}
\end{minipage}\hfill
\begin{minipage}{0.48\textwidth}
\begin{itemize}
  \item Materials: Marshes
  \item Materials: Mountains
  \item Materials: Shore
  \item Materials: Tundra
  \item Materials: Underground
\end{itemize}
\end{minipage}

\section*{Hunt Game}

\subsection*{Game}

\begin{tabularx}{\textwidth}\toprule
{}XXXXXXXXXXXXXXXX}
\midrule
d100 & DC & \multicolumn{3}{c}{Forests} & \multicolumn{3}{c}{Deserts} & \multicolumn{3}{c}{Grasslands} & \multicolumn{3}{c}{Marsh} & \multicolumn{3}{c}{Mountains} \\
\midrule
01–30 & 10 & \multicolumn{3}{c}{fresh ingredients} & \multicolumn{3}{c}{—} & \multicolumn{3}{c}{fresh ingredients} & \multicolumn{3}{c}{—} & \multicolumn{3}{c}{—} \\
\midrule
31–60 & 10 & \multicolumn{3}{c}{1d4 fresh ingredients} & \multicolumn{3}{c}{fresh ingredients} & \multicolumn{3}{c}{1d4 fresh ingredients} & \multicolumn{3}{c}{supplies} & \multicolumn{3}{c}{supplies} \\
\midrule
61–90 & 10 & \multicolumn{3}{c}{1d4 fresh ingredients 
1 hide} & \multicolumn{3}{c}{supplies} & \multicolumn{3}{c}{1d4 fresh ingredients
1 hide} & \multicolumn{3}{c}{1d4 fresh ingredients} & \multicolumn{3}{c}{1d4 supplies} \\
\midrule
91–00 & 10 & \multicolumn{3}{c}{1d8 fresh ingredients 
1d4 hides} & \multicolumn{3}{c}{1d6 fresh ingredients 
1 large carapace} & \multicolumn{3}{c}{1d8 fresh ingredients 
1d4 hides} & \multicolumn{3}{c}{1d4 fresh ingredients} & \multicolumn{3}{c}{1d6 fresh ingredients 
1 large carapace} \\
\midrule
\end{tabularx}

\subsection*{Game}

\begin{tabularx}{\textwidth}\toprule
{}XXXXXXXXXXXXXXXX}
\midrule
d100 & DC & \multicolumn{3}{c}{Caves} & \multicolumn{3}{c}{Underground} & \multicolumn{3}{c}{Graveyards} & \multicolumn{3}{c}{Shore} & \multicolumn{3}{c}{Tundra} \\
\midrule
01–30 & 12 & \multicolumn{3}{c}{fresh ingredients} & \multicolumn{3}{c}{supplies} & \multicolumn{3}{c}{—} & \multicolumn{3}{c}{fresh ingredients} & \multicolumn{3}{c}{—} \\
\midrule
31–60 & 12 & \multicolumn{3}{c}{1d4 fresh ingredients} & \multicolumn{3}{c}{fresh ingredients} & \multicolumn{3}{c}{—} & \multicolumn{3}{c}{1d4 fresh ingredients} & \multicolumn{3}{c}{fresh ingredients} \\
\midrule
61–90 & 12 & \multicolumn{3}{c}{1d4 hides} & \multicolumn{3}{c}{1d4 hides} & \multicolumn{3}{c}{fresh ingredients} & \multicolumn{3}{c}{1d8 fresh ingredients 
1 supplies} & \multicolumn{3}{c}{1d4 fresh ingredients 
1 hide} \\
\midrule
91–00 & 12 & \multicolumn{3}{c}{1d6 fresh ingredients 
1d4 hides} & \multicolumn{3}{c}{1d6 fresh ingredients 
1d4 hides} & \multicolumn{3}{c}{fresh ingredients 
1 hide scraps} & \multicolumn{3}{c}{1d6 fresh ingredients 
1 medium carapace} & \multicolumn{3}{c}{1d6 fresh ingredients 
1d4 hides} \\
\midrule
\end{tabularx}

\begin{minipage}{0.48\textwidth}
\begin{itemize}
  \item Game: Caves
  \item Game: Deserts
  \item Game: Forests
  \item Game: Grasslands
  \item Game: Graveyards
\end{itemize}
\end{minipage}\hfill
\begin{minipage}{0.48\textwidth}
\begin{itemize}
  \item Game: Marshes
  \item Game: Mountains
  \item Game: Shore
  \item Game: Tundra
  \item Game: Underground
\end{itemize}
\end{minipage}

\section{Alchemy}

% [Image Inserted Manually]

\section*{Alchemy}

Alchemy is a crafting art that almost all adventures have some degree of interest in the results of. The source of the ever in demand Potion of Healings, it is a versatile trade that fuels (sometimes quite literally) the adventuring life.

It doesn’t take many experiences with the powers of potions for an adventurer to consider if they can get away with simmering a Potion of Healing next to the stew over that night’s cooking fire... of course it’s easier said than done for the result of such things to come away not poisonous.

Alchemy tends to be a very quick form of crafting, but with this comes additional risks with most crafts resting on a single roll, and failure resulting in the loss of all materials. Although taking that chance is frequently worth it during a busy adventuring season, consider the “Taking 10” option outlined in the craft introduction when speed is not of the essence.

\section*{Related Tool \& Ability Score}

Alchemy works using alchemist’s supplies. Attempting to craft a potion without these will almost always be made with disadvantage, and proficiency with these allows you to add your proficiency to any alchemy crafting roll.

Alchemy uses your choice of your Intelligence or Wisdom modifier.

\section*{Quick Reference}

While each step will go into more depth, the quick reference allows you to at a glance follow the steps to make a potion in its most basic form:

\begin{itemize}
  \item Select a potion that you would like to craft from the “Potions Crafting Table”.
  \item Acquire the items listed in the materials column for that potion.
  \item Use your alchemist’s supplies tool to craft the potion using the number of hours listed in the Crafting Time column, or during a long rest using the crafting camp action if the crafting time is 2 hours or less. Alchemy items must be crafted in a single session.
  \item For every 2 hours, make a crafting roll of 1d20 + your Intelligence or Wisdom modifier (your choice) + your proficiency bonus with alchemist’s supplies. You can abort the craft after a bad crafting roll if you wish, this counts as a failure.
  \item On success, you mark 2 hours of completed time. Once the completed time is equal to the crafting time, the magic item is complete. On failure, the crafting time is lost and no progress has been made during the 2 hours. If you fail 3 times in a row, the crafting is a failure and all materials are lost.
\end{itemize}

\section*{Shelf Life \& Expired Potions}

A unique attribute to alchemy, potions once crafted have a shelf life of 1 year before coming expired. This shelf life is shortened to 1 month if the potion contains any reactive ingredient.

If an expired potion is used or consumed within double its shelf life, roll a d4. On a 1, you become poisoned for 1 minute. On a 2 or 3, the potion will work with reduced effect; its duration will be halved if it had a duration, and damage or healing it dealt will be halved. On a 4, it works as expected.

Any potion that is older than twice its shelf life has no effect besides causing the imbiber to become poisoned for 1 minute.

\section*{Crafting Roll}

Putting that together means that when you want to work on Alchemy, your crafting roll is as follows:

Alchemy Modifier = your Alchemist’s Supplies proficiency bonus + your Intelligence or Wisdom modifier (your choice)

\subsection*{Success and Failure}

For Alchemy, after you make the crafting roll and succeed mark your progress on a crafting project. If you succeed, you make 2 hours of progress toward the total crafting time (and have completed one of the required checks for making an item). Checks for Alchemy must be consecutive, meaning that if you wish to continue after a failed craft, you need to extend your crafting time.

Failure means that no progress is made during that time. Once an item is started, even if no progress is made, the components reserved for that item can only be recovered via salvage. If you fail three times in a row, all progress and materials are lost and can no longer be salvaged.

\section*{Alchemy Saving Throw}

Some alchemical creations trigger a saving throw. In instances that they do, the following is the formula for calculating the saving throw. The saving throw is calculated at the time of creation based on the creators attributes and proficiency, and doesn’t change once it is created.

Alchemy DC = 8 + your Alchemist’s Supplies proficiency bonus + your Intelligence or Wisdom Modifier (your choice)

% [Image Inserted Manually]

Herbalist's Limited Alchemy

While herbalism is primarily used for gathering herbs, a creature with proficiency with herbalism kits has a limited proficiency in creating potions in addition to gathering herbs. You can use proficiency with an herbalism kit in place of alchemist’s supplies when creating potions of healing , antidotes, antitoxins, and vials of poison.

When crafting in this way, you use an herbalism kit and proficiency with it in place of alchemist’s supplies. This crafting otherwise works the same as using alchemist’s supplies.

You can add additional reagent modifiers when crafting in this way.

Crafting Walk Through: Alchemy

Here’s a quick example of how alchemy will often play out in a game.

Caius the Wizard is saved from being eaten by a giant plant monster. He wants to know if his suffering was worth anything, so he asks the GM if there was any loot. There wasn’t. Was there anything to harvest from the monster? Well, that’s always possible. The GM asks for a d100. Caius rolls a 54.

It was just a CR 4 monster, so the GM consults the table and asks for a DC 8 Nature Check. Fortunately Caius the Wizard has a +3, so only needs a 5, and rolls an 8. Harvesting successful and now he has 1 common curative reagent.

Knowing that he’s going to need some healing potions to survive the next plant monster, he consults the table and realizes that with just 2 more curative reagents and a glass vial he’d have enough materials.

Next time they walk through a forest, Caius the Wizard asks the GM if there are any reagents about. Slowing down to a slow pass, he can make a gathering check with disadvantage, and attempts to gather reagents. Rolls the d100, and comes up with a 73. The GM asks for a DC 10 gathering check using herbalism kit, with disadvantage as they were travelling. Caius doesn’t have proficiency, but has +1 wisdom, so needs a 9 or higher. He rolls a 14!

The results in 1d4 common curative reagents. He rolls a 2. That’s all the curative reagents he needed! He cleans out the remains of an old healing potion to get a glass vial, and next time they camp, he spends his camp action (2 hours) trying to make a healing potion.

He needs to succeed only once, DC 13. Fortunately he has proficiency in alchemist’s supplies, and a +3 Intelligence, meaning his check is at +5. He needs an 8 or higher.

...He rolls a 3. This a failure! But all is not lost, he can keep trying until he fails 3 times in a row... but he’s already spent his 2 hours, and alchemy checks must be made consecutively.

Not wanting to lose his precious hard won reagents, he stays up late making another check. He rolls a 10! This results in a 15! He’s successfully made a Healing Potion!

Unfortunately he’s stayed up too late, but the next morning the party lets him sleep in to avoid the level of exhaustion.

Two days later, Vandrin the Cleric, the groups only healer, is on death’s door! A quick swig of the healing potion gets him back on his feet, and the day is won thanks to the hard work of making a potion, and adventurers keep a keen eye out for more of those curative reagents!

\section{Alchemy}

\section*{Alchemy Crafting Tables}

\section*{Potions}

\subsection*{Potions}

\begin{longtable}{p{2.5cm}\toprule
|p{2.5cm}|p{2.5cm}|p{2.5cm}|p{2.5cm}|p{2.5cm}|p{2.5cm}|p{2.5cm}|p{2.5cm}|p{2.5cm}|p{2.5cm}|}
\midrule
\multicolumn{2}{c}{Name} & \multicolumn{4}{c}{Materials} & Crafting Time & Checks & Difficulty & Rarity & Value \\
\midrule
\multicolumn{2}{c}{Antitoxin} & \multicolumn{4}{c}{2 common curative reagents 
1 common poisonous reagent 
1 glass vial} & 2 hours & 1 & DC 13 & common & 50 gp \\
\midrule
\multicolumn{2}{c}{Potion of Climbing} & \multicolumn{4}{c}{1 common reactive reagent 
1 common poisonous 
1 uncommon reactive reagent 
1 glass vial} & 2 hours & 1 & DC 14 & common & 85 gp \\
\midrule
\multicolumn{2}{c}{Potion of Healing} & \multicolumn{4}{c}{3 common curative reagents 
1 glass vial} & 2 hours & 1 & DC 13 & common & 50 gp \\
\midrule
\multicolumn{2}{c}{Potion of Animal Friendship} & \multicolumn{4}{c}{2 common reactive reagents 
1 common poisonous reagent 
1 uncommon curative reagent 
1 primal common essence 
1 glass vial} & 2 hours & 1 & DC 13 & uncommon & 145 gp \\
\midrule
\multicolumn{2}{c}{Potion of Fire Breath} & \multicolumn{4}{c}{1 common reactive reagent 
1 uncommon reactive reagent 
1 glass vial} & 2 hours & 1 & DC 15 & uncommon & 75 gp \\
\midrule
\multicolumn{2}{c}{Potion of Greater Healing} & \multicolumn{4}{c}{1 common curative reagent 
2 uncommon curative reagents 
1 glass vial} & 2 hours & 1 & DC 15 & uncommon & 120 gp \\
\midrule
\multicolumn{2}{c}{Potion of Growth} & \multicolumn{4}{c}{1 common reactive reagent 
1 uncommon curative reagent 
1 uncommon reactive reagent 
1 glass vial} & 2 hours & 1 & DC 14 & uncommon & 115 gp \\
\midrule
\multicolumn{2}{c}{Potion of Hill Giant Strength} & \multicolumn{4}{c}{1 uncommon primal essence 
1 uncommon reactive reagent 
1 uncommon curative reagent 
1 glass vial} & 4 hours & 2 & DC 15 & uncommon & 260 gp \\
\midrule
\multicolumn{2}{c}{Potion of Poison} & \multicolumn{4}{c}{1 common poisonous reagent 
1 uncommon poisonous reagent 
1 glass vial} & 2 hours & 1 & DC 13 & uncommon & 65 gp \\
\midrule
\multicolumn{2}{c}{Potion of Resistance} & \multicolumn{4}{c}{1 uncommon primal essence 
1 uncommon reactive reagent 
1 common curative reagent 
1 glass vial} & 2 hours & 1 & DC 15 & uncommon & 240 gp \\
\midrule
\multicolumn{2}{c}{Potion of Water Breathing} & \multicolumn{4}{c}{1 common reactive reagent 
1 uncommon poisonous reagent 
1 uncommon reactive reagent 
1 glass vial} & 2 hours & 1 & DC 15 & uncommon & 120 gp \\
\midrule
\multicolumn{2}{c}{Draught of Damnation (B)} & \multicolumn{4}{c}{1 rare reactive reagent 
1 rare poisonous reagent 
1 uncommon arcane essence 
1 glass vial} & 4 hours & 2 & DC 16 & rare & 680 gp \\
\midrule
\multicolumn{2}{c}{Potion of Clairvoyance} & \multicolumn{4}{c}{1 uncommon reactive reagent 
1 uncommon poisonous reagent 
2 rare reactive reagents 
1 common arcane essence 
1 glass vial} & 2 hours & 1 & DC 16 & rare & 570 gp \\
\midrule
\multicolumn{2}{c}{Potion of Diminution} & \multicolumn{4}{c}{1 uncommon curative reagent 
1 rare curative reagent 
1 rare poisonous reagent 
1 glass vial} & 2 hours & 1 & DC 15 & rare & 480 gp \\
\midrule
\multicolumn{2}{c}{Potion of Gaseous Form} & \multicolumn{4}{c}{2 uncommon reactive reagents 
1 rare curative reagent 
1 rare reactive reagent 
1 glass vial} & 2 hours & 1 & DC 16 & rare & 560 gp \\
\midrule
\multicolumn{2}{c}{Potion of Heroism} & \multicolumn{4}{c}{1 uncommon curative reagent 
1 uncommon reactive reagent 
2 rare curative reagents 
1 common divine essence 
1 glass vial} & 2 hours & 1 & DC 15 & rare & 480 gp \\
\midrule
\multicolumn{2}{c}{Potion of Mind Reading} & \multicolumn{4}{c}{1 uncommon poisonous reagent 
1 uncommon reactive reagent 
1 rare poisonous reagent 
1 rare reactive reagent 
1 glass vial} & 2 hours & 1 & DC 16 & rare & 550 gp \\
\midrule
\multicolumn{2}{c}{Potion of Superior Healing} & \multicolumn{4}{c}{2 uncommon curative reagents 
2 rare curative reagents 
1 glass vial} & 4 hours & 2 & DC 15 & rare & 525 gp \\
\midrule
\multicolumn{2}{c}{Potion of Flying} & \multicolumn{4}{c}{2 uncommon reactive reagents 
2 rare curative reagents 
2 very rare reactive reagents 
1 uncommon primal essences 
1 uncommon arcane essence 
1 crystal vial} & 4 hours & 2 & DC 19 & very rare & 5,500 gp \\
\midrule
\multicolumn{2}{c}{Potion of Invisibility} & \multicolumn{4}{c}{2 uncommon reactive reagents 
2 rare curative reagents 
1 very rare reactive reagent 
1 very rare curative reagent 
1 crystal vial} & 4 hours & 2 & DC 19 & very rare & 5,200 gp \\
\midrule
\multicolumn{2}{c}{Potion of Speed} & \multicolumn{4}{c}{2 uncommon reactive reagents 
2 rare reactive reagents 
1 very rare reactive reagent 
1 very rare curative reagent 
1 rare arcane essence 
1 crystal vial} & 4 hours & 2 & DC 20 & very rare & 6,150 gp \\
\midrule
\multicolumn{2}{c}{Potion of Supreme Healing} & \multicolumn{4}{c}{1 uncommon curative reagent 
1 rare curative reagent 
2 very rare curative reagents 
1 uncommon divine essence 
1 crystal vial} & 4 hours & 2 & DC 18 & very rare & 5,000 gp \\
\midrule
\multicolumn{2}{c}{Panacea (B)} & \multicolumn{4}{c}{1 legendary curative reagent 
2 very rare curative reagents 
1 legendary divine essence 
1 crystal vial} & 8 hours & 4 & DC 24 & legendary & 54,000 gp \\
\midrule
\multicolumn{2}{c}{Potion of Storm Giant Strength} & \multicolumn{4}{c}{1 legendary reactive reagent 
1 legendary curative reagent 
1 very rare primal essence 
1 crystal vial} & 8 hours & 4 & DC 23 & legendary & 25,000 gp \\
\midrule
\end{longtable}

\section*{Concoctions}

\subsection*{Concoctions}

\begin{tabularx}{\textwidth}\toprule
{}XXXXXXXXXX}
\midrule
\multicolumn{2}{c}{Name} & \multicolumn{4}{c}{Materials} & Crafting Time & Checks & Difficulty & Rarity & Value \\
\midrule
\multicolumn{2}{c}{Alchemical Acid (B)} & \multicolumn{4}{c}{2 common reactive reagents 
1 common poisonous reagent 
1 glass flask} & 2 hours & 1 & DC 13 & common & 50 gp \\
\midrule
\multicolumn{2}{c}{Alchemical Fire (B)} & \multicolumn{4}{c}{3 common reactive reagents 
1 glass flask} & 2 hours & 1 & DC 13 & common & 50 gp \\
\midrule
\multicolumn{2}{c}{Alchemical Napalm (B)} & \multicolumn{4}{c}{3 common reactive reagents 
1 common curative reagent 
1 glass flask} & 2 hours & 1 & DC 14 & common & 70 gp \\
\midrule
\multicolumn{2}{c}{Bottled Wind (B)} & \multicolumn{4}{c}{2 common reactive reagents 
1 glass flask} & 2 hours & 1 & DC 14 & common & 40 \\
\midrule
\multicolumn{2}{c}{Potent Alchemical Acid (B)} & \multicolumn{4}{c}{2 uncommon reactive reagents 
1 uncommon poisonous reagent 
1 glass flask} & 2 hours & 1 & DC 15 & uncommon & 140 gp \\
\midrule
\multicolumn{2}{c}{Potent Alchemical Fire (B)} & \multicolumn{4}{c}{3 uncommon reactive reagents 
1 glass flask} & 2 hours & 1 & DC 15 & uncommon & 140 gp \\
\midrule
\multicolumn{2}{c}{Sticky Goo Potion (B)} & \multicolumn{4}{c}{Either 
(a) 1 finely shredded scroll of web
 or 
(b) 2 uncommon poisonous reagents 
1 uncommon reactive reagent 
1 glass flask} & 2 hours & 1 & DC 14 & uncommon & 140 gp \\
\midrule
\multicolumn{2}{c}{Liquid Lightning (B)} & \multicolumn{4}{c}{2 rare reactive reagents 
1 uncommon primal essence 
1 glass vial} & 2 hours & 1 & DC 16 & rare & 640 gp \\
\midrule
\multicolumn{2}{c}{Powerful Alchemical Acid (B)} & \multicolumn{4}{c}{2 rare reactive reagents 
1 rare poisonous reagent 
1 glass flask} & 2 hours & 1 & DC 17 & rare & 690 gp \\
\midrule
\multicolumn{2}{c}{Powerful Alchemical Fire (B)} & \multicolumn{4}{c}{3 rare reactive reagents 
1 glass flask} & 2 hours & 1 & DC 17 & rare & 690 gp \\
\midrule
\end{tabularx}

\section*{Explosives}

\subsection*{Explosives}

\begin{tabularx}{\textwidth}\toprule
{}XXXXXXXXXX}
\midrule
\multicolumn{2}{c}{Name} & \multicolumn{4}{c}{Materials} & Crafting Time & Checks & Difficulty & Rarity & Value \\
\midrule
\multicolumn{2}{c}{Blasting Powder} & \multicolumn{4}{c}{2 common reactive reagents} & 2 hours & 1 & DC 14 & common & 40 gp \\
\midrule
\multicolumn{2}{c}{Dwarven Alcohol} & \multicolumn{4}{c}{1 flask of alcohol 
1 common reactive reagent 
1 sturdy metal flask} & 8 hours & 4 & DC 12 & common & 20 gp \\
\midrule
\multicolumn{2}{c}{Simple Explosive} & \multicolumn{4}{c}{2 packets blasting powder 
1 common reactive reagent} & 2 hours & 1 & DC 15 & common & 120 gp \\
\midrule
\multicolumn{2}{c}{Smoke Powder} & \multicolumn{4}{c}{2 common reactive reagents} & 2 hours & 1 & DC 15 & common & 40 gp \\
\midrule
\multicolumn{2}{c}{Grenade Casing} & \multicolumn{4}{c}{2 parts 
1 fancy parts 
1 glass flask} & 4 hours & 2 & DC 15 & uncommon & 50 gp \\
\midrule
\multicolumn{2}{c}{Nail Bomb} & \multicolumn{4}{c}{3 parts 
2 uncommon reactive reagents 
1 packet blasting powder} & 2 hours & 1 & DC 17 & uncommon & 275 gp \\
\midrule
\multicolumn{2}{c}{Potent Explosive} & \multicolumn{4}{c}{4 packets blasting powder 
1 common reactive reagent} & 2 hours & 1 & DC 17 & uncommon & 250 gp \\
\midrule
\multicolumn{2}{c}{Powerful Explosive} & \multicolumn{4}{c}{8 packets blasting powder 
1 rare reactive reagent} & 4 hours & 2 & DC 19 & rare & 750 gp \\
\midrule
\end{tabularx}

\section*{Magical Dust}

\subsection*{Magical Dust}

\begin{tabularx}{\textwidth}\toprule
{}XXXXXXXXXX}
\midrule
\multicolumn{2}{c}{Name} & \multicolumn{4}{c}{Materials} & Crafting Time & Checks & Difficulty & Rarity & Value \\
\midrule
\multicolumn{2}{c}{Dust of Disappearance} & \multicolumn{4}{c}{1 handful of sand 
1 common arcane essence 
2 common reactive reagents 
1 common curative reagent} & 4 hours & 2 & DC 15 & uncommon & 130 gp \\
\midrule
\multicolumn{2}{c}{Dust of Dryness} & \multicolumn{4}{c}{1 handful of sand 
1 common primal essence 
1 common reactive reagent 
1 common poisonous reagent} & 4 hours & 2 & DC 14 & uncommon & 100 gp \\
\midrule
\multicolumn{2}{c}{Dust of Sneezing and Choking} & \multicolumn{4}{c}{1 uncommon poisonous reagent 
1 common reactive reagent 
1 common poisonous reagent} & 4 hours & 2 & DC 14 & uncommon & 95 gp \\
\midrule
\end{tabularx}

\section*{Magical Ink}

\subsection*{Magical Ink}

\begin{tabularx}{\textwidth}\toprule
{}XXXXXXXXXX}
\midrule
\multicolumn{2}{c}{Name} & \multicolumn{4}{c}{Materials} & Crafting Time & Checks & Difficulty & Rarity & Value \\
\midrule
\multicolumn{2}{c}{Common Magical Ink} & \multicolumn{4}{c}{1 common alchemical reagent 
1 glass vial} & 2 hours & 1 & DC 10 & common & 15 gp \\
\midrule
\multicolumn{2}{c}{Uncommon Magical Ink} & \multicolumn{4}{c}{1 uncommon alchemical reagent 
1 glass vial} & 2 hours & 1 & DC 12 & uncommon & 40 gp \\
\midrule
\multicolumn{2}{c}{Rare Magical Ink} & \multicolumn{4}{c}{1 rare alchemical reagent 
1 glass vial} & 2 hours & 1 & DC 14 & rare & 200 gp \\
\midrule
\multicolumn{2}{c}{Very Rare Magical Ink} & \multicolumn{4}{c}{1 very rare alchemical reagent 
1 glass vial} & 4 hours & 2 & DC 16 & very rare & 2,000 gp \\
\midrule
\multicolumn{2}{c}{Legendary Magical Ink} & \multicolumn{4}{c}{1 legendary alchemical reagent 
1 glass vial} & 8 hours & 4 & DC 18 & legendary & 5,000 gp \\
\midrule
\end{tabularx}

\section*{Oils}

\subsection*{Oils}

\begin{tabularx}{\textwidth}\toprule
{}XXXXXXXXXX}
\midrule
\multicolumn{2}{c}{Name} & \multicolumn{4}{c}{Materials} & Crafting Time & Checks & Difficulty & Rarity & Value \\
\midrule
\multicolumn{2}{c}{Burning Oil (B)} & \multicolumn{4}{c}{2 common reactive reagents 
1 glass vial} & 2 hours & 1 & DC 13 & common & 40 gp \\
\midrule
\multicolumn{2}{c}{Frost Oil (B)} & \multicolumn{4}{c}{1 common reactive reagents 
1 common primal essence 
1 glass vial} & 2 hours & 1 & DC 14 & common & 75 gp \\
\midrule
\multicolumn{2}{c}{Silver Oil (B)} & \multicolumn{4}{c}{4 silver scraps 
1 common reactive reagent 
1 glass vial} & 2 hours & 1 & DC 12 & common & 20 gp \\
\midrule
\multicolumn{2}{c}{Flametongue Oil (B)} & \multicolumn{4}{c}{2 uncommon reactive reagents 
1 common arcane essence 
1 glass vial} & 2 hours & 1 & DC 16 & uncommon & 170 gp \\
\midrule
\multicolumn{2}{c}{Oil of Sharpness} & \multicolumn{4}{c}{1 rare poisonous reagent 
2 very rare reactive reagent 
300 gp of precious metal flakes 
1 crystal vial} & 4 hours & 2 & DC 19 & very rare & 5,200 gp \\
\midrule
\end{tabularx}

\section*{Miscellaneous}

\subsection*{Miscellaneous}

\begin{tabularx}{\textwidth}\toprule
{}XXXXXXXXXX}
\midrule
\multicolumn{2}{c}{Name} & \multicolumn{4}{c}{Materials} & Crafting Time & Checks & Difficulty & Rarity & Value \\
\midrule
\multicolumn{2}{c}{Restorative Ointment} & \multicolumn{4}{c}{1 common divine essence 
2 uncommon curative reagents 
3 common curative reagents} & 8 hours & 4 & DC 15 & uncommon & 250 gp \\
\midrule
\multicolumn{2}{c}{Sovereign Glue} & \multicolumn{4}{c}{1 legendary curative reagent 
1 legendary reactive reagent 
1 very rare divine essence} & 16 hours
 (2 days) & 8 & DC 22 & legendary & 25,000 gp \\
\midrule
\multicolumn{2}{c}{Universal Solvent} & \multicolumn{4}{c}{1 legendary poisonous reagent 
1 legendary reactive reagent 
1 very rare primal essence} & 16 hours
 (2 days) & 8 & DC 22 & legendary & 25,000 \\
\midrule
\end{tabularx}

\section*{Alchemy Recipes}

\begin{minipage}{0.48\textwidth}
\subsubsection*{Concoctions}

\begin{itemize}
  \item Alchemical Acid (B)
  \item Alchemical Fire (B)
  \item Alchemical Napalm (B)
  \item Bottled Wind (B)
  \item Liquid Lightning (B)
  \item Potent Alchemical Acid (B)
  \item Potent Alchemical Fire (B)
  \item Powerful Alchemical Acid (B)
  \item Powerful Alchemical Fire (B)
  \item Sticky Goo Potion(B)
\end{itemize}

\subsubsection*{Explosives}

\begin{itemize}
  \item Dwarven Alcohol
  \item Grenade Casing
  \item Nail Bomb
  \item Packet of Blasting Powder
  \item Potent Explosive
  \item Powerful Explosive
  \item Simple Explosive
  \item Smoke Powder
\end{itemize}

\subsubsection*{Magical Dust}

\begin{itemize}
  \item Dust of Disappearance
  \item Dust of Dryness
  \item Dust of Sneezing and Choking
\end{itemize}

\subsubsection*{Magical Ink}

\begin{itemize}
  \item Common Magical Ink
  \item Legendary Magical Ink
  \item Rare Magical Ink
  \item Uncommon Magical Ink
  \item Very Rare Magical Ink
\end{itemize}

\subsubsection*{Miscellaneous}

\begin{itemize}
  \item Restorative Ointment
  \item Sovereign Glue
  \item Universal Solvent
\end{itemize}
\end{minipage}\hfill
\begin{minipage}{0.48\textwidth}
\subsubsection*{Oils}

\begin{itemize}
  \item Burning Oil (B)
  \item Flametongue Oil (B)
  \item Frost Oil (B)
  \item Oil of Sharpness
  \item Silver Oil (B)
\end{itemize}

\subsubsection*{Potions}

\begin{itemize}
  \item Antitoxin
  \item Draught of Damnation (B)
  \item Panacea (B)
  \item Potion of Animal Friendship
  \item Potion of Clairvoyance
  \item Potion of Climbing
  \item Potion of Diminution
  \item Potion of Fire Breath
  \item Potion of Flying
  \item Potion of Gaseous Form
  \item Potion of Greater Healing
  \item Potion of Growth
  \item Potion of Healing
  \item Potion of Heroism
  \item Potion of Hill Giant Strength
  \item Potion of Invisibility
  \item Potion of Mind Reading
  \item Potion of Poison
  \item Potion of Resistance
  \item Potion of Speed
  \item Potion of Superior Healing
  \item Potion of Supreme Healing
  \item Potion of Water Breathing
\end{itemize}
\end{minipage}

\section{Alchemy}

\section*{Alchemy Items}

\section*{Potions}

% [Image Inserted Manually]

\subsection*{Draught of Damnation}

Potion, rare

This sticky red liquid has a living viscosity to it, churning slowly within the flask. When you drink this potion, you become a fiend (as if by the spell shapechange for 1 hour). The type of fiend you become is determined by your level. The new form is a random fiend with a Challenge Rating equal to your level, up to a maximum of 10. If there are multiple options at that Challenge Rating, roll to determine which option is selected.

When the effect ends and you revert to your normal form, you take 1d6 necrotic damage equal to the CR of the fiend who’s form you assumed as the toxins of pure evil attempt to destroy your body before relinquishing control of it. If this damage kills you, your soul is dragged to the realm of the fiends by the malignant will of the fiend.

\begin{itemize}
  \item Item: Draught of Damnation (B)
\end{itemize}

\begin{minipage}{0.48\textwidth}
\subsubsection*{Healing Potion}

Potion, common

You regain 2d4 + 2 Hit Points when you drink this potion. The potion’s red liquid glimmers when agitated.

\begin{tabularx}{\textwidth}\toprule
{}XX}
\midrule
Quality & Rarity & Hit Points Restored \\
\midrule
Common & Common & 2d4 + 2 \\
\midrule
Greater & Uncommon & 4d4 + 4 \\
\midrule
Superior & Rare & 8d4 + 8 \\
\midrule
Supreme & Very Rare & 10d4 + 20 \\
\midrule
\end{tabularx}

\begin{itemize}
  \item Item: Potion of Healing
  \item Item: Potion of Greater Healing
  \item Item: Potion of Superior Healing
  \item Item: Potion of Supreme Healing
\end{itemize}

\subsubsection*{Panacea}

Potion, legendary

When you drink this potion, you regain all lost hit points, all status effects are removed, all reductions to ability scores are removed, any missing limbs are restored, all diseases are cured, all curses are removed, and all levels of exhaustions are removed, and you cease to age for 1 year. This potion can remove effects that can otherwise only be removed by wish.

\begin{itemize}
  \item Item: Panacea (B)
\end{itemize}

\subsubsection*{Potion of Animal Friendship}

Potion, uncommon

When you drink this potion, you can cast the animal friendship spell (save DC 13) for 1 hour at will.

A murky, muddy potion, it leaves various animal shapes and tracks on the side of the container as it swirls.

\begin{itemize}
  \item Item: Potion of Animal Friendship
\end{itemize}

\subsubsection*{Potion of Clairvoyance}

Potion, rare

When you drink this potion, you gain the effect of the clairvoyance spell. An eyeball bobs in this yellowish liquid but vanishes when the potion is opened.

\begin{itemize}
  \item Item: Potion of Clairvoyance
\end{itemize}

\subsubsection*{Potion of Climbing}

Potion, uncommon

When you drink this potion, you gain a climbing speed equal to your walking speed for 1 hour. During this time, you have advantage on Strength (Athletics) checks you make to climb. The potion is separated into brown, silver, and gray layers resembling bands of stone. Shaking the bottle fails to mix the colors.

\begin{itemize}
  \item Item: Potion of Climbing
\end{itemize}

\subsubsection*{Potion of Diminution}

Potion, rare

When you drink this potion, you gain the “reduce” effect of the enlarge/reduce spell for 1d4 hours (no concentration required). The red in the potion’s liquid continuously contracts to a tiny bead and then expands to color the clear liquid around it. Shaking the bottle fails to interrupt this process.

\begin{itemize}
  \item Item: Potion of Diminution
\end{itemize}

\subsubsection*{Potion of Fire Breath}

Potion, uncommon

After drinking this potion, you can use a bonus action to exhale fire at a target within 30 feet of you. The target must make a DC 13 Dexterity saving throw, taking 4d6 fire on a failed save, or half as much damage on a successful one. The effect ends after you exhale the fire three times or when 1 hour has passed. This potion’s orange liquid flickers, and smoke fills the top of the container and wafts out whenever it is opened.

\begin{itemize}
  \item Item: Potion of Fire Breath
\end{itemize}
\end{minipage}\hfill
\begin{minipage}{0.48\textwidth}
\subsubsection*{Potion of Flying}

Potion, very rare

When you drink this potion, you gain a flying speed equal to your walking speed for 1 hour and can hover. If you’re in the air when the potion wears off, you fall unless you have some other means of staying aloft. This potion’s clear liquid floats at the top of its container and has cloudy white impurities drifting in it.

\begin{itemize}
  \item Item: Potion of Flying
\end{itemize}

\subsubsection*{Potion of Gaseous Form}

Potion, rare

When you drink this potion, you gain the effect of the gaseous form spell for 1 hour (no concentration required) or until you end the effect as a bonus action. This potion’s container seems to hold fog that moves and pours like water.

\begin{itemize}
  \item Item: Potion of Gaseous Form
\end{itemize}

\subsubsection*{Potion of Growth}

Potion, uncommon

When you drink this potion, you gain the “enlarge” effect of the enlarge/reduce spell for 1d4 hours (no concentration required). The red in the potion’s liquid continuously expands from a tiny bead to color the clear liquid around it and then contracts. Shaking the bottle fails to interrupt this process.

\begin{itemize}
  \item Item: Potion of Growth
\end{itemize}

\subsubsection*{Potion of Heroism}

Potion, rare

For 1 hour after drinking it, you gain 10 temporary hit points that last for 1 hour. For the same duration, you are under the effect of the bless spell (no concentration required). This blue potion bubbles and steams as if boiling.

\begin{itemize}
  \item Item: Potion of Heroism
\end{itemize}

\subsubsection*{Potion of Invisibility}

Potion, very rare

This potion’s container looks empty but feels as though it holds liquid. When you drink it, you become invisible for 1 hour. Anything you wear or carry is invisible with you. The effect ends early if you attack or cast a spell.

\begin{itemize}
  \item Item: Potion of Invisibility
\end{itemize}

\subsubsection*{Potion of Mind Reading}

Potion, rare

When you drink this potion, you gain the effect of the detect thoughts spell (save DC 13, no concentration required) for 1 hour. The potion’s dense, purple liquid has an ovoid cloud of pink floating in it.

\begin{itemize}
  \item Item: Potion of Mind Reading
\end{itemize}

\subsubsection*{Potion of Poison}

Potion, uncommon

This concoction looks, smells, and tastes like a potion of healing or other beneficial potion. However, it is actually poison masked by illusion magic. An identify spell reveals its true nature.

If you drink it, you take 3d6 poison damage, and you must succeed on a DC 13 Constitution saving throw or be poisoned. At the start of each of your turns while you are poisoned in this way, you take 3d6 poison damage. At the end of each of your turns, you can repeat the saving throw. On a successful save, the poison damage you take on your subsequent turns decreases by 1d6. The poison ends when the damage decreases to 0.

\begin{itemize}
  \item Item: Potion of Poison
\end{itemize}

\subsubsection*{Potion of Resistance}

Potion, uncommon

When you make this potion, you can make it resist one particular damage type. For necrotic or radiant resistance, substitute an uncommon divine essence for the primal essence. For force resistance, substitute an uncommon arcane essence. For psychic, substitute an uncommon psionic essence.

The color of the potion depends on what kind of resistance it is, often having that element swirling within it.

When you drink this potion, you gain resistance to the selected damage type of the potion for 1 hour.
\end{minipage}

% [Image Inserted Manually]

\subsection*{Potion of Speed}

Potion, very rare

When you drink this potion, you gain the effect of the haste spell for 1 minute (no concentration required). The potion’s yellow fluid is streaked with black and swirls on its own.

\begin{itemize}
  \item Item: Potion of Speed
\end{itemize}

\subsection*{Potion of Water Breathing}

Potion, uncommon

You can breathe underwater for 1 hour after drinking this potion. Its cloudy green fluid smells of the sea and has a jellyfish-like bubble floating in it.

\begin{itemize}
  \item Item: Potion of Water Breathing
\end{itemize}

\section*{Concoctions}

\begin{minipage}{0.48\textwidth}
\subsubsection*{Alchemical Acid}

Concoction, common

A small flask of burbling acid, a strange hissing green viscous liquid. It deals 4d4 acid damage when poured on an object. Can be used as a simple ranged weapon with the thrown (20/60) property, dealing 4d4 acid damage on hit. You do not add your modifier to the damage roll.

\begin{tabularx}{\textwidth}\toprule
{}XX}
\midrule
Quality & Rarity & Acid Damage \\
\midrule
Common & Common & 4d4 \\
\midrule
Potent & Uncommon & 6d4 \\
\midrule
Powerful & Rare & 8d4 \\
\midrule
\end{tabularx}

\begin{itemize}
  \item Item: Alchemical Acid (B)
  \item Item: Potent Alchemical Acid (B)
  \item Item: Powerful Alchemical Acid (B)
\end{itemize}
\end{minipage}\hfill
\begin{minipage}{0.48\textwidth}
\subsubsection*{Alchemical Fire}

Concoction, common

A small flask of volatile orange liquid. It deals 2d10 fire damage when poured on an object. Can be used as a simple ranged weapon with the thrown (20/60) property, dealing 2d10 fire damage on hit. You do not add your modifier to the damage roll.

\begin{tabularx}{\textwidth}\toprule
{}XX}
\midrule
Quality & Rarity & Fire Damage \\
\midrule
Common & Common & 2d10 \\
\midrule
Potent & Uncommon & 3d10 \\
\midrule
Powerful & Rare & 4d10 \\
\midrule
\end{tabularx}

\begin{itemize}
  \item Item: Alchemical Fire (B)
  \item Item: Potent Alchemical Fire (B)
  \item Item: Powerful Alchemical Fire (B)
\end{itemize}
\end{minipage}

\subsection*{Alchemical Napalm}

Concoction, common

A vicious sticky flammable substance. It deals 3d4 fire damage when poured on an object. Can be used as a simple ranged weapon with the thrown (20/60) property, dealing 3d4 fire damage on hit. You do not add your modifier to the damage roll.

On hit, the target creature or object continues to burn for one minute, taking 1d4 fire damage at the start of their turn (or at the start of your turn for an object without a turn) until a creature spends an action to put the flames out.

\begin{itemize}
  \item Item: Alchemical Napalm (B)
\end{itemize}

\subsection*{Bottled Wind}

Concoction, common

As an action, you can open this, casting gust of wind without verbal or somatic components. Alternatively, you can breath from it, letting out only a little bit at a time, breathing directly from the bottle, but each time you must make a DC 5 athletics checks. On failure, you cast gust of wind as above and all the air is lost. You can get 10 minutes of breathable air from one bottle.

\begin{itemize}
  \item Item: Bottled Wind (B)
\end{itemize}

% [Image Inserted Manually]

\subsection*{Liquid Lightning}

Concoction, rare

Drinking this potion allows you to zip around for 1 minute as pure electricity. As a bonus action on each of your turns before the effect ends, you can transform into a bolt of lightning and instantly travel in a straight line to an unoccupied space you can see within 30 feet. This movement doesn’t provoke opportunity attacks and you are immune to lightning damage while in this form.

You can pass through small holes, narrow openings, and even mere cracks, as well as through conductive materials such as metal. Each creature in your travel path must succeed on a DC 15 Dexterity saving throw or take 3d6 lightning damage. You immediately revert to your normal form upon reaching the destination. This potion’s stormy liquid arcs with tiny bolts of lightning and it numbs the tongue when sipped.

\begin{itemize}
  \item Item: Liquid Lightning (B)
\end{itemize}

\subsection*{Sticky Goo}

Concoction, common

When broken and exposed to air, it creates a very sticky rapidly expanding web like foam, with the effect of the spell web centered on where the flask breaks. You can reliably throw the flask to a target point within 30 feet, shattering it on impact.

\begin{itemize}
  \item Item: Sticky Goo (B)
\end{itemize}

\section*{Explosives}

\begin{minipage}{0.48\textwidth}
\subsubsection*{Blasting Powder}

Explosive, common

A fine grey powder with large grains and the faint smell of sulfur and charcoal that comes in small packets weighing 1/2 pound.

When ignited by 1 or more fire or lightning damage, it explodes violently. All creatures within 10 feet of it must make a DC 14 Dexterity saving throw. On failure, they take 1d4 fire + 1d4 thunder damage, or half as much on a success. The amount of damage increases by 1d4 (both the fire and thunder) and the radius increases by 5 feet for each additional packet of Blasting Powder detonated in the same spot, up to a maximum of five packets. Deals double damage to buildings and structures. Creatures in range of more than one stack of up to 5 explosives at the same time take damage only from the highest damage effect.

Frequently used for mining and other responsible things... until adventurers get their hands on it.

\begin{itemize}
  \item Item: Blasting Powder
\end{itemize}

\subsubsection*{Dwarven Alcohol}

Explosive, common

Only dwarves really know if the name of this liquid explosive is a joke or not, but most assume it is an acquired taste. An explosively flammable liquid that comes in a flask, this flask can be splashed across a 5-foot square within 5 feet. Once splashed, it can be ignited by 1 or more fire or lightning damage. When ignited it explodes in a plume of fire, dealing 2d4 fire damage to all creatures within 5 feet of the container, or within a square that has been soaked with it.

\begin{itemize}
  \item Item: Dwarven Alcohol
\end{itemize}

\subsubsection*{Grenade Casing}

Explosive, uncommon

A simple construction of a two chambered projectile (typically made of glass). When you add an Alchemist Fire and an Explosive to its separate compartments, it becomes an incredibly dangerous device. As an action, a character can light this bomb and throw it at a point up to 60 feet away. Creatures within the range of the explosive used must make a Dexterity saving throw against the DC of the explosive used, or take damage equal to the explosion + 1d4 piercing damage + 1d4 fire damage.

\begin{itemize}
  \item Item: Grenade Casing
\end{itemize}

\subsubsection*{Nail Bomb}

Explosive, uncommon

A brutal instrument, this mixes explosive powder and nails to create a devastating fragmentation device. An exceedingly dangerous device. Heavier and more deadly than other explosives, the primary damage comes from the metal shrapnel (nails) flung in all directions. It can be detonated by dealing 1 fire or lightning damage to it. As an action, a packet of this explosive can be accurately thrown 20 feet, but will not detonate on impact (usually). When it detonates, all creatures within 20 feet of the target point must make a Dexterity saving throw with a DC equal to the crafter’s Alchemy DC. On failure, they take 8d4 piercing damage, or half as much on a success.

You can fuse your explosives. When fused, intentionally dealing fire damage to the explosives (or otherwise lighting the fuse) causes it to detonate on a delay, selected from: short (the end of your turn), medium (the start of your next turn), and long (2 rounds, at the start of your turn).

\begin{itemize}
  \item Item: Nail Bomb
\end{itemize}
\end{minipage}\hfill
\begin{minipage}{0.48\textwidth}
\subsubsection*{Smoke Powder}

Explosive, common

A fine grey powder with large grains and the faint smell of sulfur and charcoal that comes in small packets weighing 1/2 pound.

When ignited by 1 or more fire or lightning damage, it releases a blast of thick black smoke that fills a 20-foot radius. Everything in this smoke is heavily obscured for 2d4 rounds. At the start of your next turn after the number of rounds rolled, the smoke begins to fade leaving everything within the radius lightly obscured, and it clears completely at the start of your next turn after that.

Additional uses of the smoke powder extend the duration of heavy obscurement for an extra 1d4 rounds.

\begin{itemize}
  \item Item: Smoke Powder
\end{itemize}

\subsubsection*{Simple/Potent/Powerful Explosive}

Explosive, common/uncommon/rare

A bundled explosive alchemical preparation. It can be detonated by dealing 1 fire or lightning damage to it. As an action, a packet of this explosive can be accurately thrown 20 feet, but will not detonate on impact (usually). When it detonates, all creatures within 10 feet (for common) of the target point must make a Dexterity saving throw with a DC equal to the crafter’s Alchemy DC. On failure, they take 1d8 fire + 1d8 thunder damage, or half as much on a success. Creatures in range of more than one explosive take damage only from the highest damage effect.

\begin{tabularx}{\textwidth}\toprule
{}XX}
\midrule
Name & Radius & Damage \\
\midrule
Common & 10 feet & 1d8 fire + 1d8 thunder. \\
\midrule
Potent & 15 feet & 2d8 fire + 2d8 thunder. \\
\midrule
Powerful & 20 feet & 4d8 fire + 4d8 thunder. \\
\midrule
\end{tabularx}

\begin{itemize}
  \item Item: Simple Explosive
  \item Item: Potent Explosive
  \item Item: Powerful Explosive
\end{itemize}
\end{minipage}

\section*{Oils}

% [Image Inserted Manually]

\subsection*{Burning Oil}

Oil, common

As an action, you can coat a weapon in this oil and ignite it. For 1 minute, the ignited weapon burns, dealing an extra 1d4 fire to attacks made with it, and providing bright light in a 20-foot radius and dim light for an additional 20 feet.

\begin{itemize}
  \item Item: Burning Oil (B)
\end{itemize}

\subsection*{Flametongue Oil}

Oil, uncommon

As an action, you can coat a weapon in this oil and ignite it. For 1 minute, the ignited weapon burns, dealing an extra 2d6 fire to attacks made with it, and providing bright light in a 20-foot radius and dim light for an additional 20 feet.

\begin{itemize}
  \item Item: Flametongue Oil (B)
\end{itemize}

\subsection*{Frost Oil}

Oil, common

As an action, you can coat a weapon in this oil, causing it freeze over, covered in icy crystals. For 1 minute, the weapon deals an extra 1d6 cold damage on hit.

\begin{itemize}
  \item Item: Frost Oil (B)
\end{itemize}

\subsection*{Silver Oil}

Oil, common

A sparkling chromatic oil. The oil can coat one slashing or piercing weapon or up to 5 pieces of slashing or piercing ammunition. Applying the oil takes 1 minute. For 1 hour, the coated item is considered silvered.

\begin{itemize}
  \item Item: Silver Oil (B)
\end{itemize}

\section*{Miscellaneous}

\subsection*{Magical Ink}

Component, common/uncommon/rare/very rare/legendary

Magical ink that is used by Enchanters to create scrolls, made by rendering down magical alchemical ingredients.

\begin{itemize}
  \item Item: Magical Ink, Common
  \item Item: Magical Ink, Uncommon
  \item Item: Magical Ink, Rare
  \item Item: Magical Ink, Very Rare
  \item Item: Magical Ink, Legendary
\end{itemize}

\section{Alchemy}

\section*{Alchemy Modifiers \& Additional Materials}

\subsection*{Alchemy Modifiers}

The following are alchemical modifiers that can be applied to a potion. All alchemical modifiers require approval from the GM if the modifier will work with a given potion or creation.

\begin{tabularx}{\textwidth}\toprule
{}XXXXXXX}
\midrule
Modifier & Difficulty Modifier & \multicolumn{6}{c}{Effect} \\
\midrule
Aerosol & +8 & \multicolumn{6}{c}{Require two additional reactive reagents of equal rarity to the rarest reagent of the potion. Rather than drinking it, when uncorked or broken (as an action) it effects all creatures in a 5-foot radius as if they’d consumed it.} \\
\midrule
Celestial & +3 & \multicolumn{6}{c}{Requires a common divine essence. Adds the effect of lesser restoration to the potion.} \\
\midrule
Divine & +6 & \multicolumn{6}{c}{Requires an rare divine essence. Adds the effect of greater restoration to the potion.} \\
\midrule
Endothermic & +4 & \multicolumn{6}{c}{Any fire damage the potion does becomes cold damage} \\
\midrule
Expansive & +5 & \multicolumn{6}{c}{Requires an additional common reactive reagent. Expands the area of effect of any area of effect the potion has by 5 feet.} \\
\midrule
Insidious & +3 & \multicolumn{6}{c}{The effect of the potions becomes shrouded from magic such as identify. You can make the potion appear as another potion of the same rarity. Another alchemist can identify it by making an Alchemy Check against your Alchemy DC} \\
\midrule
\end{tabularx}

\subsection*{Additional Materials}

You can simply load more reagents into any potion, increasing its potency in different ways, though it increases the difficulty. You can make a “custom potion” simply by using these modifiers with no base potion. When you do so, the base DC is 8.

\begin{tabularx}{\textwidth}\toprule
{}XXXXX}
\midrule
Additional Reagent & Difficulty Modifier & \multicolumn{4}{c}{Effect} \\
\midrule
Common Curative & +2 & \multicolumn{4}{c}{The potion restores +1d4 hit points when consumed.} \\
\midrule
Common Reactive & +2 & \multicolumn{4}{c}{The potion deals +1d4 fire damage when shattered (or consumed).} \\
\midrule
Common Poisonous & +2 & \multicolumn{4}{c}{The potion deals +1d4 poison damage when consumed.} \\
\midrule
Uncommon Curative & +3 & \multicolumn{4}{c}{The potion restores +2d4 hit points when consumed.} \\
\midrule
Uncommon Reactive & +3 & \multicolumn{4}{c}{The potion deals +2d4 fire damage when shattered (or consumed).} \\
\midrule
Uncommon Poisonous & +3 & \multicolumn{4}{c}{The potion deals +2d4 poison damage when consumed.} \\
\midrule
Rare Curative & +4 & \multicolumn{4}{c}{The potion restores +3d4 hit points when consumed.} \\
\midrule
Rare Reactive & +4 & \multicolumn{4}{c}{The potion deals +3d4 fire damage when shattered (or consumed).} \\
\midrule
Rare Poisonous & +4 & \multicolumn{4}{c}{The potion deals +3d4 poison damage when consumed.} \\
\midrule
Very Rare Curative & +5 & \multicolumn{4}{c}{The potion restores +4d4 hit points when consumed.} \\
\midrule
Very Rare Reactive & +5 & \multicolumn{4}{c}{The potion deals +4d4 fire damage when shattered (or consumed).} \\
\midrule
Very Rare Poisonous & +5 & \multicolumn{4}{c}{The potion deals +4d4 poison damage when consumed.} \\
\midrule
Legendary Curative & +6 & \multicolumn{4}{c}{The potion restores +5d4 hit points when consumed.} \\
\midrule
Legendary Reactive & +6 & \multicolumn{4}{c}{The potion deals +5d4 fire damage when shattered (or consumed).} \\
\midrule
Legendary Poisonous & +6 & \multicolumn{4}{c}{The potion deals +5d4 poison damage when consumed.} \\
\midrule
\end{tabularx}

\section{Poisoncraft}

\section*{Poisoncraft}

A subdomain of alchemy, the profession of poisoncraft is often seen as the “dark side” of Alchemy. While Alchemy often deals in poisonous reagents, typically speaking they aim to tame the poison, channeling it into useful effects. While capable of making crudely poisonous potions, such things are generally considered failures to an alchemist. To a poisoner, they are considered the art itself.

Poisoncraft shares Shelf Life and Reagents with Alchemy; for details regarding those, see Alchemy.

\section*{Related Tool \& Ability Score}

Poisoncraft works using a Poisoner’s Kit. Attempting to craft a potion without these will almost always be made with disadvantage, and proficiency with these allows you to add your proficiency to any poisoncraft crafting roll.

Poisoncraft uses your choice of your Intelligence or Wisdom modifier, representing your path of knowledge to the art of making deadly things deadlier.

\section*{Quick Reference}

While each step will go into more depth, the quick reference allows you to at a glance follow the steps to make a poison in its most basic form:

\begin{itemize}
  \item Select a poison that you would like to craft from the “Poison Crafting Table”.
  \item Acquire the items listed in the materials column for that potion.
  \item Use your Poisoner’s Kit tool to craft the option using the number of hours listed in the Crafting Time column, or during a long rest using the crafting camp action if the crafting time is 2 hours or less. Poisoncraft items must be crafted in a single session.
  \item For every 2 hours, make a crafting roll of 1d20 + your Intelligence or Wisdom modifier (your choice) + your proficiency bonus with a Poisoner’s Kit. You can abort the craft after a bad crafting roll if you wish, this counts as a failure.
  \item On success, you mark 2 hours of completed time. Once the completed time is equal to the crafting time, the magic item is complete. On failure, the crafting time is lost and no progress has been made during the 2 hours. If you fail 3 times in a row, the crafting is a failure and all materials are lost.
\end{itemize}

\section*{Crafting Roll}

Putting that together means that when you would like to create poison, your crafting roll is as follows:

Poisoncraft Modifier = your Poisoner’s Kit proficiency bonus + your Intelligence or Wisdom modifier (your choice).

\subsection*{Success and Failure}

For Poisoncraft, after you make the crafting roll and succeed, mark your progress on a crafting project. If you succeed, you make 2 hours of progress toward the total crafting time (and have completed one of the required checks for making an item). Checks for Poisoncraft do not need to be immediately consecutive. Failure means that no progress is made during that time. Once an item is started, even if no progress is made, the components reserved for that item can only be recovered via salvage.

If you fail three times in a row, all progress and materials are lost and can no longer be salvaged.

\section*{Poisoncraft Saving Throw}

When a poison requires a saving throw, the following is the formula for calculating the saving throw. The saving throw is calculated at the time of creation based on the creators attributes and proficiency, and doesn’t change once it is created.

Poison DC = 8 + your Poisoner’s Kit proficiency bonus + your Intelligence or Wisdom Modifier (your choice).

\section*{Poisons}

Given their insidious and deadly nature, poisons are illegal in most societies but are a favorite tool among assassins, drow, and other evil creatures.

Poisons come in the following four types.

\begin{minipage}{0.48\textwidth}
Contact. Contact poison can be smeared on an object and remains potent until it is touched or washed off. A creature that touches contact poison with exposed skin suffers its effects.

Ingested. A creature must swallow an entire dose of ingested poison to suffer its effects. The dose can be delivered in food or a liquid. You might decide that a partial dose has a reduced effect, such as allowing advantage on the saving throw or dealing only half damage on a failed save.
\end{minipage}\hfill
\begin{minipage}{0.48\textwidth}
Inhaled. These poisons are powders or gases that take effect when inhaled. Blowing the powder or releasing the gas subjects creatures in a 5-foot cube to its effect. The resulting cloud dissipates immediately afterward. Holding one’s breath is ineffective against inhaled poisons, as they affect nasal membranes, tear ducts, and other parts of the body.

Injury. Injury poison can be applied to weapons, ammunition, trap components, and other objects that deal piercing or slashing damage. An injury poison typically lasts 1 minute on a weapon, and lasts for up to 5 hits. A creature that takes piercing or slashing damage from an object coated with the poison is exposed to its effects.
\end{minipage}

\section{Poisoncraft}

\section*{Poison crafting Table}

\subsection*{Poisons}

\begin{longtable}{p{2.5cm}\toprule
|p{2.5cm}|p{2.5cm}|p{2.5cm}|p{2.5cm}|p{2.5cm}|p{2.5cm}|p{2.5cm}|p{2.5cm}|p{2.5cm}|p{2.5cm}|}
\midrule
\multicolumn{2}{c}{Name} & \multicolumn{4}{c}{Materials} & Crafting Time & Checks & Difficulty & Rarity & Value \\
\midrule
\multicolumn{2}{c}{Dizzying Touch (B)} & \multicolumn{4}{c}{1 common poisonous reagent 
1 common arcane essence 
1 glass vial} & 2 hours & 1 & DC 14 & common & 45 gp \\
\midrule
\multicolumn{2}{c}{Simple Contact Poison (B)} & \multicolumn{4}{c}{2 common poisonous reagents 
1 glass vial} & 2 hours & 1 & DC 14 & common & 40 gp \\
\midrule
\multicolumn{2}{c}{Simple Ingested Poison (B)} & \multicolumn{4}{c}{2 common poisonous reagents 
1 glass vial} & 2 hours & 1 & DC 12 & common & 35 gp \\
\midrule
\multicolumn{2}{c}{Simple Inhaled Poison (B)} & \multicolumn{4}{c}{2 common poisonous reagents 
1 common reactive reagent 
1 glass vial} & 2 hours & 1 & DC 14 & common & 60 gp \\
\midrule
\multicolumn{2}{c}{Simple Injury Poison (B)} & \multicolumn{4}{c}{2 common poisonous reagents 
1 glass vial} & 2 hours & 1 & DC 14 & common & 40 gp \\
\midrule
\multicolumn{2}{c}{Burning Wound (B)} & \multicolumn{4}{c}{1 uncommon poisonous reagent 
1 uncommon reactive reagent 
2 common reactive reagents 
1 glass vial} & 2 hours & 1 & DC 16 & uncommon & 150 gp \\
\midrule
\multicolumn{2}{c}{Old Reliable (B)} & \multicolumn{4}{c}{1 uncommon poisonous reagent 
1 common curative reagent 
1 glass vial} & 2 hours & 1 & DC 14 & uncommon & 70 gp \\
\midrule
\multicolumn{2}{c}{Potent Contact Poison (B)} & \multicolumn{4}{c}{2 uncommon poisonous reagents 
1 glass vial} & 2 hours & 1 & DC 15 & uncommon & 100 gp \\
\midrule
\multicolumn{2}{c}{Potent Ingested Poison (B)} & \multicolumn{4}{c}{2 uncommon poisonous reagents 
1 glass vial} & 2 hour & 1 & DC 14 & uncommon & 95 gp \\
\midrule
\multicolumn{2}{c}{Potent Inhaled Poison (B)} & \multicolumn{4}{c}{2 uncommon poisonous reagents 
1 uncommon reactive reagents 
1 glass vial} & 2 hours & 1 & DC 15 & uncommon & 140 gp \\
\midrule
\multicolumn{2}{c}{Potent Injury Poison (B)} & \multicolumn{4}{c}{2 uncommon poisonous reagents 
1 glass vial} & 2 hours & 1 & DC 15 & uncommon & 100 gp \\
\midrule
\multicolumn{2}{c}{Withering Soul (B)} & \multicolumn{4}{c}{1 uncommon poisonous reagent 
1 common arcane essence 
1 glass vial} & 2 hours & 1 & DC 14 & uncommon & 100 gp \\
\midrule
\multicolumn{2}{c}{Crawler Mucus} & \multicolumn{4}{c}{1 rare poisonous reagent 
1 glass vial} & 2 hours & 1 & DC 16 & rare & 250 gp \\
\midrule
\multicolumn{2}{c}{Essence of Ether} & \multicolumn{4}{c}{1 rare poisonous reagent 
1 glass vial} & 2 hours & 1 & DC 17 & rare & 270 gp \\
\midrule
\multicolumn{2}{c}{Fainting Fumes (B)} & \multicolumn{4}{c}{1 rare poisonous reagent 
1 uncommon reactive regeant 
2 uncommon poisonous reagents 
1 glass vial} & 4 hours & 2 & DC 16 & rare & 415 gp \\
\midrule
\multicolumn{2}{c}{Knockout Poison} & \multicolumn{4}{c}{3 rare poisonous reagents 
1 glass vial} & 4 hours & 2 & DC 17 & rare & 760 gp \\
\midrule
\multicolumn{2}{c}{Malice} & \multicolumn{4}{c}{1 rare poisonous reagent 
1 glass vial} & 2 hours & 1 & DC 16 & rare & 250 gp \\
\midrule
\multicolumn{2}{c}{Paralyzing Poison (B)} & \multicolumn{4}{c}{2 rare poisonous reagents 
2 uncommon poisonous reagents 
1 glass vial} & 4 hours & 2 & DC 16 & rare & 590 gp \\
\midrule
\multicolumn{2}{c}{Veins of Tar (B)} & \multicolumn{4}{c}{1 rare poisonous reagent 
1 uncommon reactive reagent 
1 uncommon supplies 
1 glass vial} & 4 hours & 2 & DC 15 & rare & 300 gp \\
\midrule
\multicolumn{2}{c}{Whispers of Madness (B)} & \multicolumn{4}{c}{1 rare poisonous reagent 
1 uncommon psionic essence 
1 glass vial} & 4 hours & 2 & DC 16 & rare & 450 gp \\
\midrule
\multicolumn{2}{c}{Endless Dreams (B)} & \multicolumn{4}{c}{1 legendary poisonous reagent 
2 very rare curative reagents 
1 crystal vial} & 8 hours & 4 & DC 20 & legendary & 11,000 gp \\
\midrule
\multicolumn{2}{c}{Grievous Injury Poison (B)} & \multicolumn{4}{c}{1 very rare poisonous reagent 
1 crystal vial} & 4 hours & 2 & DC 18 & very rare & 2,380 gp \\
\midrule
\multicolumn{2}{c}{Midnight Tears} & \multicolumn{4}{c}{1 very rare poisonous reagent 
1 crystal vial} & 4 hours & 2 & DC 16 & very rare & 2,300 \\
\midrule
\multicolumn{2}{c}{Slow Death (B)} & \multicolumn{4}{c}{2 very rare poisonous reagents 
1 very rare curative reagent 
1 crystal vial} & 6 hours & 3 & DC 18 & very rare & 6,800 gp \\
\midrule
\end{longtable}

\section*{Poisoncraft Recipes}

\begin{minipage}{0.48\textwidth}
\subsubsection*{Contact Poisons}

\begin{itemize}
  \item Crawler Mucus
  \item Dizzying Touch (B)
  \item Potent Contact Poison (B)
  \item Simple Contact Poison (B)
  \item Whispers of Madness (B)
\end{itemize}

\subsubsection*{Ingested Poisons}

\begin{itemize}
  \item Endless Dreams (B)
  \item Midnight Tears
  \item Potent Ingested Poison (B)
  \item Simple Ingested Poison (B)
  \item Slow Death (B)
  \item Veins of Tar (B)
\end{itemize}
\end{minipage}\hfill
\begin{minipage}{0.48\textwidth}
\subsubsection*{Inhaled Poisons}

\begin{itemize}
  \item Essence of Ether
  \item Fainting Fumes (B)
  \item Malice
  \item Potent Inhaled Poison (B)
  \item Simple Inhaled Poison (B)
\end{itemize}

\subsubsection*{Injury Poisons}

\begin{itemize}
  \item Burning Wound (B)
  \item Grievous Injury Poison (B)
  \item Knockout Poison
  \item Old Reliable (B)
  \item Paralyzing Poison (B)
  \item Potent Injury Poison (B)
  \item Simple Injury Poison (B)
\end{itemize}
\end{minipage}

\section{Poisoncraft}

\section*{Poisons}

\section*{Contact Poisons}

\begin{minipage}{0.48\textwidth}
\subsubsection*{Crawler Mucus}

Contact Poison, rare

Applied to an object. The first creature that touches must make a Constitution saving throw with a DC equal to the crafter's Poison DC or be Poisoned for 1 minute. The Poisoned creature is Paralyzed. The creature can repeat the saving throw at the end of each of its turns, Ending the effect on itself on a success, after which the poison is rubbed off.

\begin{itemize}
  \item Item: Crawler Mucus
\end{itemize}

\subsubsection*{Dizzying Touch}

Contact Poison, common

Applied to an object. The first creature that touches must make a Constitution saving throw with a DC equal to the crafter's Poison DC. On failure, the creature becomes poisoned for 1 minute. While poisoned in this way, they must succeed a Wisdom saving throw at the end of each of their turns or fall prone.

\begin{itemize}
  \item Item: Dizzying Touch
\end{itemize}
\end{minipage}\hfill
\begin{minipage}{0.48\textwidth}
\subsubsection*{Simple Contact Poison}

Contact Poison, common

Applied to an object. The first creature that touches must make a Constitution saving throw with a DC equal to the crafter's Poison DC or take 2d4 Poison damage, after which the poison is rubbed off.

\begin{itemize}
  \item Item: Simple Contact Poison (B)
\end{itemize}

\subsubsection*{Whispers of Madness}

Contact Poison, rare

Applied to an object. The first creature that touches must make a Constitution saving throw with a DC equal to the crafter's Poison DC. On failure, the character becomes poisoned for 10 minutes, and rolls on the short term madness table. They are under the effect of the rolled madness while poisoned.

\begin{itemize}
  \item Item: Whispers of Madness
\end{itemize}
\end{minipage}

% [Image Inserted Manually]

\subsection*{Potent Contact Poison}

Contact Poison, uncommon

Applied to an object. The first creature that touches that object after it is applied must make a Constitution saving throw with a DC equal to the crafter's Poison DC.

On failure, they take 4d4 Poison damage and become Poisoned for 1 hour. At the end of a poisoned creature’s turn, it can repeat the saving throw, ending the condition on success.

\begin{itemize}
  \item Item: Potent Contact Poison (B)
\end{itemize}

\section*{Ingested Poisons}

\begin{minipage}{0.48\textwidth}
\subsubsection*{Endless Dreams}

Ingested Poison, legendary

Applied to food or beverage. A creature that consumes this poison must make a Constitution saving throw with a DC equal to the crafter's Poison DC. On failure, the next time the creature falls asleep they enter endless slumber in stasis. They do not wake and can’t be roused by any normal means. A creature sleeping in this way doesn’t require food or drink, and doesn’t age.

This effect can only be ended by greater restoration cast at 7th level or higher or wish.

\begin{itemize}
  \item Item: Endless Dreams (B)
\end{itemize}

\subsubsection*{Midnight Tears}

Ingested Poison, uncommon

A creature that ingests this poison suffers no Effect until the stroke of midnight. If the poison has not been neutralized before then, the creature must succeed on a Constitution saving throw with a DC equal to the crafter's Poison DC, taking 31 (9d6) poison damage on a failed save, or half as much damage on a successful one.

\begin{itemize}
  \item Item: Midnight Tears
\end{itemize}

\subsubsection*{Potent Ingested Poison}

Ingested Poison, uncommon

A creature subjected to this poison must make a Constitution saving throw with a DC equal to the crafter's Poison DC. On failure, they take 3d6 Poison damage and suffer the poisoned condition for 1 hour.

\begin{itemize}
  \item Item: Potent Ingested Poison (B)
\end{itemize}
\end{minipage}\hfill
\begin{minipage}{0.48\textwidth}
\subsubsection*{Simple Ingested Poison}

Ingested Poison, common

Applied to food or beverage. A creature that consumes this poison must make a Constitution saving throw with a DC equal to the crafter's Poison DC. On failure, they take 2d6 Poison damage and suffer the poisoned condition for 1 hour.

\begin{itemize}
  \item Item: Simple Ingested Poison (B)
\end{itemize}

\subsubsection*{Slow Death}

Ingested Poison, very rare

Applied to food or beverage. A creature that consumes this poison begins to slowly die if they aren’t immune to poison. Each time that creature finishes a long rest, its hit point maximum is reduced by 1. If the creature’s hit point maximum hit is reduced to 0 by this effect, it dies. While its hit points are less than half of its original maximum, it is poisoned. This effect can be removed by effects that remove the poisoned condition and can only be identified by magic or a DC 15 Wisdom (Medicine) check.

If a creature is cured, it regains its lost hit points after it finishes a long rest.

\begin{itemize}
  \item Item: Slow Death (B)
\end{itemize}

\subsubsection*{Veins of Tar}

Ingested Poison, rare

Applied to food or beverage. A creature that consumes this poison must make a Constitution saving throw with a DC equal to the crafter's Poison DC. On failure, they are under the effect of the slow spell for 8 hours. This effect can be removed by effects that remove poison.

\begin{itemize}
  \item Item: Veins of Tar (B)
\end{itemize}
\end{minipage}

\section*{Inhaled Poisons}

\begin{minipage}{0.48\textwidth}
\subsubsection*{Essence of Ether}

Inhaled Poison, rare

When released, this poison fills a 5-foot radius around the source. You can accurately throw a vial of it (shattering on impact) at a point up to 30 feet away, or release it by other means. A creature subjected to this poison must make a Constitution saving throw with a DC equal to the crafter's Poison DC.

On failure, the creature becomes Poisoned for 8 hours. The Poisoned creature is Unconscious. The creature wakes up if it takes damage or if another creature takes an action to shake it awake.

If stored as powder, you can throw this powder 5 feet. If stored in a vial, you can throw 30 feet.

\begin{itemize}
  \item Item: Essence Of Ether
\end{itemize}

\subsubsection*{Fainting Fumes}

Inhaled Poison, rare

When released, this poison fills a 10-foot radius around the source. You can accurately throw a vial of it (shattering on impact) at a point up to 30 feet away, or release it by other means. A creature subjected to this poison must make a Constitution saving throw with a DC equal to the crafter's Poison DC. On failure, the target becomes poisoned for 1 hour. If the saving throw fails by 5 or more, the creature is also unconscious while poisoned in this way. The creature wakes up if it takes damage or if another creature takes an action to shake it awake.

\begin{itemize}
  \item Item: Fainting Fumes (B)
\end{itemize}

\subsubsection*{Malice}

Inhaled Poison, rare

When released (by throwing powder or breaking a vial of it), this poison affects a 5-foot radius from where it was released. A creature subjected to this poison must make a Constitution saving throw with a DC equal to the crafter's Poison DC.

On failure, the creature becomes Poisoned for 1 hour. The Poisoned creature is Blinded.

If stored as powder, you can throw this powder 5 feet. If stored in a vial, you can throw 20 feet.

\begin{itemize}
  \item Item: Malice
\end{itemize}
\end{minipage}\hfill
\begin{minipage}{0.48\textwidth}
\subsubsection*{Potent Inhaled Poison}

Inhaled Poison, uncommon

When released, this poison fills a 10-foot radius around the source. You can accurately throw a vial of it (shattering on impact) at a point up to 30 feet away, or release it by other means. A creature subjected to this poison must make a Constitution saving throw with a DC equal to the crafter's Poison DC.

On failure, they take 2d4 Poison damage and become Poisoned for 1 hour. At the end of a poisoned creature’s turn, it can repeat the saving throw, ending the condition on success.

This effect lingers in the area it was released for 1d4 rounds. A strong wind will clear away and disperse the poison. A creature that enters the area for the first time must save against the poison.

\begin{itemize}
  \item Item: Potent Inhaled Poison (B)
\end{itemize}

\subsubsection*{Simple Inhaled Poison}

Inhaled Poison, common

When released, this poison fills a 10-foot radius around the source. You can accurately throw a vial of it (shattering on impact) at a point up to 30 feet away or release it by other means. A creature subjected to this poison must make a Constitution saving throw with a DC equal to the crafter's Poison DC. On failure, they take 2d4 Poison damage.

\begin{itemize}
  \item Item: Simple Inhaled Poison (B)
\end{itemize}
\end{minipage}

\section*{Injury Poisons}

\begin{minipage}{0.48\textwidth}
\subsubsection*{Burning Wound}

Injury Poison, uncommon

Applied to a melee weapon or up to 5 pieces of ammunition. A creature subjected to this poison must make a Constitution saving throw with a DC equal to the crafter's Poison DC.

On failure, the target becomes poisoned for 1 minute. While poisoned this way, a creature takes 1d6 fire damage at the end of each of its turns, and any hit points regained is reduced by half. At the end of each of their turns, they can repeat the saving throw, ending the effect on success.

Once applied, the poison retains potency for 1 minute before drying, and wears off of a weapon after that weapon has delivered the effect 5 times.

\begin{itemize}
  \item Item: Burning Wound (B)
\end{itemize}

\subsubsection*{Grievious Injury Poison}

Injury Poison, very rare

Applied to a melee weapon or up to 5 pieces of ammunition.

A creature subjected to this poison must make a Constitution saving throw with a DC equal to the crafter's Poison DC, taking 6d6 poison damage on a failed save, or half as much damage on a successful save.

Once applied, the poison retains potency for 1 minute before drying, and wears off of a weapon after that weapon has delivered the effect 5 times.

\begin{itemize}
  \item Item: Grievous Injury Poison (B)
\end{itemize}

\subsubsection*{Knockout Poison}

Injury Poison, rare

Applied to a melee weapon or up to 5 pieces of ammunition. A creature subjected to this poison must make a Constitution saving throw with a DC equal to the crafter's Poison DC.

On failure, the target becomes poisoned for 1 hour. If the saving throw fails by 5 or more, the creature is also unconscious while poisoned in this way. The creature wakes up if it takes damage or if another creature takes an action to shake it awake.

Once applied, the poison retains potency for 1 minute before drying, and wears off of a weapon after that weapon has delivered the effect 5 times.

\begin{itemize}
  \item Item: Knockout Poison
\end{itemize}

\subsubsection*{Old Reliable}

Injury Poison, uncommon

Applied to a melee weapon or up to 5 pieces of ammunition. This poison is exceptionally durable, lasting on a weapon for 1 hour and an unlimited number of hits during that time. On hit, weapons with this poison applied deal 1d4 additional poison damage.

\begin{itemize}
  \item Item: Old Reliable (B)
\end{itemize}
\end{minipage}\hfill
\begin{minipage}{0.48\textwidth}
\subsubsection*{Paralyzing Poison}

Injury Poison, rare

Applied to a melee weapon or up to 5 pieces of ammunition. A creature subjected to this poison must make a Constitution saving throw with a DC equal to the crafter's Poison DC.

On failure, the target becomes poisoned for 1 minute. A creature is paralyzed while poisoned this way. At the end of each of the creature’s turns, it can repeat the saving throw, ending the effect on success.

Once applied, the poison retains potency for 1 minute before drying, and wears off of a weapon after that weapon has delivered the effect 5 times.

\begin{itemize}
  \item Item: Paralyzing Poison (B)
\end{itemize}

\subsubsection*{Potent Injury Poison}

Injury Poison, uncommon

Applied to a melee weapon or up to 5 pieces of ammunition. A creature subjected to this poison must make a Constitution saving throw with a DC equal to the crafter's Poison DC.

On failure, they take 2d6 Poison damage and become Poisoned for 1 hour. At the end of a poisoned creature’s turn, it can repeat the saving throw, ending the condition on success.

Once applied, the poison retains potency for 1 minute before drying, and wears off of a weapon after that weapon has delivered the effect 5 times.

\begin{itemize}
  \item Item: Potent Injury Poison (B)
\end{itemize}

\subsubsection*{Simple Injury Poison}

Injury Poison, common

Applied to a melee weapon or up to 5 pieces of ammunition. A creature subjected to this poison must succeed a Constitution saving throw with a DC equal to the crafter's Poison DC or take 2d4 Poison damage.

Once applied, the poison retains potency for 1 minute before drying, and wears off of a weapon after that weapon has delivered the effect 5 times.

\begin{itemize}
  \item Item: Simple Injury Poison (B)
\end{itemize}

\subsubsection*{Withering Soul}

Injury Poison, uncommon

Applied to a melee weapon or up to 5 pieces of ammunition. A creature subjected to this poison must make a Constitution saving throw with a DC equal to the crafter's Poison DC.

On failure, they become Poisoned for 1 minute. While poisoned in this way, a creatures takes 1d4 necrotic damage at the start of their turn, and are under the effect of bane. At the end of a poisoned creature’s turn, it can repeat the saving throw, ending the condition on success.

Once applied, the poison retains potency for 1 minute before drying, and wears off of a weapon after that weapon has delivered the effect 5 times.

\begin{itemize}
  \item Item: Withering Soul (B)
\end{itemize}
\end{minipage}

\section{Blacksmithing}

% [Image Inserted Manually]

\section*{Blacksmithing}

Blacksmithing is a popular professional interest of two sorts of adventurers: those that want to hit things with heavy metal objects, and those that want a heavy metal object between them and the thing hitting them.

While often relying on the town blacksmith to do their work for them is a fine option, rolling up your sleeves and doing the work yourself can allow you to express your creativity... and may save you a few coins in the process.

Blacksmithing is slow hard work, but has a higher tolerance for failure than most, and is more dependent on knowing your material, as the templates you work from tend to be common across many of them.

\section*{Related Tool \& Ability Score}

Blacksmithing works using blacksmith’s tools. Attempting to craft an item without blacksmith’s tools will often be impossible, though a GM may let you use makeshift tools to make a check with disadvantage. Proficiency in blacksmith’s tools allows you to add your proficiency bonus to any blacksmithing check.

While Blacksmiths can benefit from their skills in small ways such as sharpening their weapons and retrofitting their gear on the go, many of their crafting options require a fully equipped Forge; a fully equipped Forge entails forge, anvil, and blacksmith’s tools.

\section*{Quick Reference}

While each step will go into more depth, the quick reference allows you to at a glance follow the steps to make a blacksmith item in its most basic form:

\begin{itemize}
  \item Select the item that you would like to craft from any of the Blacksmithing Crafting Tables.
  \item Acquire the items listed in the materials column for that item.
  \item Use your blacksmith’s tools to craft the option using the number of hours listed in the Crafting Time column.
  \item For every 2 hours, make a crafting roll of 1d20 + your Strength modifier + your proficiency bonus with blacksmith’s tools.
  \item On success, you mark 2 hours of completed time. Once the completed time is equal to the crafting time, the item is complete.
  \item On failure, the crafting time is lost, and no progress has been made during the 2 hours. If you fail 3 times in a row, the crafting is a failure, and all materials are lost.
\end{itemize}

Magical Forges

The worlds of the planar multiverse are a fantastical place with many wonders. Sometimes you may find locations that have been constructed in such a way as to leverage powerful primal powers in the forging technique: for example, a forge at the heart of a volcano or atop an ever-frozen glacier, which might imbue items crafted there with special properties.

\section*{Crafting Roll}

Putting that together means that when you would like to smith an item, your crafting roll is as follows:

Blacksmithing Modifier = your Blacksmith’s Tools proficiency bonus + your Strength modifier

\subsection*{Success and Failure}

After you make a crafting roll, if you succeed, you make 2 hours of progress toward the total crafting time (and have completed one of the required checks for making an item).

Checks for Blacksmithing do not need to be immediately consecutive. If you fail three times in a row, all progress and materials are lost and can no longer be salvaged. Failure means that no progress is made during that time.

Once an item is started, even if no progress is made, the components reserved for that item can only be recovered via salvage.

\subsection*{Blacksmithing Materials}

\begin{minipage}{0.48\textwidth}
Metals

\begin{tabularx}{\textwidth}\toprule
{}XX}
\midrule
Materials & Rarity & Price \\
\midrule
Metal Scraps & Trivial & 1 sp \\
\midrule
Silver Scraps & Trivial & 1 sp \\
\midrule
Gold Scraps & Common & 1 gp \\
\midrule
Iron Ingot & Common & 1 gp \\
\midrule
Steel Chain (2 ft) & Common & 1 gp \\
\midrule
Steel Ingot & Common & 2 gp \\
\midrule
Mithril Ingot & Uncommon & 30 gp \\
\midrule
Adamant Ingot & Uncommon & 40 gp \\
\midrule
Adamantine Ingot & Uncommon & 60 gp \\
\midrule
Icesteel Ingot & Uncommon & 60 gp \\
\midrule
Darksteel Ingot & Uncommon & 60 gp \\
\midrule
Firesteel Ingot & Uncommon & 60 gp \\
\midrule
\end{tabularx}
\end{minipage}\hfill
\begin{minipage}{0.48\textwidth}
“Ice/Dark/Fire” Steel Ingots

The names for Icesteel, Darksteel, and Firesteel are intentionally generic to better map to unique metals that might be present in your setting and may have other names. Adamantine and Mithril tend to be widespread (and consequently open-source metals), but other more unique metals may vary based on setting.
\end{minipage}

\section*{Maintenance \& Modifications}

While the primary purpose of Blacksmithing is to forge armor and weapons from metal, for an adventurer such events are important milestones that generally will not occur every day. The following are some tasks that require proficiency with Blacksmith’s Tools that provide a more day-to-day utility to the proficiency, giving them minor ways to enhance or adapt their gear.

These are minor crafts that can be completed in 2 hours (or as one camp action when using the Camp Actions rules) with the expenditure of 5 gp worth of materials. They can be done as part of a long rest but have limitations the normally crafted items do not (such as a maximum stockpile of minor crafts).

The following are “minor crafting options” for Blacksmiths:

\begin{minipage}{0.48\textwidth}
\subsubsection*{Maintain Gear}

One of the perks of having a blacksmith in the field is their ability to keep gear in its best condition, giving you an edge (sometimes literally) in the quality of your gear and weapons. Over the course of 2 hours, a Blacksmith can maintain a number of weapons or sets of armor equal to their proficiency bonus, granting each weapon or armor maintained a special d6 Quality Die.

For a weapon, this can be rolled and added to an attack or damage roll, representing a case where the perfect state of the gear turned a miss into a hit or dealt a bit of extra damage. For a set of armor, the die can be rolled when hit by an attack, and the damage taken from that attack can be reduced by that amount.

Rolling this die doesn’t require an action, but once rolled it is spent and can’t be regained until the blacksmith maintains that armor or weapon again.
\end{minipage}\hfill
\begin{minipage}{0.48\textwidth}
\subsubsection*{Modify Armor}

While the field crafting of armor is often not possible, you can make smaller adjustments on the go. Over the course of 2 hours, you can turn a set of plate mail into a half plate or a breastplate, refit a set of heavy or medium armor to fit another user that is equal in size or smaller than the original user.

\subsubsection*{Modify Weapon}

Every adventure has slightly different preferences in their gear, and your skills allow you to make slight modifications to nonmagical weapons made of metal. These modifications take 2 hours, require a heat source, and require you to pass a DC 14 blacksmithing tool’s check (on failure, the weapon is damaged and has a −1 penalty to its attack rolls until fixed). You can perform one of the following modifications:
\end{minipage}

\begin{minipage}{0.48\textwidth}
Note: Imperfect Results

Using this method will make some... bad weapons. That is largely intentional. If you want to make a more functional weapon, you can make it from scratch using the Weapon Template. This represents quick hacks to an existing weapon.
\end{minipage}\hfill
\begin{minipage}{0.48\textwidth}
\begin{itemize}
  \item You can weight a weapon, giving it the heavy property. If it does not already have the two-handed property, it gains the two-handed property. • You can remove the heavy property from a weapon, reducing its damage dice by d2.
  \item You can add the light property to a weapon without the heavy property, reducing its damage dice by d2.
  \item You can silver the weapon (requires 5 silver scraps, doubled for two handed weapons).
\end{itemize}
\end{minipage}

\subsection*{Repair Gear}

Sometime in the course of adventuring, weapons or armor will become severely damaged, suffering a penalty to its attack rolls or AC. Over the course of 2 hours, you can repair this damage, though at the discretion of the GM you may need other materials to perform this task if it is heavily damaged. Weapons that are entirely broken (such as a snapped sword) are generally beyond simple repair.

\section{Blacksmithing}

\section*{Blacksmithing Crafting Tables}

\section*{Weapons}

\subsection*{Simple Weapons}

\begin{tabularx}{\textwidth}\toprule
{}XXXXXXXX}
\midrule
\multicolumn{2}{c}{Name} & \multicolumn{2}{c}{Materials} & Crafting Time & Checks & Difficulty & Rarity & Value \\
\midrule
\multicolumn{2}{c}{Dagger} & \multicolumn{2}{c}{0.5 ingots} & 2 hours & 1 & DC 10 & Common & 3 gp \\
\midrule
\multicolumn{2}{c}{Handaxe} & \multicolumn{2}{c}{1 ingot 
1 short haft} & 2 hours & 1 & DC 11 & Common & 5 gp \\
\midrule
\multicolumn{2}{c}{Light Hammer} & \multicolumn{2}{c}{1 ingot 
1 short haft} & 2 hours & 1 & DC 9 & Common & 3 gp \\
\midrule
\multicolumn{2}{c}{Javelin} & \multicolumn{2}{c}{1 ingot 
1 short haft} & 2 hours & 1 & DC 9 & Common & 3 gp \\
\midrule
\multicolumn{2}{c}{Mace} & \multicolumn{2}{c}{2 ingots 
1 short haft} & 2 hours & 1 & DC 9 & Common & 5 gp \\
\midrule
\multicolumn{2}{c}{Sickle} & \multicolumn{2}{c}{1 ingot 
1 short haft} & 2 hours & 1 & DC 10 & Common & 4 gp \\
\midrule
\multicolumn{2}{c}{Spear} & \multicolumn{2}{c}{1 ingot 
1 long haft} & 2 hours & 1 & DC 9 & Common & 3 gp \\
\midrule
\end{tabularx}

\subsection*{Martial Weapons}

\begin{tabularx}{\textwidth}\toprule
{}XXXXXXXX}
\midrule
\multicolumn{2}{c}{Name} & \multicolumn{2}{c}{Materials} & Crafting Time & Checks & Difficulty & Rarity & Value \\
\midrule
\multicolumn{2}{c}{Battleaxe} & \multicolumn{2}{c}{3 ingots 
1 short haft} & 4 hours & 2 & DC 12 & Common & 10 gp \\
\midrule
\multicolumn{2}{c}{Flail} & \multicolumn{2}{c}{2 ingots 
1 short haft 
1 steel chain} & 4 hours & 2 & DC 12 & Common & 10 gp \\
\midrule
\multicolumn{2}{c}{Glaive} & \multicolumn{2}{c}{4 ingots 
1 long haft} & 4 hours & 2 & DC 13 & Common & 20 gp \\
\midrule
\multicolumn{2}{c}{Greataxe} & \multicolumn{2}{c}{8 ingots 
1 short haft} & 4 hours & 2 & DC 13 & Common & 30 gp \\
\midrule
\multicolumn{2}{c}{Greatsword} & \multicolumn{2}{c}{10 ingots} & 4 hours & 2 & DC 15 & Common & 50 gp \\
\midrule
\multicolumn{2}{c}{Halberd} & \multicolumn{2}{c}{4 ingots 
1 long haft} & 4 hours & 2 & DC 13 & Common & 20 gp \\
\midrule
\multicolumn{2}{c}{Longsword} & \multicolumn{2}{c}{4 ingots} & 4 hours & 2 & DC 14 & Common & 15 gp \\
\midrule
\multicolumn{2}{c}{Maul} & \multicolumn{2}{c}{8 ingots 
1 short haft} & 4 hours & 2 & DC 12 & Common & 25 gp \\
\midrule
\multicolumn{2}{c}{Morningstar} & \multicolumn{2}{c}{4 ingots 
1 short haft} & 4 hours & 2 & DC 14 & Common & 15 gp \\
\midrule
\multicolumn{2}{c}{Pike} & \multicolumn{2}{c}{3 ingots 
1 long haft} & 4 hours & 2 & DC 12 & Common & 15 gp \\
\midrule
\multicolumn{2}{c}{Rapier} & \multicolumn{2}{c}{1 ingot} & 4 hours & 2 & DC 15 & Common & 25 gp \\
\midrule
\multicolumn{2}{c}{Scimitar} & \multicolumn{2}{c}{2 ingots} & 4 hours & 2 & DC 14 & Common & 25 gp \\
\midrule
\multicolumn{2}{c}{Shortsword} & \multicolumn{2}{c}{2 ingots} & 4 hours & 2 & DC 12 & Common & 10 gp \\
\midrule
\multicolumn{2}{c}{War Pick} & \multicolumn{2}{c}{2 ingots 
1 short haft} & 4 hours & 2 & DC 12 & Common & 10 gp \\
\midrule
\multicolumn{2}{c}{Warhammer} & \multicolumn{2}{c}{4 ingots 
1 short haft} & 4 hours & 2 & DC 12 & Common & 15 gp \\
\midrule
\end{tabularx}

\section*{Armor}

\subsection*{Armor}

\begin{tabularx}{\textwidth}\toprule
{}XXXXXXXX}
\midrule
\multicolumn{2}{c}{Name} & \multicolumn{2}{c}{Materials} & Crafting Time & Checks & Difficulty & Rarity & Value \\
\midrule
\multicolumn{2}{c}{Breastplate} & \multicolumn{2}{c}{10 ingots} & 16 hours & 8 & DC 16 & common & 400 gp \\
\midrule
\multicolumn{2}{c}{Chain Mail} & \multicolumn{2}{c}{9 ingots 
1 armor padding} & 14 hours & 7 & DC 13 & common & 75 gp \\
\midrule
\multicolumn{2}{c}{Chain Shirt} & \multicolumn{2}{c}{5 ingots} & 14 hours & 7 & DC 13 & common & 50 gp \\
\midrule
\multicolumn{2}{c}{Half Plate} & \multicolumn{2}{c}{16 ingots 
1 armor padding} & 28 hours & 14 & DC 17 & common & 750 gp \\
\midrule
\multicolumn{2}{c}{Plate} & \multicolumn{2}{c}{30 ingots 
1 armor padding} & 56 hours & 28 & DC 17 & common & 1,500 gp \\
\midrule
\multicolumn{2}{c}{Ring Mail} & \multicolumn{2}{c}{4 ingots 
1 armor padding} & 10 hours & 5 & DC 11 & common & 30 gp \\
\midrule
\multicolumn{2}{c}{Scale Mail} & \multicolumn{2}{c}{8 ingots 
1 armor padding} & 14 hours & 7 & DC 12 & common & 50 gp \\
\midrule
\multicolumn{2}{c}{Splint} & \multicolumn{2}{c}{12 ingots 
1 armor padding} & 28 hours & 14 & DC 14 & common & 200 gp \\
\midrule
\end{tabularx}

\section*{Defensive Items}

\subsection*{Defensive Items}

\begin{tabularx}{\textwidth}\toprule
{}XXXXXXXX}
\midrule
\multicolumn{2}{c}{Name} & \multicolumn{2}{c}{Materials} & Crafting Time & Checks & Difficulty & Rarity & Value \\
\midrule
\multicolumn{2}{c}{Adamantine Bracers (B)} & \multicolumn{2}{c}{2 adamantine ingots} & 4 hours & 2 & DC 20 & uncommon & 130 gp \\
\midrule
\multicolumn{2}{c}{Bracers (B)} & \multicolumn{2}{c}{2 ingots} & 4 hours & 2 & DC 13 & common & 15 gp \\
\midrule
\multicolumn{2}{c}{Shield} & \multicolumn{2}{c}{2 ingots} & 6 hours & 3 & DC 10 & common & 10 gp \\
\midrule
\multicolumn{2}{c}{Spiked Shield (B)} & \multicolumn{2}{c}{3 ingots} & 8 hours & 4 & DC 14 & common & 40 gp \\
\midrule
\multicolumn{2}{c}{Tower Shield (B)} & \multicolumn{2}{c}{8 ingots} & 10 hours & 5 & DC 13 & common & 50 gp \\
\midrule
\end{tabularx}

Shields and bracers can’t benefit from armor crafting modifications.

\section*{Miscellaneous Gear}

\subsection*{Miscellaneous Gear}

\begin{tabularx}{\textwidth}\toprule
{}XXXXXXXX}
\midrule
\multicolumn{2}{c}{Name} & \multicolumn{2}{c}{Materials} & Crafting Time & Checks & Difficulty & Rarity & Value \\
\midrule
\multicolumn{2}{c}{Ball Bearings} & \multicolumn{2}{c}{1 iron ingot} & 2 hours & 1 & DC 8 & common & 1 gp \\
\midrule
\multicolumn{2}{c}{Bell} & \multicolumn{2}{c}{2 metal scraps} & 2 hours & 1 & DC 9 & common & 1 gp \\
\midrule
\multicolumn{2}{c}{Bucket} & \multicolumn{2}{c}{4 metal scraps} & 2 hours & 1 & DC 5 & common & 3 sp \\
\midrule
\multicolumn{2}{c}{Caltrops} & \multicolumn{2}{c}{1 iron ingot} & 2 hours & 1 & DC 8 & common & 1 gp \\
\midrule
\multicolumn{2}{c}{Helmet*} & \multicolumn{2}{c}{2 ingots} & 8 hours & 4 & DC 12 & common & 12 gp \\
\midrule
\multicolumn{2}{c}{Horseshoes (4)} & \multicolumn{2}{c}{2 ingots} & 4 hours & 2 & DC 10 & common & 5 gp \\
\midrule
\multicolumn{2}{c}{Iron Pot} & \multicolumn{2}{c}{1 iron ingot} & 2 hours & 1 & DC 9 & common & 2 gp \\
\midrule
\multicolumn{2}{c}{Iron Spikes (10)} & \multicolumn{2}{c}{2 iron ingots} & 2 hours & 1 & DC 8 & common & 1 gp \\
\midrule
\multicolumn{2}{c}{Manacles} & \multicolumn{2}{c}{2 ingots 
1 lock 
1 chain (2 feet)} & 2 hours & 1 & DC 12 & common & 2 gp \\
\midrule
\multicolumn{2}{c}{Pitons (20)} & \multicolumn{2}{c}{1 iron ingot} & 2 hours & 1 & DC 8 & common & 1 gp \\
\midrule
\multicolumn{2}{c}{Ring} & \multicolumn{2}{c}{1 ingot} & 2 hours & 1 & DC 8 & common & 2 gp \\
\midrule
\multicolumn{2}{c}{Steel Chain (2 ft)} & \multicolumn{2}{c}{1 ingot} & 4 hours & 2 & DC 10 & common & 3 gp \\
\midrule
\end{tabularx}

*The effects of a helmet are included in armor, this entry is largely to allow for standalone helmet creation for enchanters.

\section*{Tools}

\subsection*{Tools}

\begin{tabularx}{\textwidth}\toprule
{}XXXXXXXX}
\midrule
\multicolumn{2}{c}{Name} & \multicolumn{2}{c}{Materials} & Crafting Time & Checks & Difficulty & Rarity & Value \\
\midrule
\multicolumn{2}{c}{Blacksmith’s Tools} & \multicolumn{2}{c}{4 ingots 
2 parts} & 8 hours & 4 & DC 11 & common & 20 gp \\
\midrule
\multicolumn{2}{c}{Carpenter’s Tools} & \multicolumn{2}{c}{2 ingots 
1 parts} & 6 hours & 3 & DC 12 & common & 8 gp \\
\midrule
\end{tabularx}

\section*{Components And Materials}

\subsection*{Components and Materials}

\begin{tabularx}{\textwidth}\toprule
{}XXXXXXXX}
\midrule
\multicolumn{2}{c}{Name} & \multicolumn{2}{c}{Materials} & Crafting Time & Checks & Difficulty & Rarity & Value \\
\midrule
\multicolumn{2}{c}{Adamant Ingot (S)} & \multicolumn{2}{c}{1 admant ore} & 2 hours & 1 & DC 16 & uncommon & 40 gp \\
\midrule
\multicolumn{2}{c}{Adamantine Ingot} & \multicolumn{2}{c}{1 steel ingot 
1 adamant ingot requires magical forge} & 2 hours & 1 & DC 15 & uncommon & 60 gp \\
\midrule
\multicolumn{2}{c}{Buckle (5)} & \multicolumn{2}{c}{5 metal scraps} & 2 hours & 1 & DC 8 & common & 1 gp \\
\midrule
\multicolumn{2}{c}{Darksteel Ingot (S)} & \multicolumn{2}{c}{1 darksteel ore 
1 common arcane essence} & 2 hours & 1 & DC 16 & uncommon & 60 gp \\
\midrule
\multicolumn{2}{c}{Fancy Parts} & \multicolumn{2}{c}{1 ingot} & 4 hours & 2 & DC 12 & common & 10 gp \\
\midrule
\multicolumn{2}{c}{Firesteel Ingot (S)} & \multicolumn{2}{c}{1 firesteel ore 
1 common primal essence} & 2 hours & 1 & DC 16 & uncommon & 60 gp \\
\midrule
\multicolumn{2}{c}{Gold Ingot} & \multicolumn{2}{c}{20 gold scraps} & 2 hours & 1 & DC 5 & common & 20 gp \\
\midrule
\multicolumn{2}{c}{Gold Scraps (20)} & \multicolumn{2}{c}{1 gold ingot} & 2 hours & 1 & DC 5 & common & 20 gp \\
\midrule
\multicolumn{2}{c}{Icesteel Ingot (S)} & \multicolumn{2}{c}{1 icesteel ore 
1 common primal essence} & 2 hours & 1 & DC 16 & uncommon & 60 gp \\
\midrule
\multicolumn{2}{c}{Ingot} & \multicolumn{2}{c}{20 metal scraps} & 2 hours & 1 & DC 5 & common & 2 gp \\
\midrule
\multicolumn{2}{c}{Iron Ingots (10) (S)} & \multicolumn{2}{c}{10 iron ore} & 4 hours & 2 & DC 5 & common & 10 gp \\
\midrule
\multicolumn{2}{c}{Metal Scraps (20)} & \multicolumn{2}{c}{1 ingot} & 2 hours & 1 & DC 5 & common & 2 gp \\
\midrule
\multicolumn{2}{c}{Mithral Ingot (S)} & \multicolumn{2}{c}{1 mithral ore} & 2 hours & 1 & DC 15 & uncommon & 30 gp \\
\midrule
\multicolumn{2}{c}{Parts} & \multicolumn{2}{c}{5 metal scraps} & 4 hours & 2 & DC 9 & common & 2 gp \\
\midrule
\multicolumn{2}{c}{Silver Ingot} & \multicolumn{2}{c}{20 silver scraps} & 2 hours & 1 & DC 5 & common & 2 gp \\
\midrule
\multicolumn{2}{c}{Silver Scraps (20)} & \multicolumn{2}{c}{1 silver ingot} & 2 hours & 1 & DC 5 & common & 2 gp \\
\midrule
\multicolumn{2}{c}{Steel Ingots (10) (S)} & \multicolumn{2}{c}{10 iron ore 
1 supplies} & 4 hours & 2 & DC 8 & common & 20 gp \\
\midrule
\end{tabularx}

(S) Smelting ore requires specialized facilities. This can usually be accomplished in a fully equipped smithy but consult your GM for where it might be appropriate. Adventurers rarely smelt their own ore; these are included primarily for informational purposes. Smelting magical ores may require more specialized facilities or locations.

\section*{Firearms*}

\subsection*{Firearms*}

\begin{tabularx}{\textwidth}\toprule
{}XXXXXXXX}
\midrule
\multicolumn{2}{c}{Name} & \multicolumn{2}{c}{Materials} & Crafting Time & Checks & Difficulty & Rarity & Value \\
\midrule
\multicolumn{2}{c}{Firearm Ammunition (20)} & \multicolumn{2}{c}{2 lead ingots 
1 packet of blasting powder} & 4 hours & 2 & DC 15 & uncommon & 80 gp \\
\midrule
\multicolumn{2}{c}{Musket} & \multicolumn{2}{c}{6 ingots 
4 parts 
2 fancy parts} & 16 hours & 8 & DC 17 & uncommon & 400 gp \\
\midrule
\multicolumn{2}{c}{Pistol} & \multicolumn{2}{c}{3 ingots 
4 parts 
2 fancy parts} & 16 hours & 8 & DC 16 & uncommon & 250 gp \\
\midrule
\multicolumn{2}{c}{Shotgun} & \multicolumn{2}{c}{8 ingots 
4 parts 
2 fancy parts 
2 esoteric parts} & 32 hours & 16 & DC 19 & uncommon & 2,425 gp \\
\midrule
\multicolumn{2}{c}{Thunder Cannon (B)} & \multicolumn{2}{c}{6 ingots 
2 uncommon primal essence 
2 uncommon arcane essence 
4 parts 
2 fancy parts 
1 esoteric part} & 8 hours & 4 & DC 17 & uncommon & 1,000 gp \\
\midrule
\multicolumn{2}{c}{Thunder Cannon Ammunition (B) (10)} & \multicolumn{2}{c}{2 ingots} & 2 hours & 1 & DC 15 & uncommon & 20 gp \\
\midrule
\end{tabularx}

*Firearms and Thunder Cannons are not found in all settings. Consult your GM.

\section*{Blacksmithing Recipes}

\begin{minipage}{0.48\textwidth}
\subsubsection*{Armor}

\begin{itemize}
  \item Breastplate
  \item Chain Mail
  \item Chain Shirt
  \item Half Plate
  \item Plate
  \item Ring Mail
  \item Scale Mail
  \item Splint
\end{itemize}

\subsubsection*{Defensive Items}

\begin{itemize}
  \item Adamantine Bracers
  \item Bracers
  \item Shield
  \item Spiked Shield
  \item Tower Shield
\end{itemize}

\subsubsection*{Martial Weapons}

\begin{itemize}
  \item Battleaxe
  \item Flail
  \item Glaive
  \item Greataxe
  \item Greatsword
  \item Halberd
  \item Longsword
  \item Maul
  \item Morning Star
  \item Pike
  \item Rapier
  \item Scimitar
  \item Shortsword
  \item War Pick
  \item Warhammer
\end{itemize}

\subsubsection*{Simple Weapons}

\begin{itemize}
  \item Dagger
  \item Handaxe
  \item Javelin
  \item Light Hammer
  \item Mace
  \item Sickle
  \item Spear
\end{itemize}
\end{minipage}\hfill
\begin{minipage}{0.48\textwidth}
\subsubsection*{Components \& Materials}

\begin{itemize}
  \item Adamant Ingot
  \item Adamantine Ingot
  \item Buckle (5)
  \item Darksteel Ingot
  \item Fancy Parts
  \item Firesteel Ingot
  \item Gold Ingot
  \item Gold Scraps (20)
  \item Icesteel Ingot
  \item Ingot
  \item Iron Ingots (10)
  \item Metal Scraps (20)
  \item Mithral Ingot
  \item Parts
  \item Silver Ingot
  \item Silver Scraps (20)
  \item Steel Ingots (10)
\end{itemize}

\subsubsection*{Firearms}

\begin{itemize}
  \item Firearm Ammunition (20)
  \item Musket
  \item Pistol
  \item Shotgun
  \item Thunder Cannon
  \item Thunder Cannon Ammunition (10)
\end{itemize}

\subsubsection*{Miscellaneous Gear}

\begin{itemize}
  \item Ball Bearings
  \item Bell
  \item Bucket
  \item Caltrops
  \item Helmet
  \item Horseshoes (4)
  \item Iron Pot
  \item Iron Spikes (10)
  \item Manacles
  \item Pitons (20)
  \item Ring
  \item Steel Chain (2 ft.)
\end{itemize}

\subsubsection*{Tools}

\begin{itemize}
  \item Blacksmith's Tools
  \item Carpenter's Tools
\end{itemize}
\end{minipage}

\section{Blacksmithing}

\section*{Custom Weapon Guide}

At first glance, it seems that the weapon selection in 5e is quite limited, but with a little knowledge of the system, you can largely expose that template that builds those weapons, and from there, well, the opportunities are limitless! When you would like to craft a template weapon, just follow the steps below:

\section*{Weapon Creation Template}

To create a weapon go through five steps to determine the final damage and properties of the weapon. Adding a d2 means increasing the die by one size (e.g. a d6 + d2 = d8) and the converse for subtracting a d2.

\subsection*{Step 1: Select One Of...}

\begin{tabularx}{\textwidth}\toprule
{}XXXX}
\midrule
Property & Base Damage Die & Crafting Modifier & Material Modifier & Notes \\
\midrule
Simple & d6 & 12 Base DC & 1 ingot & — \\
\midrule
Martial & d8 & 15 Base DC & 3 ingots & Becomes a martial weapon \\
\midrule
\end{tabularx}

\subsection*{Step 2: Select One Of...}

\begin{tabularx}{\textwidth}\toprule
{}XXXX}
\midrule
Property & Weapon Modifier & Crafting Modifier & Material Modifier & Notes \\
\midrule
Light & -d2 & +1 Base DC & -1 ingot & — \\
\midrule
None & — & — & — & — \\
\midrule
Versatile & — & +1 Base DC & +1 ingot & +d2 when wielded with two hands \\
\midrule
Two-Handed & +d2 & — & 2x ingots & — \\
\midrule
\end{tabularx}

\subsection*{Step 3: Select All That Apply...}

\begin{tabularx}{\textwidth}\toprule
{}XXXX}
\midrule
Property & Weapon Modifier & Crafting Modifier & Material Modifier & Notes \\
\midrule
Reach & -d2 & +2 Base DC & -1 ingot
 +1 long haft & — \\
\midrule
Finesse & -d2 & +3 Base DC & -1 ingot & No damage die cost if the weapon is light or has no other properties \\
\midrule
Thrown & — & +2 Base DC & — & — \\
\midrule
Heavy & +d2 & +1 Base DC & +4 ingots & Requires two-handed \\
\midrule
\end{tabularx}

\subsection*{Step 4: Set Damage Die/Dice...}

You can divide your damage die into smaller dice that equal the same total. For example, a d12 can become 2d6 or be reduced again to 3d4. Each time you do this, the crafting Base DC increases by +1. You can’t divide dice to be lower than d4.

\subsection*{Step 5: Select Damage Type...}

\begin{tabularx}{\textwidth}\toprule
{}X}
\midrule
Type & Effect \\
\midrule
Slashing & Deals Slashing Damage \\
\midrule
Piercing & Deals Piercing Damage \\
\midrule
Bludgeoning & Deals Bludgeoning Damage \\
\midrule
\end{tabularx}

\begin{minipage}{0.48\textwidth}
\subsubsection*{Bonus Step: Modifiers and Materials}

You can additionally add Material Modifiers and Crafting Modifiers to template weapons. These modifiers may add additional properties as listed by the modifier, and their difficult modifier is incorporated into the weapon.
\end{minipage}\hfill
\begin{minipage}{0.48\textwidth}
Notes:

\begin{itemize}
  \item Thrown can be ranged weapons instead of melee weapons. (Example: Dart)
  \item The GM can waive the restriction on heavy property requiring two-handed property but should be aware it opens the door to combining feats for great weapons with the use of shields.
  \item Add one short haft for axes, maces or similar.
  \item Weapons made of metal require a minimum of half an ingot (like a dagger), even if the table would reduce them zero.
\end{itemize}
\end{minipage}

\section*{Example Template Weapons}

\subsection*{Simple Weapons}

\begin{tabularx}{\textwidth}\toprule
{}XXXX}
\midrule
Weapon & Cost & Damage & Weight & Properties \\
\midrule
Brass Knuckles (B) & 2 sp & 1d4 bludgeoning & 2 lbs. & Light \\
\midrule
Chain (B) & 5 gp & 1d4 bludgeoning & 10 lbs. & Reach \\
\midrule
Finesse Spear (B) & 3 gp & 1d4 piercing & 2 lbs. & Finesse, Versatile (1d6) \\
\midrule
Heavy Greatclub (B) & 3 gp & 1d10 bludgeoning & 15 lbs. & Two-handed, Heavy \\
\midrule
Sturdy 10-Foot Pole (B) & 1 sp & 1d6 bludgeoning & 5 lbs. & Reach, Two-handed \\
\midrule
\end{tabularx}

\subsection*{Martial Weapons}

\begin{tabularx}{\textwidth}\toprule
{}XXXX}
\midrule
Weapon & Cost & Damage & Weight & Properties \\
\midrule
Broadsword (B) & 8 gp & 2d4 slashing & 3 lbs. & — \\
\midrule
Cestus (B) & 2 gp & 1d6 bludgeoning & 1 lb. & Light \\
\midrule
Finesse Glaive (B) & 25 gp & 1d4 slashing & 5 lbs. & Versatile (1d6), Reach, Finesse \\
\midrule
Katana (B) & 15 gp & 1d6 slashing & 2 lbs. & Versatile (2d4), Finesse \\
\midrule
Long Chain Flail (B) & 15 gp & 1d6 piercing & 12 lbs. & Reach \\
\midrule
Saber (B) & 15 gp & 1d8 slashing & 2 lbs. & Finesse \\
\midrule
War Spear (B) & 5 gp & 1d8 piercing & 2 lbs. & Versatile (1d10) \\
\midrule
\end{tabularx}

\begin{minipage}{0.48\textwidth}
\begin{itemize}
  \item Item: Brass Knuckles
  \item Item: Broadsword
  \item Item: Cestus
  \item Item: Chain
  \item Item: Finesse Glaive
  \item Item: Finesse Spear
\end{itemize}
\end{minipage}\hfill
\begin{minipage}{0.48\textwidth}
\begin{itemize}
  \item Item: Heavy Greatclub
  \item Item: Katana
  \item Item: Long Chain Flail
  \item Item: Saber
  \item Item: Sturdy 10-Foot Pole
  \item Item: War Spear
\end{itemize}
\end{minipage}

\section*{Example Template Weapon Crafting}

\subsection*{Simple Weapons}

\begin{tabularx}{\textwidth}\toprule
{}XXXXXXXX}
\midrule
\multicolumn{2}{c}{Name} & \multicolumn{2}{c}{Materials} & Crafting Time & Checks & Difficulty & Rarity & Value \\
\midrule
\multicolumn{2}{c}{Brass Knuckles (B)} & \multicolumn{2}{c}{1 ingot} & 2 hours & 1 & DC 13 & Common & 10 gp \\
\midrule
\multicolumn{2}{c}{Chain (B)} & \multicolumn{2}{c}{2 ingots} & 2 hours & 1 & DC 14 & Common & 5 gp \\
\midrule
\multicolumn{2}{c}{Finesse Spear (B)} & \multicolumn{2}{c}{1 ingot 
1 long haft} & 2 hours & 1 & DC 15 & Common & 13 gp \\
\midrule
\multicolumn{2}{c}{Heavy Greatclub (B)} & \multicolumn{2}{c}{2 ingots 
3 short hafts} & 2 hours & 1 & DC 14 & Common & 12 gp \\
\midrule
\multicolumn{2}{c}{Sturdy 10-Foot
Pole (B)} & \multicolumn{2}{c}{1 extra long haft} & 0 hours & 0 & DC 0 & Common & 1 sp \\
\midrule
\end{tabularx}

\subsection*{Martial Weapons}

\begin{tabularx}{\textwidth}\toprule
{}XXXXXXXX}
\midrule
\multicolumn{2}{c}{Name} & \multicolumn{2}{c}{Materials} & Crafting Time & Checks & Difficulty & Rarity & Value \\
\midrule
\multicolumn{2}{c}{Broadsword (B)} & \multicolumn{2}{c}{3 ingots} & 4 hours & 2 & DC 15 & Common & 42 gp \\
\midrule
\multicolumn{2}{c}{Cestus (B)} & \multicolumn{2}{c}{2 ingots} & 4 hours & 2 & DC 16 & Common & 70 gp \\
\midrule
\multicolumn{2}{c}{Finesse Glaive (B)} & \multicolumn{2}{c}{1 ingot 
1 long haft} & 4 hours & 2 & DC 21 & Common & 450 gp \\
\midrule
\multicolumn{2}{c}{Katana (B)} & \multicolumn{2}{c}{3 ingots} & 4 hours & 2 & DC 20 & Common & 175 gp \\
\midrule
\multicolumn{2}{c}{Long Chain Flail (B)} & \multicolumn{2}{c}{2 ingots 
1 short haft 
1 chain} & 4 hours & 2 & DC 17 & Common & 110 gp \\
\midrule
\multicolumn{2}{c}{Saber (B)} & \multicolumn{2}{c}{2 ingots} & 4 hours & 2 & DC 18 & Common & 170 gp \\
\midrule
\multicolumn{2}{c}{War Spear (B)} & \multicolumn{2}{c}{4 ingots 
1 long haft} & 4 hours & 2 & DC 16 & Common & 110 gp \\
\midrule
\end{tabularx}

\begin{itemize}
  \item Custom Weapons are a bit more expensive than normal weapons. This is a reflection of their customization and greater difficulty to craft is derived from their crafting DC.
\end{itemize}

\section*{Example Template Weapon Recipes}

\begin{minipage}{0.48\textwidth}
\begin{itemize}
  \item Brass Knuckles
  \item Broadsword
  \item Cestus
  \item Chain
  \item Finesse Glaive
  \item Finesse Spear
\end{itemize}
\end{minipage}\hfill
\begin{minipage}{0.48\textwidth}
\begin{itemize}
  \item Heavy Greatclub
  \item Katana
  \item Long Chain Flail
  \item Saber
  \item Sturdy 10-foot Pole
  \item War Spear
\end{itemize}
\end{minipage}

\section*{Modifiers}

% [Image Inserted Manually]

\subsection*{Material Modifiers}

\begin{tabularx}{\textwidth}\toprule
{}XXXXXXXXX}
\midrule
Metal & Difficulty Modifier & \multicolumn{4}{c}{Weapon Effect} & \multicolumn{4}{c}{Armor Effect} \\
\midrule
Adamantine & +7 & \multicolumn{4}{c}{Gains the “Special: Critical Strikes with this weapon permanently damage nonmagical weapons, shields or armor of the defending creature that are not forged from Adamantine (reducing the attack roll of a weapon or the AC of armor by 2)”.} & \multicolumn{4}{c}{While you’re wearing it, any critical hit against you becomes a normal hit.} \\
\midrule
Bronze & −3 & \multicolumn{4}{c}{Weapons forged from Bronze are inferior, having −1 to attack and damage rolls. Gains the Fragile property.} & \multicolumn{4}{c}{Armor forged from bronze is inferior, having a −1 penalty to its AC. Gains the Fragile property.} \\
\midrule
Cold Iron (Meteoric Iron) & −2 & \multicolumn{4}{c}{Gains the Fragile property.} & \multicolumn{4}{c}{Gains the Fragile property.} \\
\midrule
Dark Steel & +6 & \multicolumn{4}{c}{You have advantage on attack rolls while in darkness wielding Darksteel weapons.} & \multicolumn{4}{c}{Perception checks relying on sight have disadvantage against you when you are in dim light or darkness while wearing this armor.} \\
\midrule
Fire Steel & +6 & \multicolumn{4}{c}{A weapon forged from firesteel deals an extra 1d4 fire damage on hit.} & \multicolumn{4}{c}{Wearing armor forged from firesteel grants resistance to Cold damage.} \\
\midrule
Ice Steel & +6 & \multicolumn{4}{c}{A weapon forged from icesteel deals an extra 1d4 cold damage on hit.} & \multicolumn{4}{c}{Wearing armor forged from icesteel grants resistance to Fire damage.} \\
\midrule
Mithral & +5 & \multicolumn{4}{c}{A weapon with the heavy property forged from it loses the heavy property. If the weapon didn’t have the heavy property, it gains the light property. The DC of an Enchanter applying an Enchantment to it is reduced by 4, and it always counts has having 1 common essence of any type as part of the craft} & \multicolumn{4}{c}{If the armor normally imposes disadvantage on Dexterity (Stealth) checks or has a Strength requirement, the mithral version of the armor doesn’t. Easier for Enchanters to Enchant.} \\
\midrule
\end{tabularx}

\subsection*{Crafting Modifiers}

\begin{tabularx}{\textwidth}\toprule
{}XXXXXXXXX}
\midrule
Modifier & Difficulty Modifier & \multicolumn{4}{c}{Weapon Effect} & \multicolumn{4}{c}{Armor Effect} \\
\midrule
Aerodynamic & +4 & \multicolumn{4}{c}{The weapon gains the Thrown (10/30) property if it doesn’t have the Thrown property. If it has the Thrown property, the range increases by 10/30 feet instead.} & \multicolumn{4}{c}{Your falling speed increases to 520 feet per round while wearing this armor.} \\
\midrule
Double Bladed & +10 & \multicolumn{4}{c}{The weapon’s damage die is reduced by d2. Adds the “Special: You can use a bonus action immediately after to make a single melee attack with it. This attack deals 1d4 slashing damage on a hit”} & \multicolumn{4}{c}{—} \\
\midrule
Elven & +5 & \multicolumn{4}{c}{The weapon gains the Finesse property.} & \multicolumn{4}{c}{You are considered proficient with this armor even if you lack proficiency.} \\
\midrule
Fragile & N/A & \multicolumn{4}{c}{A Fragile weapon breaks on an attack roll of 1 against an armored target (a target wearing armor or with the natural armor property) if that armor doesn’t have the Fragile property.} & \multicolumn{4}{c}{A Fragile set of armor is destroyed when you take a critical strike from a creature wielding a weapon without the Fragile property.} \\
\midrule
Hardened & +4 & \multicolumn{4}{c}{The weapon’s Quality Die when maintained becomes a d12} & \multicolumn{4}{c}{The armor’s Quality Die when maintained becomes a d12.} \\
\midrule
Lance & +2 & \multicolumn{4}{c}{Requires two-handed weapon with reach. The weapon becomes one handed and its base damage die increases by d2, but you have disadvantage when you use a lance to attack a target within 5 feet of you. Also, a lance requires two hands to wield when you aren’t mounted.} & \multicolumn{4}{c}{—} \\
\midrule
Masterwork (MW) & +6 & \multicolumn{4}{c}{A Masterwork weapon gains +1 to attack rolls. Removes the fragile property if present.} & \multicolumn{4}{c}{A set of Masterwork armor gains a Damage Reduction (DR) value of 2. Removes the fragile property if present.} \\
\midrule
Segmented & +4 & \multicolumn{4}{c}{—} & \multicolumn{4}{c}{The armor can be donned or doffed in half as much time.} \\
\midrule
Slotted & +2 & \multicolumn{4}{c}{This weapon can hold 1 magical gem crafted by an Enchanter} & \multicolumn{4}{c}{This armor can hold 1 magical gem crafted by Enchanter.} \\
\midrule
Spiked & +4 & \multicolumn{4}{c}{If a weapon deals bludgeoning damage, it now deals piercing damage.} & \multicolumn{4}{c}{Attackers that strike you with unarmed strikes or natural weapons take 1d4 piercing damage. A creature that ends its turn while grappling you takes 1d4 piercing damage.} \\
\midrule
Weighted (Dwarven) & +4 & \multicolumn{4}{c}{A weapon with the light property forged from it loses the light property. If the weapon didn’t have the light property, it gains the heavy property.} & \multicolumn{4}{c}{If an Effect moves you against your will along the ground while wearing this armor, you can use your Reaction to reduce the distance you are moved by up to 10 feet. The weight of the armor is increased by 50\%.} \\
\midrule
\end{tabularx}

\begin{itemize}
  \item (MW) Masterwork: Failing a crafting roll for Masterwork doesn’t cause a failure, but the resulting weapon is only a Masterwork if all crafting rolls succeed pass the DC of Masterwork. An item is automatically masterwork if every roll qualified for a Masterwork version.
  \item (DR) Damage Reduction: While you are wearing armor, bludgeoning, piercing, and slashing damage that you take from nonmagical weapons is reduced by the value of your Damage Reduction to a minimum of 1.
\end{itemize}

\subsection*{Supplemental Modifiers}

\begin{tabularx}{\textwidth}\toprule
{}XXXXXXXXX}
\midrule
Modifier & \multicolumn{2}{c}{Materials Needed} & Difficulty Modifier & \multicolumn{3}{c}{Weapon Effect} & \multicolumn{3}{c}{Armor Effect} \\
\midrule
Magical & \multicolumn{2}{c}{2 common arcane essences 
2 uncommon arcane essences*} & +8 (Magic) & \multicolumn{3}{c}{Weapon adds +1 to attack and damage rolls.} & \multicolumn{3}{c}{Armor AC is increased by +1.} \\
\midrule
Silvered & \multicolumn{2}{c}{5 silver scraps*} & +2 & \multicolumn{3}{c}{This weapon is considered silvered for the purposes of overcoming damage resistance.} & \multicolumn{3}{c}{This armor is shiny} \\
\midrule
\end{tabularx}

\begin{itemize}
  \item *Supplemental Materials are doubled for weapons with the two-handed property or armor.
  \item (Magic) Magic: Difficulty modifier is reduced to +3 if combined with Mithral or Adamantine material modifiers.
\end{itemize}

\section{Blacksmithing}

\section*{Additional Items}

\begin{minipage}{0.48\textwidth}
\subsubsection*{Adamantine Bracers}

Uncommon

While wearing bracers and not using a shield, as a reaction to being hit by an attack, you can attempt to parry the attack with your bracer, adding +2 AC bonus against the triggering attack. If this causes your AC to be exactly equal to the attack roll and the attacking weapon is made of a common metal, the attacking weapon is destroyed.

\begin{itemize}
  \item Item: Adamantine Bracers (B)
\end{itemize}

\subsubsection*{Bracers}

Common

While wearing bracers and not using a shield, as a reaction to being hit by an attack, you can attempt to parry the attack with your bracer, adding +2 AC bonus against the triggering attack.

\begin{itemize}
  \item Item: Bracers (B)
\end{itemize}

\subsubsection*{Spiked Shield}

Common

Your shield is considered a martial melee weapon, dealing 1d4 piercing damage on hit.

\begin{itemize}
  \item Item: Spiked Shield (B)
\end{itemize}
\end{minipage}\hfill
\begin{minipage}{0.48\textwidth}
\subsubsection*{Thunder Cannon}

Requires Attunement

The principle weapon of a Thundersmith. Deals 1d12 piercing damage, and has the Ammunition (60/180), Two-Handed, Loud, and Stormcharged properties.

\begin{itemize}
  \item Stormcharged. When you use an action, bonus action, or reaction to attack with a Stormcharged Weapon, you can make only one Attack regardless of the number of attacks you can normally make. If you could otherwise make additional attacks with that action, the weapon deals an extra 3d6 lightning or thunder damage per attack that was foregone.
  \item Loud. Your weapon rings with thunder that is audible within 300 feet of you whenever it makes an attack
\end{itemize}

\begin{itemize}
  \item Item: Thunder Cannon (B)
\end{itemize}

\subsubsection*{Tower Shield}

Common

This is a massive, unwieldy shield. While carrying it, your speed is reduced by 10 feet. Wielding a tower shield increases your Armor Class by 2. At the end of each of your turns, pick a direction. You have half cover from attacks in a cone that direction. Alternatively, you can pick a single target, tracking the movement. You have half cover against attacks from that target (and only that target).

\begin{itemize}
  \item Item: Tower Shield (B)
\end{itemize}
\end{minipage}

\section*{Beyond the System}

\begin{minipage}{0.48\textwidth}
While blacksmithing allows for a huge amount of custom creation, there will always be new things beyond the system.

The following are some guidelines for how to make something beyond the system.

\begin{itemize}
  \item Trivial items should be composed of metal scraps and have a DC in the range of 8–12. These represent things blacksmiths can easily make. Common items should require ingots, and have a DC of 12–18 depending on if they are simple or exotic. Uncommon and rarer items should have uncommon and rarer materials matching their rarity.
  \item Blacksmithing alone can make magical weapons only to the extent that magical components are used.
  \item If you would like Blacksmithing alone to make fully magical swords, you can largely just take the materials from an equivalent enchanting recipe and add them to materials of the weapon, and then take the higher of the difficulty and crafting time of the projects.
\end{itemize}
\end{minipage}\hfill
\begin{minipage}{0.48\textwidth}
Work Together and be Reasonable

The system is, by its nature, extremely extensible. Great pains have been taken to make things as “balanced” as possible. But this doesn’t mean the rules transcend common sense. As a player, tell your GM what you want to make and be open to adjustments to how it would work. As a GM, tell players how their projects will work from the start, and be open to the idea they can make cool things... these crafted items are part of their “loot” and don’t need to be “power neutral” as long as they aren’t breaking anything.
\end{minipage}

% [Image Inserted Manually]

% [Image Inserted Manually]

\section{Cooking}

% [Image Inserted Manually]

\section*{Cooking}

Adventures need to eat. While some are content to eat the same stale old rations, some prefer to apply their skills to be kept well fed... even in the depths of a dungeon where fresh ingredients can be... strange.

Not only can someone with talent produce tasty treats, but being well fed can have a variety of benefits... particularly when eating correctly prepared magical ingredients.

\section*{Related Tool \& Ability Score}

Cooking works using cook’s utensils. Attempting to craft a meal without these will almost always be made with disadvantage, and proficiency with these allows you to add your proficiency to any cooking crafting roll. Cooking uses your Wisdom modifier.

\section*{Quick Reference}

While each step will go into more depth, the quick reference allows you to at a glance follow the steps to make a meal in its most basic form: • Acquire a fresh ingredient by harvesting or finding it.

\begin{itemize}
  \item Review the Recipe Tables for Feasts, Snacks or Rations that ingredient would qualify for, and gather the other materials needed listed in the Materials column.
  \item Use your cook’s utensils tool to craft the option using the number of hours listed in the Crafting Time column, or use the crafting camp action, during a long rest, if the crafting time is 2 hours or less. Meals must be crafted in a single crafting session. All meals require a heat source (such as a campfire).
  \item For every 2 hours, make a crafting roll of 1d20 + your Wisdom modifier + your proficiency bonus with cook’s utensils. You can abort the craft after a bad crafting roll if you wish, this counts as a failure.
  \item On success, you mark 2 hours of completed time. Once the completed time is equal to the crafting time, the meal is complete. On failure, the crafting time is lost, and no progress has been made during the 2 hours. If you fail 3 times in a row, the crafting is a failure, and all materials are lost.
\end{itemize}

\section*{Duration}

The duration a meal remains edible depends on the type of meal made. A feast must be consumed within an hour, a snack lasts for 1 day, and rations last 1 month.

\section*{Ingredients}

\begin{minipage}{0.48\textwidth}
The materials for cooking are fresh ingredients and supplies. A fresh ingredient is something you harvest that can increase the quality of your food and sometimes provide supernatural boosts beyond mere satiation. Staples include seasonings, spices, flour or even turnips! Foods that last awhile can be cooked into perfectly fine meals but have less benefits than eating a good meal. Some ingredients additionally have the exotic property and may confer special effects.

\subsubsection*{Ingredient Expiration}

Normal Ingredients expire very quickly and must be used within 24 hours of being harvested or they become unusable. Preserved rations would fall into the staples category and cannot typically be used to cook anything besides basic meals.

The ability to gain any sort of magical benefit is linked to its freshness; even well-prepared preserved foods provide only the benefit of a Basic Meal (which is a satisfying and good tasting meal, but its benefits are not otherwise magical).

Unlike Fresh Ingredients, Staples last significantly longer and do not need to have their expiration tracked for the purposes of this system.
\end{minipage}\hfill
\begin{minipage}{0.48\textwidth}
General System, Specific Examples

Like with all aspects, this crafting system does not attempt to provide systematic specific examples (though they are provided in the appendix), but rather a system that allows you to know the outcome of anything you could cook. For example, a party could acquire 1 common fresh ingredient by harvesting a Hook Horror, and then combine that with 1 common supplies to cook either Hook Turkey Sandwiches or Murder Chicken Tenders, but the benefit from either example would still fall into a Monstrosity Meat Quality Meal category (one step above standard rations).
\end{minipage}

\subsection*{Satiation}

The magical benefits you get consuming rare and magical foods cannot be gained again until 24 hours have passed for any particular food. When combined with the ingredient expiration above, this typically means that any ingredient collected is only good for a single meal; this is intentional. You can harvest more if you wish, but cooking naturally is a profession of fleeting achievement and fickle opportunity.

\section*{Crafting Roll}

\begin{minipage}{0.48\textwidth}
Putting that together means that when you would like to create a meal, your crafting roll is as follows:

Cooking Modifier = your Cook’s Utensils proficiency bonus + your Wisdom modifier
\end{minipage}\hfill
\begin{minipage}{0.48\textwidth}
\subsubsection*{Success and Failure}

For cooking, all crafting rolls must be made consecutively. Make all checks listed for the items. If you succeed a greater number of times than you fail, the meal is successfully completed, otherwise, it is inedible, and all ingredients are lost.
\end{minipage}

\section*{Exotic Ingredients}

While standard meals are made from a selection of ingredient types and generally a seasoning, exotic ingredients have specialized effects. When making a meal from these ingredients, the meal’s effect is a combination of the effect of the exotic ingredients added.

An Exotic Meal (meal cooked entirely from exotic ingredients) doesn’t need a recipe and has a crafting time of 1 hour, and the difficulty is the difficulty of all the exotic ingredients added together, with 1 check needed per exotic ingredient added.

Exotic ingredients can be combined with a standard meal by adding the DC of the standard meal to the combined difficulty of the exotic ingredients added. This can cause unusual meals and frequently has unattainably high difficulty to make it work, as adding random new components to meals typically wrecks the taste.

\section*{Basic Cooking \& Camp Actions}

While the rules present a handful of ways to use cooking for more exotic ends, the most common application of cooking is just to produce an edible meal during a long rest—something any adventuring group would welcome. This is called a Quality Meal and provides greater benefit than rations, though the benefits it provides are not magical, and merely stem from it being a satisfying meal. You can do so by expending 1 fresh ingredient of any type and 1 common supply to feed up to 5 allies or willing creatures, or by spending common supplies per creature being cooked for. These materials must have been purchased within the last week. Any creature, even one not proficient with cook’s utensils, can take this action.

You and all willing creatures (willing to eat your cooking) regain an additional Hit Die (up to their maximum). If you have proficiency with cook’s utensils, creatures regain additional Hit Dice equal to your Proficiency bonus.

\section*{Purchasing Materials}

Due to the rules on ingredient freshness, typically ingredients cannot be purchased in a way that is relevant for cooking recipes. You can purchase Supplies (of any type) that can be used to make Basic Meals.

\begin{tabularx}{\textwidth}\toprule
{}X}
\midrule
Rarity & Material Price \\
\midrule
Supplies (Salt, Staples, etc) & 1 gp \\
\midrule
Uncommon Supplies (Uncommon spices, oils, rare seeds, etc) & 10 gp \\
\midrule
Rare Supplies (Hard to get luxury goods) & 100 gp \\
\midrule
\end{tabularx}

\section*{Harvesting Ingredients}

Unlike other things, you can harvest from monsters, as there’s little chance of failure in harvesting. There’s generally more to harvest than meaningfully used. You can consume Monstrosity, Dragon, Beast and Plant Type creatures for magical benefits. However, your GM may allow other creature types as special delicacies at their discretion.

\begin{tabularx}{\textwidth}\toprule
{}X}
\midrule
Monster CR & Gathered Ingredients \\
\midrule
1/4–4 & Common \\
\midrule
5–8 & Uncommon \\
\midrule
9–12 & Rare \\
\midrule
13–16 Very & Rare \\
\midrule
16–20 & Legendary \\
\midrule
\end{tabularx}

\section{Cooking}

\section*{Cooking Crafting Tables}

\section*{Feasts}

\subsection*{Feasts}

\begin{tabularx}{\textwidth}\toprule
{}XXXXXXXXXX}
\midrule
\multicolumn{2}{c}{Name} & \multicolumn{4}{c}{Materials} & Crafting Time & Checks & Difficulty & Rarity & Value \\
\midrule
\multicolumn{2}{c}{Common Feast (Quality Meal)} & \multicolumn{4}{c}{1 common fresh ingredient 
1 common supplies} & 1 hour & 1 & DC 8 & common & 3 gp \\
\midrule
\multicolumn{2}{c}{Enhancing Feast} & \multicolumn{4}{c}{1 uncommon fresh ingredient 
1 uncommon supplies 
2 common supplies} & 2 hours & 1 & DC 14 & uncommon & 15 gp \\
\midrule
\multicolumn{2}{c}{Meat Feast} & \multicolumn{4}{c}{1 uncommon meat 
1 uncommon supplies 
2 common supplies} & 2 hours & 1 & DC 14 & uncommon & 15 gp \\
\midrule
\multicolumn{2}{c}{What Doesn’t Kill
You... Feast} & \multicolumn{4}{c}{1 uncommon meat from a creature that deals poison damage 
2 uncommon poisonous reagents 
1 uncommon supplies 
2 common supplies} & 2 hours & 1 & DC 16 & uncommon & 110 gp \\
\midrule
\multicolumn{2}{c}{Seaworthy Bouillabaisse} & \multicolumn{4}{c}{1 uncommon meat from a creature with a swimming speed greater than its walking speed 
2 uncommon supplies 
2 common supplies} & 2 hours & 1 & DC 12 & uncommon & 25 gp \\
\midrule
\multicolumn{2}{c}{Wondrous Feast} & \multicolumn{4}{c}{1 rare fresh ingredient 
1 uncommon reagent (any) 
1 rare supplies 
1 uncommon supplies 
2 common supplies} & 2 hours & 1 & DC 16 & rare & 150 gp \\
\midrule
\multicolumn{2}{c}{Hearty Meat Feast} & \multicolumn{4}{c}{1 rare meat 
1 uncommon reagent (any) 
1 rare supplies 
1 uncommon supplies 
2 common supplies} & 2 hours & 1 & DC 16 & rare & 150 gp \\
\midrule
\multicolumn{2}{c}{Elementally Fortifying Feast} & \multicolumn{4}{c}{1 rare meat from a creature with an elemental resistance or immunity 
2 uncommon reactive reagents 
1 common primal essence 
2 rare supplies 
2 uncommon supplies 
2 common supplies} & 2 hours & 1 & DC 16 & rare & 325 gp \\
\midrule
\multicolumn{2}{c}{Heroes’ Feast} & \multicolumn{4}{c}{4 rare curative reagents 
2 uncommon divine essences 
4 rare supplies 
4 supplies} & 4 hours & 2 & DC 18 & rare & 1,500 gp \\
\midrule
\multicolumn{2}{c}{Superb Feast} & \multicolumn{4}{c}{1 very rare fresh ingredient 
1 rare reagent (any) 
2 rare supplies 
2 uncommon supplies 
2 common supplies} & 4 hours & 2 & DC 18 & very rare & 300 gp \\
\midrule
\multicolumn{2}{c}{Superb Meat Feast} & \multicolumn{4}{c}{1 very rare meat 
1 rare reagent (any) 
2 rare supplies 
2 uncommon supplies 
2 common supplies} & 4 hours & 2 & DC 18 & very rare & 300 gp \\
\midrule
\multicolumn{2}{c}{Legendary Meat Feast} & \multicolumn{4}{c}{1 legendary meat 
1 very rare reagent (any) 
3 rare supplies 
3 uncommon supplies 
1 common supplies} & 6 hours & 3 & DC 20 & legendary & 3,000 gp \\
\midrule
\multicolumn{2}{c}{Legendary Feast} & \multicolumn{4}{c}{1 legendary fresh ingredient 
1 very rare reagent (any) 
3 rare supplies 
3 uncommon supplies 
1 common supplies} & 6 hours & 3 & DC 20 & legendary & 3,000 gp \\
\midrule
\end{tabularx}

\section*{Snacks}

\subsection*{Snacks}

\begin{tabularx}{\textwidth}\toprule
{}XXXXXXXXXX}
\midrule
\multicolumn{2}{c}{Name} & \multicolumn{4}{c}{Materials} & Crafting Time & Checks & Difficulty & Rarity & Value \\
\midrule
\multicolumn{2}{c}{Flame Breathing Jerky (5)} & \multicolumn{4}{c}{1 uncommon or rarer meat from a creature that is immune to fire damage 
2 uncommon reactive reagents 
1 rare supplies} & 6 hours & 3 & DC 15 & uncommon & 250 gp \\
\midrule
\multicolumn{2}{c}{Mint Chew (5)} & \multicolumn{4}{c}{2 uncommon curative reagent 
1 uncommon supplies} & 4 hours & 2 & DC 14 & uncommon & 100 gp \\
\midrule
\multicolumn{2}{c}{Morph Cookies (5)} & \multicolumn{4}{c}{1 ingredient harvested from a shapeshifter 
1 rare supplies 
1 uncommon supplies 
1 common supplies} & 4 hours & 2 & DC 15 & uncommon & 125 gp \\
\midrule
\multicolumn{2}{c}{Seeing Sticks (5)} & \multicolumn{4}{c}{1 ingredient from a creature with blindsight or tremorsense 
1 uncommon reactive reagent 
1 rare supplies 
1 common supplies (optional)} & 4 hours & 2 & DC 15 & uncommon & 150 gp \\
\midrule
\multicolumn{2}{c}{Quickening Candies (5)} & \multicolumn{4}{c}{1 rare supplies 
2 uncommon supplies 
1 common supplies} & 4 hours & 2 & DC 18 & rare & 300 gp \\
\midrule
\end{tabularx}

\section*{Rations}

\subsection*{Rations}

\begin{tabularx}{\textwidth}\toprule
{}XXXXXXXXXX}
\midrule
\multicolumn{2}{c}{Name} & \multicolumn{4}{c}{Materials} & Crafting Time & Checks & Difficulty & Rarity & Value \\
\midrule
\multicolumn{2}{c}{Elvish Bread (10)} & \multicolumn{4}{c}{1 uncommon curative reagent 
1 uncommon supplies 
1 common supplies} & 6 hours & 3 & DC 15 & uncommon & 60 gp \\
\midrule
\multicolumn{2}{c}{Iron Rations (10)} & \multicolumn{4}{c}{2 common supplies} & 1 hour & 1 & DC 8 & common & 2 gp \\
\midrule
\end{tabularx}

\section*{Cooking Recipes}

\begin{minipage}{0.48\textwidth}
\subsubsection*{Feasts}

\begin{itemize}
  \item Common Feast
  \item Elementally Fortifying Feast
  \item Enhancing Feast
  \item Hearty Meat Feast
  \item Heroes' Feast
  \item Legendary Feast
  \item Legendary Meat Feast
  \item Meat Feast
  \item Seaworth Bouillabaisse
  \item Superb Feast
  \item Superb Meast Feast
  \item What Doesn't Kill You...Feast
  \item Wondrous Feast
\end{itemize}
\end{minipage}\hfill
\begin{minipage}{0.48\textwidth}
\subsubsection*{Rations}

\begin{itemize}
  \item Elvish Bread
  \item Iron Rations
\end{itemize}

\subsubsection*{Snacks}

\begin{itemize}
  \item Flame Breathing Jerky
  \item Mint Chew
  \item Morph Cookies
  \item Quickening Candies
  \item Seeing Sticks
\end{itemize}
\end{minipage}

\section{Cooking}

\section*{Food}

\section*{Feasts}

\begin{minipage}{0.48\textwidth}
\subsubsection*{Common Feast}

Food (feast), common

Up to 5 creatures can consume this feast within an hour of it being prepared. After consuming this hearty meal, you become satiated for the next 24 hours. This is a hearty meal well surpassing the benefits of normal meals or rations, and each creature that consumes it regains an additional Hit Die during the next long rest.

If cooked by a creature with proficiency in cook’s utensils, creatures that consume this feast regain additional Hit Dice equal to the cook’s proficiency bonus during the next long rest instead.

\begin{itemize}
  \item Item: Common Feast
\end{itemize}

\subsubsection*{Uncommon Feast}

Enhancing/Wondrous/Superb/Legendary Food (feast), uncommon/rare/very rare/legendary

Up to 5 creatures can consume this feast within an hour of it being prepared. After consuming this hearty meal, you become satiated for the next 24 hours. You gain the benefits for a common feast but can gain additional benefits by trading in Hit Dice that would be gained as part of your next long rest after consuming this meal. These Hit Dice are consumed after calculating how many you would have after the rest (including the benefits of the feast), but you can trade Hit Dice in this way you would normally lose if you have more than your maximum Hit Dice.

You can trade Hit Dice in this way up to a number equal to your proficiency bonus. The benefits scale is based on the rarity on the feast.

\begin{tabularx}{\textwidth}\toprule
{}}
\midrule
Benefits [Uncommon/Rare/Very Rare/Legendary] \\
\midrule
Roll a [ d4/d6/d8/d12] per Hit Die traded and gain temporary hit points equal to the value rolled. \\
\midrule
Trade three Hit Dice for an additional spell slot of [ 1st/2nd/3rd/4th] level \\
\midrule
Gain a point that can be expended like Inspiration by trading [ 5/4/3/2] Hit Dice per point. \\
\midrule
\end{tabularx}

Any benefit from a feast fade after 24 hours.

\begin{itemize}
  \item Item: Enhancing Feast
  \item Item: Wondrous Feast
  \item Item: Superb Feast
  \item Item: Legendary Feast
\end{itemize}
\end{minipage}\hfill
\begin{minipage}{0.48\textwidth}
\subsubsection*{Elementally Fortifying Feast}

Food (feast), rare

Up to 5 creatures can consume this feast within an hour of it being prepared. After consuming this hearty meal, you become satiated for the next 24 hours. The magical properties of the meal confer the following special benefits:

\begin{tabularx}{\textwidth}\toprule
{}XX}
\midrule
Element & \multicolumn{2}{c}{Benefit} \\
\midrule
Cold & \multicolumn{2}{c}{Advantage on saves against cold climates and resistance to cold damage.} \\
\midrule
Fire & \multicolumn{2}{c}{Advantage on saves against hot climates and resistance to fire damage.} \\
\midrule
Lightning & \multicolumn{2}{c}{Advantage on Constitution saving throws against the stunned condition and resistance to lightning damage.} \\
\midrule
\end{tabularx}

These benefits fade after 24 hours. You can’t gain the benefits of another feast until these benefits fade and you are no longer satiated.

\begin{itemize}
  \item Item: Elementally Fortifying Feast
\end{itemize}

\subsubsection*{Seaworthy Bouillabaisse}

Food (feast), uncommon

Up to 5 creatures can consume this feast within an hour of it being prepared. After consuming this hearty meal, you become satiated for the next 24 hours. The magical properties of the meal grant you immunity to sea sickness as well as advantage on Dexterity or Constitution saving throws involving the motion of sea vessels.

Additionally, if you do not have proficiency in Water Vehicles, you can add half your proficiency bonus (rounded down) to any check involving them for the duration.

\begin{itemize}
  \item Item: Seaworthy Boullabaisse
\end{itemize}

\subsubsection*{Heroes’ Feast}

Food (feast), rare

A magnificent spread of food. Cooking this has the effect of casting the spell heroes’ feast, except it’s made of real food and not poofed into existence by strange magics.

\begin{itemize}
  \item Item: Heroes' Feast
\end{itemize}
\end{minipage}

\begin{minipage}{0.48\textwidth}
How Feast Spent Dice Work

If, for example, at level 4, if you start a rest with 2 Hit Dice, and consume a meal that would give an additional Hit Die, you would end the rest with 5, but your maximum is 4, so you would lose the 5th. This allows you to spend that extra fifth Hit Die for additional benefits. You could spend 3 Hit Dice on these benefits, but doing so would mean you only have 2 Hit Dice after the long rest.
\end{minipage}\hfill
\begin{minipage}{0.48\textwidth}
\subsubsection*{What Doesn’t Kill You... Feast}

Food (feast), uncommon

Up to 5 creatures can consume this feast within an hour of it being prepared. After consuming this dubious meal, you gain resistance to poison damage, and gain advantage on saving throws against being poisoned for 24 hours.

If the source of a poison saving throw is the same as the source of the meat or poisonous reagent used to make the feast, the advantage against being poisoned by that creature or effect lasts one week.

\begin{itemize}
  \item Item: What Doesn't Kill You...Feast
\end{itemize}
\end{minipage}

% [Image Inserted Manually]

\subsection*{Meat Feast}

Meat/Hearty/Superb/Legendary Food (feast), uncommon/rare/very rare/legendary

Up to 5 creatures can consume this feast within an hour of it being prepared. After consuming this hearty meal, you become satiated for the next 24 hours. Consuming the essence of a properly prepared creature of great power confers some of its power to you while satiated by the feast.

You can add +1/+2/+3/+4 to ability checks and saving throws of the prepared creatures highest stat (if multiple stats are tied, the cook chooses which when preparing the meal). Alternatively, the cook can bring forth exotic properties of the creature, conferring one trait of the cooked animal to those that consume it as per the following table (this replaces the benefit to ability checks and saves):

\begin{tabularx}{\textwidth}\toprule
{}XX}
\midrule
Minimum Rarity & \multicolumn{2}{c}{Creature Boon} \\
\midrule
uncommon & \multicolumn{2}{c}{Hold Breath, Keen Senses, Pounce, Stone Camouflage} \\
\midrule
rare & \multicolumn{2}{c}{Amphibious/Water Breathing, Web Walker} \\
\midrule
very rare & \multicolumn{2}{c}{Pack Tactics, Spider Climb} \\
\midrule
legendary & \multicolumn{2}{c}{Magic Resistance, Regeneration, Shapechanger(Mimic)} \\
\midrule
\end{tabularx}

The details from the creature boon can be found on the creature stat block. The cook must be aware of the creature’s property to make the meal imbue that property, and the GM may adjust the property in cases where it would not work for players or be too powerful as written. At the GM’s discretion, any ability can be added to this list.

\begin{itemize}
  \item Item: Meat Feast
  \item Item: Hearty Meat Feast
  \item Item: Superb Meat Feast
  \item Item: Legendary Meat Feast
\end{itemize}

\section*{Snacks}

% [Image Inserted Manually]

\subsection*{Flame Breathing Jerky}

Food (snack), uncommon

A tough jerky with exotic flavoring. Very spicy, uncomfortably so. After a creature consumes this snack as an action, they gain the following benefits for the next 10 minutes:

\begin{itemize}
  \item You can’t be put to sleep by magical means.
  \item You have disadvantage on Wisdom checks.
  \item You have disadvantage on Constitution saving throws to maintain concentration on a spell.
  \item You can use a bonus action to exhale fire at a target within 10 feet of you. The target must make a DC 13 Dexterity saving throw, taking 2d6 fire damage on a failed save, or half as much damage on a successful one.
\end{itemize}

You can eat a number of pieces of this jerky equal to your Constitution modifier per day.

\begin{itemize}
  \item Item: Flame Breathing Jerky
\end{itemize}

\begin{minipage}{0.48\textwidth}
\subsubsection*{Mint Chew}

Food (snack), uncommon

A chewy minty candy. You can pop one into your mouth as a bonus action. While being chewed (for up to 10 minutes), you are energized and can ignore the effects of up to 3 levels of Exhaustion for the duration and can’t be put to sleep by magical means.

You can only gain this benefit once per day, after which consuming additional pieces of the candy has no effect until you finish a long rest.

\begin{itemize}
  \item Item: Mint Chew
\end{itemize}

\subsubsection*{Morph Cookies}

Food (snack), uncommon

Consuming one of these has the effects of casting alter self, however you can’t change the adaptation unless you consume another cookie. This effect doesn’t require concentration to maintain and can be extended by consuming an additional morph cookie.

If the ingredient came from a mimic, you can additionally transform your appearance to creatures with a different basic shape than you, though you remain the same size.

\begin{itemize}
  \item Item: Morph Cookie
\end{itemize}
\end{minipage}\hfill
\begin{minipage}{0.48\textwidth}
\subsubsection*{Seeing Sticks}

Food (snack), uncommon

A stick of hard tacky substance. You can pop one into your mouth as a bonus action. While being sucked (for up to 10 minutes), you gain expanded senses and have advantage on Wisdom (Perception) checks, saving throws against Illusion spells, and Intelligence (Investigation) checks to see through illusions, though because of the bad taste inherent to the formulation, continuing to suck on the seeing stick requires concentration, as if concentrating on a spell.

By adding the optional common supplies ingredient, they can be sweetened, granting you advantage on Constitution saving throws made to maintain your concentration against spitting them out; this increases the DC of the recipe by 1.

\begin{itemize}
  \item Item: Seeing Stick
\end{itemize}

\subsubsection*{Quickening Candies}

Food (snack), rare

A small hard candy ball, with extreme caffeinated properties. When you pop one of these into your mouth as a bonus action, you are under the effect of haste for 1d4 rounds. You still suffer the normal effect of haste ending when the effect ends.

\begin{itemize}
  \item Item: Quickening Candy
\end{itemize}
\end{minipage}

\section*{Rations}

\begin{minipage}{0.48\textwidth}
\subsubsection*{Elvish Bread}

Food (ration), uncommon

A creature can use its action to eat one bite of this bread (1 ration of it). Eating a piece restores 1 hit point, and the bread provides enough nourishment to sustain a creature for one day.

\begin{itemize}
  \item Item: Elvish Bread
\end{itemize}
\end{minipage}\hfill
\begin{minipage}{0.48\textwidth}
\subsubsection*{Iron Rations}

Food (ration), common

Iron rations refers to field rations involving dried and preserved food. A basic food ration that will keep you from starving for a day after consuming it. Common side effects include a craving for real food after prolonged exposure.

\begin{itemize}
  \item Item: Iron Rations
\end{itemize}
\end{minipage}

\section{Enchanting}

% [Image Inserted Manually]

\section*{Enchanting}

Enchanting is a hard and expensive profession, but one eagerly pursued by many all the same. The makers of miracles, the craftsmen of wonder, no other profession holds the fascination of adventurers quite like Enchanter, for their domain encompasses the large majority of magical items.

An item need not pass through an enchanter’s hands to be magical, indeed many a blacksmith has forged a magical blade with the right materials, but the true wonder of enchantment is to turn the mundane magical. An enchanter can turn even the most base and commonplace item into something wonderful and powerful, and when given the head start of working with an already well-crafted item can craft things of legend.

Many enchanters further specialize in subdomains such as Scroll Scribing or Wand Whittling for more specialized good that require more specialized tools, with many even pursuing such things as Jewelry Crafting in order to create the precious items that most easily enchant, but the general field of Enchanting still covers a large swath of the wondrous.

Scrolls are heavily featured as a component of nearly all magical items, forming the basis for the powerful enchantments that imbue them with their magic. These are templates of a sort, and thus the ability to craft scrolls with Scroll Scribing is often the most desired of the subdomains for an Enchanter.

\section*{Related Tool \& Ability Score}

\begin{minipage}{0.48\textwidth}
Rather than any one tool, Enchanting primarily uses the Arcana skill. Due to the subdomains of scrolls (Scroll Scribing) and runes (Runecraft) being part of enchanting, proficiency in Calligraphy Tools is often useful. 0

Enchanting uses your Intelligence modifier. While magic comes in many forms (Arcane, Primal, Divine) and many casters are able to control it with other aspects of their talent, the ability to systematically bind it into magic items requires a deep understanding of its inner workings that can only be accomplished through meticulous study and knowledge.
\end{minipage}\hfill
\begin{minipage}{0.48\textwidth}
Psionic Items

The items listed on the Psionic Items table are the exception to this rule. These are items infused with psionic power, and instead use the “Psionics” skill, a skill that any psionic character can be considered proficient if not using special psionics rules.

These items are something of a subtype of enchanting and may not exist in your setting if psionics aren’t included. Consult your GM.
\end{minipage}

\section*{Quick Reference}

While each step will go into more depth, the quick reference allows you to at a glance follow the steps to make a magic item in its most basic form:

\begin{itemize}
  \item Select the magic item that you would like to craft from any of the Magic Item Tables.
  \item Acquire the items listed in the materials column for that item.
  \item Use your Arcana skill to infuse the option using the number of hours listed in the Crafting Time column, or during a long rest using the crafting camp action if the crafting time is 2 hours or less.
  \item For every 2 hours, make a crafting roll of 1d20 + your Intelligence modifier + your Arcana proficiency.
  \item On success, you mark 2 hours of completed time. Once the completed time is equal to the crafting time, the magic item is complete. On failure, the crafting time is lost, and no progress has been made during the 2 hours. If you fail 3 times in a row, the crafting is a failure, and all materials are lost.
\end{itemize}

\section*{Materials: Essences \& Components}

The materials of enchanting are Essences and Components. Essences come in three different types: Arcane, Primal, and Divine—as well as five rarities: common, uncommon, rare, very rare, and legendary. What an Essence is can vary greatly, as they are things of innate magic that is used to power the Enchanter’s creations.

They could be organs of magical monsters (such as the heart of a dragon which would be a rare primal essence) or they can be synthesized from magical reagents into a magical compound. Components are a broad category of items ranging from the base item you are enchanting to any extraneous bits needed to be added. One unique component that is present in many enchantments is a Spell Scroll of various types of spells that form the basis of the sort of the magic the item has.

\section*{Replacing Spell Scrolls}

\begin{minipage}{0.48\textwidth}
A crafter that is capable of casting magic can replace the spell scroll in an enchantment with the ability to cast that spell, but when doing so they must cast that spell for each crafting check they make on that item. This is an exhausting process, draining their magic far more deeply than normal casting, and confers a level of exhaustion each time this technique is used to replace a crafting check. If a magic item requires multiple scrolls, only one of them can be replaced in this way, though if an additional spell caster that can cast the necessary spell can assist you, they can replace a second scroll, though suffering the same penalty.
\end{minipage}\hfill
\begin{minipage}{0.48\textwidth}
Exhausting \& Difficult Method

This mechanic is intentionally quite difficult to use—even for casters that are capable of casting the spell. The typical process would be to make the scroll first to formalize their thoughts and process, laying down the patterns and templates for the magic item. It also removes a very costly gate in the process, so should not be easily bypassed.
\end{minipage}

\section*{Crafting Roll}

\begin{minipage}{0.48\textwidth}
Putting that together means that when you would like to enchant an item, your crafting roll is as follows:

Enchanting Modifier = your Arcana proficiency bonus + your Intelligence modifier
\end{minipage}\hfill
\begin{minipage}{0.48\textwidth}
\subsubsection*{Success and Failure}

After you make a crafting roll, if you succeed, you make 2 hours of progress toward the total crafting time (and have completed one of the required checks for making an item).

Checks for Enchanting do not need to be immediately consecutive. If you fail three times in a row, all progress and materials are lost and can no longer be salvaged. Failure means that no progress is made during that time.

Once an item is started, even if no progress is made, the components reserved for that item can only be recovered via salvage.
\end{minipage}

Enchanting Example and Walkthrough

Caius the Wizard has a keen interest in magic items. At the start, all he has is the Arcana skill and big dreams. Let’s walk through how turn those into a magic item.

Caius the Wizard happens to be a level 4 wizard at this moment in time. Strangely, he’s the only human in his party, and his lack of darkvision has been holding them back.

So, Caius decides to embark on making a set of Goggles of Night. Let’s follow his journey:

First, he'll need to gather those materials. He needs googles, a scroll of darkvision, a common primal essence, and an uncommon arcane essence. Goggles are easily acquired from the local merchants for a few gold pieces, the rest will be a bit trickier.

A common primal essence isn't that hard to come by, and can be found by harvesting elementals, dragons, giants, or monstrosities, even at the humble 0-4 CR rating of monsters Caius has been dealing with.

The uncommon arcane essence will require a hunt. His party takes on a CR 5 undead, but on their d100, they roll a 64... no luck, all that was left was some mangled bones and 3 uncommon poisonous reagents. After a few more tries, it seems luck isn't with them. Fortunately, Caius has a plan. That uncommon poisonous reagent he got can be combined with an uncommon curative and uncommon reactive reagent in a simple 4-hour process with Alchemist's supplies and a head source.

Boiling them all down, he's left with some magical residue forming an uncommon arcane essence. Now all he needs is the scroll of darkvision. Caius really should have had this spell long since given his darkvision shortcomings, but never got around to it.

So, Caius shells out 90 gp to buy a scroll of it. Not to learn it though, of course, as soon he'll never need that spell again!

All the pieces are assembled. As a level 4 Wizard with 16 Intelligence and proficiency in the Arcana skill, Caius Enchanting Modifier is 2 (his Arcana proficiency bonus) + 3 (his intelligence modifier), so is 5 total.

During his next long rest, Caius makes his first crafting check, and rolls a d20! It's a 10. 10 + 5 = 15, so he has a success. He's 1/4th of the way done but needs to spend the rest of that long rest sleeping.

The next day he rolls again, 6! That's only 11! One failure. The next night it's a 15 for a check of 20! That's a second success. Tragedy almost strikes with the next two checks being a 4 and 2, with results of 9 and 7 respectively, that's two more failures! All his hard work is on the edge of being lost.

But Caius asks to take an extra-long break at the next town. He uses the 4 hours to make his next check, taking a 10 on the roll to forestall disaster. As he wasn't too ambitious and the Goggles of Night are only DC 13, that means by taking 10, Caius has a minimum check of a 15, and it's a pass. Feeling better, now that he's no longer one checked from the whole project failing, he makes his next check on the road, and gets an 18, for a total of 24. A fourth success, and he can take his watch that night with darkvision at long last...

...unfortunately he'd been up too late working and fell asleep, so his party got ambushed anyway, but that's a story for another time!

Caius the Wizard now can see in the dark as well as any pointy eared elf or long bearded dwarf! It took a bit longer than he'd hoped, but now Caius has his very first shiny magic item made by his own hands!

\section{Enchanting}

\section*{Enchanting Crafting Tables}

\subsection*{Common \& Uncommon Wondrous Items}

\begin{longtable}{p{2.5cm}\toprule
|p{2.5cm}|p{2.5cm}|p{2.5cm}|p{2.5cm}|p{2.5cm}|p{2.5cm}|p{2.5cm}|p{2.5cm}|p{2.5cm}|p{2.5cm}|}
\midrule
\multicolumn{2}{c}{Name} & \multicolumn{4}{c}{Materials} & Crafting Time & Checks & Difficulty & Rarity & Value \\
\midrule
\multicolumn{2}{c}{Doodle Ring (B)} & \multicolumn{4}{c}{1 ring 
1 scroll of illusory script 
1 common magical ink} & 8 hours & 4 & DC 12 & common & 100 gp \\
\midrule
\multicolumn{2}{c}{Bag of Beans} & \multicolumn{4}{c}{1 bag of beans 
1 scroll of plant growth 
1 scroll of conjure animals 
1 uncommon primal essence 
6 common primal essences} & 24 hours (3 days) & 12 & DC 15 & uncommon & 1,750 gp \\
\midrule
\multicolumn{2}{c}{Bag of Holding} & \multicolumn{4}{c}{1 bag 
1 scroll of secret chest 
2 uncommon arcane essences} & 16 hours (2 days) & 8 & DC 16 & uncommon & 1,000 gp \\
\midrule
\multicolumn{2}{c}{Bag of Tricks} & \multicolumn{4}{c}{1 bag 
4 common arcane essences 
1 scroll of conjure animals} & 12 hours (1.5 days) & 6 & DC 14 & uncommon & 520 gp \\
\midrule
\multicolumn{2}{c}{Boots of Elvenkind} & \multicolumn{4}{c}{1 pair of boots worth at least 50 gp 
1 scroll of silence 
1 scroll of pass without a trace 
1 uncommon primal essence 
1 common primal essence} & 8 hours & 4 & DC 14 & uncommon & 470 gp \\
\midrule
\multicolumn{2}{c}{Boots of Striding and Springing} & \multicolumn{4}{c}{1 pair of boots worth at least 50 gp 
1 common arcane essence 
1 common primal essence 
1 scroll of longstrider 
1 scroll of jump} & 8 hours & 4 & DC 15 & uncommon & 325 gp \\
\midrule
\multicolumn{2}{c}{Boots of the Winterlands} & \multicolumn{4}{c}{1 pair of boots worth at least 50 gp 
1 scroll of protection from energy 
1 scroll of arctic breath(B) 
2 uncommon primal essences} & 12 hours (1.5 days) & 6 & DC 15 & uncommon & 760 gp \\
\midrule
\multicolumn{2}{c}{Bracers of Archery} & \multicolumn{4}{c}{1 set of bracers 
1 scroll of seeking shot(B) 
1 uncommon primal essence 
1 uncommon arcane essence} & 12 hours (1.5 days) & 6 & DC 15 & uncommon & 500 gp \\
\midrule
\multicolumn{2}{c}{Broom of Flying} & \multicolumn{4}{c}{1 broom 
1 scroll of levitate 
1 scroll of fly 
1 scroll of animate object(B) 
2 uncommon primal essences} & 16 hours (2 days) & 8 & DC 16 & uncommon & 1,050 gp \\
\midrule
\multicolumn{2}{c}{Cap of Water Breathing} & \multicolumn{4}{c}{1 leather cap 
1 scroll of water breathing 
1 common arcane essence 
2 common primal essences} & 8 hours & 4 & DC 14 & uncommon & 270 gp \\
\midrule
\multicolumn{2}{c}{Circlet of Blasting} & \multicolumn{4}{c}{1 circlet worth at least 50 gp 
1 scroll of scorching ray 
1 common arcane essence 
1 common primal essence} & 8 hours & 4 & DC 12 & uncommon & 250 gp \\
\midrule
\multicolumn{2}{c}{Cloak of Elvenkind} & \multicolumn{4}{c}{1 cloak 
1 scroll of pass without a trace 
1 uncommon primal essence 
1 common primal essence} & 8 hours & 4 & DC 14 & uncommon & 335 gp \\
\midrule
\multicolumn{2}{c}{Cloak of the Manta Ray} & \multicolumn{4}{c}{1 cloak 
1 scroll of water breathing 
1 scroll of alter self 
2 common primal essences} & 8 hours & 4 & DC 14 & uncommon & 600 gp \\
\midrule
\multicolumn{2}{c}{Cloak of Protection} & \multicolumn{4}{c}{1 cloak 
1 scroll of shield of faith 
1 scroll of mage armor 
1 scroll of protection from energy 
1 scroll of shield 
1 scroll of false life 
1 uncommon arcane essence 
1 uncommon divine essence} & 16 hours (2 days) & 8 & DC 16 & uncommon & 1,150 gp \\
\midrule
\multicolumn{2}{c}{Decanter of Endless Water} & \multicolumn{4}{c}{1 decanter 
1 scroll of create or destroy water 
1 uncommon primal essence 
1 common divine essence} & 8 hours & 4 & DC 14 & uncommon & 300 gp \\
\midrule
\multicolumn{2}{c}{Deck of Illusions} & \multicolumn{4}{c}{34 cards 
1 scroll of major image 
1 scroll of silent image 
34 common arcane essences} & 8 hours & 4 & DC 15 & uncommon & 330 gp \\
\midrule
\multicolumn{2}{c}{Efficient Quiver} & \multicolumn{4}{c}{1 quiver worth 25 gp 
1 scroll of secret chest 
1 common arcane essence} & 8 hours & 4 & DC 15 & uncommon & 430 gp \\
\midrule
\multicolumn{2}{c}{Eversmoking Bottle} & \multicolumn{4}{c}{1 bottle 
1 scroll of fog cloud 
1 scroll of produce flame 
1 common arcane essence 
1 common primal essence} & 8 hours & 4 & DC 13 & uncommon & 210 gp \\
\midrule
\multicolumn{2}{c}{Eyes of Charming} & \multicolumn{4}{c}{1 crystal lenses (glasses) worth 50 gp 
1 scroll of charm person 
1 uncommon arcane essence} & 4 hours & 2 & DC 14 & uncommon & 300 gp \\
\midrule
\multicolumn{2}{c}{Eyes of Minute Seeing} & \multicolumn{4}{c}{1 crystal lenses (glasses) worth 50 gp 
1 scroll of identify 
1 common arcane essence} & 4 hours & 2 & DC 14 & uncommon & 190 gp \\
\midrule
\multicolumn{2}{c}{Eyes of the Eagle} & \multicolumn{4}{c}{1 crystal lenses (glasses) worth 50 gp 
1 scroll of far sight(B) 
1 common primal essence} & 4 hours & 2 & DC 14 & uncommon & 190 gp \\
\midrule
\multicolumn{2}{c}{Figurine of Wondrous Power (Silver Raven)} & \multicolumn{4}{c}{1 figurine of a raven worth at least 10 gp 
1 scroll of find familiar 
1 scroll of animal messenger 
1 common primal essence 
1 common arcane essence} & 8 hours & 4 & DC 13 & uncommon & 270 gp \\
\midrule
\multicolumn{2}{c}{Gauntlets of Ogre Power} & \multicolumn{4}{c}{1 pair of gauntlets worth 50 gp 
1 scroll of enlarge/reduce 
1 scroll of enhance ability 
1 uncommon primal essence 
1 common arcane essence} & 12 hours (1.5 days) & 6 & DC 14 & uncommon & 500 gp \\
\midrule
\multicolumn{2}{c}{Gem of Brightness} & \multicolumn{4}{c}{1 cut gem worth at least 50 gp 
1 scroll of light 
1 scroll of daylight 
1 uncommon divine essence 
1 common arcane essence} & 8 hours & 4 & DC 14 & uncommon & 585 gp \\
\midrule
\multicolumn{2}{c}{Gloves of Missile Snaring} & \multicolumn{4}{c}{1 pair of gloves 
1 scroll of attract/repel 
1 uncommon arcane essence 
1 common primal essence} & 8 hours & 4 & DC 15 & uncommon & 365 gp \\
\midrule
\multicolumn{2}{c}{Gloves of Swimming and Climbing} & \multicolumn{4}{c}{1 pair of gloves 
1 scroll of enhance ability 
1 common arcane essence 
1 common primal essence} & 8 hours & 4 & DC 14 & uncommon & 225 gp \\
\midrule
\multicolumn{2}{c}{Goggles of Night} & \multicolumn{4}{c}{1 pair of goggles 
1 scroll of darkvision 
1 common primal essence 
1 uncommon arcane essence} & 8 hours & 4 & DC 13 & uncommon & 300 gp \\
\midrule
\multicolumn{2}{c}{Hat of Disguise} & \multicolumn{4}{c}{1 hat 
1 scroll of disguise self 
1 scroll of minor illusion 
1 uncommon arcane essence 
1 common arcane essence} & 8 hours & 4 & DC 14 & uncommon & 340 gp \\
\midrule
\multicolumn{2}{c}{Headband of Intellect} & \multicolumn{4}{c}{1 headband worth at least 25 gp 
1 scroll of enhance ability 
2 uncommon arcane essences 
1 common divine essence} & 12 hours (1.5 days) & 6 & DC 14 & uncommon & 540 gp \\
\midrule
\multicolumn{2}{c}{Helm of Comprehending Language} & \multicolumn{4}{c}{1 helm worth at least 25 gp 
1 scroll of comprehend languages 
1 common arcane essence 
1 common divine essence} & 16 hours (2 days) & 8 & DC 14 & uncommon & 280 gp \\
\midrule
\multicolumn{2}{c}{Helm of Telepathy} & \multicolumn{4}{c}{1 helm worth at least 50 gp 
1 scroll of detect thoughts 
1 scroll of suggestion 
1 uncommon arcane essence 
1 common arcane essence} & 16 hours (2 days) & 8 & DC 15 & uncommon & 575 gp \\
\midrule
\multicolumn{2}{c}{Lantern of Revealing} & \multicolumn{4}{c}{1 lantern 
1 scroll of light 
1 scroll of see invisible 
1 uncommon arcane essence 
1 uncommon divine essence} & 8 hours & 4 & DC 15 & uncommon & 500 gp \\
\midrule
\multicolumn{2}{c}{Luckstone} & \multicolumn{4}{c}{1 polished agate worth 50 gp 
1 scroll of imbue luck(B) 
1 uncommon divine essence 
1 common primal essence} & 8 hours & 4 & DC 15 & uncommon & 415 gp \\
\midrule
\multicolumn{2}{c}{Pearl of Power} & \multicolumn{4}{c}{1 pearl worth at least 100 gp 
5 common arcane essences} & 8 hours & 4 & DC 12 & uncommon & 350 gp \\
\midrule
\multicolumn{2}{c}{Pipes of Haunting} & \multicolumn{4}{c}{1 pipes worth at least 25 gp 
1 scroll of frighten(B) 
1 uncommon arcane essence 
1 uncommon divine essence} & 8 hours & 4 & DC 15 & uncommon & 485 gp \\
\midrule
\multicolumn{2}{c}{Pipes of the Sewers} & \multicolumn{4}{c}{1 pipes worth at least 25 gp 
1 scroll of animal friendship 
1 uncommon primal essence 
1 common primal essence} & 8 hours & 4 & DC 14 & uncommon & 345 gp \\
\midrule
\multicolumn{2}{c}{Robe of Useful Items (with all patches)} & \multicolumn{4}{c}{1 robe worth 100 gp 
1 silver coffer worth 500 gp 
1 iron door 
10 gems worth 100 gp each 
1 wooden ladder 
1 picture of a horse worth 75 gp 
1 saddle bag 
1 scroll of create hole(B) 
4 potions of healing 
1 rowboat 
1 1st level scroll 
1 picture of mastiffs worth at least 25 gp 
1 window 
1 portable ram 
13 common arcane essences} & 8 hours & 4 & DC 14 & uncommon & 3,560 gp \\
\midrule
\multicolumn{2}{c}{Rope of Climbing} & \multicolumn{4}{c}{1 60-foot rope 
1 scroll of awaken rope(B) 
1 uncommon arcane essence 
1 common arcane essence} & 4 hours & 2 & DC 14 & uncommon & 300 gp \\
\midrule
\multicolumn{2}{c}{Sending Stones} & \multicolumn{4}{c}{1 set of the same kind of stones 
1 scroll of sending 
2 common arcane essences} & 4 hours & 2 & DC 14 & uncommon & 380 gp \\
\midrule
\multicolumn{2}{c}{Shawm of Sundering (B)} & \multicolumn{4}{c}{1 shawm worth 50 gp 
1 scroll of thunderwave 
1 scroll of shatter 
1 scroll of lightning bolt 
1 uncommon primal essence 
2 common arcane essences} & 8 hours & 4 & DC 14 & uncommon & 600 gp \\
\midrule
\multicolumn{2}{c}{Slippers of Spider Climbing} & \multicolumn{4}{c}{1 slippers 
1 scroll of spider climbing 
1 common arcane essence 
1 common primal essence} & 8 hours & 4 & DC 14 & uncommon & 225 gp \\
\midrule
\multicolumn{2}{c}{Winged Boots} & \multicolumn{4}{c}{1 pair of boots worth at least 50 gp 
1 scroll of fly 
1 scroll of levitate 
1 scroll of feather fall 
1 uncommon arcane essence 
1 uncommon primal essence} & 16 hours (2 days) & 8 & DC 15 & uncommon & 1,000 gp \\
\midrule
\multicolumn{2}{c}{Yve’s Thieves’ Tools (B)} & \multicolumn{4}{c}{1 thieves’ tools 
1 scroll of enhance ability 
1 uncommon arcane essence 
2 common arcane essences} & 8 hours & 4 & DC 15 & uncommon & 440 gp \\
\midrule
\end{longtable}

\subsection*{Rare Wondrous Items}

\begin{longtable}{p{2.5cm}\toprule
|p{2.5cm}|p{2.5cm}|p{2.5cm}|p{2.5cm}|p{2.5cm}|p{2.5cm}|p{2.5cm}|p{2.5cm}|p{2.5cm}|p{2.5cm}|}
\midrule
\multicolumn{2}{c}{Name} & \multicolumn{4}{c}{Materials} & Crafting Time & Checks & Difficulty & Rarity & Value \\
\midrule
\multicolumn{2}{c}{Bag of Beans} & \multicolumn{4}{c}{1 bag of beans 
1 scroll of plant growth 
1 scroll of conjure animals 
1 uncommon primal essence 
6 common primal essences} & 24 hours (3 days) & 12 & DC 15 & rare & 1,750 gp \\
\midrule
\multicolumn{2}{c}{Bead of Force} & \multicolumn{4}{c}{8 beads 
8 scrolls of resilient sphere} & 16 hours (2 days) & 8 & DC 16 & rare & 2,900 gp \\
\midrule
\multicolumn{2}{c}{Belt of Dwarvenkind} & \multicolumn{4}{c}{1 belt worth at least 200 gp
 200 gp worth of quality ale 
1 rare primal essence 
2 uncommon primal essences 
1 scroll of stoneskin 
1 scroll of alter self} & 16 hours (2 days) & 8 & DC 17 & rare & 2,400 gp \\
\midrule
\multicolumn{2}{c}{Belt of Hill Giant Strength} & \multicolumn{4}{c}{1 belt 
1 scroll of enhance ability 
1 scroll of enlarge/reduce 
3 rare primal essences} & 16 hours (2 days) & 8 & DC 18 & rare & 3,500 gp \\
\midrule
\multicolumn{2}{c}{Boots of Levitation} & \multicolumn{4}{c}{1 pair of boots worth at least 50 gp 
1 scroll of levitate 
2 rare arcane essences 
2 uncommon primal essences} & 16 hours (2 days) & 8 & DC 17 & rare & 2,600 gp \\
\midrule
\multicolumn{2}{c}{Boots of Speed} & \multicolumn{4}{c}{1 pair of boots worth at least 200 gp 
1 scroll of haste 
1 rare primal essence 
2 rare arcane essences 
2 uncommon arcane essences} & 16 hours (2 days) & 8 & DC 16 & rare & 3,400 gp \\
\midrule
\multicolumn{2}{c}{Bracers of Defense} & \multicolumn{4}{c}{1 set of bracers worth at least 200 gp 
1 scroll of shield 
1 scroll of shield of faith 
1 rare divine essence 
1 rare arcane essence} & 16 hours (2 days) & 8 & DC 17 & rare & 2,300 gp \\
\midrule
\multicolumn{2}{c}{Brazier of Commanding Fire Elementals} & \multicolumn{4}{c}{1 brazier worth 200 gp 
1 scroll of conjure elemental 
2 rare primal essences 
2 rare reactive reagents} & 16 hours (2 days) & 8 & DC 17 & rare & 2,400 gp \\
\midrule
\multicolumn{2}{c}{Bowl of Commanding Water Elementals} & \multicolumn{4}{c}{1 bowl worth 200 gp 
1 scroll of conjure elemental 
2 rare primal essences 
2 rare curative reagents} & 16 hours (2 days) & 8 & DC 17 & rare & 2,400 gp \\
\midrule
\multicolumn{2}{c}{Censer of Controlling Air Elementals} & \multicolumn{4}{c}{1 censer worth 200 gp 
1 scroll of conjure elemental 
2 rare primal essences 
1 rare reactive reagent 
1 uncommon arcane essence} & 16 hours (2 days) & 8 & DC 17 & rare & 2,400 gp \\
\midrule
\multicolumn{2}{c}{Cape of the Mountebank} & \multicolumn{4}{c}{1 cape worth at least 200 gp 
1 scroll of dimension door 
1 scroll of pyrotechnics 
1 rare arcane essence 
1 uncommon primal essence} & 16 hours (2 days) & 8 & DC 18 & rare & 2,000 gp \\
\midrule
\multicolumn{2}{c}{Chime of Opening} & \multicolumn{4}{c}{1 chime worth 50 gp 
1 scroll of knock 
10 common arcane essences} & 10 hours & 5 & DC 14 & 14 rare & 666 gp \\
\midrule
\multicolumn{2}{c}{Cloak of Displacement} & \multicolumn{4}{c}{1 cloak worth 50 gp 
1 scroll of mirror image 
1 rare arcane essence 
1 uncommon arcane essence} & 12 hours (1.5 days) & 6 & DC 16 & rare & 1,270 gp \\
\midrule
\multicolumn{2}{c}{Cloak of the Bat} & \multicolumn{4}{c}{1 cloak worth at least 50 gp 
1 scroll of form of the familiar(B) 
1 scroll of darkness 
1 scroll of fly 
1 rare primal essence 
1 uncommon arcane essence} & 12 hours (1.5 days) & 6 & DC 15 & rare & 1,500 gp \\
\midrule
\multicolumn{2}{c}{Cube of Force} & \multicolumn{4}{c}{1 metal cube of mithril, adamantine, or gold 
1 scroll of wall of force 
1 scroll of antilife shell 
1 scroll of gaseous form 
1 scroll of antimagic field 
2 rare arcane essences 
3 uncommon arcane essences} & 16 hours (2 days) & 8 & DC 17 & rare & 20,000 gp \\
\midrule
\multicolumn{2}{c}{Dimensional Shackles} & \multicolumn{4}{c}{1 set of manacles 
1 scroll of hold monster 
1 scroll of forbiddance 
1 rare divine essence 
1 rare arcane essence} & 16 hours (2 days) & 8 & DC 17 & rare & 4,900 gp \\
\midrule
\multicolumn{2}{c}{Feather Token (Anchor)} & \multicolumn{4}{c}{1 fletching 
1 scroll of entangle 
1 scroll of web 
1 scroll of binding curse(B) 
1 uncommon arcane essence} & 6 hours & 3 & DC 15 & rare & 435 gp \\
\midrule
\multicolumn{2}{c}{Feather Token (Bird)} & \multicolumn{4}{c}{1 fletching 
1 scroll of conjure animals 
2 scrolls of enlarge reduce 
1 uncommon primal essence} & 6 hours & 3 & DC 17 & rare & 770 gp \\
\midrule
\multicolumn{2}{c}{Feather Token (Fan)} & \multicolumn{4}{c}{1 fletching 
1 scroll of gust of wind 
1 scroll of animate object(B) 
1 uncommon arcane essence 
1 uncommon primal essence} & 6 hours & 3 & DC 15 & rare & 525 gp \\
\midrule
\multicolumn{2}{c}{Feather Token (Swan Boat)} & \multicolumn{4}{c}{1 fletching 
1 boat 50 feet long and 20 feet wide 
1 scroll of dimension door 
1 uncommon arcane essence} & 8 hours & 4 & DC 16 & rare & 4,000 gp \\
\midrule
\multicolumn{2}{c}{Feather Token (Tree)} & \multicolumn{4}{c}{1 fletching 
1 scroll of plant growth 
1 uncommon primal essence} & 6 hours & 3 & DC 15 & rare & 475 gp \\
\midrule
\multicolumn{2}{c}{Feather Token (Whip)} & \multicolumn{4}{c}{1 fletching 
1 whip 
1 scroll of animate objects 
1 uncommon arcane essence} & 6 hours & 3 & DC 14 & rare & 275 gp \\
\midrule
\multicolumn{2}{c}{Figurine of Wondrous Power (Bronze Griffon)} & \multicolumn{4}{c}{1 figurine of a griffon worth at least 20 gp 
1 scroll of summon greater steed 
1 rare divine essence 
1 uncommon divine essence} & 16 hours (2 days) & 8 & DC 17 & rare & 2,700 gp \\
\midrule
\multicolumn{2}{c}{Figurine of Wondrous Power (Ebony Fly)} & \multicolumn{4}{c}{1 figurine of a fly worth at least 20 gp 
1 scroll of giant insect 
1 rare primal essence 
1 uncommon primal essence} & 8 hours & 4 & DC 15 & rare & 1,300 gp \\
\midrule
\multicolumn{2}{c}{Figurine of Wondrous Power (Golden Lions)} & \multicolumn{4}{c}{2 figurines of a lion worth at least 20 gp 
1 scroll of conjure animals 
1 uncommon primal essence 
1 uncommon divine essence} & 12 hours (1.5 days) & 6 & DC 15 & rare & 730 gp \\
\midrule
\multicolumn{2}{c}{Figurine of Wondrous Power (Ivory Goats)} & \multicolumn{4}{c}{3 figurines of a goat worth at least 20 gp 
1 scroll of conjure animals 
1 rare arcane essence 
1 uncommon divine essence 
1 uncommon arcane essence} & 16 hours (2 days) & 8 & DC 16 & rare & 1,700 gp \\
\midrule
\multicolumn{2}{c}{Figurine of Wondrous Power (Marble Elephant)} & \multicolumn{4}{c}{1 figurine of an elephant worth at least 20 gp 
1 scroll of conjure animals 
1 scroll of enlarge reduce 
1 uncommon primal essence} & 8 hours & 4 & DC 15 & rare & 600 gp \\
\midrule
\multicolumn{2}{c}{Figurine of Wondrous Power (Onyx Dog)} & \multicolumn{4}{c}{1 figurine of a dog worth at least 20 gp 
1 scroll of conjure animals 
1 uncommon primal essence} & 8 hours & 4 & DC 15 & rare & 515 gp \\
\midrule
\multicolumn{2}{c}{Figurine of Wondrous Power (Serpentine Owl)} & \multicolumn{4}{c}{1 figurine of an owl worth at least 20 gp 
1 scroll of find familiar 
1 scroll of enlarge reduce 
1 uncommon arcane essence} & 8 hours & 4 & DC 16 & rare & 475 gp \\
\midrule
\multicolumn{2}{c}{Gem of Seeing} & \multicolumn{4}{c}{1 cut gem worth at least 50 gp 
1 scroll of true seeing 
1 rare arcane essence 
1 uncommon arcane essence} & 16 hours (2 days) & 8 & DC 17 & rare & 3,050 gp \\
\midrule
\multicolumn{2}{c}{Handy Haversack} & \multicolumn{4}{c}{1 backpack 
1 scroll of secret chest 
1 scroll of instant summons 
1 uncommon arcane essence 
1 rare arcane essence} & 16 hours (2 days) & 8 & DC 18 & rare & 3,600 gp \\
\midrule
\multicolumn{2}{c}{Helm of Teleportation} & \multicolumn{4}{c}{1 helm worth at least 200 gp 
1 scroll of teleport 
2 rare arcane essences 
1 rare divine essence} & 24 hours (3 days) & 12 & DC 18 & rare & 16,700 gp \\
\midrule
\multicolumn{2}{c}{Horn of Blasting} & \multicolumn{4}{c}{1 horn worth at least 100 gp 
1 scroll of shockwave(B) 
1 scroll of sonic shriek(B)} & 24 hours (3 days) & 12 & DC 17 & rare & 3,370 gp \\
\midrule
\multicolumn{2}{c}{Horn of Valhalla (Brass)} & \multicolumn{4}{c}{1 brass horn worth at least 50 gp 
1 scroll of spirit guardians 
1 scroll of guardian of faith 
1 rare primal essence 
1 rare divine essence} & 24 hours (3 days) & 12 & DC 17 & rare & 2,800 gp \\
\midrule
\multicolumn{2}{c}{Horn of Valhalla (Silver)} & \multicolumn{4}{c}{1 silver horn worth at least 50 gp 
1 scroll of spirit guardians 
1 scroll of guardian of faith 
1 rare primal essence 
1 rare divine essence} & 24 hours (3 days) & 12 & DC 17 & rare & 2,800 gp \\
\midrule
\multicolumn{2}{c}{Horseshoes of Speed} & \multicolumn{4}{c}{4 horseshoes worth 10 gp each 
4 scrolls of longstrider 
4 uncommon arcane essences 
4 common primal essences} & 24 hours (3 days) & 12 & DC 16 & rare & 1,570 gp \\
\midrule
\multicolumn{2}{c}{Helm of Heroes (B)} & \multicolumn{4}{c}{1 helm worth 50 gp 
1 scroll of heroism 
1 scroll of shield of faith 
1 rare divine essence 
1 uncommon divine essence} & 16 hours (2 days) & 8 & DC 16 & rare & 1,390 gp \\
\midrule
\multicolumn{2}{c}{Instant Fortress} & \multicolumn{4}{c}{1 adamantine cube 
100 adamantine ingots 
1 scroll of magnificent mansion 
2 very rare arcane essence 
4 rare arcane essence 
2 rare divine essence} & 40 hours (5 days) & 20 & DC 18 & rare & 42,000 gp \\
\midrule
\multicolumn{2}{c}{Iron Bands of Binding} & \multicolumn{4}{c}{30 feet of chain 
1 scroll of entangle 
1 scroll of awaken rope(B) 
1 rare arcane essence 
2 uncommon divine essences} & 16 hours (2 days) & 8 & DC 16 & rare & 740 gp \\
\midrule
\multicolumn{2}{c}{Mantle of Spell Resistance} & \multicolumn{4}{c}{1 mantle worth at least 100 gp 
1 scroll of dispel magic 
3 rare arcane essences 
1 uncommon divine essence} & 16 hours (2 days) & 8 & DC 17 & rare & 3,280 gp \\
\midrule
\multicolumn{2}{c}{Portable Hole} & \multicolumn{4}{c}{1 large cloth 
1 scroll of rope trick 
1 scroll of passwall 
1 scroll of dimension door 
2 rare arcane essences 
1 rare primal essence} & 16 hours (2 days) & 8 & DC 18 & rare & 5,000 gp \\
\midrule
\multicolumn{2}{c}{Robe of Eyes} & \multicolumn{4}{c}{1 robe worth at least 100 gp 
1 scroll of darkvision 
1 scroll of see invisibility 
1 scroll of arcane eye 
1 rare arcane essence 
1 uncommon psionic essence} & 24 hours (3 days) & 12 & DC 17 & rare & 2,170 gp \\
\midrule
\multicolumn{2}{c}{Rope of Entanglement} & \multicolumn{4}{c}{1 30-foot long rope 
1 scroll of awaken rope(B) 
1 scroll of entangle 
1 rare arcane essence} & 16 hours (2 days) & 8 & DC 17 & rare & 1,325 gp \\
\midrule
\multicolumn{2}{c}{Squall Collar (B)} & \multicolumn{4}{c}{1 collar worth 100 gp 
1 scroll of enhance ability 
1 scroll of lightning charged(B) 
1 rare primal essence 
1 uncommon primal essence} & 12 hours (1.5 days) & 6 & DC 16 & rare & 1,400 gp \\
\midrule
\multicolumn{2}{c}{Stone of Controlling Earth Elementals} & \multicolumn{4}{c}{1 stone worth 200 gp 
1 scroll of conjure elemental 
2 rare primal essences 
2 rare poisonous reagents} & 16 hours (2 days) & 8 & DC 17 & rare & 2,400 gp \\
\midrule
\multicolumn{2}{c}{Wings of Flying} & \multicolumn{4}{c}{1 cloak worth 50 gp 
1 scroll of fly 
1 scroll of polymorph 
1 rare arcane essence 
1 uncommon arcane essence} & 16 hours (2 days) & 8 & DC 17 & rare & 3,000 gp \\
\midrule
\end{longtable}

\subsection*{Very Rare Wondrous Items}

\begin{tabularx}{\textwidth}\toprule
{}XXXXXXXXXX}
\midrule
\multicolumn{2}{c}{Name} & \multicolumn{4}{c}{Materials} & Crafting Time & Checks & Difficulty & Rarity & Value \\
\midrule
\multicolumn{2}{c}{Bag of Devouring} & \multicolumn{4}{c}{1 bag of holding 
1 scroll of plane shift 
1 scroll of hunger of hadar} & 16 hours (2 days) & 8 & DC 17 & very rare & 15,000 gp \\
\midrule
\multicolumn{2}{c}{Belt of Fire Giant Strength} & \multicolumn{4}{c}{1 belt 
1 scroll of enhance ability 
1 scroll of transformation 
1 very rare arcane essence 
3 very rare primal essences} & 32 hours (4 days) & 16 & DC 22 & very rare & 47,000 gp \\
\midrule
\multicolumn{2}{c}{Belt of Stone Giant Strength} & \multicolumn{4}{c}{1 belt 
1 scroll of enhance ability 
1 scroll of enlarge/reduce 
1 scroll of stoneskin 
3 very rare primal essences} & 24 hours (3 days) & 12 & DC 20 & very rare & 26,000 gp \\
\midrule
\multicolumn{2}{c}{Candle of Invocation} & \multicolumn{4}{c}{1 candle worth 100 gp 
4 rare divine essences 
1 scroll of holy aura 1 scroll of gate} & 32 hours (4 days) & 16 & DC 19 & very rare & 65,000 gp \\
\midrule
\multicolumn{2}{c}{Carpet of Flying} & \multicolumn{4}{c}{1 fancy carpet worth 1000 gp 
1 scroll of fly 
1 scroll of levitate 
1 scroll of animate objects 
1 very rare arcane essence 
1 very rare primal essence} & 32 hours (4 days) & 16 & DC 19 & very rare & 20,000 gp \\
\midrule
\multicolumn{2}{c}{Cloak of Arachnida} & \multicolumn{4}{c}{1 cloak worth 200 gp 
1 scroll of web 
1 scroll of spider climb 
1 scroll of protection from poison 
1 very rare arcane essence 
1 rare arcane essence} & 24 hours (3 days) & 12 & DC 18 & very rare & 10,000 gp \\
\midrule
\multicolumn{2}{c}{Crystal Ball} & \multicolumn{4}{c}{1 crystal worth at least 1,000 gp 
1 scroll of scrying 
1 very rare arcane essence 
2 rare arcane essences} & 24 hours (3 days) & 12 & DC 18 & very rare & 12,600 gp \\
\midrule
\multicolumn{2}{c}{Efreeti Bottle} & \multicolumn{4}{c}{1 brass bottle 
1 scroll of conjure elemental 
1 scroll of planar binding 
2 very rare primal essences 
3 rare primal essences 
3 rare arcane essences} & 32 hours (4 days) & 16 & DC 18 & very rare & 24,000 gp \\
\midrule
\multicolumn{2}{c}{Figurine of Wondrous Power (Obsidian Steed)} & \multicolumn{4}{c}{1 figurine of a horse worth at least 50 gp 
1 scroll of summon greater steed 
1 very rare arcane essence 
1 rare divine essence} & 16 hours (2 days) & 8 & DC 17 & very rare & 10,250 gp \\
\midrule
\multicolumn{2}{c}{Helm of Brilliance (with all gems)} & \multicolumn{4}{c}{1 helm worth at least 100 gp 
5 diamonds worth 50 gp 
10 rubies worth 20 gp 
15 fire opals worth 10 gp 
20 opals worth 10 gp 
1 scroll of daylight 
1 scroll of fireball 
1 scroll of prismatic spray 
1 wall of fire 
1 rare arcane essence 
1 uncommon arcane essence} & 24 hours (3 days) & 12 & DC 18 & very rare & 17,000 gp \\
\midrule
\multicolumn{2}{c}{Horn of Valhalla (Bronze)} & \multicolumn{4}{c}{1 bronze horn worth at least 100 gp 
1 scroll of spirit guardians 
1 scroll of guardian of faith 
1 scroll of conjure celestial 
2 rare primal essences 
2 rare divine essences} & 32 hours (4 days) & 16 & DC 19 & very rare & 5,600 gp \\
\midrule
\multicolumn{2}{c}{Horseshoes of a Zephyr} & \multicolumn{4}{c}{4 horseshoes worth 10 gp each 
4 scrolls of levitate 
1 scroll of fly 
1 scroll of longstrider 
4 rare arcane essences 
4 uncommon primal essences} & 24 hours (3 days) & 12 & DC 18 & very rare & 5,500 gp \\
\midrule
\multicolumn{2}{c}{Mirror of Life Trapping} & \multicolumn{4}{c}{1 4 foot tall mirror worth at least 50 gp 
1 scroll of demiplane 
1 scroll of banishment 
1 very rare arcane essence 
2 rare arcane essence} & 32 hours (4 days) & 16 & DC 19 & very rare & 25,000 gp \\
\midrule
\multicolumn{2}{c}{Robe of Scintillating Colors} & \multicolumn{4}{c}{1 robe worth at least 200 gp 
1 scroll of daylight 
1 scroll of prismatic spray 
1 scroll of wall of light 
1 very rare arcane essence 
1 rare arcane essence 
1 rare divine essence} & 24 hours (3 days) & 12 & DC 18 & very rare & 25,000 gp \\
\midrule
\multicolumn{2}{c}{Robe of Stars} & \multicolumn{4}{c}{1 black or blue robe worth 200 gp 
1 scroll of magic missile 
1 scroll of etherealness 
1 scroll of blink 
1 very rare arcane essence 
1 very rare divine essence 
6 rare arcane essences} & 32 hours (4 days) & 16 & DC 19 & very rare & 32,800 gp \\
\midrule
\end{tabularx}

\subsection*{Legendary Wondrous Items}

\begin{tabularx}{\textwidth}\toprule
{}XXXXXXXXXX}
\midrule
\multicolumn{2}{c}{Name} & \multicolumn{4}{c}{Materials} & Crafting Time & Checks & Difficulty & Rarity & Value \\
\midrule
\multicolumn{2}{c}{Belt of Cloud Giant Strength} & \multicolumn{4}{c}{1 belt 
1 scroll of enhance ability 
1 scroll of transformation 
1 scroll of investiture of wind 
1 very rare arcane essence 
3 very rare primal essences 
1 legendary primal essence} & 40 hours (5 days) & 16 & DC 23 & legendary & 83,000 gp \\
\midrule
\multicolumn{2}{c}{Belt of Storm Giant Strength} & \multicolumn{4}{c}{1 belt 
1 scroll of enhance ability 
1 scroll of transformation 
1 scroll of invulnerability 
1 very rare arcane essence 
3 very rare primal essences 
2 legendary primal essences} & 48 hours (6 days) & 24 & DC 25 & legendary & 200,000 gp \\
\midrule
\multicolumn{2}{c}{Crystal Ball of Mind Reading} & \multicolumn{4}{c}{1 very rare magical crystal ball 
1 scroll of detect thoughts 
1 very rare arcane essence} & 8 hours & 4 & DC 19 & legendary & 22,200 gp \\
\midrule
\multicolumn{2}{c}{Crystal Ball of Telepathy} & \multicolumn{4}{c}{1 very rare magical crystal ball 
1 scroll of suggestion 
1 very rare psionic essence} & 8 hours & 4 & DC 20 & legendary & 22,600 gp \\
\midrule
\multicolumn{2}{c}{Crystal Ball of Thieving (B)} & \multicolumn{4}{c}{1 crystal worth at least 1,000 gp 
1 scroll of scrying 
1 scroll of mage hand 
1 legendary arcane essence 
1 very rare arcane essence} & 24 hours (3 days) & 12 & DC 19 & legendary & 40,000 gp \\
\midrule
\multicolumn{2}{c}{Crystal Ball of True Seeing} & \multicolumn{4}{c}{1 very rare magical crystal ball 
1 scroll of true seeing 
1 legendary arcane essence} & 8 hours & 4 & DC 21 & legendary & 50,000 gp \\
\midrule
\multicolumn{2}{c}{Cubic Gate} & \multicolumn{4}{c}{1 3 inch cube worth at least 500 gp 
1 scroll of gate 
1 scroll of plane shift 
1 very rare arcane essence 
1 very rare divine essence} & 32 hours (4 days) & 16 & DC 20 & legendary & 75,000 gp \\
\midrule
\multicolumn{2}{c}{Deck of Many Things} & \multicolumn{4}{c}{22 cards 
1 tears of a dungeon master} & 22 hours & 11 & DC 100 & legendary & ??? \\
\midrule
\multicolumn{2}{c}{Iron Flask (Empty)} & \multicolumn{4}{c}{1 iron flask worth at least 200 gp 
1 scroll of imprisonment 
1 scroll of planar blinding 
1 legendary arcane essence 
1 very rare divine essence} & 40 hours (5 days) & 20 & DC 21 & legendary & 98,000 gp \\
\midrule
\multicolumn{2}{c}{Horn of Valhalla (Iron)} & \multicolumn{4}{c}{1 iron horn worth at least 500 gp 
1 scroll of spirit guardians 
1 scroll of conjure celestial 
1 very rare primal essence 
1 very rare divine essence} & 40 hours (5 days) & 20 & DC 20 & legendary & 33,800 gp \\
\midrule
\multicolumn{2}{c}{Robe of the Archmagi} & \multicolumn{4}{c}{1 white, gray, or black robe worth at least 500 gp 
1 scroll of mage armor 
1 scroll of antimagic field 
1 legendary arcane essences 
5 very rare arcane essences} & 40 hours (5 days) & 20 & DC 21 & legendary & 100,000 gp \\
\midrule
\multicolumn{2}{c}{Sphere of Annihilation} & \multicolumn{4}{c}{1 legendary arcane essence 
1 scroll of disintegrate 
1 scroll of levitate 
1 scroll of demiplane 
2 very rare arcane essences 
1 very rare divine essence} & 40 hours (5 days) & 20 & DC 21 & legendary & 81,000 gp \\
\midrule
\multicolumn{2}{c}{Well of Many Worlds} & \multicolumn{4}{c}{1 fine black cloth worth 100 gp 
1 scroll of plane shift 
1 scroll of demiplane 
1 legendary divine essence 
1 very rare arcane essence} & 40 hours (5 days) & 20 & DC 20 & legendary & 75,000 gp \\
\midrule
\end{tabularx}

\section{Enchanting}

\section*{Enchanting Crafting Tables}

\subsection*{Psionic Items}

\begin{tabularx}{\textwidth}\toprule
{}XXXXXXXXXX}
\midrule
\multicolumn{2}{c}{Name} & \multicolumn{4}{c}{Materials} & Crafting Time & Checks & Difficulty & Rarity & Value \\
\midrule
\multicolumn{2}{c}{Vision Stone} & \multicolumn{4}{c}{1 crystal worth at least 10 gp 
1 common psion essence} & 4 hours & 2 & DC 13 & common & 70 gp \\
\midrule
\multicolumn{2}{c}{+1 Amplifying Crystal} & \multicolumn{4}{c}{1 crystal worth at least 20 gp 
2 uncommon psionic essences 
2 common psionic essences} & 8 hours & 4 & DC 14 & uncommon & 480 gp \\
\midrule
\multicolumn{2}{c}{+1 Psi Blade Crystal} & \multicolumn{4}{c}{1 crystal worth at least 20 gp 
1 uncommon psionic essence 
1 common psionic essence 
1 common primal essence} & 8 hours & 4 & DC 14 & uncommon & 320 gp \\
\midrule
\multicolumn{2}{c}{Blasting Crystal} & \multicolumn{4}{c}{1 crystal worth at least 20 gp 
1 uncommon psionic essence 
1 uncommon primal essence} & 8 hours & 4 & DC 15 & uncommon & 410 gp \\
\midrule
\multicolumn{2}{c}{Focusing Crystal} & \multicolumn{4}{c}{1 crystal worth at least 20 gp 
2 uncommon psionic essences 
1 common psionic essence} & 8 hours & 4 & DC 15 & uncommon & 460 gp \\
\midrule
\multicolumn{2}{c}{Psionically Attuned Weapon} & \multicolumn{4}{c}{1 weapon 
1 uncommon psionic essence 
1 common psionic essence} & 4 hours & 2 & DC 13 & uncommon & 220 gp \\
\midrule
\multicolumn{2}{c}{Resonating Crystal} & \multicolumn{4}{c}{1 crystal worth at least 20 gp 
2 uncommon psionic essences} & 8 hours & 4 & DC 15 & uncommon & 410 gp \\
\midrule
\multicolumn{2}{c}{+2 Amplifying Crystal} & \multicolumn{4}{c}{1 crystal worth at least 100 gp 
2 rare psionic essences 
1 uncommon psionic essence} & 12 hours & 6 & DC 16 & rare & 2,000 gp \\
\midrule
\multicolumn{2}{c}{+2 Psi Blade Crystal} & \multicolumn{4}{c}{1 crystal worth at least 100 gp 
1 rare psionic essence 
1 rare primal essence 
2 uncommon psionic essences} & 12 hours & 6 & DC 16 & rare & 2,100 gp \\
\midrule
\multicolumn{2}{c}{Imprint Crystal} & \multicolumn{4}{c}{1 crystal worth at least 100 gp 
2 rare psionic essences 
2 uncommon psionic essences} & 12 hours & 6 & DC 16 & rare & 2,100 gp \\
\midrule
\multicolumn{2}{c}{Purity of Mind} & \multicolumn{4}{c}{1 blindfold worth 10 gp 
2 rare psionic essence 
1 uncommon divine essence} & 16 hours (2 days) & 8 & DC 16 & rare & 2,000 gp \\
\midrule
\multicolumn{2}{c}{+3 Amplying Crystal} & \multicolumn{4}{c}{1 crystal worth at least 200 gp 
4 very rare psionic essences 
2 rare psionic essences} & 24 hours (3 days) & 12 & DC 18 & very rare & 33,000 gp \\
\midrule
\multicolumn{2}{c}{+3 Psi Blade Crystal} & \multicolumn{4}{c}{1 crystal worth at least 200 gp 
2 very rare psionic essences 
1 very rare primal essence 
1 very rare arcane essence} & 24 hours (3 days) & 12 & DC 18 & very rare & 32,000 gp \\
\midrule
\multicolumn{2}{c}{Mind Shard} & \multicolumn{4}{c}{1 crystal worth at least 50 gp 
3 very rare psionic essences 
2 rare psionic essences} & 32 hours (4 days) & 16 & DC 19 & very rare & 26,000 gp \\
\midrule
\multicolumn{2}{c}{Soul Shard} & \multicolumn{4}{c}{1 crystal worth at least 20 gp 
1 psionic creature of CR 7 or higher* 
1 legendary psion essence 
2 very rare psionic essences} & 40 hours (5 days) & 20 & DC 20 & legendary & 55,000 gp \\
\midrule
\end{tabularx}

*The creature can be alive or dead, but must have died within 24 hours of starting the craft. On success, the creatures soul is consumed, and it can’t be resurrected while the soul shard exists. The discipline and effect can be tailored to the creature used as per the GM’s discretion.

\section{Enchanting}

\section*{Enchanting Crafting Tables}

\subsection*{Magic Weapons}

\begin{longtable}{p{2.5cm}\toprule
|p{2.5cm}|p{2.5cm}|p{2.5cm}|p{2.5cm}|p{2.5cm}|p{2.5cm}|p{2.5cm}|p{2.5cm}|p{2.5cm}|p{2.5cm}|}
\midrule
\multicolumn{2}{c}{Name} & \multicolumn{4}{c}{Materials} & Crafting Time & Checks & Difficulty & Rarity & Value \\
\midrule
\multicolumn{2}{c}{+1 Ammunition} & \multicolumn{4}{c}{1 piece of ammunition 
1 common arcane essence} & 2 hours & 1 & DC 14 & uncommon & 60 gp \\
\midrule
\multicolumn{2}{c}{+1 Weapon} & \multicolumn{4}{c}{1 weapon 
1 scroll of magic weapon 
2 common arcane essences 
2 uncommon arcane essences} & 8 hours & 4 & DC 14 & uncommon & 540 gp \\
\midrule
\multicolumn{2}{c}{Berserker Axe} & \multicolumn{4}{c}{1 axe 
1 scroll of crown of madness 
1 scroll of magic weapon 
1 scroll of aid 
2 uncommon primal essences} & 8 hours & 4 & DC 14 & uncommon & 500 gp \\
\midrule
\multicolumn{2}{c}{Dagger of the Ogre Mage (B)} & \multicolumn{4}{c}{1 shortsword (or dagger for a Large-sized creature) worth at least 50 gp 
1 scroll of magic weapon 
3 scrolls of cantrip 
1 uncommon arcane essence 
3 common arcane essences} & 12 hours & 6 & DC 15 & uncommon & 612 gp \\
\midrule
\multicolumn{2}{c}{Javelin of Lightning} & \multicolumn{4}{c}{1 javelin 
1 scroll of lightning bolt 
1 uncommon primal essence} & 8 hours & 4 & DC 15 & uncommon & 400 gp \\
\midrule
\multicolumn{2}{c}{Squall Caller (B)} & \multicolumn{4}{c}{1 battle axe 
1 scroll of windborne weapon(B) 
1 scroll of returning weapon(B) 
1 uncommon primal essence 
1 common arcane essence} & 8 hours & 4 & DC 15 & uncommon & 380 gp \\
\midrule
\multicolumn{2}{c}{Trident of Fish Command} & \multicolumn{4}{c}{1 trident 
1 scroll of dominate beast 
1 common primal essence 
1 uncommon primal essence} & 8 hours & 4 & DC 15 & uncommon & 560 gp \\
\midrule
\multicolumn{2}{c}{+2 Ammunition} & \multicolumn{4}{c}{1 piece of ammunition 
2 common arcane essences} & 2 hours & 1 & DC 16 & rare & 130 gp \\
\midrule
\multicolumn{2}{c}{+2 Weapon} & \multicolumn{4}{c}{1 weapon worth at least 100 gp 
1 scroll of magic weapon 
2 uncommon arcane essences 
2 rare arcane essences 
2 rare divine essences 
2 rare primal essences} & 16 hours (2 days) & 8 & DC 16 & rare & 5,300 gp \\
\midrule
\multicolumn{2}{c}{Dagger of Venom} & \multicolumn{4}{c}{1 +1 dagger 
1 potent injury poison 
1 scroll of nauseating poison(B) 
1 rare primal essence 
1 rare arcane essence} & 12 hours (1.5 days) & 6 & DC 16 & rare & 2,500 gp \\
\midrule
\multicolumn{2}{c}{Dragon Slayer} & \multicolumn{4}{c}{1 sword worth at least 1,000 gp 
2 rare primal essences 
1 rare divine essence 
1 scroll of bestow curse} & 16 hours (2 days) & 8 & DC 18 & rare & 4,400 gp \\
\midrule
\multicolumn{2}{c}{Flame Tongue} & \multicolumn{4}{c}{1 sword worth at least 100 gp 
1 scroll of prismatic weapon(B) 
1 flametongue oil(B) 
5 rare primal essences 
1 rare arcane essence} & 16 hours (2 days) & 8 & DC 18 & rare & 5,800 gp \\
\midrule
\multicolumn{2}{c}{Giant Slayer} & \multicolumn{4}{c}{1 axe or sword worth at least 100 gp 
1 scroll of magic weapon 
1 rare primal essence from a giant 
2 uncommon arcane essences 
2 uncommon primal essences} & 16 hours (2 days) & 8 & DC 16 & rare & 1,800 gp \\
\midrule
\multicolumn{2}{c}{Javelin of the Harpy Eagle (B)} & \multicolumn{4}{c}{1 javelin worth at least 100 gp 
1 scroll of returning weapon(B) 
1 rare primal essence 
1 uncommon arcane essence} & 12 hours (1.5 days) & 6 & DC 16 & rare & 1,300 gp \\
\midrule
\multicolumn{2}{c}{Mace of Disruption} & \multicolumn{4}{c}{1 mace worth at least 100 gp 
1 scroll of banishment 
2 rare divine essence 
1 uncommon divine essence 
1 uncommon arcane essence} & 16 hours (2 days) & 8 & DC 16 & rare & 2,400 gp \\
\midrule
\multicolumn{2}{c}{Mace of Smiting} & \multicolumn{4}{c}{1 mace worth at least 100 gp 
1 scroll of dispel construct(B) 
1 rare arcane essence 
1 uncommon arcane essence 
1 uncommon divine essence} & 12 hours (1.5 days) & 6 & DC 16 & rare & 1,500 gp \\
\midrule
\multicolumn{2}{c}{Mace of Terror} & \multicolumn{4}{c}{1 mace worth at least 100 gp 
1 scroll of fear 
1 rare arcane essence 
2 uncommon arcane essences} & 12 hours (1.5 days) & 6 & DC 17 & rare & 1,680 gp \\
\midrule
\multicolumn{2}{c}{Sun Blade} & \multicolumn{4}{c}{1 sword hilt worth 200 gp 
1 scroll of vorpal weapon(B) 
1 scroll of magic weapon 
1 scroll of daylight 
3 rare divine essences 
2 uncommon arcane essences 
2 uncommon primal essences} & 24 hours (3 days) & 12 & DC 17 & rare & 5,500 gp \\
\midrule
\multicolumn{2}{c}{Sword of Life Stealing} & \multicolumn{4}{c}{1 sword worth at least 200 gp 
1 scroll of vampiric touch 
1 rare arcane essence 
2 uncommon arcane essences} & 12 hours (1.5 days) & 6 & DC 16 & rare & 1,800 gp \\
\midrule
\multicolumn{2}{c}{Sword of Wounding} & \multicolumn{4}{c}{1 sword worth at least 200 gp 
1 scroll of decaying touch(B) 
1 scroll of rotting curse(B) 
2 rare arcane essences 
2 very rare poisonous reagents 
1 rare poisonous reagent} & 12 hours (1.5 days) & 6 & DC 17 & rare & 5,000 gp \\
\midrule
\multicolumn{2}{c}{Vicious Weapon} & \multicolumn{4}{c}{1 weapon worth at least 100 gp 
1 scroll of vorpal weapon(B) 
3 uncommon arcane essences} & 8 hours & 4 & DC 16 & rare & 1,650 gp \\
\midrule
\multicolumn{2}{c}{+3 Ammunition} & \multicolumn{4}{c}{1 piece of ammunition 
1 uncommon arcane essence 
1 scroll of magic weapon} & 4 hours & 2 & DC 18 & very rare & 585 gp \\
\midrule
\multicolumn{2}{c}{+3 Weapon} & \multicolumn{4}{c}{1 weapon worth at least 1,000 gp 
1 scroll of magic weapon 
1 scroll of prismatic weapon(B) 
1 scroll of vorpal weapon(B) 
1 very rare arcane essence 
1 very rare divine essence 
1 very rare primal essence} & 24 hours (3 days) & 12 & DC 18 & very rare & 26,000 gp \\
\midrule
\multicolumn{2}{c}{Arrow of Slaying} & \multicolumn{4}{c}{1 arrow 
1 scroll of bestow curse 
2 rare primal essences 
2 uncommon primal essences} & 6 hours & 3 & DC 18 & very rare & 2,500 gp \\
\midrule
\multicolumn{2}{c}{Bow of Magic Missiles (B)} & \multicolumn{4}{c}{1 bow (short or long) 
1 scroll of magic missile 
1 scroll of magic weapon 
3 rare arcane essences 
1 very rare arcane essence} & 16 hours (2 days) & 8 & DC 18 & very rare & 10,000 gp \\
\midrule
\multicolumn{2}{c}{Dancing Sword} & \multicolumn{4}{c}{1 sword worth 100 gp 
1 scroll of animate objects 
2 very rare arcane essences 
4 uncommon arcane essences} & 24 hours (3 days) & 12 & DC 19 & very rare & 18,500 gp \\
\midrule
\multicolumn{2}{c}{Dwarven Thrower} & \multicolumn{4}{c}{1 +3 warhammer 
1 scroll of weapon of throwing 
1 very rare primal essence} & 16 hours (2 days) & 8 & DC 18 & very rare & 37,000 gp \\
\midrule
\multicolumn{2}{c}{Frost Brand} & \multicolumn{4}{c}{1 sword worth at least 500 gp 
1 scroll of prismatic weapon(B) 
1 freezing oil 
1 very rare primal essence 
3 rare primal essences 
1 rare arcane essence} & 16 hours (2 days) & 8 & DC 19 & very rare & 11,600 gp \\
\midrule
\multicolumn{2}{c}{Nine Lives Stealer} & \multicolumn{4}{c}{1 sword worth at least 500 gp 
1 scroll of power word kill 
1 scroll of magic jar} & 16 hours (2 days) & 8 & DC 17 & very rare & 46,600 gp \\
\midrule
\multicolumn{2}{c}{Oathbow} & \multicolumn{4}{c}{1 longbow worth 500 gp 
4 very rare primal essences 
1 scroll of true strike 
1 scroll of hunter’s mark 
1 scroll of hex} & 24 hours (3 days) & 12 & DC 19 & very rare & 17,700 gp \\
\midrule
\multicolumn{2}{c}{Scimitar of Speed} & \multicolumn{4}{c}{1 scimitar worth 500 gp 
1 scroll of haste 
1 very rare arcane essence 
1 rare primal essence} & 16 hours (2 days) & 8 & DC 18 & very rare & 10,000 gp \\
\midrule
\multicolumn{2}{c}{Sword of Sharpness} & \multicolumn{4}{c}{1 sword worth at least 500 gp 
1 scroll of vorpal weapon 
1 very rare arcane essence 
1 rare arcane essence} & 16 hours (2 days) & 8 & DC 18 & very rare & 11,000 gp \\
\midrule
\multicolumn{2}{c}{Defender} & \multicolumn{4}{c}{1 sword worth at least 1,000 gp 
1 legendary divine essence 
1 scroll of magic weapon 
1 scroll of shield of faith 
1 +3 shield} & 32 hours (4 days) & 16 & DC 20 & legendary & 59,000 gp \\
\midrule
\multicolumn{2}{c}{Dragon Tamer Lance (B)} & \multicolumn{4}{c}{1 lance worth 1,000 gp 
1 scroll of summon dragon 
1 scroll of chromatic orb 
1 legendary primal essence 
2 very rare primal essences 
8 uncommon primal essences} & 32 hours (4 days) & 16 & DC 20 & legendary & 50,000 gp \\
\midrule
\multicolumn{2}{c}{Hammer of Thunderbolts} & \multicolumn{4}{c}{1 maul worth at least 1,000 gp 
1 legendary primal essence 
2 very rare primal essences 
1 scroll of thunderwave} & 40 hours (5 days) & 20 & DC 20 & legendary & 64,000 gp \\
\midrule
\multicolumn{2}{c}{Holy Avenger} & \multicolumn{4}{c}{1 sword worth at least 10,000 gp 
3 legendary divine essences 
1 scroll of holy weapon 
1 scroll of holy aura 
1 scroll of magic weapon 
3 very rare divine essences} & 40 hours (5 days) & 20 & DC 22 & legendary & 158,000 gp \\
\midrule
\multicolumn{2}{c}{Luck Blade} & \multicolumn{4}{c}{1 +1 sword 
2 scrolls of wish} & 32 hours (4 days) & 16 & DC 20 & legendary & 92,000 gp \\
\midrule
\end{longtable}

\section{Enchanting}

\section*{Enchanting Crafting Tables}

\subsection*{Magic Armor}

\begin{longtable}{p{2.5cm}\toprule
|p{2.5cm}|p{2.5cm}|p{2.5cm}|p{2.5cm}|p{2.5cm}|p{2.5cm}|p{2.5cm}|p{2.5cm}|p{2.5cm}|p{2.5cm}|}
\midrule
\multicolumn{2}{c}{Name} & \multicolumn{4}{c}{Materials} & Crafting Time & Checks & Difficulty & Rarity & Value \\
\midrule
\multicolumn{2}{c}{+1 Shield} & \multicolumn{4}{c}{1 shield 
1 scroll of shield 
1 scroll of shield of faith 
1 uncommon arcane essence 
1 uncommon divine essence} & 8 hours & 4 & DC 15 & uncommon & 500 gp \\
\midrule
\multicolumn{2}{c}{+1 Armor} & \multicolumn{4}{c}{1 set of armor 
1 scroll of mage armor 
1 scroll of shield 
1 scroll of shield of faith 
1 rare arcane essence 
1 rare divine essence 
1 rare primal essence} & 24 hours (3 days) & 12 & DC 16 & rare & 3,000 gp \\
\midrule
\multicolumn{2}{c}{+2 Shield} & \multicolumn{4}{c}{1 shield 
1 scroll of shield 
1 scroll of shield of faith 
1 scroll of glyph of warding 
2 rare arcane essences 
1 rare divine essence} & 16 hours (2 days) & 8 & DC 17 & rare & 3,200 gp \\
\midrule
\multicolumn{2}{c}{Armor of Resistance} & \multicolumn{4}{c}{1 set of armor 
1 scroll of protection from energy 
2 rare primal essences 
1 uncommon primal essence} & 16 hours (2 days) & 8 & DC 17 & rare & 2,400 gp \\
\midrule
\multicolumn{2}{c}{Arrow-Catching Shield} & \multicolumn{4}{c}{1 shield 
1 scroll of warding wind 
1 rare primal essence 
1 uncommon arcane essence} & 12 hours (1.5 days) & 6 & DC 17 & rare & 1,300 gp \\
\midrule
\multicolumn{2}{c}{Captain’s Coat (B)} & \multicolumn{4}{c}{1 fine long coat worth 100 gp 
1 scroll of enhance ability 
1 scroll of vicious mockery 
1 scroll of dancing wave(B) 
1 rare primal essence 
1 uncommon arcane essence} & 12 hours (1.5 days) & 6 & DC 16 & rare & 1,400 gp \\
\midrule
\multicolumn{2}{c}{Glamoured Studded Leather} & \multicolumn{4}{c}{1 +1 studded leather armor 
1 scroll of disguise self 
1 scroll of silent image 
1 rare arcane essence} & 16 hours (2 days) & 8 & DC 15 & rare & 4,200 gp \\
\midrule
\multicolumn{2}{c}{Scale Mail of the Pangolin (B)} & \multicolumn{4}{c}{1 set of scale mail 
1 large carapace 
1 scroll of move earth 
1 rare primal essence 
2 uncommon primal essences} & 12 hours (1.5 days) & 6 & DC 16 & rare & 3,000 gp \\
\midrule
\multicolumn{2}{c}{Shield of Missile Attraction} & \multicolumn{4}{c}{1 shield 
1 scroll of warding wind 
1 rare arcane essence 
1 uncommon primal essence} & 12 hours (1.5 days) & 6 & DC 16 & rare & 1,200 gp \\
\midrule
\multicolumn{2}{c}{+2 Armor} & \multicolumn{4}{c}{1 set of armor worth at least 2,000 gp 
1 scroll of globe of invulnerability 
1 scroll of stoneskin 
1 scroll of mage armor 
1 scroll of shield 
1 scroll of shield of faith 
1 very rare arcane essence 
1 rare divine essence 
1 rare primal essence} & 32 hours (4 days) & 16 & DC 20 & very rare & 17,000 gp \\
\midrule
\multicolumn{2}{c}{+3 Shield} & \multicolumn{4}{c}{1 shield worth at least 1,000 gp 
1 scroll of wall of stone 
1 scroll of wall of force
 1 scroll of wind wall 
1 scroll of shield 
1 scroll of shield of faith 
1 very rare arcane essence 
1 very rare divine essence 
2 rare arcane essences 
2 rare divine essences} & 24 hours (3 days) & 12 & DC 20 & very rare & 24,000 gp \\
\midrule
\multicolumn{2}{c}{Animated Shield} & \multicolumn{4}{c}{1 shield worth at least 200 gp 
1 scroll of animate objects 
1 very rare arcane essence 
2 rare arcane essences 
1 rare divine essence} & 24 hours (3 days) & 12 & DC 18 & very rare & 12,500 gp \\
\midrule
\multicolumn{2}{c}{Breastplate of the Golden Retriever (B)} & \multicolumn{4}{c}{1 ornate breastplate worth at least 500 gp 
1 scroll of warding bond 
1 scroll of faithful hound} & 16 hours (2 days) & 8 & DC 16 & very rare & 9,700 gp \\
\midrule
\multicolumn{2}{c}{Dark Fathom Armor (B)} & \multicolumn{4}{c}{1 set of studded leather armor 
1 scroll of mage armor 
1 scroll of shield 
1 scroll of black tentacles 
1 very rare arcane essence 
2 rare primal essences 
1 scroll of water breathing} & 24 hours (3 days) & 12 & DC 18 & very rare & 11,000 gp \\
\midrule
\multicolumn{2}{c}{Demon Armor} & \multicolumn{4}{c}{1 plate armor worth at least 1,500 gp 
1 scroll of summon fiend 
1 rare arcane essence} & 16 hours (2 days) & 8 & DC 15 & very rare & 4,000 gp \\
\midrule
\multicolumn{2}{c}{Dwarven Plate} & \multicolumn{4}{c}{1 plate armor with the dwarven modifier 
1 scroll of globe of invulnerability 
1 scroll of stoneskin 
1 scroll of mage armor 
1 scroll of shield 
1 scroll of shield of faith 
1 very rare arcane essence 
1 rare divine essence 
1 rare primal essence} & 32 hours (4 days) & 16 & DC 20 & very rare & 28,000 gp \\
\midrule
\multicolumn{2}{c}{Raiment of the Racoon (B)} & \multicolumn{4}{c}{1 set of studded leather 
1 scroll of enhance ability 
1 scroll of polymorph 
1 very rare primal essence 
1 rare arcane essence} & 16 hours (2 days) & 8 & DC 17 & very rare & 9,350 gp \\
\midrule
\multicolumn{2}{c}{Spellguard Shield} & \multicolumn{4}{c}{1 shield worth at least 200 gp 
1 scroll of antimagic field 
1 very rare divine essence 
1 rare arcane essence} & 32 hours (4 days) & 16 & DC 20 & very rare & 28,000 gp \\
\midrule
\multicolumn{2}{c}{+3 Armor} & \multicolumn{4}{c}{1 set of armor worth at least 4,000 gp 
1 scroll of invulnerability 
1 legendary arcane essence 
1 very rare primal essence 
1 very rare divine essence} & 48 hours (6 days) & 24 & DC 22 & legendary & 100,000 gp \\
\midrule
\multicolumn{2}{c}{Plate Armor of Invulnerability} & \multicolumn{4}{c}{1 set of plate armor worth at least 4,000 gp 
1 scroll of invulnerability 
1 scroll of stone skin 
1 legendary divine essence 
2 very rare arcane essences} & 48 hours (6 days) & 24 & DC 23 & legendary & 125,000 gp \\
\midrule
\multicolumn{2}{c}{Plate Armor of Etherealness} & \multicolumn{4}{c}{1 set of plate armor worth at least 1,500 gp 
1 scroll of etherealness 
1 legendary arcane essence 
1 very rare arcane essence} & 24 hours (3 days) & 12 & DC 19 & legendary & 50,000 gp \\
\midrule
\end{longtable}

\section{Enchanting}

\section*{Enchanting Crafting Tables}

\subsection*{Rings}

\begin{longtable}{p{2.5cm}\toprule
|p{2.5cm}|p{2.5cm}|p{2.5cm}|p{2.5cm}|p{2.5cm}|p{2.5cm}|p{2.5cm}|p{2.5cm}|p{2.5cm}|p{2.5cm}|}
\midrule
\multicolumn{2}{c}{Name} & \multicolumn{4}{c}{Materials} & Crafting Time & Checks & Difficulty & Rarity & Value \\
\midrule
\multicolumn{2}{c}{Ring of Jumping} & \multicolumn{4}{c}{1 ring worth at least 10 gp 
1 scroll of jump 
1 common primal essence} & 8 hours & 4 & DC 12 & uncommon & 140 gp \\
\midrule
\multicolumn{2}{c}{Ring of Mind Shielding} & \multicolumn{4}{c}{1 ring worth at least 20 gp 
1 scroll of protection from good and evil 
1 scroll of detect good and evil 
1 scroll of detect thoughts 
1 common psionic essence} & 12 hours (1.5 days) & 6 & DC 15 & uncommon & 350 gp \\
\midrule
\multicolumn{2}{c}{Ring of Swimming} & \multicolumn{4}{c}{1 ring 
1 scroll of alter self 
1 common primal essence} & 4 hours & 2 & DC 14 & uncommon & 150 gp \\
\midrule
\multicolumn{2}{c}{Ring of Warmth} & \multicolumn{4}{c}{1 ring 
1 scroll of create bonfire 
1 scroll of protection from energy 
2 common primal essences 
1 uncommon primal essence} & 16 hours (2 days) & 8 & DC 14 & uncommon & 580 gp \\
\midrule
\multicolumn{2}{c}{Ring of Water Walking} & \multicolumn{4}{c}{1 ring worth at least 10 gp 
1 scroll of water walking 
2 common primal essences} & 8 hours & 4 & DC 14 & uncommon & 400 gp \\
\midrule
\multicolumn{2}{c}{Ring of Animal Influence} & \multicolumn{4}{c}{1 ring worth at least 200 gp 
1 scroll of animal friendship 
1 scroll of fear 
1 scroll of speak with animals 
1 rare primal essence} & 12 hours (1.5 days) & 6 & DC 16 & rare & 1,600 gp \\
\midrule
\multicolumn{2}{c}{Ring of Evasion} & \multicolumn{4}{c}{1 ring worth at least 400 gp 
1 scroll of haste 
1 rare primal essence 
1 rare arcane essence} & 16 hours (2 days) & 8 & DC 17 & rare & 2,600 gp \\
\midrule
\multicolumn{2}{c}{Ring of Feather Falling} & \multicolumn{4}{c}{1 ring worth at least 50 gp 
1 scroll of feather fall 
1 scroll of levitate 
1 uncommon primal essence 
1 uncommon arcane essence} & 8 hours & 4 & DC 14 & rare & 500 gp \\
\midrule
\multicolumn{2}{c}{Ring of Free Action} & \multicolumn{4}{c}{1 ring worth at least 200 gp 
1 scroll of freedom of movement 
2 rare divine essences 
2 rare arcane essences 
1 rare primal essence} & 24 hours (3 days) & 12 & DC 17 & rare & 5,000 gp \\
\midrule
\multicolumn{2}{c}{Ring of Protection} & \multicolumn{4}{c}{1 ring worth at least 400 gp 
1 scroll of shield of faith 
1 scroll of mage armor 
1 scroll of protection from energy 
1 scroll of shield 
1 scroll of false life 
1 rare arcane essence 
1 rare divine essence} & 16 hours (2 days) & 8 & DC 17 & rare & 3,500 gp \\
\midrule
\multicolumn{2}{c}{Ring of Resistance} & \multicolumn{4}{c}{1 ring 
1 gem worth 50 gp 
1 scroll of protection from energy 
1 common primal essence 
1 uncommon primal essence 
1 rare primal essence} & 16 hours (2 days) & 8 & DC 16 & rare & 1,500 gp \\
\midrule
\multicolumn{2}{c}{Ring of Spell Storing} & \multicolumn{4}{c}{1 ring worth at least 400 gp 
1 empty wizard’s spell book (50 gp) 
4 rare arcane essences 
4 uncommon arcane essences 
4 common arcane essences} & 16 hours (2 days) & 8 & DC 17 & rare & 5,000 gp \\
\midrule
\multicolumn{2}{c}{Ring of the Ram} & \multicolumn{4}{c}{1 ring worth at least 200 gp 
1 scroll of galebolt(B) 
1 scroll of shatter 
1 rare arcane essence 
1 uncommon primal essence} & 12 hours (1.5 days) & 6 & DC 15 & rare & 1,400 gp \\
\midrule
\multicolumn{2}{c}{Ring of X-Ray Vision} & \multicolumn{4}{c}{1 ring worth at least 200 gp 
1 scroll of true seeing 
1 scroll of find traps 
1 uncommon arcane essence} & 12 hours (1.5 days) & 6 & DC 16 & rare & 2,300 gp \\
\midrule
\multicolumn{2}{c}{Ring of Regeneration} & \multicolumn{4}{c}{1 ring worth at least 400 gp 
1 scroll of regeneration 
1 rare divine essence 
1 rare arcane essence 
1 rare primal essence} & 16 hours (2 days) & 8 & DC 18 & very rare & 16,600 gp \\
\midrule
\multicolumn{2}{c}{Ring of Shooting Stars} & \multicolumn{4}{c}{1 ring worth at least 400 gp 
1 scroll of field of stars(B) 
1 very rare arcane essence 
1 rare divine essence} & 16 hours (2 days) & 8 & DC 18 & very rare & 11,000 gp \\
\midrule
\multicolumn{2}{c}{Ring of Telekinesis} & \multicolumn{4}{c}{1 ring worth at least 400 gp 
1 scroll of telekinesis 
1 very rare arcane essence 
1 very rare psionic essence} & 16 hours (2 days) & 8 & DC 19 & very rare & 18,250 gp \\
\midrule
\multicolumn{2}{c}{Ring of Djinni Summoning} & \multicolumn{4}{c}{1 ring worth at least 400 gp 
1 scroll of gate 
1 scroll of conjure elemental 
1 very rare primal essence
 the true name of a djinni} & 24 hours (3 days) & 12 & DC 20 & legendary & 56,000 gp \\
\midrule
\multicolumn{2}{c}{Ring of Elemental Command} & \multicolumn{4}{c}{1 ring worth at least 400 gp 
1 scroll of dominate monster 
1 scroll of conjure elemental 
1 legendary primal essence 
3 very rare primal essences} & 24 hours (3 days) & 12 & DC 19 & legendary & 55,000 gp \\
\midrule
\multicolumn{2}{c}{Ring of Invisibility} & \multicolumn{4}{c}{1 ring worth at least 400 gp 
1 scroll of invisibility 
1 legendary arcane essence 
1 very rare arcane essence} & 32 hours (4 days) & 16 & DC 22 & legendary & 50,000 gp \\
\midrule
\multicolumn{2}{c}{Ring of Spell Turning} & \multicolumn{4}{c}{1 ring worth at least 400 gp 
1 scroll of antimagic field 
1 scroll of counterspell 
1 legendary arcane essence} & 32 hours (4 days) & 16 & DC 22 & legendary & 57,000 gp \\
\midrule
\multicolumn{2}{c}{Ring of Three Wishes} & \multicolumn{4}{c}{1 ring 
3 scrolls of wish} & 16 hours (2 days) & 8 & DC 18 & legendary & 133,333 gp \\
\midrule
\end{longtable}

\section{Enchanting}

\section*{Enchanting Crafting Tables}

\subsection*{Magical Necklaces}

\begin{tabularx}{\textwidth}\toprule
{}XXXXXXXXXX}
\midrule
\multicolumn{2}{c}{Name} & \multicolumn{4}{c}{Materials} & Crafting Time & Checks & Difficulty & Rarity & Value \\
\midrule
\multicolumn{2}{c}{Amulet of Proof against Detection and Location} & \multicolumn{4}{c}{1 amulet 
1 scroll of nondetection} & 8 hours & 4 & DC 14 & uncommon & 350 gp \\
\midrule
\multicolumn{2}{c}{Brooch of Shielding} & \multicolumn{4}{c}{1 brooch or amulet worth 10 gp 
1 scroll of magic missile 
1 scroll of shield 
1 scroll of protection from energy} & 6 hours & 3 & DC 14 & uncommon & 444 gp \\
\midrule
\multicolumn{2}{c}{Medallion of Thoughts} & \multicolumn{4}{c}{1 medallion worth 25 gp 
1 scroll of detect thoughts 
1 uncommon arcane essence 
1 uncommon psionic essence} & 8 hours & 4 & DC 15 & uncommon & 500 gp \\
\midrule
\multicolumn{2}{c}{Necklace of Adaptation} & \multicolumn{4}{c}{1 necklace worth at least 25 gp 
1 scroll of protection from poison 
1 common divine essence} & 6 hours & 3 & DC 12 & uncommon & 170 gp \\
\midrule
\multicolumn{2}{c}{Periapt of Health} & \multicolumn{4}{c}{1 necklace worth 50 gp 
1 scroll of purify food and drink 
1 scroll of lesser restoration 
2 common divine essences} & 12 hours (1.5 days) & 6 & DC 12 & uncommon & 325 gp \\
\midrule
\multicolumn{2}{c}{Periapt of Wound Closure} & \multicolumn{4}{c}{1 amulet worth at least 50 gp 
1 uncommon divine essence 
1 scroll of cure wounds 
1 scroll of prayer of healing} & 8 hours & 4 & DC 14 & uncommon & 450 gp \\
\midrule
\multicolumn{2}{c}{Savage Talisman (B)} & \multicolumn{4}{c}{1 necklace worth at least 10 gp 
1 scroll of alter self 
2 common arcane essences 
2 uncommon arcane essences} & 8 hours & 4 & DC 14 & uncommon & 550 gp \\
\midrule
\multicolumn{2}{c}{Amulet of Heath} & \multicolumn{4}{c}{1 amulet worth 200 gp 
1 rare divine essence 
1 rare primal essence} & 16 hours (2 days) & 8 & DC 18 & rare & 2,500 gp \\
\midrule
\multicolumn{2}{c}{Necklace of Prayer Beads *} & \multicolumn{4}{c}{6 gems worth 50 gp each 
1 scroll of planar ally 
1 scroll of wind walk 
1 scroll of branding smite 
1 scroll of greater restoration 
1 scroll of cure wounds 
1 scroll of lesser restoration 
1 scroll of bless 
6 rare divine essences} & 12 hours (1.5 days) & 6 & DC 16 & rare & 10,000 gp \\
\midrule
\multicolumn{2}{c}{Amulet of the Planes} & \multicolumn{4}{c}{1 amulet worth 650 gp 
1 scroll of plane shift 
1 rare arcane essence 
1 very rare divine essence} & 24 hours (3 days) & 12 & DC 18 & very rare & 23,400 gp \\
\midrule
\multicolumn{2}{c}{Scarab of Protection} & \multicolumn{4}{c}{1 scarab shaped medallion worth at least 500 gp 
1 scroll of holy aura 
1 very rare divine essence} & 24 hours (3 days) & 12 & DC 19 & legendary & 25,000 gp \\
\midrule
\multicolumn{2}{c}{Talisman of Pure Good} & \multicolumn{4}{c}{1 talisman worth at least 1,000 gp 
1 legendary divine essence from a good aligned source 
1 scroll of fissure(B) 
1 scroll of gate} & 40 hours (5 days) & 20 & DC 21 & legendary & 88,000 gp \\
\midrule
\multicolumn{2}{c}{Talisman of Ultimate Evil} & \multicolumn{4}{c}{1 talisman worth at least 1,000 gp 
1 legendary divine essence from an evil-aligned source 
1 scroll of fissure(B) 
1 scroll of gate} & 40 hours (5 days) & 20 & DC 21 & legendary & 88,000 gp \\
\midrule
\multicolumn{2}{c}{Talisman of the Sphere} & \multicolumn{4}{c}{1 talisman worth at least 1,000 gp 
1 legendary arcane essence} & 24 hours (3 days) & 12 & DC 22 & legendary & 37,000 gp \\
\midrule
\end{tabularx}

* Necklaces of Prayer Beads crafted in this way have all the possible beads.

\section{Enchanting}

\section*{Enchanting Crafting Tables}

\subsection*{Staves}

\begin{tabularx}{\textwidth}\toprule
{}XXXXXXXXXX}
\midrule
\multicolumn{2}{c}{Name} & \multicolumn{4}{c}{Materials} & Crafting Time & Checks & Difficulty & Rarity & Value \\
\midrule
\multicolumn{2}{c}{Staff of the Python} & \multicolumn{4}{c}{1 uncommon branch 
1 scroll of conjure animals 
1 common primal essence} & 8 hours & 4 & DC 12 & uncommon & 350 gp \\
\midrule
\multicolumn{2}{c}{Staff of Charming} & \multicolumn{4}{c}{1 rare branch 
2 rare divine essences 
2 rare primal essences 
4 uncommon arcane essences 
1 scroll of command 
1 scroll of comprehend language 
1 scroll of charm person} & 16 hours (2 days) & 8 & DC 18 & rare & 4,600 gp \\
\midrule
\multicolumn{2}{c}{Staff of Healing} & \multicolumn{4}{c}{1 rare branch 
3 rare divine essences 
3 uncommon divine essences 
1 scroll of mass cure wounds 
1 scroll of cure wounds 
1 scroll of lesser restoration} & 12 hours (1.5 days) & 6 & DC 18 & rare & 5,000 gp \\
\midrule
\multicolumn{2}{c}{Staff of Swarming Insects} & \multicolumn{4}{c}{1 rare branch 
1 scroll of giant insect 
1 scroll of insect plague 
1 rare primal essences 
2 uncommon primal essences 
1 uncommon divine essence} & 12 hours (1.5 days) & 6 & DC 16 & rare & 3,200 gp \\
\midrule
\multicolumn{2}{c}{Staff of the Woodlands} & \multicolumn{4}{c}{1 rare branch 
4 rare primal essences 
8 uncommon primal essences} & 16 hours (2 days) & 8 & DC 18 & rare & 5,100 gp \\
\midrule
\multicolumn{2}{c}{Staff of Withering} & \multicolumn{4}{c}{1 rare branch 
1 uncommon primal essence 
1 uncommon arcane essence 
1 scroll of blight} & 8 hours & 4 & DC 15 & rare & 780 gp \\
\midrule
\multicolumn{2}{c}{Staff of Fire} & \multicolumn{4}{c}{1 very rare branch 
1 ruby worth 500 gp 
3 very rare primal essences 
6 rare primal essences 
1 scroll of burning hands 
1 scroll of fireball 
1 scroll of wall of fire} & 16 hours (2 days) & 8 & DC 19 & very rare & 31,000 gp \\
\midrule
\multicolumn{2}{c}{Staff of Frost} & \multicolumn{4}{c}{1 very rare branch 
1 sapphire worth 500 gp 
1 very rare primal essence 
1 very rare arcane essence 
4 rare primal essences 
2 rare arcane essences 
1 scroll of cone of cold 
1 scroll of fog cloud 
1 scroll of icestorm 
1 scroll of wall of ice} & 16 hours (2 days) & 8 & DC 19 & very rare & 26,000 gp \\
\midrule
\multicolumn{2}{c}{Staff of Power} & \multicolumn{4}{c}{1 +2 quarterstaff 
1 diamond worth 500 gp 
1 legendary arcane essence 
1 scroll of cone of cold 
1 scroll of fireball 
1 scroll of globe of invulnerability 
1 scroll of hold monster 
1 scroll of levitate 
1 scroll of lightning bolt 
1 scroll of magic missile 
1 scroll of ray of enfeeblement 
1 scroll of wall of force 
10 rare arcane essences} & 32 hours (4 days) & 16 & DC 20 & very rare & 50,000 gp \\
\midrule
\multicolumn{2}{c}{Staff of Striking} & \multicolumn{4}{c}{1 +3 quarterstaff 
10 rare arcane essences} & 16 hours (2 days) & 8 & DC 18 & very rare & 37,000 gp \\
\midrule
\multicolumn{2}{c}{Staff of Thunder and Lightning} & \multicolumn{4}{c}{1 very rare branch 
1 very rare primal essence 
1 scroll of lightning bolt 
1 scroll of thunder pulse(B)} & 16 hours (2 days) & 8 & DC 16 & very rare & 9,300 gp \\
\midrule
\multicolumn{2}{c}{Staff of the Magi} & \multicolumn{4}{c}{1 legendary branch 
2 legendary arcane essences 
1 wizard’s spellbook containing all the spells of a staff of magi 
4 very rare arcane essences} & 40 hours (5 days) & 20 & DC 22 & legendary & 114,000 gp \\
\midrule
\end{tabularx}

\section{Enchanting}

\section*{Enchanting Crafting Tables}

\subsection*{Rods}

\begin{tabularx}{\textwidth}\toprule
{}XXXXXXXXXX}
\midrule
\multicolumn{2}{c}{Name} & \multicolumn{4}{c}{Materials} & Crafting Time & Checks & Difficulty & Rarity & Value \\
\midrule
\multicolumn{2}{c}{Immovable Rod} & \multicolumn{4}{c}{1 rod worth at least 100 gp 
1 scroll of gravity surge(B) 
4 uncommon arcane essences} & 8 hours & 4 & DC 15 & uncommon & 800 gp \\
\midrule
\multicolumn{2}{c}{Rod of the Pact Keeper, +1} & \multicolumn{4}{c}{1 rod worth at least 100 gp
 Either 
(a) 1 entrapped humanoid soul
 or 
(b) 3 uncommon arcane essences} & 8 hours & 4 & DC 15 & uncommon & 650 gp \\
\midrule
\multicolumn{2}{c}{Rod of Rulership} & \multicolumn{4}{c}{1 rod worth at least 500 gp 
1 scroll of command 
1 scroll of charm person 
1 scroll of suggestion 
1 scroll of charm monster 
2 rare arcane essences} & 16 hours (2 days) & 8 & DC 17 & rare & 3,100 gp \\
\midrule
\multicolumn{2}{c}{Rod of the Pact Keeper, +2} & \multicolumn{4}{c}{1 rod worth at least 500 gp
 Either 
(a) 3 entrapped humanoid souls of CR/Level 5 or higher
 or 
(b) 3 rare arcane essences} & 16 hours (2 days) & 8 & DC 17 & rare & 3,300 gp \\
\midrule
\multicolumn{2}{c}{Tentacle Rod} & \multicolumn{4}{c}{1 rod worth at least 500 gp 
1 scroll of black tentacle 
3 tentacles at least 5 feet long 
2 rare arcane essences} & 8 hours & 4 & DC 16 & rare & 2,500 gp \\
\midrule
\multicolumn{2}{c}{Rod of Absorption} & \multicolumn{4}{c}{1 rod worth at least 3,000 gp 
1 scroll of spelltrap(B) 
1 scroll of counterspell 
2 rare arcane essences} & 8 hours & 4 & DC 16 & very rare & 7,500 gp \\
\midrule
\multicolumn{2}{c}{Rod of Alertness} & \multicolumn{4}{c}{1 rod worth at least 3,000 gp 
1 scroll of alarm 
1 scroll of detect evil and good 
1 scroll of detect magic 
1 scroll of detect poison and disease 
1 scroll of see invisibility 
3 rare arcane essences} & 16 hours (2 days) & 8 & DC 17 & very rare & 6,400 gp \\
\midrule
\multicolumn{2}{c}{Rod of Security} & \multicolumn{4}{c}{1 rod worth at least 5,000 gp 
1 scroll of demiplane 
1 very rare divine essence} & 24 hours (3 days) & 12 & DC 18 & very rare & 29,000 gp \\
\midrule
\multicolumn{2}{c}{Rod of the Pact Keeper, +3} & \multicolumn{4}{c}{1 rod worth at least 5,000 gp
 Either 
(a) 1 entrapped soul of a devil or demon CR 15 or higher 
2 very rare arcane essences
 or 
(b) 4 very rare arcane essence} & 24 hours (3 days) & 12 & DC 19 & very rare & 38,000 gp \\
\midrule
\multicolumn{2}{c}{Rod of Lordly Might} & \multicolumn{4}{c}{1 rod worth at least 10,000 gp 
1 scroll of magic weapon 
1 scroll of prismatic weapon(B) 
1 scroll of fear 
1 scroll of hold monster 
1 scroll of vampiric touch 
1 very rare primal essence 
1 very rare arcane essence 
1 very rare divine essence 
1 +3 mace or +3 battleaxe} & 40 hours (5 days) & 20 & DC 22 & legendary & 84,000 gp \\
\midrule
\multicolumn{2}{c}{Rod of Resurrection} & \multicolumn{4}{c}{1 rod worth at least 10,000 gp 
1 scroll of revivify 
1 scroll of raise dead 
1 scroll of resurrection 
1 scroll of true resurrection 
1 very rare divine essence 
1 legendary divine essence} & 80 hours (10 days) & 40 & DC 24 & legendary & 120,000 gp \\
\midrule
\end{tabularx}

\section{Enchanting}

\section*{Enchanting Crafting Tables}

\subsection*{Magical Manuals \& Tomes}

\begin{tabularx}{\textwidth}\toprule
{}XXXXXXXXXX}
\midrule
\multicolumn{2}{c}{Name} & \multicolumn{4}{c}{Materials} & Crafting Time & Checks & Difficulty & Rarity & Value \\
\midrule
\multicolumn{2}{c}{Manual of Golems*} & \multicolumn{4}{c}{1 blank book worth 250 gp 
1 scroll of awaken 
1 scroll of scroll of animate objects 
2 very rare arcane essences 
1 rare divine essence} & 24 hours (3 days) & 12 & DC 19 & very rare & 12,400 gp \\
\midrule
\multicolumn{2}{c}{Manual of Bodily Health} & \multicolumn{4}{c}{1 blank book worth 500 gp 
1 scroll of enhance ability 
1 legendary primal essence 
1 legendary divine essence} & 40 hours (5 days) & 20 & DC 21 & legendary & 68,500 gp \\
\midrule
\multicolumn{2}{c}{Manual of Gainful Exercise} & \multicolumn{4}{c}{1 blank book worth 500 gp 
1 scroll of enhance ability 
2 legendary primal essences} & 40 hours (5 days) & 20 & DC 21 & legendary & 68,500 gp \\
\midrule
\multicolumn{2}{c}{Manual of Quickness of Action} & \multicolumn{4}{c}{1 blank book worth 500 gp 
1 scroll of enhance ability 
1 legendary arcane essence 
1 legendary divine essence} & 40 hours (5 days) & 20 & DC 21 & legendary & 68,500 gp \\
\midrule
\multicolumn{2}{c}{Tome of Clear Thought} & \multicolumn{4}{c}{1 blank book worth 500 gp 
1 scroll of enhance ability 
2 legendary arcane essences} & 40 hours (5 days) & 20 & DC 21 & legendary & 68,500 gp \\
\midrule
\multicolumn{2}{c}{Tome of Leadership and Influence} & \multicolumn{4}{c}{1 blank book worth 500 gp 
1 scroll of enhance ability 
2 legendary divine essences} & 40 hours (5 days) & 20 & DC 21 & legendary & 68,500 gp \\
\midrule
\multicolumn{2}{c}{Tome of Understanding} & \multicolumn{4}{c}{1 blank book worth 500 gp 
1 scroll of enhance ability 
1 legendary divine essence 
1 legendary primal essence} & 40 hours (5 days) & 20 & DC 21 & legendary & 68,500 gp \\
\midrule
\end{tabularx}

*Note that this creates the item, not the golem. Creating the golem requires a sum of gold pieces and time beyond the item. When making a manual of golems in this way, you can pick which golem type it is for.

\section{Enchanting}

\section*{Enchanting Crafting Tables}

\subsection*{Ioun Stones}

\begin{tabularx}{\textwidth}\toprule
{}XXXXXXXXXX}
\midrule
\multicolumn{2}{c}{Name} & \multicolumn{4}{c}{Materials} & Crafting Time & Checks & Difficulty & Rarity & Value \\
\midrule
\multicolumn{2}{c}{Ioun Stone (Awareness)} & \multicolumn{4}{c}{1 dark blue gem worth at least 200 gp 
1 scroll of enhance ability 
1 rare primal essence 
1 rare arcane essence} & 16 hours (2 days) & 8 & DC 17 & rare & 2,260 gp \\
\midrule
\multicolumn{2}{c}{Ioun Stone (Protection)} & \multicolumn{4}{c}{1 rose gem worth at least 200 gp 
1 scroll of mage armor 
3 rare arcane essences 
1 rare divine essence} & 16 hours (2 days) & 8 & DC 17 & rare & 3,800 gp \\
\midrule
\multicolumn{2}{c}{Ioun Stone (Reserve)} & \multicolumn{4}{c}{1 purple gem worth at least 200 gp 
5 rare arcane essences 
4 uncommon arcane essences} & 16 hours (2 days) & 8 & DC 17 & rare & 5,100 gp \\
\midrule
\multicolumn{2}{c}{Ioun Stone (Sustenance)} & \multicolumn{4}{c}{1 clear gem worth at least 200 gp 
1 scroll of create food and water 
2 rare divine essences 
1 uncommon arcane essence} & 16 hours (2 days) & 8 & DC 17 & rare & 2,290 gp \\
\midrule
\multicolumn{2}{c}{Ioun Stone (Absorption)} & \multicolumn{4}{c}{1 pale lavender gem worth at least 500 gp 
1 scroll of dispel magic 
1 very rare arcane essence 
1 very rare divine essence} & 24 hours (3 days) & 12 & DC 19 & very rare & 18,000 gp \\
\midrule
\multicolumn{2}{c}{Ioun Stone (Agility)} & \multicolumn{4}{c}{1 deep red gem worth at least 500 gp 
1 scroll of enhance ability 
2 very rare primal essences 
2 very rare divine essences} & 24 hours (3 days) & 12 & DC 19 & very rare & 33,000 gp \\
\midrule
\multicolumn{2}{c}{Ioun Stone (Fortitude)} & \multicolumn{4}{c}{1 pink gem worth at least 500 gp 
1 scroll of enhance ability 
2 very rare divine essences 
2 very rare primal essences} & 24 hours (3 days) & 12 & DC 19 & very rare & 33,000 gp \\
\midrule
\multicolumn{2}{c}{Ioun Stone (Insight)} & \multicolumn{4}{c}{1 incandescent blue gem worth at least 500 gp 
1 scroll of enhance ability 
2 very rare divine essences 
2 very rare primal essences} & 24 hours (3 days) & 12 & DC 19 & very rare & 33,000 gp \\
\midrule
\multicolumn{2}{c}{Ioun Stone (Intellect)} & \multicolumn{4}{c}{1 incandescent blue gem worth at least 500 gp 
1 scroll of enhance ability 
3 very rare arcane essences 
1 very rare primal essence} & 24 hours (3 days) & 12 & DC 19 & very rare & 33,000 gp \\
\midrule
\multicolumn{2}{c}{Ioun Stone (Leadership)} & \multicolumn{4}{c}{1 pink gem at least 500 gp 
1 scroll of enhance ability 
4 very rare divine essences} & 24 hours (3 days) & 12 & DC 19 & very rare & 33,000 gp \\
\midrule
\multicolumn{2}{c}{Ioun Stone (Strength)} & \multicolumn{4}{c}{1 pale blue gem at least 500 gp 
1 scroll of enhance ability 
4 very rare primal essences} & 24 hours (3 days) & 12 & DC 19 & very rare & 33,000 gp \\
\midrule
\multicolumn{2}{c}{Ioun Stone (Greater Absorption)} & \multicolumn{4}{c}{1 lavender gem worth at least 500 gp 
1 green gem worth at least 500 gp 
2 very rare divine essences 
2 very rare arcane essences 
1 scroll of antimagic sphere} & 40 hours (5 days) & 20 & DC 21 & legendary & 50,000 gp \\
\midrule
\multicolumn{2}{c}{Ioun Stone (Mastery)} & \multicolumn{4}{c}{1 green gem worth at least 1,000 gp 
1 scroll of enhance ability 
1 legendary divine essence 
1 very rare arcane essence 
1 very rare primal essence} & 40 hours (5 days) & 20 & DC 21 & legendary & 55,800 gp \\
\midrule
\multicolumn{2}{c}{Ioun Stone (Regeneration)} & \multicolumn{4}{c}{1 pearl worth at least 1,000 gp 
1 scroll of regeneration 
2 very rare divine essences 
2 rare divine essences} & 40 hours (5 days) & 20 & DC 21 & legendary & 32,800 gp \\
\midrule
\end{tabularx}

\section{Enchanting}

\section*{Enchanting Crafting Tables}

\subsection*{Infused Gems}

\begin{tabularx}{\textwidth}\toprule
{}XXXXXXXXXX}
\midrule
\multicolumn{2}{c}{Name} & \multicolumn{4}{c}{Materials} & Crafting Time & Checks & Difficulty & Rarity & Value \\
\midrule
\multicolumn{2}{c}{Brilliant Diamond} & \multicolumn{4}{c}{1 diamond worth at least 25 gp 
1 common arcane essence} & 4 hours & 2 & DC 14 & Common & 95 gp \\
\midrule
\multicolumn{2}{c}{Effervescent Emerald} & \multicolumn{4}{c}{1 emerald worth at least 25 gp 
1 common primal essence} & 4 hours & 2 & DC 13 & Common & 70 gp \\
\midrule
\multicolumn{2}{c}{Flickering Ruby} & \multicolumn{4}{c}{1 ruby worth at least 50 gp 
1 common primal essence} & 4 hours & 2 & DC 14 & Common & 125 gp \\
\midrule
\multicolumn{2}{c}{Glittering Garnet} & \multicolumn{4}{c}{1 garnet worth at least 100 gp 
1 common divine essence 
1 uncommon arcane essence} & 6 hours & 3 & DC 16 & Uncommon & 420 gp \\
\midrule
\multicolumn{2}{c}{Magic Diamond} & \multicolumn{4}{c}{1 diamond worth at least 100 gp 
1 uncommon arcane essence} & 6 hours & 3 & DC 16 & Uncommon & 370 gp \\
\midrule
\multicolumn{2}{c}{Perfect Infusion} & \multicolumn{4}{c}{1 cut gem worth at least 50 gp 
1 common arcane essence 
1 common primal essence} & 6 hours & 3 & DC 18 & Uncommon & 400 gp \\
\midrule
\multicolumn{2}{c}{Sparkling Sapphire} & \multicolumn{4}{c}{1 sapphire worth at least 100 gp 
1 uncommon arcane essence} & 6 hours & 3 & DC 15 & Uncommon & 320 gp \\
\midrule
\end{tabularx}

\section{Enchanting}

\section*{Magic Items}

% [Image Inserted Manually]

\subsection*{Arcblade}

Weapon (longsword), very rare (requires attunement)

This item appears to be a longsword hilt. While grasping the hilt, you can use a bonus action to cause a blade of dark lightning to spring into existence, or make the blade disappear. While the blade exists, this magic longsword has the finesse property. If you are proficient with shortswords or longswords, you are proficient with the arcblade.

You gain a +2 bonus to attack and damage rolls made with this weapon, which deals lightning damage instead of slashing damage. When you hit a creature with it, you can cause lightning to arc from the target to a different creature of your choice that you can see within 5 feet of it. The second creature takes 1d8 lightning damage.

The arcblade is able to absorb ambient electrical energy to briefly enter a supercharged state. If you take lightning damage while this weapon’s blade is active, you can use your reaction to capture some of that energy and store it for up to 1 minute. The first time you hit with the arcblade before the end of the duration, the target takes additional lightning damage equal to half of the damage you took. This property can’t be used again until the next dusk.

\begin{itemize}
  \item Item: Arcblade (B)
\end{itemize}

\subsection*{Boarding Party}

Weapon (any ammunition), uncommon

This is a magical piece of ammunition carved with pirate motifs. This piece of ammunition changes size to fit the size of the weapon used (becoming large when loaded into a large weapon). You can use this piece of ammunition to cast translocating shot(B) once without expending a spell, targeting this piece of ammunition. Once used in this way, it can’t be used again until the next dawn.

\begin{itemize}
  \item Item: Boarding Party (B)
\end{itemize}

\subsection*{Bow of Magic Missiles}

Weapon (any bow), very rare (requires attunement by a creature that is proficient with shortbows or longbows)

You gain a +1 bonus to attack and damage rolls made with this magic weapon. In addition, it has 5 charges. While holding it, you can expend 1 charge as an action and draw the bowstring, causing two gleaming arrows of magical force to materialize. You immediately fire each arrow at a creature that you can see within 600 feet of you. You can direct the arrows at one creature or multiple, and each arrow automatically hits its target, striking simultaneously.

You can choose to expend additional charges as part of the same action to fire one extra arrow per charge expended. Each arrow deals force damage equal to 1d6 + your proficiency bonus. The bow regains 1d4 + 1 expended charges daily at dawn. If you expend the bow’s last charge, roll a d20. On a 1, the bow retains its +1 bonus to attack and damage rolls but loses all other properties.

Item by Bakku

\begin{itemize}
  \item Item: Longbow of Magic Missiles (B)
  \item Item: Shortbow of Magic Missiles (B)
\end{itemize}

% [Image Inserted Manually]

\subsection*{Breastplate of the Golden Retriever}

Armor (breastplate), very rare (requires attunement)

You have a +1 bonus to AC and advantage on saving throws to resist being frightened while wearing this armor.

This armor has 3 charges, which it regains daily at dawn. As an action, you can expend 1 charge to cast the warding bond spell or 2 charges to cast the faithful hound spell from the armor without requiring material components. When warding bond is cast in this way, a spectral golden retriever appears next to the creature you are bonded with and follows it dutifully. The golden retriever is incorporeal, and it can shed bright light in a 20-foot radius and dim light for an additional 20 feet upon the bonded creature’s request (no action required).

\begin{itemize}
  \item Item: Breastplate of the Golden Retriever (B)
\end{itemize}

% [Image Inserted Manually]

\subsection*{Captain's Coat}

Armor (studded leather), rare (requires attunement)

Functional and fashionable, this long pirate captain’s coat is reinforced for combat and draws the eyes to its ornate design. While wearing this coat, you have proficiency in navigator’s tools, vehicles (water) and you have a swimming speed equal to your walking speed.

Multiple pockets are hidden on the inside of the coat, able to conceal small objects. A creature attempting to search your person must succeed on a DC 15 Intelligence (Investigation) check in order to discover the hidden pockets.

Sailor’s Mouth. Once per short or long rest, you can use a bonus action to cast the vicious mockery spell (save DC 15) while wearing the coat. When cast in this way, the target takes extra psychic damage equal to your Charisma modifier.

\begin{itemize}
  \item Item: Captain's Coat (B)
\end{itemize}

\subsection*{Cascade Catalyst}

Wondrous Item, rare (requires attunement)

While holding this catalyst, you can use it as spell casting focus for your spells. When you roll damage for a spell, for each die that rolls its highest value, you can roll one additional die of the same type and add it to the damage dealt.

\begin{itemize}
  \item Item: Cascade Catalyst (B)
\end{itemize}

\subsection*{Crystal Key}

Wondrous item, very rare

An intricate crystalline doorknob with intricate carved patterns. This crystal contains a psionically constructed space that can be manifested by placing it against any flat vertical surface large enough for a Medium-sized door to appear on and speaking the command word. It creates a door that leads into this interdimensional space. The size of this space depends on the size of the object the doorknob is placed again. If the space is at least 10 feet long by 10 feet wide, it creates a simply furnished bedroom with a bed, light, chair, and desk. The atmosphere in the room is comfortable and the air is breathable.

For each additional 10 cubic feet of space in the object the doorknob is placed in, additional rooms appear, in the following order:

\begin{itemize}
  \item A bathroom furnished with wash tub and running water (the water is potable).
  \item A comfortable study furnished with a writing table, chair, bookshelves and fireplace.
  \item A workshop with basic artisan tools and work benches.
  \item A kitchen that contains no food, but all basic cooking tools, and an always hot oven.
  \item Additional bedrooms, up to four additional rooms.
  \item Anything created by the extradimensional space can’t leave the extradimensional space.
\end{itemize}

These rooms are extradimensional and do not truly exist within the object and can’t be accessed by any means other than through the door (they do not structurally affect the object, or allow you to access entrances or areas within the object).

\begin{itemize}
  \item Item: Crystal Key (B)
\end{itemize}

% [Image Inserted Manually]

\subsection*{Crystal Ball of Thieving}

Wondrous item, legendary (requires attunement)

This crystal ball is about 6 inches in diameter. While touching it, you can cast the scrying spell (save DC 17) with it.

You can use an action to cast the mage hand spell while you are scrying with the crystal ball, the spectral hand appearing within 30 feet of the spell’s sensor. When cast in this way you can make the spectral hand invisible. When the scrying ends, so does the spell, and any object held by the spectral hand is teleported into your open hand or to an unoccupied space at your feet. Teleporting an object via this method is unreliable and has a chance to fail. Roll a d20 to determine if you are successful. If you have had physical contact with the object before add 5 to your roll. If the object has spent more than 24 hours in your possession, add 10. The DC is 15. On a failure, the object reappears in the space that it occupied before the spectral hand interacted with it, or in the nearest unoccupied space if that space is now occupied. Once used, the thieving power of the crystal ball can’t be used again until the next dawn.

Curse. Each time you successfully steal an item using this crystal ball, the owner of the item immediately sees an image of your face and hears your name in their mind. They don’t necessarily know the context, but if your theft arouses suspicion, it’s a likely deduction that you are related to the disappearance of their possession.

\begin{itemize}
  \item Item: Crystal Ball of Thieving (B)
\end{itemize}

% [Image Inserted Manually]

\subsection*{Dagger of the Ogre Mage}

Weapon (shortsword), uncommon (requires attunement)

This blade was once used as a dagger by an ogre mage but is nevertheless large enough to be a shortsword. You gain a +1 bonus to attack and damage rolls made with this magic weapon.

This blade’s hilt is hollow and can hold up to 3 cantrip spell scrolls. Cantrip scrolls left within the compartment for 1 minute activate one of the three clear gems adorning its grip. Once activated, you can cast these cantrips at will using your spellcasting modifier. If you don’t have a spellcasting modifier, use your Intelligence modifier instead. You have proficiency with these cantrips. Cantrips cast from the blade use the weapon as an arcane focus and allow you to perform any somatic components with the weapon instead of requiring a free hand. Scrolls contained within the sword are not destroyed after using them in this way.

When you cast a cantrip using the sword that deals damage, the sword’s damage type changes to match the cantrip’s for 1 minute or until you end it early (no action required).

\begin{itemize}
  \item Item: Dagger of the Ogre Mage (B)
\end{itemize}

% [Image Inserted Manually]

\subsection*{Dark Fathom Armor}

Armor (studded leather), very rare (requires attunement)

This dark, sea-soaked suit of armor smells of brine and is covered in eldritch runes. You have a +1 bonus to AC and can breathe normally underwater while wearing this armor.

In addition, while wearing this armor, you can use an action to cast the black tentacles spell (save DC 15) from it. When cast in this way, the spell targets a 15-foot square centered on your location that does not follow you when you move. In addition, you do not need to concentrate on the spell and are immune to its effects when cast in this way. Once this property has been used, it can’t be used again until the next dusk.

\begin{itemize}
  \item Item: Dark Fathom Armor (B)
\end{itemize}

\subsection*{Decombobulator}

Wondrous item, rare

This small mysterious device can emit a projected field within 5 feet of it that heats matter until it undergoes a state change, from a solid to a liquid, or from a liquid to a gas. It has no effect on gas. The target area is always in the shape of a cube and can be as small as a 1-inch cube or as large as a 5-foot cube. The process takes 1 minute. If a creature ends its turn in the projected field, it takes 2d6 fire damage. This damage doubles each consecutive turn it ends in the area. This damage is also taken by any creature attempting to hold or touch something in the area of effect. This field may burn away the effects of some spells that would be subject to its effects at the discretion of the GM.

\begin{itemize}
  \item Item: Decombobulator (B)
\end{itemize}

% [Image Inserted Manually]

\subsection*{Doodle Ring}

Ring, common

This magic wooden ring is always covered in splotches of charcoal, ink, or paint, regardless of how well or often it’s cleaned. While wearing the ring on a finger, you can touch that finger to a solid surface and begin to draw. The drawing uses your finger to make marks as if it were a piece of charcoal, ink quill, or paintbrush (your choice). The marks you make can be in any color. Any drawing made by the ring can be easily smudged or washed away without leaving any marks behind. The ring can have up to a total of 5 square feet covered with drawings in this way at any time. Drawing more than that begins to erase the previous marks, and any mark left after 24 hours of being made is magically erased.

\begin{itemize}
  \item Item: Doodle Ring (B)
\end{itemize}

% [Image Inserted Manually]

\subsection*{Dragon Tamer Lance}

Weapon (lance), legendary (requires attunement)

This slender, iridescent lance is covered in an enchanted finish that resembles the hardened scales of many dragons. You gain a +1 bonus to attack and damage rolls made with this magic weapon. While attuned to the weapon, you can understand and speak Draconic.

This lance has 8 charges and regains 1d4 + 4 expended charges each day at dawn. When you hit a target with the lance, you can expend 1 of its charges to deal an extra 3d6 acid, cold, fire, lightning, or poison damage (your choice) to the target. If you expend a charge in this way when you hit a dragon, it must also succeed on a DC 17 Wisdom saving throw or become frightened of you until the beginning of your next turn. For the purpose of this weapon, “dragon” refers to any creature with the dragon type, including dragon turtles and wyverns.

Whenever you expend a charge in this way to deal extra damage, you can choose to expend a 2nd charge and deal an extra 1d6 damage of the same type to the target and prevent it from using its reaction until the start of its next turn.

Wyrmling Steed. In addition, while holding the lance, you can use an action to cast the find steed spell from it. Once this property has been used, it can’t be used again until the next dawn. When you use the lance to cast this spell, you can choose to summon a dragon wyrmling instead of a normal mount by expending a number of charges equal to its challenge rating (maximum 2). You determine the kind of dragon summoned, although its alignment towards good or evil must be the same as your own. If your alignment is neutral, you can summon a wyrmling of either alignment. The wyrmling counts as one size larger when used as a mount and cannot use its breath weapons.

Forceful Impact. If you move at least 20 feet straight toward a Medium or smaller target and then make a melee attack with the lance against it while within 5 feet of the target, you make the attack with advantage instead of disadvantage. If you hit, you can immediately roll to attack another target 5 feet directly behind the first, without advantage, as part of the initial attack.

\begin{itemize}
  \item Item: Dragon Tamer Lance (B)
\end{itemize}

\begin{minipage}{0.48\textwidth}
\subsubsection*{Essence Crystal}

Wondrous item, rare (requires attunement)

A translucent crystal capsule that contains a small amount of a mysterious dark colored liquid.

While in possession of this crystal, an attuned creature gains a +1 bonus to to ability checks and saving throws of the crystals attribute type.

These crystals can be cracked, and the liquid inside makes a potent essence for forging powerful runes.

Curse. While attuned to the crystal, you have vivid dreams of lives that are not your own, seeing glimpses of the souls that contributed their essence to the liquid within the crystal.

\begin{itemize}
  \item Item: Essence Crystal (B) (Deep Crimson)
  \item Item: Essence Crystal (B) (Forest Green)
  \item Item: Essence Crystal (B) (Amber Orange)
  \item Item: Essence Crystal (B) (Honey Yellow)
  \item Item: Essence Crystal (B) (Azure Blue)
  \item Item: Essence Crystal (B) (Rich Purple)
\end{itemize}
\end{minipage}\hfill
\begin{minipage}{0.48\textwidth}
\begin{tabularx}{\textwidth}\toprule
{}XXXXXXXXXX}
\midrule
\multicolumn{11}{c}{Essence Crystal Color \& Attribute} \\
\midrule
1d6 & \multicolumn{5}{c}{Essence Color} & \multicolumn{5}{c}{Attribute} \\
\midrule
1 & \multicolumn{5}{c}{Deep Crimson} & \multicolumn{5}{c}{Strength} \\
\midrule
2 & \multicolumn{5}{c}{Forest Green} & \multicolumn{5}{c}{Dexterity} \\
\midrule
3 & \multicolumn{5}{c}{Amber Orange} & \multicolumn{5}{c}{Constitution} \\
\midrule
4 & \multicolumn{5}{c}{Honey Yellow} & \multicolumn{5}{c}{Wisdom} \\
\midrule
5 & \multicolumn{5}{c}{Azure Blue} & \multicolumn{5}{c}{Intelligence} \\
\midrule
6 & \multicolumn{5}{c}{Rich Purple} & \multicolumn{5}{c}{Charisma} \\
\midrule
\end{tabularx}

\begin{itemize}
  \item Table: Essence Crystal Color \& Attribute
\end{itemize}
\end{minipage}

\subsection*{Galvanic Spine}

Weapon (whip), rare (requires attunement)

A segmented metallic whip that crackles with lightning. You have a +1 bonus to attack and damage rolls made with this magic whip, and it deals an extra 1d4 lightning damage on hit.

This whip is highly articulated and can move with surgical precision, allowing it to serve as an extended reach for certain tasks. As an action, you can attempt to grapple a creature, entangling with the whip. When you do so, you can make a Dexterity (Athletics) check to initiate and maintain the grapple (instead of Strength). A creature that starts its turn while grappled with this whip takes 1d4 lightning damage.

Additionally, you can pick up objects weighing 10 pounds or less within 10 feet, pulling them toward you, or interact with objects within 10 feet. You can use the whip as a set of thieves’ tools +1 when interacting with objects within 10 feet.

\begin{itemize}
  \item Item: Galvanic Spine (B)
\end{itemize}

% [Image Inserted Manually]

\subsection*{Helm of Heroes}

Wondrous item, rare (requires attunement)

You gain a +1 bonus to AC if you’re wearing no armor while wearing this helmet. You can still use a shield and gain this benefit. In addition, while wearing no armor, the helm becomes suffused with glory and glimmers for 1 minute whenever you roll a 20 on an attack roll. While the helm is glimmering in this way, you gain temporary hit points at the start of each of your turns equal to half your level, and friendly creatures within 10 feet of you can’t be frightened.

\begin{itemize}
  \item Item: Helm of Heroes (B)
\end{itemize}

\subsection*{Hero’s Sheath}

Wondrous item, rare (requires attunement by a paladin)

An ornate sheath built for a longsword; this sheath infuses a blade stored within with divine power. When you draw a weapon stored in this sheath, the first attack you make with it after drawing it forth has a range of 30/60 (using the weapons normal attack modifier) and deals radiant damage instead of slashing damage.

If you expend a spell slot to empower this special attack using your Divine Smite feature, the blade emits a holy blast that deals damage equal to the Divine Smite targeting all creatures within 15 feet. The range of the cone increases by 15 feet for each level of spell slot expended on the Divine Smite. The target of the attack takes the Divine Smite damage as normal if the attack hits, while all other creatures in the area must make a Dexterity saving throw with a DC of 16. On failure, they take radiant damage equal to the damage dice of the Divine Smite damage (this is not doubled if the attack is a critical hit), or half as much damage on a successful saving throw.

A longsword that is sheathed in this sheath for at least 24 hours is infused with divine power. Its damage die is increased by one step (from a d8 to a d10, and its versatile damage from a d10 to a d12) and it counts as magical for the purposes of overcoming resistance to damage. This bonus lasts until another weapon is sheathed in this scabbard.

\begin{itemize}
  \item Item: Hero's Sheath (B)
\end{itemize}

\subsection*{Holy Symbol of the Arcane}

Wondrous item, rare (requires attunement by a cleric)

This is a special holy symbol often belonging to the order of a god dedicated to knowledge or magic, imbued with special power by the god or their agents with the ability to confer arcane magic to a disciple.

When an attuned user is preparing spells for the day, they can prepare spells from the wizard spell list in addition to the cleric spell list, but any spell prepared in this way is prepared at one level higher than it normally would be. For example, if they were to use this holy symbol to prepare the fireball spell from the wizard list, it would be prepared as if were normally a 4th-level spell (not gaining benefits from being up-cast as a 4th-level spell).

An attuned cleric can have a number of spells prepared this way equal to their Wisdom modifier.

\begin{itemize}
  \item Item: Holy Symbol of the Arcane (B)
\end{itemize}

\subsection*{Hulking Bracers}

Wondrous item, uncommon (requires attunement)

While wearing these bracers, as an action you can gain the Enlarge effect from the enlarge/reduce spell without requiring a spell slot or concentration. The effect lasts until deactivated as a bonus action. You can remain enlarged for up to 1 hour per day, all at once or in several shorter uses. If you run out of time while enlarged, you shrink to your normal size. The bracers regain their 1 hour of usage at dawn each day.

\begin{itemize}
  \item Item: Hulking Bracers (B)
\end{itemize}

% [Image Inserted Manually]

\subsection*{Javelin of the Harpy Eagle}

Weapon (javelin), rare (requires attunement)

This barbed javelin embeds itself into creatures and hinders their movement. A creature hit by this javelin has its speed reduced by 10 feet until the javelin is removed. While you are within 120 feet of the javelin and you can see it, you can use a bonus action to speak its command word, causing the javelin to viciously remove itself from the target and fly back to your open hand or to an unoccupied space at your feet. You can choose to remove the javelin as a part of the attack while wielding it in melee. Alternatively, any creature can remove the javelin with an action. When the javelin is removed, the barbs deal an extra 1d6 piercing damage to the creature.

\begin{itemize}
  \item Item: Javelin of the Harpy Eagle (B)
\end{itemize}

% [Image Inserted Manually]

\subsection*{Lost Era Emanator}

Wondrous item, very rare (requires attunement)

This ancient device is fueled by a mysterious power source. It hums to life and glows softly when held. The device’s arcane design is nearly inscrutable and requires careful examination to decipher its function. In order to attune to the device, you must first succeed on a DC 16 Arcana check, otherwise the attunement fails, and you can’t attempt it again for 24 hours.

The device has 12 charges. While holding it, you can use the attack action to emit a ray of concentrated heat from the device, expending 1 charge for each attack. The device functions as a ranged weapon with the ammunition and two-handed properties, a normal range of 60 feet, and a long range of 240 feet. While you are attuned to it, you can add your proficiency bonus to attack rolls with the device. Additionally, when you attack with the device, you can use your Intelligence modifier, instead of Dexterity, for the attack rolls. The device deals 3d8 fire damage (this damage doesn’t benefit from ability score modifiers) to any target it hits.

As an action, you can expend 3 charges to fire a supercharged ray from the device. The ray is 90 feet long and 6 inches wide. Any creature within this area must make a DC 16 Dexterity saving throw and take 8d8 fire damage on a failure, or half damage on a success. This ray is able to penetrate through 1 inch of metal, 6 inches of stone, or 1 foot of wood, leaving a scorched hole 6 inches in diameter.

The device regains charges by storing arcane energy. Any creature can cast a spell of 1st through 6th level into the device by touching it as the spell is cast. The spell has no effect, other than to be stored in the device. If the device can’t hold the spell, the spell is expended without effect. The device regains 1 charge for each level of spell slot used.

Orville examined the strange device that he had just unearthed. It gleamed and glinted in the torchlight, despite its apparent age, and the craftsmanship was like nothing he had ever seen. He traced his finger along the intricate grooves and amber-like appendages, daydreaming about the people who could have crafted such a curious contraption.

\begin{itemize}
  \item Item: Lost Era Emanator (B)
\end{itemize}

% [Image Inserted Manually]

\subsection*{Raiment of the Raccoon}

Armor (studded leather), very rare (requires attunement by a rogue)

You have a +1 bonus to AC while wearing this armor. While wearing it, you are imbued with the hardiness of a raccoon. Your rogue feature Evasion now applies to Constitution saving throws as well.

Keen Sense of Touch. While wearing this armor you develop a hypersensitive tactile awareness. You can add your Wisdom modifier to Dexterity (Sleight of Hand) checks, thieves’ tools checks and Intelligence (Investigation) checks that use your sense of touch. You can reliably detect the surface details, such as engraved writing or material properties, of an object by touching it.

Curse. While attuned to this armor you become obsessively hygienic. You must spend at least 10 minutes washing your hands to benefit from a short or long rest and you prefer to dunk your food into water before eating it.

\begin{itemize}
  \item Item: Raiment of the Raccoon (B)
\end{itemize}

\subsection*{Savage Talisman}

Wondrous item, uncommon (requires attunement)

While wearing the talisman, you gain a +1 bonus to the attack rolls and the damage rolls you make with unarmed strikes and natural weapons. Such attacks are considered to be magical.

\begin{itemize}
  \item Item: Savage Talisman (B)
\end{itemize}

% [Image Inserted Manually]

\subsection*{Scale Mail of the Pangolin}

Armor (scale mail), rare (requires attunement)

While wearing this armor you gain a burrowing speed of 10 feet. You can use your burrowing speed to move through sand, loose earth, mud, or ice, not solid rock.

This armor can adjust itself to provide superior protection, allowing you to curl up into a ball as an action. Until you emerge (on your turn, no action required), you gain a +5 bonus to AC, and you have advantage on Strength and Constitution saving throws. While curled up, you are prone, your speed is 0 and can’t increase, you have disadvantage on Dexterity saving throws, and you are incapacitated. Attack rolls made against you do not have advantage normally granted by the prone condition. Additionally, a creature can roll you along the ground in front of itself if it succeeds on a DC 10 Strength check (no action required).

\begin{itemize}
  \item Item: Scale Mail of the Pangolin (B)
\end{itemize}

% [Image Inserted Manually]

\subsection*{Shawm of Sundering}

Wondrous item, uncommon (requires attunement by a bard)

When a creature adds one of your Bardic Inspiration dice to an attack or damage roll, and they hit a target within 60 feet of you, you can play the shawm as a reaction to deal thunder damage equal to one roll of your bardic inspiration die to the target.

Additionally, you can play the shawm as an action to cast thunderwave, shatter, or lightning bolt, and that spell can’t be cast from it again until the next dawn.

\begin{itemize}
  \item Item: Shawm of Sundering (B)
\end{itemize}

% [Image Inserted Manually]

\subsection*{Squall Caller}

Weapon (battleaxe), uncommon (requires attunement)

This battleaxe has the thrown property with a normal range of 20 feet and a long range of 60 feet. It has 3 charges. While holding the battleaxe, you can use an action and expend 1 charge to cast gust of wind (save DC 13) from it. The Squall Caller regains all expended charges daily at dawn.

Amplified. If the Squall Collar is within 60 feet of this battleaxe, it gains the following properties:

\begin{itemize}
  \item You gain a +1 bonus to attack and damage rolls made with this magic weapon.
  \item Immediately after being thrown, the battleaxe flies back to your hand.
  \item The saving throw DC for spells cast from the battleaxe increases to 15.
  \item When you cast gust of wind from the battleaxe, the distance pushed increases by 5 feet, and you can choose any number of creatures within the area of effect to automatically succeed on their saving throw. Additionally, you can choose to cast gust of wind without requiring concentration, but the spell ends at the start of your next turn.
\end{itemize}

\begin{itemize}
  \item Item: Squall Caller (B)
\end{itemize}

% [Image Inserted Manually]

\subsection*{Squall Collar}

Wondrous item, rare (requires attunement)

This cold steel choker is forged into the shape of a roiling storm. It gives you a static shock when touched. It has 3 charges. While wearing the collar, you can use an action and expend 1 charge to cast sleet storm (save DC 15) from it. The Squall Collar regains all expended charges daily at dawn.

Amplified. If the Squall Caller is within 60 feet of this collar, it gains the following properties:

\begin{itemize}
  \item Wearing the collar allows you to see normally through heavily obscured areas caused by weather or spells such as sleet storm or fog cloud.
  \item The saving throw DC for spells cast from the collar increases to 16.
  \item When you are targeted with an attack that deals thunder or lightning damage, you can use your reaction to gain resistance to the triggering damage type until the start of your next turn.
  \item Additionally, the Squall Caller deals an extra 1d8 of the triggering damage type for 1 minute. Once the collar is used in this way, the ability cannot be used again until the next dawn.
\end{itemize}

\begin{itemize}
  \item Item: Squall Collar (B)
\end{itemize}

\begin{minipage}{0.48\textwidth}
\subsubsection*{Quantum Chaos Box}

Wondrous item, rare

This is a mysterious box that harnesses primordial chaos to manipulate aspects of reality within it, creating random chaotic things. When opening this box, roll a 1d12 to determine the results. If opened again, the box is empty until the next dawn.

Additional Effects

An ambitious GM can replace any effect rolled with a new effect (either positive or negative) to keep the box indefinitely chaotic. Ideas that might lead to further adventures can even be included. Imagine the ramifications of opening the box and finding a lich’s phylactery.

\begin{itemize}
  \item Item: Quantum Chaos Box(B)
\end{itemize}
\end{minipage}\hfill
\begin{minipage}{0.48\textwidth}
\begin{tabularx}{\textwidth}\toprule
{}XXXXXXXXXXXXXXXX}
\midrule
\multicolumn{17}{c}{Quantum Chaos Box Contents} \\
\midrule
\multicolumn{2}{c}{1d12} & \multicolumn{15}{c}{Contents of Box} \\
\midrule
\multicolumn{2}{c}{1} & \multicolumn{15}{c}{1d4 hostile swarms of rats under the effect of fire shield} \\
\midrule
\multicolumn{2}{c}{2} & \multicolumn{15}{c}{1d12 berries created by the goodberry spell.} \\
\midrule
\multicolumn{2}{c}{3} & \multicolumn{15}{c}{A small piece of a star. It explodes violently. All creatures within 20 feet must make a DC 15 Dexterity saving throw or take 3d12 fire damage, or half as much on a successful save.} \\
\midrule
\multicolumn{2}{c}{4} & \multicolumn{15}{c}{The last consumable item you used, of uncommon rarity or less, or a potion of healing (if no consumable item applies).} \\
\midrule
\multicolumn{2}{c}{5} & \multicolumn{15}{c}{A small wooden toy. It looks quite old.} \\
\midrule
\multicolumn{2}{c}{6} & \multicolumn{15}{c}{A key. Someone probably lost it.} \\
\midrule
\multicolumn{2}{c}{7} & \multicolumn{15}{c}{1d4 delicious looking fresh fish. Cooking and eating them gives a creature up to 10 temporary hit points, as long as they are cooked within 1 hour of opening the box.} \\
\midrule
\multicolumn{2}{c}{8} & \multicolumn{15}{c}{A diamond worth 1d100 gp.} \\
\midrule
\multicolumn{2}{c}{9} & \multicolumn{15}{c}{Another box that looks identical to the Quantum Chaos Box, just slightly smaller. It is a hostile mimic.} \\
\midrule
\multicolumn{2}{c}{10} & \multicolumn{15}{c}{A letter written in an incomprehensible language. If somehow decoded it might be a mundane letter, or a spell scroll for a spell on the Warlock spell list of 3rd level or lower.} \\
\midrule
\multicolumn{2}{c}{11} & \multicolumn{15}{c}{Nothing} \\
\midrule
\multicolumn{2}{c}{12} & \multicolumn{15}{c}{A single bean from a magical bag of beans.} \\
\midrule
\end{tabularx}

\begin{itemize}
  \item Table: Quantum Chaos Box Contents
\end{itemize}
\end{minipage}

Additional Effects

An ambitious GM can replace any effect rolled with a new effect (either positive or negative) to keep the box indefinitely chaotic. Ideas that might lead to further adventures can even be included. Imagine the ramifications of opening the box and finding a lich’s phylactery.

\subsection*{Theurge’s Librum}

Wondrous items, very rare (requires attunement by a wizard)

This hefty ornate spell book remains pristine in all conditions, its cover depicting detailed sacred inscriptions. After studying the contents of this book for 8 hours, the user can comprehend a unifying theory of divine and arcane power, though such a comprehension mystically fades from the mind if they ever lose possession or un-attune from the librum, until they are once more in possession and attuned to the librum.

When found, the librum contains the following spells: cure wounds, bless, lesser restoration, gentle repose, revivify, spirit guardians, commune, guardian of faith, dawn, and mass cure wounds. The spells contained within can be prepared as Wizard spells for the attuned user but can’t be copied out of the spell book. Casting these divine spells exerts a special strain on a wizard, however, and they can only cast a total number of spell slots of spells that do not appear on the Wizard list from this book equal to their Wizard level, after which they must finish a long rest, refreshing the number of spells they can cast.

In addition, it contains 10 empty pages, each of which can contain one additional spell from the cleric list. If you come across a cleric spell in written form (such as a scroll) the spell can be permanently copied into the librum, expending one of the empty pages. The process of copying the spell into your ritual book takes 2 hours per level of the spell, and costs 50 gp per level. The cost represents the material components you expend as you experiment with the spell to master it, as well as the fine inks you need to record it.

\begin{itemize}
  \item Item: Theurge's Librum (B)
\end{itemize}

% [Image Inserted Manually]

\subsection*{Yves’ Thieves’ Tools}

Wondrous item, uncommon (requires attunement)

You have a +1 bonus to ability checks using these thieves’ tools. As an action you can hold these tools up to a door to hear through it, as long as it’s no thicker than 1 foot, alerting you to the location of any creatures within 20 feet of the door. If you set off a trap while using these tools to pick a lock, you have advantage on any saving throws to resist the trap.

\begin{itemize}
  \item Item: Yves' Thieves' Tools (B)
\end{itemize}

\section{Scrollscribing}

\section*{Scrollscribing}

Scrollscribing is the process of creating magical scrolls. Critical both as useful ways to cast spells and as the core of the magic formula used in enchanting, every adventuring group’s magic blokes are well served by the ability to make scrolls.

\section*{Related Tool \& Ability Score}

\begin{minipage}{0.48\textwidth}
Scrollscribing works using Calligraphy Tools. Attempting to craft a scroll without these is impossible.

The related ability score is Intelligence. While spellcasters of any stripe can make scrolls of the spells they know, the process is one of systematic application of magical theory to lay down the spell in a function that can later be used.

Additionally, like its kin Wand Whittling and Enchanting, proficiency in Arcana is required; without proficiency in arcana, you can’t add your Tool proficiency to the crafting roll.
\end{minipage}\hfill
\begin{minipage}{0.48\textwidth}
Scriptures

A GM can choose to allow a cleric to use Wisdom in place of Intelligence and require proficiency in Religion instead of Arcana. Scrolls produced this way are called Scriptures and can only be used by others of the same faith.
\end{minipage}

\section*{Quick Reference}

While each step will go into more depth, the quick reference allows you to at a glance follow the steps to make a scroll in its most basic form:

\begin{itemize}
  \item Select a spell you know that you would like to craft from spells you are able to cast, or through Alternate Methods (see “Magic Formula”).
  \item Acquire the items listed in the materials column for a scroll of that level and type.
  \item Use your Calligraphy Tools to write the scroll using the number of hours listed in the Crafting Time column, or during a long rest using the crafting camp action if the crafting time is 2 hours or less.
  \item For every 2 hours, make a crafting roll of 1d20 + your Intelligence modifier + your proficiency with Calligraphy Tools (if you have proficiency in the Arcana skill)
  \item On success, you mark 2 hours of completed time. Once the completed time is equal to the crafting time, the magic item is complete. On failure, the crafting time is lost, and no progress has been made during the 2 hours. If you fail 3 times in a row, the crafting is a failure, and all materials are lost.
\end{itemize}

\section*{Materials: Ink \& Parchment}

\begin{minipage}{0.48\textwidth}
The materials for Scrollscribing are Ink and Parchment. Ink and Parchment used in scrolls is typically purchased, and below are the price tables. Some types of rare parchment may be processed from rare alchemical ingredients by an alchemist or from the hides of magical creatures by a leatherworker. If they are found as part of treasure, they are calculated as any other precious non-currency treasure would be calculated.

The ink used to create scrolls must be a special formulation that allows it to contain the magical essence behind the glyphs, script, runes, and words that make up a magical scroll. This ink is created by alchemists, but can be purchased at the below rates:
\end{minipage}\hfill
\begin{minipage}{0.48\textwidth}
\begin{tabularx}{\textwidth}\toprule
{}XX}
\midrule
\multicolumn{2}{c}{Component} & Price \\
\midrule
\multicolumn{2}{c}{Common Magical Ink} & 15 gp \\
\midrule
\multicolumn{2}{c}{Common Parchment} & 1 sp \\
\midrule
\multicolumn{2}{c}{Uncommon Magical Ink} & 40 gp \\
\midrule
\multicolumn{2}{c}{Uncommon Parchment} & 40 gp \\
\midrule
\multicolumn{2}{c}{Rare Magical Ink} & 200 gp \\
\midrule
\multicolumn{2}{c}{Rare Parchment} & 200 gp \\
\midrule
\multicolumn{2}{c}{Very Rare Magical Ink} & 2,000 gp \\
\midrule
\multicolumn{2}{c}{Very Rare Parchment} & 2,000 gp \\
\midrule
\multicolumn{2}{c}{Legendary Magical Ink} & 5,000 gp \\
\midrule
\multicolumn{2}{c}{Legendary Parchment} & 5,000 gp \\
\midrule
\end{tabularx}
\end{minipage}

\section*{Crafting Roll}

\begin{minipage}{0.48\textwidth}
When you would like to create a scroll, it will depend on your Intelligence and skill with Calligrapher’s tools:

Scrollcrafting Modifier = your Calligrapher’s Tools proficiency bonus + your Intelligence modifier.
\end{minipage}\hfill
\begin{minipage}{0.48\textwidth}
\subsubsection*{Success and Failure}

After making a crafting roll, if you succeed, you make 2 hours of progress toward the total crafting time (and have completed one of the required checks for making an item).

Checks for Scrollscribing do not need to be immediately consecutive. If you fail three times in a row, all progress and materials are lost and can no longer be salvaged. Failure means that no progress is made during that time.

Once an item is started, even if no progress is made, the components reserved for that item can only be recovered via salvage.
\end{minipage}

\subsection*{Crafting Without Essence}

\begin{minipage}{0.48\textwidth}
A crafter that is capable of casting magic can replace the essence when crafting a spell scroll with the ability to cast that spell, but when doing so they must cast that spell for each crafting check they make on that item. This is an exhausting process, draining their magic far more deeply than normal casting, and confers a level of exhaustion for each crafting check made this way.
\end{minipage}\hfill
\begin{minipage}{0.48\textwidth}
A Difficult Process

This is intentionally a difficult process, as stockpiling spell scrolls is something that should be challenging, otherwise magic can end up trivializing many encounters, and this method of crafting removes a large potential cost barrier.
\end{minipage}

\section*{Magical Formula}

To craft a spell scroll, you must know the Magic Formula of the spell you want to make a Spell Scroll of. The easiest way to do this is to be able to cast the spell. You always know the Magic Formula of a spell you know how to cast. Otherwise, you need to have deep knowledge of the spell to be able to make a scroll of it. The following are some ways you can gain that knowledge:

\begin{itemize}
  \item Have it in your spellbook as a Wizard.
  \item Have it in your spell manual as an Infusionsmith Inventor.
  \item Have it in your ritual book as a ritual caster.
  \item Have a spell scroll of it (DC +2)
  \item Study its magical formula and record it. To learn a spell in this way requires proficiency in arcana and 1 day (8 hours per day) of study per level of the spell, as well as access to a teacher or book that records the spell. Once learned, you can record it in a Magical Formula book and can subsequently make scrolls of it.
\end{itemize}

\subsection*{Scroll Essence Type}

\begin{minipage}{0.48\textwidth}
The type of Essence is determined by the spell list the spell comes from; if it is on multiple spell lists, it is determined by how you have access to the spell. If you have access to the spell via multiple lists or the written form of the spell, you can pick which Essence to use for spells that have multiple options.
\end{minipage}\hfill
\begin{minipage}{0.48\textwidth}
\begin{tabularx}{\textwidth}\toprule
{}XX}
\midrule
Essence Type & \multicolumn{2}{c}{Spell List} \\
\midrule
Arcane & \multicolumn{2}{c}{Inventor, Bard, Occultist, Sorcerer, Warlock, Wizard} \\
\midrule
Divine & \multicolumn{2}{c}{Cleric, Occultist, Paladin} \\
\midrule
Primal & \multicolumn{2}{c}{Druid, Occultist, Ranger} \\
\midrule
Psionic & \multicolumn{2}{c}{Monk, Psion} \\
\midrule
\end{tabularx}
\end{minipage}

\section{Scrollscribing}

\section*{Scrollscribing Crafting Table}

\begin{tabularx}{\textwidth}\toprule
{}XXXXXXXXXX}
\midrule
\multicolumn{2}{c}{Name} & \multicolumn{4}{c}{Materials} & Crafting Time & Checks & Difficulty & Rarity & Value \\
\midrule
\multicolumn{2}{c}{Cantrip} & \multicolumn{4}{c}{1 common magical ink 
1 common parchment} & 2 hours & 1 & DC 12 & Common & 20 gp \\
\midrule
\multicolumn{2}{c}{1st-Level Spell} & \multicolumn{4}{c}{1 common essence 
1 common magical ink 
1 common parchment} & 2 hours & 1 & DC 12 & Uncommon & 65 gp \\
\midrule
\multicolumn{2}{c}{2nd-Level Spell} & \multicolumn{4}{c}{1 common essence 
2 common magical ink 
1 common parchment} & 2 hours & 1 & DC 14 & Uncommon & 90 gp \\
\midrule
\multicolumn{2}{c}{3rd-Level Spell} & \multicolumn{4}{c}{1 uncommon essence 
1 uncommon magical ink 
1 uncommon parchment} & 4 hours & 2 & DC 14 & Rare & 250 gp \\
\midrule
\multicolumn{2}{c}{4th-Level Spell} & \multicolumn{4}{c}{1 uncommon essence 
2 uncommon magical ink 
1 uncommon parchment} & 4 hours & 2 & DC 14 & Rare & 300 gp \\
\midrule
\multicolumn{2}{c}{5th-Level Spell} & \multicolumn{4}{c}{1 rare essence 
1 rare magical ink 
1 rare parchment} & 4 hours & 2 & DC 15 & Rare & 1,200 gp \\
\midrule
\multicolumn{2}{c}{6th-Level Spell} & \multicolumn{4}{c}{1 rare essence 
2 rare magical ink 
1 rare parchment} & 4 hours & 2 & DC 16 & Rare & 1,500 gp \\
\midrule
\multicolumn{2}{c}{7th-Level Spell} & \multicolumn{4}{c}{1 very rare essence 
1 very rare magical ink 
1 very rare parchment} & 8 hours & 4 & DC 17 & Very Rare & 12,000 gp \\
\midrule
\multicolumn{2}{c}{8th-Level Spell} & \multicolumn{4}{c}{1 very rare essence 
2 very rare magical ink 
1 very rare parchment} & 8 hours & 4 & DC 18 & Very Rare & 14,000 gp \\
\midrule
\multicolumn{2}{c}{9th-Level Spell} & \multicolumn{4}{c}{1 legendary essence 
1 legendary magical ink 
1 legendary parchment} & 24 hours (3 days) & 12 & DC 20 & Legendary & 40,000 gp \\
\midrule
\end{tabularx}

\section*{Scrollscribing Recipes}

\begin{minipage}{0.48\textwidth}
\begin{itemize}
  \item Cantrip Spell Scroll
  \item 1st-Level Spell Scroll
  \item 2nd-Level Spell Scroll
  \item 3rd-Level Spell Scroll
  \item 4th-Level Spell Scroll
\end{itemize}
\end{minipage}\hfill
\begin{minipage}{0.48\textwidth}
\begin{itemize}
  \item 5th-Level Spell Scroll
  \item 6th-Level Spell Scroll
  \item 7th-Level Spell Scroll
  \item 8th-Level Spell Scroll
  \item 9th-Level Spell Scroll
\end{itemize}
\end{minipage}

\section{Wand Whittling}

% [Image Inserted Manually]

\section*{Wand Whittling}

Unlike many magic items that are crafted and then enchanted with magic, a wand is purpose built to store the magic it will contain, worked from wood and infused with magic as a single process, with different intricacies based on the type of magic it will wield.

Wands are very valuable to adventurers, and the power they wield should never be underestimated, as they can save precious resources. While the typical wand is an attunement item that can recharge, there is a weaker variety of lesser wands that are consumable items, more akin to multi-use scrolls that are easier to make.

\section*{Related Tool \& Ability Score}

Wand whittling works using Woodcarver’s tools. Attempting to craft a wand without these is impossible.

The related ability score is Dexterity. While spellcasters of any stripe can make wands of the spells they know, the process is one of systematic application of magical theory to integrate the spell into a wand in a function that can later be used.

Additionally, like Scroll Scribing and Enchanting, proficiency in Arcana is required; without proficiency in arcana, you can’t add your Tool proficiency to the crafting roll.

\section*{Quick Reference}

While each step will go into more depth, the quick reference allows you at a glance to follow the steps to make a wand in its most basic form:

\begin{itemize}
  \item Select a wand from the Greater Wand Table you would like to make, or a spell you would like to make a Lesser Wand of.
  \item Acquire the items listed in the materials column for a Wand from the appropriate table.
  \item Use your Woodcarver’s tools to create the wand using the number of hours listed in the Crafting Time column, or during a long rest using the crafting camp action if the crafting time is 2 hours or less.
  \item For every 2 hours, make a crafting roll of 1d20 + your Dexterity modifier + your proficiency with Woodcarver’s tools. On success, you mark 2 hours of completed time. Once the completed time is equal to the crafting time, the wand is complete. On failure, the crafting time is lost, and no progress has been made during the 2 hours. If you fail 3 times in a row, the crafting is a failure, and all materials are lost.
\end{itemize}

\section*{Materials: Wood \& Essence}

\begin{minipage}{0.48\textwidth}
The materials for Wand Whittling are wood and essences. The wood used in wands is typically purchased, and below are the table of prices. If they are found as part of treasure, they are calculated as any other precious non-currency treasure would be calculated. Low-level wands can use common woods that are not magically attuned, but more powerful magic will cause such mundane wood to instantly splinter or catch fire, requiring the use of rare and exotic woods.

Essences are varied in nature but are what power wands. Essences are shared with Enchanting, and more details can be gleaned from the Essences section under Enchanting.
\end{minipage}\hfill
\begin{minipage}{0.48\textwidth}
What Are Rare Woods?

The most common examples would be from trees in exotic locations—wood from the Feywild or from the outer planes. The exact nature of the wands can match the spell for flavor, but the details beyond rarity are not considered by default for the system. Some examples are provided at the end of this section.
\end{minipage}

\subsection*{Purchasing Wooden Branches}

\begin{minipage}{0.48\textwidth}
Like with many materials, one popular method of acquisition is to simply spend gold. A piece of wood here is typically a branch about 2–3 feet long and moderately narrow, suitable for a wand with some whittling.
\end{minipage}\hfill
\begin{minipage}{0.48\textwidth}
\begin{tabularx}{\textwidth}\toprule
{}XX}
\midrule
\multicolumn{2}{c}{Rarity} & Material Price \\
\midrule
\multicolumn{2}{c}{Common Branch} & 1 sp \\
\midrule
\multicolumn{2}{c}{Uncommon Branch} & 25 gp \\
\midrule
\multicolumn{2}{c}{Rare Branch} & 80 gp \\
\midrule
\multicolumn{2}{c}{Very Rare Branch} & 800 gp \\
\midrule
\multicolumn{2}{c}{Legendary Branch} & 2,000 gp \\
\midrule
\end{tabularx}
\end{minipage}

\section*{Crafting Roll}

\begin{minipage}{0.48\textwidth}
When you would like to make a wand, it relies on your woodcarver’s tools, and Dexterity to use them.

Wand Whittling Crafting Modifier = your Woodcarver’s Tools proficiency bonus + your Dexterity modifier
\end{minipage}\hfill
\begin{minipage}{0.48\textwidth}
\subsubsection*{Success and Failure}

For Wand Whittling, after you make the crafting roll and succeed, mark your progress on a crafting project. If you succeed, you make 2 hours of progress toward the total crafting time (and have completed one of the required checks for making an item). Checks for Wand Whittling do not need to be immediately consecutive. Failure means that no progress is made during that time. As usual, failing three times in a row results in the crafting project being a failure, and all materials are lost.
\end{minipage}

\section*{Wand Whittling Saving Throw}

Some wands require a saving throw, the following is the formula for calculating the saving throw. The saving throw is calculated at the time of creation based on the creator’s attributes and proficiency, and doesn’t change once it is created. A saving throw doesn’t include any expertise or other bonuses a crafter has to the crafting roll.

Wand DC = 8 + your related tool proficiency bonus + your Intelligence modifier

\section*{Lesser \& Greater Wands}

\begin{minipage}{0.48\textwidth}
Many adventurers are most familiar with greater wands, forged with such skill and power that their power regenerates each day at dawn, providing nearly unlimited ability to wield magic. These are, however, the rarer exception: Greater Wands. Their weaker cousins (Lesser Wands) work in a similar fashion, but do not naturally recharge, and must be recharged to be used again.
\end{minipage}\hfill
\begin{minipage}{0.48\textwidth}
\subsubsection*{Recharging Lesser Wands}

Recharging lesser wands is fairly simple; you can either expend 1 essence of the rarity required to craft that wand and perform a ritual that takes 2 hours, expending the essence and restoring all charges to the Lesser Wand (this only works with Lesser Wands), or you can expend 2 spell slots equal to the level of the spell the wand can cast (1st level for cantrip) during this ritual to recharge a single charge; you can repeat this ritual to recharge the wand multiple times, up to its maximum charges.
\end{minipage}

\section*{Magical Formula}

\begin{minipage}{0.48\textwidth}
For Greater Wands, they have known formulas that can be deduced as part of crafting (as with Enchanting), and you do not need to know the spell or effect before crafting the Wand, though they are typically harder to craft.

To craft a Lesser Wand, you must know the Magic Formula of the spell you want to make a Lesser Wand of. The easiest way to do this is to be able to cast the spell. You always know the Magic Formula of a spell you know how to cast. Otherwise, you need to have deep knowledge of the spell to be able to make a Lesser Wand of it. The following are some ways you can gain that knowledge:

\begin{itemize}
  \item Have it in your spellbook as a Wizard.
  \item Have it in your spell manual as an Infusionsmith Inventor.
  \item Have it in your ritual book as a ritual caster.
  \item Have a spell scroll of it (DC +2)
  \item Study its magical formula and record it. To learn a spell in this way requires proficiency in arcana and 1 day (8 hours per day) of study per level of the spell, as well as access to a teacher or book that records the spell. Once learned, you can record it in a Magical Formula book and can subsequently make scrolls of it.
\end{itemize}
\end{minipage}\hfill
\begin{minipage}{0.48\textwidth}
\subsubsection*{Lesser Wand Essence Type}

The type of Essence is determined by the spell list the spell comes from; if it is on multiple spell lists, it is determined by how you have access to the spell. If you have access to the spell via multiple lists or the written form of the spell, you can pick which Essence to use for spells that have multiple options.

\begin{tabularx}{\textwidth}\toprule
{}XX}
\midrule
Essence Type & \multicolumn{2}{c}{Spell List} \\
\midrule
Arcane & \multicolumn{2}{c}{Inventor, Bard, Occultist, Sorcerer, Warlock, Wizard} \\
\midrule
Divine & \multicolumn{2}{c}{Cleric, Occultist, Paladin} \\
\midrule
Primal & \multicolumn{2}{c}{Druid, Occultist, Ranger} \\
\midrule
Psionic & \multicolumn{2}{c}{Monk, Psion} \\
\midrule
\end{tabularx}
\end{minipage}

\section{Wand Whittling}

\section*{Wand Whittling Crafting Tables}

\section*{Lesser Wands}

\subsection*{Lesser Wands}

\begin{tabularx}{\textwidth}\toprule
{}XXXXXXXXXX}
\midrule
\multicolumn{2}{c}{Name} & \multicolumn{4}{c}{Materials} & Crafting Time & Checks & Difficulty & Rarity & Value \\
\midrule
\multicolumn{2}{c}{Lesser Wand of Cantrip} & \multicolumn{4}{c}{1 scroll of the spell 
1 common branch} & 2 hours & 1 & DC 12 & common & 30 gp \\
\midrule
\multicolumn{2}{c}{Lesser Wand of 1st-Level Spell} & \multicolumn{4}{c}{1 common essence 
1 scroll of the spell 
1 common branch} & 4 hours & 2 & DC 12 & uncommon & 100 gp \\
\midrule
\multicolumn{2}{c}{Lesser Wand of 2nd-Level Spell} & \multicolumn{4}{c}{1 common essence 
1 scroll of the spell 
1 common branch} & 4 hours & 2 & DC 14 & uncommon & 160 gp \\
\midrule
\multicolumn{2}{c}{Lesser Wand of 3rd-Level Spell} & \multicolumn{4}{c}{1 uncommon essence 
1 scroll of the spell 
1 uncommon branch} & 4 hours & 2 & DC 15 & rare & 500 gp \\
\midrule
\multicolumn{2}{c}{Lesser Wand of 4th-Level Spell} & \multicolumn{4}{c}{1 uncommon essence 
1 scroll of the spell 
1 uncommon branch} & 4 hours & 2 & DC 16 & rare & 600 gp \\
\midrule
\multicolumn{2}{c}{Lesser Wand of 5th-Level Spell} & \multicolumn{4}{c}{1 rare essence 
1 scroll of the spell 
1 rare branch} & 8 hours & 4 & DC 16 & rare & 2,300 gp \\
\midrule
\multicolumn{2}{c}{Lesser Wand of 6th-Level Spell} & \multicolumn{4}{c}{1 rare essence 
1 scroll of the spell 
1 rare branch} & 8 hours & 4 & DC 17 & rare & 2,700 gp \\
\midrule
\multicolumn{2}{c}{Lesser Wand of 7th-Level Spell} & \multicolumn{4}{c}{1 very rare essence 
1 scroll of the spell 
1 very rare branch} & 16 hours & 8 & DC 17 & very rare & 22,000 gp \\
\midrule
\multicolumn{2}{c}{Lesser Wand of 8th-Level Spell} & \multicolumn{4}{c}{1 very rare essence 
1 scroll of the spell 
1 very rare branch} & 16 hours & 8 & DC 18 & very rare & 32,000 gp \\
\midrule
\multicolumn{2}{c}{Lesser Wand of 9th-Level Spell} & \multicolumn{4}{c}{1 legendary essence 
1 scroll of the spell 
1 legendary branch} & 24 hours & 12 & DC 20 & legendary & 76,000 gp \\
\midrule
\end{tabularx}

\section*{Greater Wands}

\subsection*{Greater Wands}

\begin{longtable}{p{2.5cm}\toprule
|p{2.5cm}|p{2.5cm}|p{2.5cm}|p{2.5cm}|p{2.5cm}|p{2.5cm}|p{2.5cm}|p{2.5cm}|p{2.5cm}|p{2.5cm}|}
\midrule
\multicolumn{2}{c}{Name} & \multicolumn{4}{c}{Materials} & Crafting Time & Checks & Difficulty & Rarity & Value \\
\midrule
\multicolumn{2}{c}{Blast Stick (B)} & \multicolumn{4}{c}{2 common arcane essences 
1 common branch} & 8 hours & 4 & DC 14 & common & 215 gp \\
\midrule
\multicolumn{2}{c}{Goodberry Dispenser (B)} & \multicolumn{4}{c}{1 scroll of goodberry 
2 common primal essences 
1 common branch} & 8 hours & 4 & DC 15 & common & 240 gp \\
\midrule
\multicolumn{2}{c}{Magician’s Wand (B)} & \multicolumn{4}{c}{1 scroll of minor illlusion 
1 hat worth at least 5 gp 
2 common arcane essences 
1 common branch} & 8 hours & 4 & DC 14 & common & 165 gp \\
\midrule
\multicolumn{2}{c}{Wand of Chores (B)} & \multicolumn{4}{c}{1 scroll of prestidigitation 
1 common arcane essence 
1 common branch} & 8 hours & 4 & DC 14 & common & 160 gp \\
\midrule
\multicolumn{2}{c}{Whisperstick (B)} & \multicolumn{4}{c}{1 scroll of message 
1 common arcane essence 
1 common psionic essence 
1 common branch} & 8 hours & 4 & DC 15 & common & 190 gp \\
\midrule
\multicolumn{2}{c}{Wand of Magic Detection} & \multicolumn{4}{c}{1 scroll of detect magic 
1 uncommon arcane essence 
1 uncommon branch} & 8 hours & 4 & DC 15 & uncommon & 325 gp \\
\midrule
\multicolumn{2}{c}{Wand of Magic Missiles} & \multicolumn{4}{c}{1 scroll of magic missile 
3 uncommon arcane essences 
1 gem worth 50 gp 
1 uncommon branch} & 12 hours & 6 & DC 17 & uncommon & 950 gp \\
\midrule
\multicolumn{2}{c}{Wand of Secrets} & \multicolumn{4}{c}{1 scroll of find traps 
1 common arcane essence 
1 common primal essence 
1 uncommon branch} & 8 hours & 4 & DC 15 & uncommon & 250 gp \\
\midrule
\multicolumn{2}{c}{Wand of the War Mage, +1} & \multicolumn{4}{c}{1 uncommon arcane essence 
1 uncommon primal essence 
1 uncommon divine essence 
1 jewel worth 150 gp 
1 uncommon branch} & 8 hours & 4 & DC 16 & uncommon & 800 gp \\
\midrule
\multicolumn{2}{c}{Wand of Web} & \multicolumn{4}{c}{1 scroll of web 
1 uncommon arcane essence 
1 uncommon primal essence 
1 uncommon branch} & 12 hours & 6 & DC 16 & uncommon & 600 gp \\
\midrule
\multicolumn{2}{c}{Wand of Binding} & \multicolumn{4}{c}{1 scroll of hold monster 
1 scroll of hold person 
4 rare arcane essences 
1 rare branch} & 24 hours (3 days) & 12 & DC 17 & rare & 4,600 gp \\
\midrule
\multicolumn{2}{c}{Wand of Enemy Detection} & \multicolumn{4}{c}{1 scroll of see invisbility 
1 scroll of true seeing 
1 rare arcane essence 
1 rare branch} & 16 hours (2 days) & 8 & DC 17 & rare & 3,000 gp \\
\midrule
\multicolumn{2}{c}{Wand of Fear} & \multicolumn{4}{c}{1 scroll of fear 
2 rare arcane essences 
1 uncommon divine essence 
1 rare branch} & 16 hours (2 days) & 8 & DC 18 & rare & 2,400 gp \\
\midrule
\multicolumn{2}{c}{Wand of Fireballs} & \multicolumn{4}{c}{1 scroll of fireball 
3 rare arcane essences 
2 rare primal essences 
1 ruby worth 500 gp 
1 rare branch} & 24 hours (3 days) & 12 & DC 18 & rare & 5,300 gp \\
\midrule
\multicolumn{2}{c}{Wand of Lightning Bolts} & \multicolumn{4}{c}{1 scroll of lightning bolt 
2 rare arcane essences 
3 rare primal essences 
1 topaz worth 500 gp 
1 rare branch} & 24 hours (3 days) & 12 & DC 18 & rare & 5,300 gp \\
\midrule
\multicolumn{2}{c}{Wand of Paralysis} & \multicolumn{4}{c}{1 wand of binding 
4 rare arcane essences 
1 rare branch} & 24 hours (3 days) & 12 & DC 17 & rare & 8,000 gp \\
\midrule
\multicolumn{2}{c}{Wand of the Warmage, +2} & \multicolumn{4}{c}{2 rare arcane essences 
2 rare primal essences 
2 rare divine essences 
1 jewel worth 300 gp 
1 rare branch} & 24 hours (3 days) & 12 & DC 19 & rare & 5,400 gp \\
\midrule
\multicolumn{2}{c}{Wand of Wonder} & \multicolumn{4}{c}{1 scroll of prestidigitation 
1 scroll of faerie fire 
1 scroll of stinking cloud 
1 scroll of darkness 
1 scroll of enlarge/reduce 
1 scroll of invisibility 
1 scroll of lightning bolt
 A handful of colorful gems worth 250 gp 
1 uncommon primal essence 
1 uncommon divine essence 
1 uncommon arcane essence 
1 rare branch} & 16 hours (2 days) & 8 & DC 18 & rare & 2,100 gp \\
\midrule
\multicolumn{2}{c}{Wand of Polymorph} & \multicolumn{4}{c}{1 scroll of polymorph 
2 very rare arcane essences 
1 very rare primal essence 
1 very rare branch} & 24 hours (3 days) & 12 & DC 18 & very rare & 25,000 gp \\
\midrule
\multicolumn{2}{c}{Wand of the Warmage, +3} & \multicolumn{4}{c}{1 legendary arcane essence 
1 very rare primal essence 
1 very rare divine essence 
1 jewel worth 500 gp 
1 very rare branch} & 48 hours (6 days) & 24 & DC 21 & very rare & 50,000 gp \\
\midrule
\end{longtable}

\section*{Scroll Scribing Recipes}

\subsection*{Lesser Wands}

\begin{minipage}{0.48\textwidth}
\begin{itemize}
  \item Lesser Wand of Cantrip
  \item Lesser Wand of 1st-Level Spell
  \item Lesser Wand of 2nd-Level Spell
  \item Lesser Wand of 3rd-Level Spell
  \item Lesser Wand of 4th-Level Spell
\end{itemize}
\end{minipage}\hfill
\begin{minipage}{0.48\textwidth}
\begin{itemize}
  \item Lesser Wand of 5th-Level Spell
  \item Lesser Wand of 6th-Level Spell
  \item Lesser Wand of 7th-Level Spell
  \item Lesser Wand of 8th-Level Spell
  \item Lesser Wand of 9th-Level Spell
\end{itemize}
\end{minipage}

\subsection*{Greater Wands}

\begin{minipage}{0.48\textwidth}
\begin{itemize}
  \item Blast Stick
  \item Goodberry Dispenser
  \item Magician's Wand
  \item Wand of Binding
  \item Wand of Chores
  \item Wand of Enemy Detection
  \item Wand of Fear
  \item Wand of Fireballs
  \item Wand of Lightning Bolts
  \item Wand of Magic Detection
\end{itemize}
\end{minipage}\hfill
\begin{minipage}{0.48\textwidth}
\begin{itemize}
  \item Wand of Magic Missiles
  \item Wand of Paralysis
  \item Wand of Polymorph
  \item Wand of Secrets
  \item Wand of the War Mage, +1
  \item Wand of the Warmage, +2
  \item Wand of the Warmage, +3
  \item Wand of Web
  \item Wand of Wonder
  \item Whisperstick
\end{itemize}
\end{minipage}

\section{Wand Whittling}

\section*{Wands}

\subsection*{Lesser Wand}

Wand, common/uncommon/rare/very rare/legendary

This wand has 3 charges. While holding it, you can use an action to expend 1 or more of its charges to cast the spell infused in it it. For 1 charge, you cast the base version of the spell. You can increase the spell slot level by one for each additional charge you expend (if applicable for the spell). The DC of the spell is crafter’s Wand DC.

The wand doesn’t regain charges naturally but can be recharged (see recharge rules for Lesser Wands).

Lesser wands that cast 3rd-level spells or higher require attunement to use.

\begin{itemize}
  \item Item: Lesser Wand
\end{itemize}

\subsection*{Blast Stick}

Wand, common

This wand has 20 charges. This wand is considered a ranged weapon with a range of 60/180, and the Light property. It deals 1d6 force damage on hit. Each time you make an attack with it, it expends one charge. If you have a spellcasting modifier, you can use your spellcasting ability in place of your Dexterity modifier when making an attack with this wand.

The wand regains 1d10 + 10 expended charges daily at dawn. If you expend the wand’s last charge, roll a d20. On a 1, the wand breaks.

\begin{itemize}
  \item Item: Blast Stick (B)
\end{itemize}

\subsection*{Goodberry Dispenser}

Wand, common

This wand has 10 charges. While holding it as an an action you can tap the tip to your palm and expend 1 of its charges to create a single berry as if from the goodberry spell in your hand.

The wand regains 1d10 expended charges daily at dawn. If you expend the wand’s last charge, roll a d20. On a 1, the wand explodes, producing a splattering burst of sticky and sour juice.

\begin{itemize}
  \item Item: Goodberry Dispenser (B)
\end{itemize}

\subsection*{Magician’s Wand}

Wand, common

This wand has 6 charges. While holding it, you can use an action to expend 1 of its charges to cast minor illusion. When you cast minor illusion using this wand, you have advantage on Dexterity (Sleight of Hand) checks involving tricks or illusions until the end of your turn.

The wand regains 1d4 + 2 expended charges daily at dawn. If you expend the wand’s last charge, roll a d20. On a 1, the wand breaks.

\begin{itemize}
  \item Item: Magician's Wand (B)
\end{itemize}

\subsection*{Wand of Chores}

Wand, common

This wand has 6 charges. While holding it, you can use an action to expend 1 of its charges to cast prestidigitation.

The wand regains 1d4 + 2 expended charges daily at dawn. If you expend the wand’s last charge, roll a d20. On a 1, the wand breaks, dumping the last 10 instances of filth it has cleaned up.

\begin{itemize}
  \item Item: Wand of Chores (B)
\end{itemize}

\subsection*{Whisperstick}

Wand, common

This wand has 6 charges. While holding it, you can use an action to expend 1 of its charges to cast message. You need to point the wand at the target to cast it in this way.

The wand regains 1d4 + 2 expended charges daily at dawn. If you expend the wand’s last charge, roll a d20. On a 1, the wand breaks, making a loud bang audible up to 100 feet.

\begin{itemize}
  \item Item: Whisperstick (B)
\end{itemize}

\section{Leatherworking}

% [Image Inserted Manually]

\section*{Leatherworking}

Leatherworking is often seen as something of the “light armor” equivalent to a blacksmith, but it covers quite a bit more ground than that. While it may be the unsung hero, an adventurer’s best friend is a study leather backpack. Backpacks, belts, waterskins, quivers and more all fall to these artisans to make and can prove essential to everyday life.

In addition to their more mundane wares, however, leather, hide, scales, and carapaces of monsters in the fantastical settings these craftsmen find themselves in often provide more opportunity than the basic components of mundane items. A leatherworker is essential if you wish to get the most mileage out of your harvested monsters.

\section*{Quick Reference}

While each step will go into more depth, the quick reference allows you to at a glance follow the steps to work items from leather:

\begin{itemize}
  \item Select the item that you would like to craft from any of the Leatherworking Crafting Tables.
  \item Acquire the items listed in the materials column for that item.
  \item Use your leatherworker’s tools to craft the option using the number of hours listed in the Crafting Time column, or during a long rest using the crafting camp action if the crafting time is 2 hours or less.
  \item For every 2 hours, make a crafting roll of 1d20 + your Dexterity + your proficiency bonus with leatherworker’s tools.
  \item On success, you mark 2 hours of completed time toward the total crafting time.
  \item On failure, the crafting time is lost, and no progress has been made during the 2 hours. If you fail 3 times in a row, the crafting is a failure, and all materials are lost.
\end{itemize}

\section*{Related Tool \& Ability Score}

Leatherworking works using leatherworker’s tools. Attempting craft items with Leatherworking without these will almost always be made with disadvantage, and proficiency with these allows you to add your proficiency in them to any Leatherworking crafting roll.

\section*{Materials: Leather \& Hides}

Leatherworking uses leather and hides, primarily harvested from monsters, however sometimes they work with heavy quilted clothes, metal pieces, and tough carapaces.

\section*{Crafting Roll}

\begin{minipage}{0.48\textwidth}
Putting that together means that when you would like to make an item, your crafting roll is as follows:

Leatherworking Modifier = your Leatherworker’s Tools proficiency bonus + your Dexterity modifier
\end{minipage}\hfill
\begin{minipage}{0.48\textwidth}
\subsubsection*{Success and Failure}

For Leatherworking, after you make the crafting roll and succeed, mark your progress on a crafting project. If you succeed, you make 2 hours of progress toward the total crafting time (and have completed one of the required checks for making an item). Checks for Leatherworking do not need to be immediately consecutive. Failure means that no progress is made during that time.

Once an item is started, even if no progress is made, the components reserved for that item can only be recovered via salvage. If you fail three times in a row, all progress and materials are lost and can no longer be salvaged.
\end{minipage}

\section{Leatherworking}

\section*{Leatherworking Crafting Tables}

\section*{Armor}

\subsection*{Armor}

\begin{tabularx}{\textwidth}\toprule
{}XXXXXXXXXX}
\midrule
\multicolumn{2}{c}{Name} & \multicolumn{4}{c}{Materials} & Crafting Time & Checks & Difficulty & Rarity & Value \\
\midrule
\multicolumn{2}{c}{Carapace Breastplate} & \multicolumn{4}{c}{1 large carapace 
2 leather (any) 
2 buckles} & 8 hours & 4 & DC 12 & common & 50 gp \\
\midrule
\multicolumn{2}{c}{Carapace Shield} & \multicolumn{4}{c}{1 medium carapace 
1 leather piece 
4 leather scraps} & 4 hours & 2 & DC 10 & common & 10 gp \\
\midrule
\multicolumn{2}{c}{Hide Armor} & \multicolumn{4}{c}{2 rawhide leather 
1 hide 
2 buckles} & 4 hours & 2 & DC 10 & common & 10 gp \\
\midrule
\multicolumn{2}{c}{Leather Armor} & \multicolumn{4}{c}{3 rawhide leather 
2 buckles} & 4 hours & 2 & DC 12 & common & 10 gp \\
\midrule
\multicolumn{2}{c}{Leather Buckler (B)} & \multicolumn{4}{c}{2 boiled leather 
2 leather scraps} & 4 hours & 2 & DC 10 & common & 10 gp \\
\midrule
\multicolumn{2}{c}{Scale Mail} & \multicolumn{4}{c}{25 scales 
5 leather scraps 
1 armor padding} & 12 hours (1.5 days) & 6 & DC 12 & common & 50 gp \\
\midrule
\multicolumn{2}{c}{Studded Leather Armor} & \multicolumn{4}{c}{3 rawhide leather 
6 metal scraps 
2 buckles} & 6 hours & 3 & DC 14 & common & 45 gp \\
\midrule
\multicolumn{2}{c}{Tough Carapace Breastplate} & \multicolumn{4}{c}{1 large tough carapace 
2 leather (any) 
2 buckles} & 12 hours (1.5 days) & 6 & DC 15 & common & 400 gp \\
\midrule
\end{tabularx}

\section*{Weapons}

\subsection*{Weapons}

\begin{tabularx}{\textwidth}\toprule
{}XXXXXXXXXX}
\midrule
\multicolumn{2}{c}{Name} & \multicolumn{4}{c}{Materials} & Crafting Time & Checks & Difficulty & Rarity & Value \\
\midrule
\multicolumn{2}{c}{Scourge (B)} & \multicolumn{4}{c}{1 tanned leather 
3 metal scraps} & 6 hours & 3 & DC 14 & common & 40 gp \\
\midrule
\multicolumn{2}{c}{Whip} & \multicolumn{4}{c}{1 tanned leather} & 4 hours & 2 & DC 9 & common & 4 gp \\
\midrule
\end{tabularx}

\section*{Miscellaneous}

\subsection*{Miscellaneous}

\begin{longtable}{p{2.5cm}\toprule
|p{2.5cm}|p{2.5cm}|p{2.5cm}|p{2.5cm}|p{2.5cm}|p{2.5cm}|p{2.5cm}|p{2.5cm}|p{2.5cm}|p{2.5cm}|}
\midrule
\multicolumn{2}{c}{Name} & \multicolumn{4}{c}{Materials} & Crafting Time & Checks & Difficulty & Rarity & Value \\
\midrule
\multicolumn{2}{c}{Armor Padding} & \multicolumn{4}{c}{1 tanned leather 
2 buckles} & 2 hours & 1 & DC 10 & common & 5 gp \\
\midrule
\multicolumn{2}{c}{Backpack} & \multicolumn{4}{c}{1 sheet of leather 
4 leather scraps 
2 buckles} & 2 hours & 1 & DC 14 & common & 5 gp \\
\midrule
\multicolumn{2}{c}{Bag} & \multicolumn{4}{c}{10 leather scraps 
1 buckles} & 2 hours & 1 & DC 10 & common & 2 gp \\
\midrule
\multicolumn{2}{c}{Belt} & \multicolumn{4}{c}{4 leather scraps 
1 buckle} & 2 hours & 1 & DC 9 & common & 1 gp \\
\midrule
\multicolumn{2}{c}{Bit and Bridle} & \multicolumn{4}{c}{4 leather scraps 
1 wood scraps 
1 metal scraps} & 2 hours & 1 & DC 10 & common & 2 gp \\
\midrule
\multicolumn{2}{c}{Boiled Leather} & \multicolumn{4}{c}{1 hide or rawhide} & 16 hours & 8 & DC 6 & common & 3 gp \\
\midrule
\multicolumn{2}{c}{Dice Bag} & \multicolumn{4}{c}{1 leather scrap} & 2 hours & 1 & DC 9 & common & 1 gp \\
\midrule
\multicolumn{2}{c}{Hide} & \multicolumn{4}{c}{20 hide scraps} & 2 hours & 1 & DC 10 & common & 2 gp \\
\midrule
\multicolumn{2}{c}{Hide Scraps (20)} & \multicolumn{4}{c}{1 hide} & 2 hours & 1 & DC 4 & common & 2 gp \\
\midrule
\multicolumn{2}{c}{Holster} & \multicolumn{4}{c}{2 leather scraps} & 2 hours & 1 & DC 9 & common & 5 sp \\
\midrule
\multicolumn{2}{c}{Leather Scraps (20)} & \multicolumn{4}{c}{1 leather (any)} & 2 hours & 1 & DC 4 & common & 2 gp \\
\midrule
\multicolumn{2}{c}{Parchment (10)} & \multicolumn{4}{c}{10 leather scraps} & 2 hours & 1 & DC 8 & common & 1 gp \\
\midrule
\multicolumn{2}{c}{Quiver} & \multicolumn{4}{c}{5 leather scraps} & 2 hours & 1 & DC 9 & common & 1 gp \\
\midrule
\multicolumn{2}{c}{Rawhide} & \multicolumn{4}{c}{1 hide} & 8 hours & 4 & DC 6 & common & 2 gp \\
\midrule
\multicolumn{2}{c}{Saddle, Exotic} & \multicolumn{4}{c}{4 rawhide 
4 tanned leather 
2 parts 
1 fancy parts} & 6 hours & 3 & DC 14 & common & 60 gp \\
\midrule
\multicolumn{2}{c}{Saddle, Military} & \multicolumn{4}{c}{2 rawhide 
2 tanned leather 
2 parts} & 4 hours & 2 & DC 12 & common & 20 gp \\
\midrule
\multicolumn{2}{c}{Saddle, Pack} & \multicolumn{4}{c}{1 tanned leather 
2 leather scraps} & 2 hours & 1 & DC 10 & common & 5 gp \\
\midrule
\multicolumn{2}{c}{Saddle, Riding} & \multicolumn{4}{c}{2 tanned leather 
1 parts} & 4 hours & 2 & DC 10 & common & 10 gp \\
\midrule
\multicolumn{2}{c}{Saddlebags} & \multicolumn{4}{c}{1 tanned leather} & 2 hours & 1 & DC 10 & common & 4 gp \\
\midrule
\multicolumn{2}{c}{Sheath} & \multicolumn{4}{c}{4 leather scraps} & 2 hours & 1 & DC 9 & common & 6 sp \\
\midrule
\multicolumn{2}{c}{Tanned Leather} & \multicolumn{4}{c}{1 hide or rawhide} & 16 hours & 8 & DC 6 & common & 3 gp \\
\midrule
\multicolumn{2}{c}{Throwing Bandolier (B)} & \multicolumn{4}{c}{1 tanned leather 
3 leather scraps 
1 buckle} & 4 hours & 2 & DC 12 & common & 20 gp \\
\midrule
\multicolumn{2}{c}{Waterskin} & \multicolumn{4}{c}{2 leather scraps} & 2 hours & 1 & DC 8 & common & 2 sp \\
\midrule
\multicolumn{2}{c}{Uncommon Parchment} & \multicolumn{4}{c}{1 tanned leather 
2 common alchemical reagents (any)} & 2 hours & 1 & DC 12 & uncommon & 40 gp \\
\midrule
\multicolumn{2}{c}{Rare Parchment} & \multicolumn{4}{c}{1 tanned leather 
1 uncommon arcane essence} & 2 hours & 1 & DC 16 & rare & 200 gp \\
\midrule
\multicolumn{2}{c}{Very Rare Parchment} & \multicolumn{4}{c}{1 tough leather 
1 rare arcane essence 
2 uncommon arcane essence} & 4 hours & 2 & DC 18 & very rare & 2,000 gp \\
\midrule
\multicolumn{2}{c}{Legendary Parchment} & \multicolumn{4}{c}{1 tough leather 
2 rare arcane essence 
2 very rare alchemical reagents (any)} & 4 hours & 2 & DC 20 & legendary & 5,000 gp \\
\midrule
\end{longtable}

\section*{Leatherworking Recipes}

\begin{minipage}{0.48\textwidth}
\subsubsection*{Armor}

\begin{itemize}
  \item Carapace Breastplate
  \item Carapace Shield
  \item Hide Armor
  \item Leather Armor
  \item Leather Buckler
  \item Scale Mail
  \item Studded Leather Armor
  \item Tough Carapace Breastplate
\end{itemize}

\subsubsection*{Weapons}

\begin{itemize}
  \item Scourge
  \item Whip
\end{itemize}
\end{minipage}\hfill
\begin{minipage}{0.48\textwidth}
\subsubsection*{Miscellaneous}

\begin{itemize}
  \item Armor Padding
  \item Backpack
  \item Bag
  \item Belt
  \item Bit and Bridle
  \item Boiled Leather
  \item Dice Bag
  \item Hide
  \item Hide Scraps (20)
  \item Holster
  \item Leather Scraps (20)
  \item Legendary Parchment
  \item Parchment (10)
  \item Quiver
  \item Rare Parchment
  \item Rawhide
  \item Saddle, Exotic
  \item Saddle, Military
  \item Saddle, Pack
  \item Saddle, Riding
  \item Saddlebag
  \item Sheath
  \item Tanned Leather
  \item Throwing Bandolier (B)
  \item Uncommon Parchment
  \item Very Rare Parchment
  \item Waterskin
\end{itemize}
\end{minipage}

\section{Leatherworking}

\section*{Leather Items}

\subsection*{Leather Buckler}

Shield, common

A small, simple, and light shield. Wielding this shield increases your armor class by 1. You can benefit from only one shield at a time. You can hold items in the same hand as this light buckler, but any weapon wielded with the same hand has disadvantage on weapon attacks, and any grappling attempts with that hand are made with disadvantage.

\begin{itemize}
  \item Item: Leather Buckler (B)
\end{itemize}

\subsection*{Scourge}

Weapon, common

This heavy metal-tipped whip is a martial weapon with the reach property. It deals 2d4 slashing damage on a hit. Due to its unwieldy nature, you have disadvantage when you use a scourge to attack a target within 5 feet of you.

\begin{itemize}
  \item Item: Scourge (B)
\end{itemize}

\subsection*{Throwing Bandolier}

Adventuring gear, common

This is a quick access bandolier that can old weapons with the thrown property. It can be configured to hold 2 javelins, handaxes, or light hammers, or 4 daggers or darts. Weapons held in it can be drawn as part of making an attack with them without using an object interaction to draw them as long as you have a free hand to draw them.

\begin{itemize}
  \item Item: Throwing Bandolier (B)
\end{itemize}

\begin{minipage}{0.48\textwidth}
\begin{itemize}
  \item Item: Carapace Breastplate
  \item Item: Carapace Shield
  \item Item: Dice Bag
\end{itemize}
\end{minipage}\hfill
\begin{minipage}{0.48\textwidth}
\begin{itemize}
  \item Item: Holster
  \item Item: Sheath
  \item Item: Tough Carapace Breastplate
\end{itemize}
\end{minipage}

\section{Tinkering}

% [Image Inserted Manually]

\section*{Tinkering}

Tinkering is applying creativity to junk to make new things. Sometimes even useful new things. Ranging from the humble crafts to complex contraptions, tinkering is a broad category that any adventuring party can benefit from.

Oft the purview of peddlers and wanderers, they have a broad skill set and tend to excel at working with limited resources and their wit rather than expensive shopping lists of materials, though many will say they have a bad habit of collecting too much junk with the idea that things can be handy when you would least expect it...

\section*{Quick Reference}

While each step will go into more depth, the quick reference allows you to at a glance follow the steps to tinker up an item in its most basic form:

\begin{itemize}
  \item Select the item that you would like to craft from any of the Tinkering Crafting Tables.
  \item Acquire the items listed in the materials column for that item.
  \item Use your tinker’s tools to craft the option using the number of hours listed in the Crafting Time column, or during a long rest using the crafting camp action if the crafting time is 2 hours or less.
  \item For every 2 hours, make a crafting roll of 1d20 + your Intelligence + your proficiency bonus with tinker’s tools.
  \item On success, you mark 2 hours of completed time. Once the completed time is equal to the crafting time, the item is complete. On failure, the crafting time is lost, and no progress has been made during the 2 hours. If you fail 3 times in a row, the crafting is a failure, and all materials are lost.
\end{itemize}

\section*{Related Tool \& Ability Score}

Tinkering works using tinker’s tools. Attempting to tinker item without these will almost always be made with disadvantage, and proficiency with these allows you to add your proficiency in them to any Tinkering crafting roll.

Most of the time tinkers need only the minimal heat of a basic fire and their tools to work, though any craft that requires an ingot may require a forge at the discretion of the GM.

\section*{Materials: Parts \& Scrap}

Tinkering uses metal scraps, miscellaneous parts (simply referred to as “parts”), and, in cases of making more magically functional things, essences to imbue them with their power. The term “parts” is used to refer to gears, wires, springs, windy bits, screws, nails, and doodads. Parts can be either found or salvaged or forged from metal scraps (or even straight from ingots by a Blacksmith for those that really want to be industrial about it). The exact nature of each item making up this collection is left abstracted.

In addition, metal scraps are collections of salvaged material that generally fall into the category of things “too small to track” which can then be used for the creations of tinkerers. In addition to all of this, occasionally tinkers will use ingots... particularly ones of tin (which is their namesake, after all).

Like other crafting branches, there are also named components for more iconic pieces of gear—the stock of a crossbow, for example, or other items. The cost for these items can be found on the common component table and are generally minor.

Lastly, Tinkerers use essences when constructing things that push beyond the mundane principles of plausibility, crafting magical properties into their inventions.

Named Components

In almost all cases, named components (such as a “wooden stock” for a crossbow) can be simply abstracted out as a minor cost, but, as always, the level of abstraction is up to the GM.

\section*{Crafting Roll}

\begin{minipage}{0.48\textwidth}
Putting that together means that when you would like to smith an item, your crafting roll is as follows:

Tinkering Modifier = your Tinker’s Tools proficiency bonus + your Intelligence modifier
\end{minipage}\hfill
\begin{minipage}{0.48\textwidth}
\subsubsection*{Success and Failure}

For Tinkering, after you make the crafting roll and succeed, mark your progress on a crafting project. If you succeed, you make 2 hours of progress toward the total crafting time (and have completed one of the required checks for making an item). Checks for Tinkering do not need to be immediately consecutive. Failure means that no progress is made during that time. Once an item is started, even if no progress is made, the components reserved for that item can only be recovered via salvage.

If you fail three times in a row, all progress and materials are lost and can no longer be salvaged.
\end{minipage}

\section*{Tinkering Saving Throw}

Some gadgets a Tinkerer creates require a saving throw, the following is the formula for calculating the saving throw. The saving throw is calculated at the time of creation based on the creator’s attributes and proficiency and doesn’t change once it is created. A saving throw doesn’t include any expertise or other bonuses a crafter has to the crafting roll.

Tinkering DC = 8 + your Tinker’s Tools proficiency bonus + your Intelligence modifier

\section{Tinkering}

\section*{Tinkering Crafting Tables}

\section*{Adventuring Gear}

\begin{tabularx}{\textwidth}\toprule
{}XXXXXXXXXX}
\midrule
\multicolumn{2}{c}{Name} & \multicolumn{4}{c}{Materials} & Crafting Time & Checks & Difficulty & Rarity & Value \\
\midrule
\multicolumn{2}{c}{Climber’s Kit} & \multicolumn{4}{c}{10 pitons 
50 feet rope 
4 parts 
1 fancy parts} & 2 hours & 1 & DC 12 & common & 25 gp \\
\midrule
\multicolumn{2}{c}{Clockwork Toy} & \multicolumn{4}{c}{2 metal scraps 
3 parts} & 2 hours & 1 & DC 12 & common & 10 gp \\
\midrule
\multicolumn{2}{c}{Drill} & \multicolumn{4}{c}{2 metal scraps 
1 parts} & 2 hours & 1 & DC 12 & common & 5 gp \\
\midrule
\multicolumn{2}{c}{Grappling Hook} & \multicolumn{4}{c}{1 rope 
2 metal scraps 
1 parts} & 2 hours & 1 & DC 12 & common & 7 gp \\
\midrule
\multicolumn{2}{c}{Lamp} & \multicolumn{4}{c}{2 metal scraps} & 4 hours & 2 & DC 10 & common & 5 sp \\
\midrule
\multicolumn{2}{c}{Lantern (Bullseye)} & \multicolumn{4}{c}{3 metal scraps 
2 parts 
1 glass flask} & 4 hours & 2 & DC 11 & common & 10 gp \\
\midrule
\multicolumn{2}{c}{Lantern (Hooded)} & \multicolumn{4}{c}{3 metal scraps 
1 parts 
1 glass flask} & 4 hours & 2 & DC 9 & common & 5 gp \\
\midrule
\multicolumn{2}{c}{Lock} & \multicolumn{4}{c}{2 metal scraps 
3 parts} & 8 hours & 4 & DC 17 & common & 10 gp \\
\midrule
\multicolumn{2}{c}{Merchant’s Scales} & \multicolumn{4}{c}{1 metal scraps 
2 parts} & 2 hours & 1 & DC 10 & common & 5 gp \\
\midrule
\multicolumn{2}{c}{Pocket Watch} & \multicolumn{4}{c}{1 metal scraps 
3 parts 
1 fancy parts 
1 esoteric parts} & 8 hours & 4 & DC 14 & common & 150 gp \\
\midrule
\multicolumn{2}{c}{Spyglass} & \multicolumn{4}{c}{2 metal scraps 
2 fancy parts 
5 esoteric parts} & 12 hours (1.5 days) & 6 & DC 18 & common & 1,000 gp \\
\midrule
\multicolumn{2}{c}{Tinderbox} & \multicolumn{4}{c}{1 metal scraps 
1 parts} & 2 hours & 1 & DC 10 & common & 2 gp \\
\midrule
\multicolumn{2}{c}{Wind Up Timer} & \multicolumn{4}{c}{2 metal scraps 
1 parts} & 2 hours & 1 & DC 12 & common & 5 gp \\
\midrule
\multicolumn{2}{c}{Underwater Breathing Apparatus} & \multicolumn{4}{c}{4 metal scraps 
3 common primal essences 
2 fancy parts} & 8 hours & 4 & DC 16 & uncommon & 300 gp \\
\midrule
\end{tabularx}

\section*{Miscellaneous}

\begin{tabularx}{\textwidth}\toprule
{}XXXXXXXXXX}
\midrule
\multicolumn{2}{c}{Name} & \multicolumn{4}{c}{Materials} & Crafting Time & Checks & Difficulty & Rarity & Value \\
\midrule
\multicolumn{2}{c}{Miscellaneous Parts} & \multicolumn{4}{c}{5 metal scraps} & 4 hours & 2 & DC 12 & common & 2 gp \\
\midrule
\multicolumn{2}{c}{Autoloader} & \multicolumn{4}{c}{2 metal scraps 
2 parts 
5 fancy parts} & 8 hours & 4 & DC 17 & uncommon & 250 gp \\
\midrule
\end{tabularx}

\section*{Traps}

\begin{tabularx}{\textwidth}\toprule
{}XXXXXXXXXX}
\midrule
\multicolumn{2}{c}{Name} & \multicolumn{4}{c}{Materials} & Crafting Time & Checks & Difficulty & Rarity & Value \\
\midrule
\multicolumn{2}{c}{Hunting Trap} & \multicolumn{4}{c}{4 metal scraps 
2 parts} & 2 hours & 1 & DC 13 & common & 10 gp \\
\midrule
\multicolumn{2}{c}{Noise Trap} & \multicolumn{4}{c}{2 metal scraps 
2 parts} & 2 hours & 1 & DC 13 & common & 10 gp \\
\midrule
\multicolumn{2}{c}{Trip Wire} & \multicolumn{4}{c}{2 metal scraps 
1 parts} & 2 hours & 1 & DC 12 & common & 5 gp \\
\midrule
\end{tabularx}

\section*{Tools \& Instruments}

\begin{tabularx}{\textwidth}\toprule
{}XXXXXXXXXX}
\midrule
\multicolumn{2}{c}{Name} & \multicolumn{4}{c}{Materials} & Crafting Time & Checks & Difficulty & Rarity & Value \\
\midrule
\multicolumn{2}{c}{Alchemist’s Supplies} & \multicolumn{4}{c}{4 metal scraps 
2 fancy parts} & 8 hours & 4 & DC 14 & common & 50 gp \\
\midrule
\multicolumn{2}{c}{Bagpipes} & \multicolumn{4}{c}{1 tanned leather 
2 metal scraps 
1 fancy parts} & 6 hours & 3 & DC 13 & common & 30 gp \\
\midrule
\multicolumn{2}{c}{Brewer’s Supplies} & \multicolumn{4}{c}{4 metal scraps 
2 fancy parts} & 6 hours & 3 & DC 10 & common & 20 gp \\
\midrule
\multicolumn{2}{c}{Cartographer’s Tools} & \multicolumn{4}{c}{2 metal scraps 
1 parts 
1 fancy parts} & 6 hours & 3 & DC 13 & common & 15 gp \\
\midrule
\multicolumn{2}{c}{Cobbler’s Tools} & \multicolumn{4}{c}{3 metal scraps 
1 parts} & 4 hours & 2 & DC 12 & common & 5 gp \\
\midrule
\multicolumn{2}{c}{Cook’s Utensils} & \multicolumn{4}{c}{5 metal scraps} & 4 hours & 2 & DC 12 & common & 1 gp \\
\midrule
\multicolumn{2}{c}{Glassblower’s Tools} & \multicolumn{4}{c}{3 metal scraps 
2 fancy parts} & 6 hours & 3 & DC 12 & common & 30 gp \\
\midrule
\multicolumn{2}{c}{Herbalism Kit} & \multicolumn{4}{c}{4 metal scraps 
1 parts} & 4 hours & 2 & DC 12 & common & 5 gp \\
\midrule
\multicolumn{2}{c}{Horn} & \multicolumn{4}{c}{4 metal scraps 
1 parts} & 4 hours & 2 & DC 10 & common & 4 gp \\
\midrule
\multicolumn{2}{c}{Jeweler’s Tools} & \multicolumn{4}{c}{2 metal scraps 
2 fancy parts} & 4 hours & 2 & DC 12 & common & 25 gp \\
\midrule
\multicolumn{2}{c}{Leatherworker’s Tools} & \multicolumn{4}{c}{4 metal scraps 
1 parts} & 4 hours & 2 & DC 12 & common & 5 gp \\
\midrule
\multicolumn{2}{c}{Mason’s Tools} & \multicolumn{4}{c}{5 metal scraps 
2 parts} & 6 hours & 3 & DC 10 & common & 10 gp \\
\midrule
\multicolumn{2}{c}{Navigator’s Tools} & \multicolumn{4}{c}{3 metal scraps 
2 parts 
1 fancy parts} & 6 hours & 3 & DC 12 & common & 25 gp \\
\midrule
\multicolumn{2}{c}{Thieves’ Tools} & \multicolumn{4}{c}{4 metal scraps 
2 parts 
1 fancy parts} & 6 hours & 3 & DC 12 & common & 25 gp \\
\midrule
\multicolumn{2}{c}{Tinker’s Tools} & \multicolumn{4}{c}{5 metal scraps 
3 fancy parts} & 8 hours & 4 & DC 12 & common & 50 gp \\
\midrule
\multicolumn{2}{c}{Weaver’s Tools} & \multicolumn{4}{c}{4 metal scraps} & 4 hours & 2 & DC 12 & common & 1 gp \\
\midrule
\multicolumn{2}{c}{Woodcarver’s Tools} & \multicolumn{4}{c}{4 metal scraps} & 4 hours & 2 & DC 12 & common & 1 gp \\
\midrule
\end{tabularx}

\section*{Weapons}

\begin{tabularx}{\textwidth}\toprule
{}XXXXXXXXXX}
\midrule
\multicolumn{2}{c}{Name} & \multicolumn{4}{c}{Materials} & Crafting Time & Checks & Difficulty & Rarity & Value \\
\midrule
\multicolumn{2}{c}{Hand Crossbow} & \multicolumn{4}{c}{1 wooden stock 
4 metal scraps 
2 parts 
3 fancy parts} & 8 hours & 4 & DC 16 & common & 75 gp \\
\midrule
\multicolumn{2}{c}{Heavy Crossbow} & \multicolumn{4}{c}{1 wooden stock 
8 metal scraps 
6 parts 
2 fancy parts} & 6 hours & 3 & DC 13 & common & 50 gp \\
\midrule
\multicolumn{2}{c}{Light Crossbow} & \multicolumn{4}{c}{1 wooden stock 
4 metal scraps 
6 parts} & 4 hours & 2 & DC 13 & common & 25 gp \\
\midrule
\multicolumn{2}{c}{Ordinary Walking Stick (B)} & \multicolumn{4}{c}{1 hand crossbow 
1 quarterstaff 
3 parts} & 6 hours & 3 & DC 14 & common & 120 gp \\
\midrule
\multicolumn{2}{c}{Lantern Shield (B)} & \multicolumn{4}{c}{1 shield 
1 hooded lantern 
1 +1 shortsword 
3 esoteric parts 
5 fancy parts} & 16 hours & 8 & DC 18 & very rare & 1,700 gp \\
\midrule
\multicolumn{2}{c}{Rapid-Fire Heavy Crossbow (B)} & \multicolumn{4}{c}{1 +2 heavy crossbow 
1 autoloader 
2 fancy parts} & 8 hours & 4 & DC 18 & very rare & 6,500 gp \\
\midrule
\end{tabularx}

\section*{Advanced Ammunition}

\begin{tabularx}{\textwidth}\toprule
{}XXXXXXXXXX}
\midrule
\multicolumn{2}{c}{Name} & \multicolumn{4}{c}{Materials} & Crafting Time & Checks & Difficulty & Rarity & Value \\
\midrule
\multicolumn{2}{c}{Adaptable Shot} & \multicolumn{4}{c}{1 piece of ammunition 
1 parts} & 2 hours & 1 & DC 14 & common & 13 gp \\
\midrule
\multicolumn{2}{c}{Bola Shot} & \multicolumn{4}{c}{1 piece of ammunition 
1 net 
2 parts} & 2 hours & 1 & DC 14 & common & 15 gp \\
\midrule
\multicolumn{2}{c}{Bouncing Shot} & \multicolumn{4}{c}{1 piece of ammunition 
1 fancy parts} & 2 hours & 1 & DC 15 & common & 30 gp \\
\midrule
\multicolumn{2}{c}{Payload Shot} & \multicolumn{4}{c}{1 piece of ammunition 
1 item weighing less than 1 lb. 
1 parts} & 2 hours & 1 & DC 16 & common & 35 gp \\
\midrule
\multicolumn{2}{c}{Propelled Shot (10)} & \multicolumn{4}{c}{10 pieces of ammunition 
1 packet of blasting powder 
5 parts} & 2 hours & 1 & DC 15 & common & 80 gp \\
\midrule
\multicolumn{2}{c}{Ricochet Shot} & \multicolumn{4}{c}{1 piece of ammunition 
1 fancy parts} & 2 hours & 1 & DC 16 & common & 40 gp \\
\midrule
\multicolumn{2}{c}{Whistling Shot} & \multicolumn{4}{c}{1 piece of ammunition 
2 parts} & 2 hours & 1 & DC 12 & common & 8 gp \\
\midrule
\multicolumn{2}{c}{Spell Shot} & \multicolumn{4}{c}{1 piece of ammunition 
1 fancy parts
 Either 
(a) 1 scroll of fog cloud
 or 
(b) 1 scroll of entangle
 or 
(c) 1 scroll of multishot} & 2 hours & 1 & DC 15 & uncommon & 80 gp \\
\midrule
\end{tabularx}

\section*{Mechanical Prosthetics}

\begin{tabularx}{\textwidth}\toprule
{}XXXXXXXXXX}
\midrule
\multicolumn{2}{c}{Name} & \multicolumn{4}{c}{Materials} & Crafting Time & Checks & Difficulty & Rarity & Value \\
\midrule
\multicolumn{2}{c}{Basic Leg Prosthetic} & \multicolumn{4}{c}{8 metal scraps 
1 parts} & 4 hours & 2 & DC 10 & common & 5 gp \\
\midrule
\multicolumn{2}{c}{Mechanical Arm} & \multicolumn{4}{c}{6 metal scraps 
4 parts 
2 fancy parts 
1 common arcane essence} & 8 hours & 4 & DC 14 & common & 125 gp \\
\midrule
\multicolumn{2}{c}{Mechanical Leg} & \multicolumn{4}{c}{8 metal scraps 
4 parts 
2 fancy parts 
1 common arcane essence} & 8 hours & 4 & DC 14 & common & 125 gp \\
\midrule
\multicolumn{2}{c}{Weaponized Arm} & \multicolumn{4}{c}{1 one handed melee weapon 
8 metal scraps 
4 parts 
2 fancy parts 
1 esoteric parts} & 8 hours & 4 & DC 15 & common & 180 gp \\
\midrule
\multicolumn{2}{c}{Specialized Mechanical Arm} & \multicolumn{4}{c}{1 tool of your choice 
6 metal scraps 
4 fancy parts 
1 esoteric parts 
1 common arcane essence} & 8 hours & 4 & DC 15 & uncommon & 270 gp \\
\midrule
\multicolumn{2}{c}{Spring Loaded Leg} & \multicolumn{4}{c}{8 metal scraps 
4 parts 
2 esoteric parts 
1 common arcane essence} & 8 hours & 4 & DC 16 & uncommon & 300 gp \\
\midrule
\end{tabularx}

\section*{Vehicles}

\begin{tabularx}{\textwidth}\toprule
{}XXXXXXXXXX}
\midrule
\multicolumn{2}{c}{Name} & \multicolumn{4}{c}{Materials} & Crafting Time & Checks & Difficulty & Rarity & Value \\
\midrule
\multicolumn{2}{c}{Folding Cart} & \multicolumn{4}{c}{1 cart (not included in cost) 
10 parts 
3 esoteric parts 
2 uncommon arcane essences 
1 rare arcane essence} & 16 hours (2 days) & 8 & DC 15 & Rare & 1,500 gp \\
\midrule
\multicolumn{2}{c}{Folding Boat} & \multicolumn{4}{c}{1 boat (not included in cost) 
10 parts 
3 esoteric parts 
2 uncommon primal essences 
1 rare arcane essence} & 16 hours (2 days) & 8 & DC 17 & Rare & 1,800 gp \\
\midrule
\end{tabularx}

% [Image Inserted Manually]

\section*{Tinkering Recipes}

\begin{minipage}{0.48\textwidth}
\subsubsection*{Advanced Ammunition}

\begin{itemize}
  \item Adaptable Shot
  \item Bola Shot
  \item Bouncing Shot
  \item Payload Shot
  \item Propelled Shot
  \item Ricochet Shot
  \item Spell Shot
  \item Whistling Shot
\end{itemize}

\subsubsection*{Adventuring Gear}

\begin{itemize}
  \item Climber's Kit
  \item Clockwork Toy
  \item Drill
  \item Grappling Hook
  \item Lamp
  \item Lantern (Bullseye)
  \item Lantern (Hooded)
  \item Lock
  \item Merchant's Scales
  \item Pocket Watch
  \item Spyglass
  \item Tinderbox
  \item Underwater Breathing Apparatus
  \item Wind Up Timer
\end{itemize}

\subsubsection*{Mechanical Prosthetics}

\begin{itemize}
  \item Basic Leg Prosthetic
  \item Mechanical Arm
  \item Mechanical Leg
  \item Specialized Mechanical Arm
  \item Spring Loaded Leg
  \item Weaponized Arm
\end{itemize}

\subsubsection*{Miscellaneous}

\begin{itemize}
  \item Autoloader
  \item Miscellaneous Parts
\end{itemize}
\end{minipage}\hfill
\begin{minipage}{0.48\textwidth}
\subsubsection*{Tools \& Instruments}

\begin{itemize}
  \item Alchemist's Supplies
  \item Bagpipes
  \item Brewer's Supplies
  \item Cartographer's Tools
  \item Cobbler's Tools
  \item Cook's Utensils
  \item Glassblower's Tools
  \item Herbalism Kit
  \item Horn
  \item Jeweler's Tools
  \item Leatherworker's Tools
  \item Mason's Tools
  \item Navigator's Tools
  \item Thieves' Tools
  \item Tinker's Tools
  \item Weaver's Tools
  \item Woodcarver's Tools
\end{itemize}

\subsubsection*{Traps}

\begin{itemize}
  \item Hunting Trap
  \item Noise Trap
  \item Trip Wire
\end{itemize}

\subsubsection*{Vehicles}

\begin{itemize}
  \item Folding Cart
  \item Folding Boat
\end{itemize}

\subsubsection*{Weapons}

\begin{itemize}
  \item Hand Crossbow
  \item Heavy Crossbow
  \item Lantern Shield (B)
  \item Light Crossbow
  \item Ordinary Walking Stick (B)
  \item Rapid-Fire Heavy Crossbow (B)
\end{itemize}
\end{minipage}

\section{Tinkering}

\section*{Tinkering Items}

\subsection*{Autoloader}

Item, common

An attachment to crossbows. When equipped, the crossbow no longer has the loading property, though gains a “reload 5” feature, and must be reloaded as an action or a bonus action after firing 5 times.

\begin{itemize}
  \item Item Template: Autoloader
\end{itemize}

\subsection*{Clockwork Toy}

Wondrous Item, common

This toy is a clockwork animal, monster, or person, such as a frog, mouse, bird, Dragon, or Soldier. When placed on the ground, the toy moves 5 feet across the ground on each of your turns in a random direction. It makes Noises as appropriate to the creature it represents.

\begin{itemize}
  \item Item: Clockwork Toy
\end{itemize}

\subsection*{Drill}

Item, common

Can make holes in things. Can destroy a lock with a DC of 14 or lower to pick or break with 10 minutes of work.

\begin{itemize}
  \item Item: Drill
\end{itemize}

\subsection*{Pocket Watch}

Item, common

A small timepiece that accurately tracks time. Must be wound up every day or it will cease to track time.

\begin{itemize}
  \item Item: Pocket Watch
\end{itemize}

\subsection*{Underwater Breathing Apparatus}

Wondrous Item, uncommon

The wearer can breathe underwater for up to 1 hour. You can’t (intelligibly) speak or perform verbal spell components while wearing this device.

\subsection*{Basic Leg Prosthetic}

Wondrous Item, common

A functional replacement leg. While using this as a replacement for one of your legs, your speed is reduced by 10 feet, and you have disadvantage on Dexterity saving throws and Dexterity (Acrobatics) checks.

\begin{itemize}
  \item Item: Basic Leg Prosthetic
\end{itemize}

\subsection*{Mechanical Arm}

Wondrous Item, common

A functional mechanical arm that can replace a missing one. This works for either a biological creature or a construct. This item follows the rules for attunement but doesn’t require an attunement slot once attuned. While attuned in this way, it functions the same as the limb it replaces.

\begin{itemize}
  \item Item: Mechanical Arm
\end{itemize}

\subsection*{Mechanical Leg}

Wondrous Item, common

A functional mechanical leg that can replace a missing one. This works for either a biological creature or a construct. This item follows the rules for attunement but doesn’t require an attunement slot once attuned. While attuned in this way, it functions the same as the limb it replaces.

\begin{itemize}
  \item Item: Mechanical Leg
\end{itemize}

\subsection*{Weaponized Arm}

Wondrous Item, common

A functional mechanical arm that can replace a missing one. This works for either a biological creature or a construct. This item follows the rules for attunement but doesn’t require an attunement slot once attuned. While attuned in this way, it functions the same as the limb it replaces. Additionally, this arm can house a one-handed melee weapon of your choice. This weapon can be swapped out as part of a short or long rest.

\begin{itemize}
  \item Item: Weaponized Arm
\end{itemize}

\subsection*{Specialized Mechanical Arm}

Wondrous Item, uncommon

A functional mechanical arm that can replace a missing one. This works for either a biological creature or a construct. This item follows the rules for attunement but doesn’t require an attunement slot once attuned. While attuned in this way, it functions the same as the limb it replaces. Additionally, this arm can house a tool of your choice.

\begin{itemize}
  \item Item: Specialized Mechanical Arm
\end{itemize}

\subsection*{Spring Loaded Leg}

Wondrous Item, uncommon (requires attunement)

A functional mechanical leg that can replace a missing one. While attuned in this way, it functions the same as the limb it replaces. Additionally, it can absorb a great deal of impact when falling and spring to new heights. You can reduce all falling damage by 20 feet, and your jumping distance is doubled. After falling, your jumping distance is further increased by a quarter of the distance fallen.

\begin{itemize}
  \item Item: Spring Loaded Leg
\end{itemize}

\subsection*{Adaptable Shot}

Weapon (ammunition), common

You can select a different damage type for the attack from bludgeoning, piercing, slashing, acid, cold, fire, or lightning when firing this shot.

When you recover this arrow, roll a d6. On a 1, it is broken.

\begin{itemize}
  \item Item: Adaptable Shot
\end{itemize}

\subsection*{Bola Shot}

Weapon (ammunition), common

This special piece of ammunition entangles a target it hits. On hit, a Large or smaller target must make a DC 12 Dexterity saving throw, or become restrained, as if by a net.
 You can select a different damage type for the attack from bludgeoning, piercing, slashing, acid, cold, fire, or lightning when firing this shot.

When you recover this arrow, roll a d6. On a 1, it is broken.

\begin{itemize}
  \item Item: Bola Shot
\end{itemize}

\subsection*{Bouncing Shot}

Weapon (ammunition), common

You can attack a target out of line of sight of with it if you know their location by bouncing it off a surface. They have the benefit of half cover instead of full cover when firing this shot.

When you recover this arrow, roll a d6. On a 1, it is broken.

\begin{itemize}
  \item Item: Bouncing Shot
\end{itemize}

\subsection*{Payload Shot}

Weapon (ammunition), common

This shot can be fired at half the normal range of the weapon firing it. You can deliver any object less than 2 inches in any dimension weighing less than 2 pounds to a target creature or point within that range.

When you recover this arrow, roll a d6. On a 1, it is broken.

\begin{itemize}
  \item Item: Payload Shot
\end{itemize}

\subsection*{Propelled Shot}

Weapon (ammunition), common

This is a special shot that contains a dangerous rocket like propellant accelerating it to greater speed and distance. This shot can be fired at a weapon’s long range without disadvantage, and does an additional weapon die of damage on hit. However, targets gain twice the bonus to AC from cover against this shot as they are difficult to arc.

When you recover this arrow, roll a d6. On a 1, it is broken.

\begin{itemize}
  \item Item: Propelled Shot
\end{itemize}

\subsection*{Ricochet Shot}

Weapon (ammunition), common

If you hit an attack roll with it, you can make another attack roll against another target within 10 feet of the first as part of the same attack when firing this shot.

When you recover this arrow, roll a d6. On a 1, it is broken.

\begin{itemize}
  \item Item: Ricochet Shot
\end{itemize}

\subsection*{Spell Shot}

Weapon (ammunition), uncommon

A magically infused arrow. It comes in three different types, based on what kind of magic is infused in it.

\begin{itemize}
  \item Fog Cloud. You can target a creature or a point within the normal range of your weapon with this arrow. This functions as a normal piece of ammunition, but casts fog cloud at the point of impact. The spell lasts 1d4 rounds and doesn’t require concentration.
  \item Entangle. You can target a creature or a point within the normal range of your weapon with this arrow. This functions as a normal piece of ammunition, but casts entangle at the point of impact. The spell lasts 1d4 rounds and doesn’t require concentration.
  \item Multishot. When you fire this arrow, you cast multishot.
\end{itemize}

Once the magic effect is discharged by firing it, the magic of the arrow is expended, and it becomes a mundane arrow.

\begin{itemize}
  \item Item: Spell Shot
\end{itemize}

\subsection*{Whistling Shot}

Weapon (ammunition), common

When fired, this shot emits a shrill whistle that can be heard from up to 500 feet from where it is fired and its target point.

When you recover this arrow, roll a d6. On a 1, it is broken.

\begin{itemize}
  \item Item: Whistling Shot
\end{itemize}

% [Image Inserted Manually]

\subsection*{Rapid-Fire Heavy Crossbow}

% [Image Inserted Manually]

This lever-action heavy crossbow is auto-drawing and magazine-fed. A revolving cylinder ringed with magnetic chambers grabs and nocks bolts from the magazine to be fired in rapid succession.

This crossbow lacks the loading property and is fitted with a cartridge that can hold up to twenty crossbow bolts. It automatically reloads after firing until the cartridge runs out of ammunition. Reloading the cartridge takes 1 minute.

You gain a +2 bonus to attack and damage rolls made with this magic weapon. When you use the Attack action with this crossbow, you can use a bonus action to attack again with it.

\begin{itemize}
  \item Item: Rapid-Fire Heavy Crossbow
\end{itemize}

\subsection*{Lantern Shield}

% [Image Inserted Manually]

This bizarre buckler has a lantern, a gauntlet, defensive spikes, and a short sword built into it. You must wear the gauntlet in order to wield it, and it only grants a +1 bonus to AC. The bullseye lantern concealed within the shield casts bright light in a 60-foot cone and dim light for an additional 60 feet. Once lit, it burns for 6 hours on a flask (1 pint) of oil. The aperture that sheds light through the shield can be opened or closed with a bonus action while wearing the gauntlet, instantly igniting or extinguishing the lantern.

While the lantern is covered and you are attacked in darkness by a creature within 60 feet of you that is also in darkness, you can reveal the light (no action required) to momentarily blind the attacker, imposing disadvantage on the attack. A creature can’t be blinded in this way again for 24 hours.

If a creature within 5 feet of you misses you with a melee attack, you can use your reaction to deal 1d4 piercing damage to the attacker with the shield’s defensive spikes.

Additionally, the shield has a shortsword built into it that can be wielded while wearing the gauntlet. You can deploy or retract it with a bonus action. You have a +1 bonus to attack and damage rolls with this shortsword and you are proficient with this weapon if you are proficient with shields. Attacks with this weapon are made with disadvantage if you are holding a weapon without the light property, or an object longer than 1 foot or heavier than 5 pounds in your gauntlet hand.

\begin{itemize}
  \item Item: Lantern Shield
\end{itemize}

\subsection*{Ordinary Walking Stick}

% [Image Inserted Manually]

You can use a bonus action to twist the shaft of this quarterstaff and deploy or retract the limbs of a light crossbow built into its head. It is indistinguishable from a walking stick while the limbs are retracted. It functions as a quarterstaff in either mode, however it only functions as a crossbow when the limbs are deployed. This crossbow lacks the loading property and is fitted with an internal magazine that can hold up to six crossbow bolts. It automatically reloads after firing until the magazine runs out of ammunition. Reloading the magazine takes an action. You are proficient with this magical weapon if you are proficient in either light crossbows or quarterstaffs.

\begin{itemize}
  \item Item: Ordinary Walking Stick
\end{itemize}

\section{Woodcarving}

\section*{Woodcarving}

Woodcarving is the branch of crafting concerned with creating objects from wood, primarily working with branches to carve bows, musical instruments, and some simple armor. A simple branch of crafting, it provides many of an adventurer's common necessities, particularly for those with more ranged styles. A good woodcarver never fears running out of arrows and can find their materials quite readily.

\section*{Quick Reference}

While each step will go into more depth, the quick reference allows you to at a glance follow the steps to work items from leather.

\begin{itemize}
  \item Select the item that you would like to craft from any of the Woodcarving Tables.
  \item Acquire the items listed in the materials column for that item.
  \item Use your woodcarver’s tools to craft the option using the number of hours listed in the Crafting Time column, or during a long rest using the crafting camp action if the crafting time is 2 hours or less.
  \item For every 2 hours, make a crafting roll of 1d20 + your Dexterity + your proficiency bonus with woodcarver’s tools.
  \item On success, you mark 2 hours of completed time. Once the completed time is equal to the crafting time, the item is complete. On failure, the crafting time is lost, and no progress has been made during the 2 hours. If you fail 3 times in a row, the crafting is a failure, and all materials are lost.
\end{itemize}

\section*{Related Tool \& Ability Score}

Woodworking works using woodcarver’s tools. Attempting to craft items with Woodworking without these will almost always be made with disadvantage, and proficiency with these allows you to add your proficiency in them to any Woodworking crafting roll.

\section*{Materials: Wood}

Woodworking is a bit more straight forward than other types of crafting in that it primarily has only category of material: wood. This can come in several different form factors, but is measured in branches, consistent pieces of useful wood.

\section*{Crafting Roll}

\begin{minipage}{0.48\textwidth}
Putting that together means that when you would like to make an item, your crafting roll is as follows:

Woodworking Modifier = your Woodcarver’s Tools proficiency bonus + your Dexterity modifier
\end{minipage}\hfill
\begin{minipage}{0.48\textwidth}
\subsubsection*{Success and Failure}

For Woodworking, after you make the crafting roll and succeed, mark your progress on a crafting project. If you succeed, you make 2 hours of progress toward the total crafting time (and have completed one of the required checks for making an item). Checks for Woodworking do not need to be immediately consecutive. Failure means that no progress is made during that time. Once an item is started, even if no progress is made, the components are reserved for that item.
\end{minipage}

\subsection*{Carved Branches}

Woodworkers can increase the rarity of a branch through expert carving techniques. A wooden branch of higher quality produced through woodcarving can’t be used as the material to make a branch of even higher rarity, but otherwise a carved branch serves as a branch of their finished rarity for crafting (such as enchanting or other woodworking items that call for a branch of that rarity).

Effectively, carving the branch can only increase the rarity of a branch by one step.

\section{Woodcarving}

\section*{Woodcarving Crafting Tables}

\section*{Weapons}

\begin{tabularx}{\textwidth}\toprule
{}XXXXXXXXXX}
\midrule
\multicolumn{2}{c}{Name} & \multicolumn{4}{c}{Materials} & Crafting Time & Checks & Difficulty & Rarity & Value \\
\midrule
\multicolumn{2}{c}{Shortbow} & \multicolumn{4}{c}{1 quality branch 
1 leather scraps 
1 length of string} & 12 hours (1.5 days) & 6 & DC 12 & common & 25 gp \\
\midrule
\multicolumn{2}{c}{Longbow} & \multicolumn{4}{c}{1 quality branch 
1 leather scraps 
1 length of string} & 14 hours (1.75 days) & 7 & DC 13 & common & 50 gp \\
\midrule
\multicolumn{2}{c}{Composite Bow (B)} & \multicolumn{4}{c}{1 common branch 
4 leather scraps 
1 length of string} & 24 hours (3 days) & 12 & DC 12 & common & 50 gp \\
\midrule
\multicolumn{2}{c}{Quarterstaff} & \multicolumn{4}{c}{1 common branch} & 2 hours & 1 & DC 10 & common & 2 sp \\
\midrule
\end{tabularx}

\section*{Armor}

\begin{tabularx}{\textwidth}\toprule
{}XXXXXXXXXX}
\midrule
\multicolumn{2}{c}{Name} & \multicolumn{4}{c}{Materials} & Crafting Time & Checks & Difficulty & Rarity & Value \\
\midrule
\multicolumn{2}{c}{Wooden Shield} & \multicolumn{4}{c}{4 common branches 
1 metal scraps 
1 leather scraps} & 8 hours & 4 & DC 12 & common & 10 gp \\
\midrule
\end{tabularx}

\section*{Musical Instruments}

\begin{tabularx}{\textwidth}\toprule
{}XXXXXXXXXX}
\midrule
\multicolumn{2}{c}{Name} & \multicolumn{4}{c}{Materials} & Crafting Time & Checks & Difficulty & Rarity & Value \\
\midrule
\multicolumn{2}{c}{Flute} & \multicolumn{4}{c}{1 quality branch} & 8 hours & 4 & DC 9 & common & 4 gp \\
\midrule
\multicolumn{2}{c}{Harp} & \multicolumn{4}{c}{3 quality branches 
4 lengths of string} & 16 hours (2 days) & 8 & DC 12 & common & 35 gp \\
\midrule
\multicolumn{2}{c}{Lute} & \multicolumn{4}{c}{2 quality branches 
2 lengths of string 
1 fancy parts} & 16 hours (2 days) & 8 & DC 13 & common & 35 gp \\
\midrule
\multicolumn{2}{c}{Drum} & \multicolumn{4}{c}{4 common branches 
1 rawhide leather 
1 parts} & 8 hours & 4 & DC 8 & common & 6 gp \\
\midrule
\multicolumn{2}{c}{Dulcimer} & \multicolumn{4}{c}{2 quality branches 
2 lengths of string 
1 fancy parts} & 8 hours & 4 & DC 11 & common & 25 gp \\
\midrule
\multicolumn{2}{c}{Lyre} & \multicolumn{4}{c}{2 quality branches 
2 lengths of string 
1 fancy parts} & 10 hours & 5 & DC 12 & common & 35 gp \\
\midrule
\multicolumn{2}{c}{Pan Flute} & \multicolumn{4}{c}{1 quality branch} & 8 hours & 4 & DC 11 & common & 12 gp \\
\midrule
\multicolumn{2}{c}{Shawm} & \multicolumn{4}{c}{1 quality branch} & 8 hours & 4 & DC 8 & common & 2 gp \\
\midrule
\multicolumn{2}{c}{Viol} & \multicolumn{4}{c}{3 quality branches 
2 lengths of string 
1 fancy parts} & 8 hours & 4 & DC 12 & common & 30 gp \\
\midrule
\end{tabularx}

\section*{Ammunition}

\begin{tabularx}{\textwidth}\toprule
{}XXXXXXXXXX}
\midrule
\multicolumn{2}{c}{Name} & \multicolumn{4}{c}{Materials} & Crafting Time & Checks & Difficulty & Rarity & Value \\
\midrule
\multicolumn{2}{c}{Arrow (10)} & \multicolumn{4}{c}{1 common branch 
1 metal scraps 
1 fletching} & 2 hours & 1 & DC 10 & common & 5 sp \\
\midrule
\multicolumn{2}{c}{Bolt (10)} & \multicolumn{4}{c}{1 common branch 
1 metal scraps 
1 fletching} & 2 hours & 1 & DC 10 & common & 5 sp \\
\midrule
\multicolumn{2}{c}{Dart (10)} & \multicolumn{4}{c}{2 wood scraps 
1 metal scraps 
1 fletching} & 2 hours & 1 & DC 11 & common & 5 sp \\
\midrule
\multicolumn{2}{c}{Blowgun Needle (10)} & \multicolumn{4}{c}{2 wood scraps 
1 fletching} & 2 hours & 1 & DC 8 & common & 1 sp \\
\midrule
\end{tabularx}

\section*{Miscellaneous}

\begin{tabularx}{\textwidth}\toprule
{}XXXXXXXXXX}
\midrule
\multicolumn{2}{c}{Name} & \multicolumn{4}{c}{Materials} & Crafting Time & Checks & Difficulty & Rarity & Value \\
\midrule
\multicolumn{2}{c}{Short Haft} & \multicolumn{4}{c}{1 common branch} & 2 hours & 1 & DC 8 & common & 1 sp \\
\midrule
\multicolumn{2}{c}{Long Haft} & \multicolumn{4}{c}{1 common branch} & 2 hours & 1 & DC 8 & common & 2 sp \\
\midrule
\multicolumn{2}{c}{Wooden Stock} & \multicolumn{4}{c}{1 common branch} & 2 hours & 1 & DC 9 & common & 5 sp \\
\midrule
\multicolumn{2}{c}{10 Foot Pole} & \multicolumn{4}{c}{1 common branch} & 2 hours & 1 & DC 8 & common & 3 sp \\
\midrule
\multicolumn{2}{c}{Fishing Pole} & \multicolumn{4}{c}{1 quality branch 
1 parts 
3 lengths of string} & 4 hours & 2 & DC 8 & common & 5 gp \\
\midrule
\multicolumn{2}{c}{Carved Figurine} & \multicolumn{4}{c}{1 wood scraps} & 8 hours & 4 & DC 8 & common & 4 sp \\
\midrule
\multicolumn{2}{c}{Quality Figurine} & \multicolumn{4}{c}{1 quality branch} & 8 hours & 4 & DC 14 & common & 50 gp \\
\midrule
\multicolumn{2}{c}{Superb Figurine} & \multicolumn{4}{c}{1 rare branch} & 8 hours & 4 & DC 18 & common & 400 gp \\
\midrule
\multicolumn{2}{c}{Wood Scraps (5)} & \multicolumn{4}{c}{1 common branch} & 2 hours & 1 & DC 5 & common & 1 sp \\
\midrule
\end{tabularx}

\section*{Carved Branches}

\begin{tabularx}{\textwidth}\toprule
{}XXXXXXXXXX}
\midrule
\multicolumn{2}{c}{Name} & \multicolumn{4}{c}{Materials} & Crafting Time & Checks & Difficulty & Rarity & Value \\
\midrule
\multicolumn{2}{c}{Quality Branch} & \multicolumn{4}{c}{1 common branch} & 4 hours & 2 & DC 9 & common & 2 gp \\
\midrule
\multicolumn{2}{c}{Uncommon Branch} & \multicolumn{4}{c}{1 quality branch} & 4 hours & 2 & DC 14 & uncommon & 25 gp \\
\midrule
\multicolumn{2}{c}{Rare Branch} & \multicolumn{4}{c}{1 uncommon branch} & 6 hours & 3 & DC 15 & rare & 70 gp \\
\midrule
\multicolumn{2}{c}{Very Rare Branch} & \multicolumn{4}{c}{1 rare branch} & 12 hours (1.5 days) & 6 & DC 19 & very rare & 800 gp \\
\midrule
\multicolumn{2}{c}{Legendary Branch} & \multicolumn{4}{c}{1 very rare branch} & 12 hours (1.5 days) & 6 & DC 20 & legendary & 2,000 gp \\
\midrule
\end{tabularx}

\begin{itemize}
  \item The material used can’t be the result of carving a branch of a lower quality.
\end{itemize}

\section*{Exotic Wood}

\begin{tabularx}{\textwidth}\toprule
{}XXXXX}
\midrule
Modifier & \multicolumn{4}{c}{Effect} & Difficulty Modifier \\
\midrule
Resonant & \multicolumn{4}{c}{Instruments made from these special types of wood provide +1 bonus to your spell save DC when used as a spellcasting focus.} & +8 \\
\midrule
Brittle & \multicolumn{4}{c}{This inferior type of wood causes weapons made of it to break when rolling a 1, or armor made of it to break when you are struck by a critical hit.} & +0 \\
\midrule
Featherlight & \multicolumn{4}{c}{This unique light but sturdy wood reduces the weight of things made of it by half. Ammunition doesn’t have disadvantage when attacking at long range.} & +4 \\
\midrule
\end{tabularx}

\section*{Woodcarving Recipes}

\begin{minipage}{0.48\textwidth}
\subsubsection*{Ammunition}

\begin{itemize}
  \item Arrow (10)
  \item Blowgun Needle (10)
  \item Bolts (10)
  \item Dart (10)
\end{itemize}

\subsubsection*{Armor}

\begin{itemize}
  \item Wooden Shield
\end{itemize}

\subsubsection*{Carved Branches}

\begin{itemize}
  \item Legendary Branch
  \item Quality Branch
  \item Rare Branch
  \item Uncommon Branch
  \item Very Rare Branch
\end{itemize}

\subsubsection*{Miscellaneous}

\begin{itemize}
  \item Carved Figurine
  \item Fishing Pole
  \item Long Haft
  \item Quality Figurine
  \item Short Haft
  \item Superb Figurine
  \item 10 Foot Pole
  \item Wooden Stock
  \item Wood Scraps
\end{itemize}
\end{minipage}\hfill
\begin{minipage}{0.48\textwidth}
\subsubsection*{Musical Instruments}

\begin{itemize}
  \item Drum
  \item Dulcimer
  \item Flute
  \item Harp
  \item Lute
  \item Lyre
  \item Pan Flue
  \item Shawm
  \item Viol
\end{itemize}

\subsubsection*{Weapons}

\begin{itemize}
  \item Composite Bow
  \item Longbow
  \item Quarterstaff
  \item Shortbow
\end{itemize}
\end{minipage}

\section*{Woodcarving Items}

\subsection*{Composite Bow}

% [Image Inserted Manually]

Martial Ranged Weapon, Ammunition, Heavy, (Range 150/600), Two-Handed

A bow made from laminated layers of horn, wood, and sinew. Deals 2d4 piercing damage on hit.

\begin{itemize}
  \item Item: Composite Bow
\end{itemize}

\section{Runecarving}

\section*{Runecarving}

Runecarving is the art of marking special magical enhancements in the form of magical runes. These are symbolic representations that channel and focus power to confer a special magical effect to the item they are marked on.

\section*{Quick Reference}

The following is a quick reference to follow for each step of carving runes:

\begin{itemize}
  \item Select a basic item you would like to mark a rune on. This can be an mundane item, gem, nonmagical weapon, or nonmagical armor.
  \item Select a rune you would like to mark on that item from the table based on the item type.
  \item Acquire the materials listed in the materials column for that rune.
  \item Use your Related Tool (based on your Runecraving tradition) to craft the option using the number of hours listed in the crafting time column. You can make progress in 2 hour increments.
  \item For every 2 hour increment, make a crafting roll of 1d20 + your Wisdom or Intelligence (based on your Tradition) + your Proficiency bonus with your related tool.
  \item On success, you mark 2 hours of completed time. Once the completed time is equal to the crafting time, the item is complete. On failure, the crafting time is lost and no progress has been made during the 2 hours. If you fail three times in a row, the crafting is a failure and all materials are lost.
\end{itemize}

\section*{Related Tool \& Ability Score}

Rune carving comes in two different traditions; you can mark runes from either tradition, but they require different tool proficiency and ability modifiers.

\begin{minipage}{0.48\textwidth}
\subsubsection*{Ancient Tradition}

The runes of the Giants, Dwarves, and other ancient traditions. Your ability modifier for this tradition is Wisdom, and you can select your related tool from Mason's Tools or Painter's Supplies.
\end{minipage}\hfill
\begin{minipage}{0.48\textwidth}
\subsubsection*{Academic Tradition}

The runes of mages, scholors, and elves, these traditions are not necessarily any less ancient, but runecarvers of these traditions seek new knowledge and expression, rather than being rooted in the old ways. Your ability modifier for this tradition is Intelligence, and you can select your related tool from Calligrapher's Supplies or Woodcarver's Tools.
\end{minipage}

\section*{Materials}

Runecarving uses primarily magical ink and essences to imbue their runes with the prerequisite magical properties.

\section*{Crafting Roll}

\begin{minipage}{0.48\textwidth}
Puting that together means that when you would like to make a rune, your crafting roll is as follows based on your tradition:

Ancient Tradition Runecarver Modifier = your Mason's Tools or Painter's Supplies (your choice) proficiency bonus + your Wisdom modifier

Academic Tradition Runecarver Modifier = your Calligrapher's Supplies or Woodcarver's Tools (your choice) proficiency bonus + your Intelligence modifier
\end{minipage}\hfill
\begin{minipage}{0.48\textwidth}
\subsubsection*{Success and Failure}

After you make a crafting roll, if you succeed, you make 2hours of progress toward the total crafting time (and have completed one of the required checks for making an item).

Checks for Runecarving do not need to be immediately consecutive. If you fail three times in a row, all progress and materials are lost and can no longer be salvaged. Failure means that no progress is made during that time.

Once an item is started, even if no progress is made, the components reserved for that item can only be recovered via salvage.
\end{minipage}

\section*{Runecarving Saving Throw}

Some runes require a saving throw, the following is the formula for calculating the saving throw. The saving throw is calculated at the time of creation based on the creators attributes and proficiency, and doesn't change once it is created. A saving throw doesn't include any expertise or other bonuses a crafter has to the crafting roll.

Rune DC = 8 + your related tool proficiency bonus + your Wisdom or Intelligence modifier (based on your Tradition).

Rune Attack Roll Modifier = your related tool proficiency bonus + your Wisdom or Intelligence modifier (based on Tradition).

\section*{Placing a Rune}

You can place a rune a nonmagical weapon, set of armor, or item. When you place an rune on an item, that item requires attunement. When a creature is attuned to it, the rune grants the wielder certain benefits as defined by the rune. An item becomes a magical item of the rarity of the rune placed on it when a rune is palced on it

\begin{minipage}{0.48\textwidth}
\subsubsection*{Runes on Magical Items}

Placing a rune on a magical item is exceedingly difficult. Magical items resist modification, and their magic interfers with the rune. It can be done with additional difficult as per the table below. If the item has attunement, the rune still requires seperate attunement.
\end{minipage}\hfill
\begin{minipage}{0.48\textwidth}
\begin{tabularx}{\textwidth}\toprule
{}XX}
\midrule
Item Rarity & \multicolumn{2}{c}{Rune Difficulty Modifier} \\
\midrule
Commmon & \multicolumn{2}{c}{+5} \\
\midrule
Uncommon & \multicolumn{2}{c}{+7} \\
\midrule
Rare & \multicolumn{2}{c}{+10} \\
\midrule
Very Rare & \multicolumn{2}{c}{+14} \\
\midrule
Legendary & \multicolumn{2}{c}{+20} \\
\midrule
\end{tabularx}
\end{minipage}

\section{Runecarving}

\section*{Runecarving Crafting Tables}

\section*{Ancient Tradition}

\subsection*{Ancient Tradition}

\begin{tabularx}{\textwidth}\toprule
{}XXXXXXXXXX}
\midrule
\multicolumn{2}{c}{Name} & \multicolumn{4}{c}{Materials} & Crafting Time & Checks & Difficulty & Rarity & Value \\
\midrule
\multicolumn{2}{c}{Light} & \multicolumn{4}{c}{1 common primal essence 
1 common magical ink} & 4 hours & 2 & DC 12 & common & 70 gp \\
\midrule
\multicolumn{2}{c}{Fire} & \multicolumn{4}{c}{1 uncommon primal essence 
1 uncommon magical ink 
2 uncommon reactive reagents} & 6 hours & 3 & DC 14 & uncommon & 320 gp \\
\midrule
\multicolumn{2}{c}{Frost} & \multicolumn{4}{c}{1 uncommon primal essence 
1 uncommon magical ink 
2 common primal essence} & 6 hours & 3 & DC 14 & uncommon & 330 gp \\
\midrule
\multicolumn{2}{c}{Lightning} & \multicolumn{4}{c}{1 uncommon primal essence 
1 uncommon magical ink 
1 common primal essence 
1 common arcane essence} & 6 hours & 3 & DC 14 & uncommon & 330 gp \\
\midrule
\multicolumn{2}{c}{Size} & \multicolumn{4}{c}{1 uncommon primal essence 
1 uncommon magical ink 
1 scroll of enlarge/reduce} & 6 hours & 3 & DC 14 & uncommon & 320 gp \\
\midrule
\multicolumn{2}{c}{Vision} & \multicolumn{4}{c}{1 uncommon primal essence 
1 uncommon magical ink 
1 common arcane essence 
1 common divine essence} & 6 hours & 3 & DC 14 & uncommon & 330 gp \\
\midrule
\multicolumn{2}{c}{Brutality} & \multicolumn{4}{c}{3 rare primal ressences 
1 rare magical ink} & 12 hours (1.5 days) & 6 & DC 16 & rare & 3,300 gp \\
\midrule
\multicolumn{2}{c}{Destruction} & \multicolumn{4}{c}{3 rare primal essence 
3 uncommon primal essence 
1 rare magical ink} & 12 hours (1.5 days) & 6 & DC 16 & rare & 3,500 gp \\
\midrule
\multicolumn{2}{c}{Vigor} & \multicolumn{4}{c}{1 rare primal essence 
1 uncommon magical ink 
4 rare curative reagents 
1 uncommon divine essence} & 12 hours (1.5 days) & 6 & DC 15 & rare & 1,900 gp \\
\midrule
\multicolumn{2}{c}{Wrath} & \multicolumn{4}{c}{1 very rare primal essence 
2 rare primal essence 
1 very rare magical ink} & 16 hours (2 days) & 8 & DC 17 & very rare & 12,000 gp \\
\midrule
\multicolumn{2}{c}{Death} & \multicolumn{4}{c}{1 legendary divine essence 
1 legendary primal essence 
1 legendary magical ink 
4 rare primal essences} & 24 hours (3 days) & 12 & DC 19 & legendary & 65,000 gp \\
\midrule
\end{tabularx}

% [Image Inserted Manually]

\subsection*{Academic Tradition}

\begin{tabularx}{\textwidth}\toprule
{}XXXXXXXXXX}
\midrule
\multicolumn{2}{c}{Name} & \multicolumn{4}{c}{Materials} & Crafting Time & Checks & Difficulty & Rarity & Value \\
\midrule
\multicolumn{2}{c}{Color} & \multicolumn{4}{c}{1 common magical ink} & 2 hours & 1 & DC 13 & common & 20 gp \\
\midrule
\multicolumn{2}{c}{Comprehension} & \multicolumn{4}{c}{1 scroll of comprehend languages 
1 uncommon arcane essence 
1 common divine essence 
1 uncommon magical ink} & 4 hours & 2 & DC 14 & uncommon & 350 gp \\
\midrule
\multicolumn{2}{c}{Connection} & \multicolumn{4}{c}{1 scroll of detect thoughts 
1 scroll of calm emotions 
1 uncommon divine essence 
1 uncommon magical ink} & 4 hours & 2 & DC 14 & uncommon & 390 gp \\
\midrule
\multicolumn{2}{c}{Gravity} & \multicolumn{4}{c}{1 uncommon arcane essence 
1 uncommon magical ink 
1 scroll of levitate} & 4 hours & 2 & DC 15 & uncommon & 320 gp \\
\midrule
\multicolumn{2}{c}{Protection} & \multicolumn{4}{c}{1 uncommon arcane essence 
1 uncommon magical ink 
1 scroll of shield} & 4 hours & 2 & DC 15 & uncommon & 305 gp \\
\midrule
\multicolumn{2}{c}{Skill} & \multicolumn{4}{c}{1 uncommon arcane essence 
1 scroll of enhance ability 
1 uncommon magical ink} & 4 hours & 2 & DC 15 & uncommon & 490 gp \\
\midrule
\multicolumn{2}{c}{Blood} & \multicolumn{4}{c}{1 rare arcane essence 
1 scroll of vampiric touch 
1 rare magical ink} & 8 hours & 4 & DC 16 & rare & 1,400 gp \\
\midrule
\multicolumn{2}{c}{Power} & \multicolumn{4}{c}{1 rare arcane essence 
1 rare magical ink 
2 uncommon arcane essence} & 10 hours & 5 & DC 17 & rare & 1,600 gp \\
\midrule
\multicolumn{2}{c}{Space} & \multicolumn{4}{c}{1 rare arcane essence 
1 scroll of dimension door 
2 uncommon arcane essence 
1 rare magical ink} & 10 hours & 5 & DC 17 & rare & 2,000 gp \\
\midrule
\multicolumn{2}{c}{Reality} & \multicolumn{4}{c}{1 scroll of major image 
1 scroll of creation 
1 very rare arcane essence 
2 rare arcane essence 
1 very rare magical ink} & 16 hours (2 days) & 8 & DC 18 & very rare & 14,000 gp \\
\midrule
\multicolumn{2}{c}{Speed} & \multicolumn{4}{c}{1 very rare arcane essence 
1 very rare magical ink 
1 scroll of haste} & 12 hours (1.5 days) & 6 & DC 17 & very rare & 10,500 gp \\
\midrule
\multicolumn{2}{c}{Time} & \multicolumn{4}{c}{1 scroll of time stop 
1 legendary arcane essence 
1 very rare arcane essence 
1 legendary magical ink 
2 rare divine essences} & 24 hours (3 days) & 12 & DC 21 & legendary & 100,000
gp \\
\midrule
\end{tabularx}

\section*{Runecarving Recipes}

\begin{minipage}{0.48\textwidth}
\subsubsection*{Ancient Tradition}

\begin{itemize}
  \item Rune of Brutality
  \item Rune of Death
  \item Rune of Destruction
  \item Rune of Fire
  \item Rune of Frost
  \item Rune of Light
  \item Rune of Lightning
  \item Rune of Size
  \item Rune of Vigor
  \item Rune of Vision
  \item Rune of Wrath
\end{itemize}
\end{minipage}\hfill
\begin{minipage}{0.48\textwidth}
\subsubsection*{Academic Tradition}

\begin{itemize}
  \item Rune of Color
  \item Rune of Comprehension
  \item Rune of Connection
  \item Rune of Blood
  \item Rune of Gravity
  \item Rune of Power
  \item Rune of Protection
  \item Rune of Reality
  \item Rune of Skill
  \item Rune of Space
  \item Rune of Speed
  \item Rune of Time
\end{itemize}
\end{minipage}

\section{Runecarving}

\section*{Runes}

\section*{Ancient Tradition}

\begin{minipage}{0.48\textwidth}
\subsubsection*{Light}

Rune, common

A simple circular rune that shines a constant white light. A creature that touches an item bearing this rune can cast the light spell targeting that item.

\begin{itemize}
  \item Item: Rune of Light
\end{itemize}

\subsubsection*{Fire}

Rune, uncommon (requires attunement)

A chaotic flowing rune with no straight lines. It pulses with a flickering orange energy.

Passive Effect. When marked on armor or an item, an attuned wearer gains resistance to fire damage. When marked on a weapon, an attuned wielder deals 1d4 additional fire damage on hit with that weapon.

Active Effect. While attuned to an item with this rune, you can use that item to cast burning hands (2nd level) as a bonus action without expending a spell slot. This uses the crafter’s Rune DC (set at the time of carving the rune). Once a rune’s power is used, the rune’s active can’t be used again until the next dawn.

\begin{itemize}
  \item Item: Rune of Fire
\end{itemize}

\subsubsection*{Lightning}

Rune, uncommon (requires attunement)

A jagged rune that flashes and crackles with blue light.

Passive Effect. When marked on armor or an item, the wearer’s speed is increased by 10 feet. When marked on a weapon, an attuned wielder deals an extra 1d4 lightning damage on hit with that weapon.

Active Effect. While attuned to an item with this rune, you can use that item to cast crackle(B) as a bonus action without expending a spell slot. This uses the crafter’s Rune Attack Modifier and Rune DC (set at the time of carving the rune). Once a rune’s power is used, the rune’s active can’t be used again until the next dawn.

\begin{itemize}
  \item Item: Rune of Lightning
\end{itemize}

\subsubsection*{Frost}

Rune, uncommon (requires attunement)

An angular geometric rune that softly pulses with a faint white light.

Passive Effect. When marked on armor or an item, an attuned wearer gains resistance to cold damage. When marked on a weapon, an attuned wielder deals 1d4 additional cold damage on hit with that weapon.

Active Effect. While attuned to an item with this rune, you can grant yourself 10 temporary hit points as a bonus action, lasting for 1 hour. If a creature hits you with a melee attack while you have these hit points, the creature takes 5 cold damage. Once a rune’s power is used, the rune’s active can’t be used again until the next dawn.

\begin{itemize}
  \item Item: Rune of Frost
\end{itemize}

\subsubsection*{Vision}

Rune, uncommon (requires attunement)

An eye shaped rune that dimly pulses a faint purple light.

Passive Effect. When marked on armor or an item, an attuned wearer grains dark vision with a range of 30 feet. If the attuned wearer already has darkvision, the range of the darkvision increases by 30 feet.

Active Effect. While attuned to an item with this rune, you can grant yourself 60 feet of blindsight as a bonus action. This lasts until the start of your next turn. Once a rune’s power is used, the rune’s active can’t be used again until the next dawn.

\begin{itemize}
  \item Item: Rune of Vision
\end{itemize}

\subsubsection*{Size}

Rune, uncommon (requires attunement)

A rune of layered circles with a pale green light flowing through them.

Passive Effect. When marked on armor or an item, an attuned wearer grows 1 foot, counts as one size larger when determining your carrying capacity and the weight you can push, drag, or lift. When marked on a weapon, the weapon grows larger when wielded by an attuned wielder and gains the heavy property if it doesn’t already have it (losing the light property if it has the light property).

Active Effect. While attuned to an item with this rune, you can use that item to cast enlarge/reduce as a bonus action without expending a spell slot. Once a rune’s power is used, the rune’s active can’t be used again until the next dawn.

\begin{itemize}
  \item Item: Rune of Size
\end{itemize}
\end{minipage}\hfill
\begin{minipage}{0.48\textwidth}
\subsubsection*{Brutality}

Rune, rare (requires attunement)

A rune made of large chaotic interlocking pattern that burns with a dull orange light.

Passive Effect. When marked on a weapon, an attuned wielder of the weapon is healed for 1d12 hit points when dealing a critical strike with the marked weapon.

Active Effect. An attuned wielder can activate the rune (no action required) during their turn. Until the end of their turn, whenever they miss an attack roll against a target within their reach, the attack still deals half damage to the target. Once activated, it can’t be activated until the next dawn.

\begin{itemize}
  \item Item: Rune of Brutality
\end{itemize}

\subsubsection*{Destruction}

Rune, rare (requires attunement)

A simple rune with bold slashing struct that glow with vibrant and pulsing red light.

Passive Effect. When marked on a weapon, the weapon deals an extra 1d6 damage on hit when wielded by an attuned wearer. Its damage overcomes all resistance to damage.

Active Effect. When the attuned wielder makes an attack with a weapon marked with this rune, they can make the attack a critical hit. Once activated, it can’t be activated again until the next dawn.

\begin{itemize}
  \item Item: Rune of Destruction
\end{itemize}

\subsubsection*{Vigor}

Rune, rare (requires attunement)

A bold rune with intertwined lines that hums slightly, beating with a faint red light.

Passive Effect. When marked on armor or an item, an attuned wearer can spend a Hit Die as a bonus action, rolling it and regaining hit points as normal. When marked on a weapon, an attuned wielder gains temporary hit points equal to the Constitution modifier when dealing damage to a living target.

Active Effect. While attuned to an item with this rune, you can use that item to cast cure wounds (2nd level) as a bonus action without expending a spell slot. Once a rune’s power is used, the rune’s active can’t be used again until the next dawn.

\begin{itemize}
  \item Item: Rune of Vigor
\end{itemize}

\subsubsection*{Wrath}

Rune very rare (requires attunement)

A jagged chaotic rune that spreads out in all directions and flashes with angry red light.

Passive Effect. When marked on an item, the attuned wearer can use their reaction to inflict 1d6 lightning damage on any creature within 30 feet of them that deals damage to them.

Active Effect. While attuned to an item with this rune, you can use that item to cast chain lightning without expending a spell slot. Once activated, it can’t be activated again until the next dawn.

\begin{itemize}
  \item Item: Rune of Wrath
\end{itemize}

\subsubsection*{Death}

Rune, legendary (requires attunement)

A rune of ominous complexity wreathed in shadows.

Passive Effect. While marked on armor or any item, the attuned wearer can’t die unless the rune is removed from the possession. They are still incapacitated and unconscious when reduced to 0 hit points but reaching 3 death saving throws has no effect while the rune is being worn or carried on their person. While attuned to this rune, undead of CR 5 or lower treat you as another undead, and only attack you if otherwise commanded to by a controlling entity or attacked by you.

Active Effect. While attuned to an item with this rune, you can use item to cast power word kill without expending a spell slot. Once this rune’s power is used, you can’t use it again until the next dawn.

\begin{itemize}
  \item Item: Rune of Death
\end{itemize}
\end{minipage}

\section*{Academic Tradition}

\begin{minipage}{0.48\textwidth}
\subsubsection*{Color}

Rune, common

A multifaced rune that glimmers a shifting color. A creature that touches an item with this rune can change the color of that item as an action.

\begin{itemize}
  \item Item: Rune of Color
\end{itemize}

\subsubsection*{Protection}

Rune, uncommon (requires attunement)

A complicated rune with twisting multilayered geometric markings that glimmers a dull grey.

Passive Effect. When marked on armor or an item, any time an attuned wearer takes damage, that damage is reduced by 1.

Active Effect. While attuned to an item with this rune, you can use that item to cast shield without expending a spell slot. Once a rune’s power is used, the rune’s active can’t be used again until the next dawn.

\begin{itemize}
  \item Item: Rune of Protection
\end{itemize}

\subsubsection*{Gravity}

Rune, uncommon (requires attunement)

A rune formed of detailed concentric circles that faintly glimmers with soft purple glow.

Passive Effect. When marked on armor or an item, or a weapon, an attuned wielder takes no falling damage.

Active Effect. While attuned to an item with this rune, you can use that item to cast levitate as a bonus action without expending a spell slot. This uses the creator’s Rune DC (set at the time of carving the rune). Once a rune’s power is used, the rune’s active can’t be used again until the next dawn.

\begin{itemize}
  \item Item: Rune of Gravity
\end{itemize}

\subsubsection*{Comprehension}

Rune, uncommon (requires attunement)

A rune that resembles a script that no one can determine the nature of, glowing a steady white-blue light.

Passive Effect. When marked on armor or an item, the attuned wearer can read all writing, even if it is not in a language they would normally understand. This confers only the literal meaning of the text and doesn’t break a cipher or code.

Active Effect. While attuned to an item with this rune, you can cast comprehend languages without expending a spell slot. Once activated, it can’t be activated again until the next dawn.

\begin{itemize}
  \item Item: Rune of Comprehension
\end{itemize}

\subsubsection*{Connection}

Rune, uncommon (requires attunement)

A rune made up of an interlocking square, circle, and triangle, that courses with pure white light.

Passive Effect. When marked on an item, an attuned wearer can spend an action to gain supernatural insight into other sentient creatures for 10 minutes. You gain advantage on all Wisdom (Insight) checks but dealing damage to another sentient creature while in this heightened state of awareness causes you to take 1 psychic damage.

Active Effect. While attuned to an item with this rune, you can force another creature to experience a brief flash of connection to your state of mind. The effects of this may vary, but most often allows you to gain advantage on a Charisma (Persuasion) check with them. If used as a reaction when taking damage, you can force the creature that dealt the damage to you to take psychic damage equal to half the damage dealt. Once activated, it can’t be activated again until the next dawn.

\begin{itemize}
  \item Item: Rune of Connection
\end{itemize}

\subsubsection*{Skill}

Rune, uncommon (requires attunement)

An intricate rune that pulses with swirling lights, varying in design to incorporate a mark of the skill granted.

Passive Effect. When marked on an item, an attuned wearer gains proficiency in one skill, selected when the rune is marked by the creator of the rune.

Active Effect. While attuned to an item with this rune, you can activate this rune gain advantage on an ability check for the skill granted by this rune. Once activated, it can’t be activated again until the next dawn.

\begin{itemize}
  \item Item: Rune of Skill
\end{itemize}

\subsubsection*{Blood}

Rune, rare (requires attunement)

A swirling rune with repeating spiraling patterns that pulses with a deep red light.

Passive Effect. When marked on a weapon, if an attuned wielder deals damage with that weapon to a living creature of CR 1/4 or higher, the weapon deals an extra 1d4 necrotic damage, and the wielder regains hit points equal to the necrotic damage dealt.

Active Effect. After dealing necrotic damage with this weapon, an attuned wielder can use their bonus action to activate the rune, drawing the blood of the creature, dealing an extra 3d4 necrotic damage, and the wielder regains hit points equal to the additional necrotic damage dealt. Once a rune’s power is used, the rune’s active can’t be used again until the next dawn.

\begin{itemize}
  \item Item: Rune of Blood
\end{itemize}
\end{minipage}\hfill
\begin{minipage}{0.48\textwidth}
\subsubsection*{Power}

Rune, rare (requires attunement by a spellcaster)

A vibrant rune with intricate line work that shines a bright cyan.

Passive Effect. When wearing armor or an item marked with this rune, once per turn when the wearer rolls for damage with a spell, they can maximize one die of the damage roll. When wielding a weapon marked with this rune, the damage of that weapon is considered magical, and once per turn, when the wielder rolls damage for an attack with that weapon, they can maximize one damage die.

Active Effect. While attuned to an item with this rune, when you cast a spell, you can use this rune to empower it, casting it as if it was one level higher. Once activated, it can’t be activated again until the next dawn.

\begin{itemize}
  \item Item: Rune of Power
\end{itemize}

\subsubsection*{Speed}

Rune, very rare (requires attunement)

A flickering white rune with jagged lines.

Passive Effect. When marked on armor or an item, an attuned wearer can take the dash action as a bonus action on their turn. When marked on a weapon with light property, an attuned wearer can make a single weapon attack with the bonus action on their turn.

Active Effect. While attuned to an item with this rune, you can use that item to cast haste without expending a spell slot. Once activated, it can’t be activated again until the next dawn.

\begin{itemize}
  \item Item: Rune of Speed
\end{itemize}

\subsubsection*{Space}

Rune, rare (requires attunement)

A deceptively simply rune made of up lines of varying lengths in an odd arrangement that glows a steady dim grey light.

Passive Effect. On your turn, you can replace your movement by teleporting 10 feet in any direction to an empty space you can see.

Active Effect. You can cast dimension door without expending a spell slot. Once activated, it can’t be activated again until the next dawn.

\begin{itemize}
  \item Item: Rune of Space
\end{itemize}

\subsubsection*{Reality}

Rune, very rare (requires attunement)

A fanciful twisting rune that confuses the eye with ever shifting patterns of colors.

Passive Effect. When marked on an item, an attuned wearer can cast ever more convincing illusions. Creatures have disadvantage on their first Intelligence (Investigation) check to determine if a spell you cast is an illusion (the first check it makes against that spell).

Active Effect. You can cause one illusion you’ve made to become real. This affects a 10-foot by 10-foot cube of the illusion if it is larger than that. This can’t create an object worth more than 100 gp, lasts 1 hour, and can’t create a creature of a CR greater than 5 (when attempting to, it may create a weakened version of that creature, or simply fail). If the illusion directly causes damage (from falling on a creature, or otherwise being harmful) it can deal a maximum of 8d8 damage (potentially split across multiple creatures). This limit doesn’t apply to creatures created through it. Once activated, it can’t be activated again until the next dawn.

\begin{itemize}
  \item Item: Rune of Reality
\end{itemize}

\subsubsection*{Time}

Rune, legendary (requires attunement)

A complicated geometric rune that slowly pulses with pale grey light.

Passive Effect. When marked on armor or an item, your speed increases by 20 feet, you gain a +2 bonus to AC and Dexterity saving throws, and you can take two reactions per round, though you can’t take multiple reactions on the same turn. You age twice as fast while attuned to this rune.

Active Effect. While attuned to an item with this rune, you can use that item to cast time stop without expending a spell slot. Alternatively, you can use this rune as a reaction failing a Dexterity saving throw or being hit by an attack to automatically pass a Dexterity saving throw or cause an attack to miss. You can use this ability after the roll to change its outcome. Once activated, it can’t be activated again until the next dawn.

\begin{itemize}
  \item Item: Rune of Time
\end{itemize}
\end{minipage}

\subsection*{Essence Rune}

Rune, very rare (requires attunement)

A special rune forged with the liquid of an essence crystal. This rune empowers an item or creature it is marked on, increasing the Ability Score and maximum for the Ability Score associated with the Essence Crystal used by the rune carver.

\begin{itemize}
  \item Item: Essence Rune
\end{itemize}

\section{Engineering}

% [Image Inserted Manually]

\section*{Engineering}

Engineering is an academic art of turning labor and materials into grand creations. At its smallest scale, it is used for pratical applications like siege equipment, but scales up to building bridges, buildings, and even ships. An engineer is a deeply sought after resource for any lord or kingdom, but has knowledge that savvy adventurers can put to great use on occasion.

\section*{Related Tool \& Ability Score}

Engineering uses carpenter's tools. Checks can be made with a substitute tool at your GM's discretion, but are made with disadvantage.

\section*{Quick Reference}

The following is a quick reference to follow for each step of an engineering project:

\begin{itemize}
  \item Select the construction project you would like to begin.
  \item Gather the required labor if necessary.
  \item Ensure you have all materials for the construction project available.
  \item Use your carpenter's tools to begin work, taking a number of hours listed in the crafting time column. You can make progress in 2 hours increments.
  \item For each 2 hour increment, make a crafting check of 1d20 + your Intelligence modifier + your proficiency with carpenter's tools.
  \item On success, you mark 2 hours of completed time. Once the completed time is equal to the crafting time, the construction project is complete. On failure, the crafting time is lost and no progress has been made during the 2 hours. If you fail three times in a row, the construction project is a failure and all materials are lost.
\end{itemize}

\section*{Materials}

\subsection*{Bulk Materials Table}

\begin{tabularx}{\textwidth}\toprule
{}X}
\midrule
Material & Price \\
\midrule
Unit of Lumber & 1 gp \\
\midrule
Unit of Stone & 1 gp \\
\midrule
Cement (Mortar) & 1 sp \\
\midrule
\end{tabularx}

\begin{minipage}{0.48\textwidth}
\subsubsection*{Harvesting Trees}

You can convert trees that are at least 20 feet tall into units of lumber. A 20 foot tall tree will produce 1 unit of lumber, and an additional unit of lumber for each 10 feet of tree height. At your GM's discrection, some trees may not be suitable lumber. It takes 4 hours of work by a laborer to convert trees to units of lumber.

A unit of lumber can be converted to 10 common branches
\end{minipage}\hfill
\begin{minipage}{0.48\textwidth}
\subsubsection*{Quarrying Stone}

Quarrying Stone is significantly harder and at the edge of what this system attempts to define. As a rule of thumb, if a quarry that can produce stone is available, 1 laborer can produce 1 unit of stone in 8 hours of hard labor (1 unit per day). A laborer with a Strength ability score of 18 or higher can produce twice as much stone in the same time period.

Setting up a quarry exceeds the scope of this system, but would take days or weeks.
\end{minipage}

\section*{Crafting Roll}

\begin{minipage}{0.48\textwidth}
When you would like to build an engineering project, it relies on your carpenter's tools, and the Intelligence to use them.

Engineering Modifier = your Carpenter’s Tools proficiency bonus + your Intelligence modifier
\end{minipage}\hfill
\begin{minipage}{0.48\textwidth}
\subsubsection*{Success and Failure}

After making a crafting roll, if you succeed, you make 2 hours of progress toward the total crafting time (and have completed one of the required checks for making an item).

Checks for Engineering do not need to be immediately consecutive. If you fail three times in a row, all progress and materials are lost and can no longer be salvaged. Failure means that no progress is made during that time.

Once an item is started, even if no progress is made, the components reserved for that item can only be recovered via salvage.
\end{minipage}

\section*{Labor}

Unlike most crafting, Engineering can encompass large projects that require additional laborers. A laborer can be any creature with a Strength of 12 or higher that can understand you, lift and carry objects, and is willing to work for you for the duration. You can be your own laborer if you fit the requirements. A creature with a Strength of 18 or higher or proficiency with carpenter's tools can count as a skilled laborer, and is worth two laborers. Two creatures of insufficient Strength ability score but with 8 or more Strength ability score working together can be counted as single laborer, but only for 4 hours a day, after which more labor confers a level of exhaustion.

Additional labor doesn't inherently speed a project up, though a GM may rule that it is possible in a case by case basis. Checks can be made with insufficient labor as long as you have half the labor pool needed, but each crafting check takes 4 hours in this case.

The number of laborers required is based on the scale of the project, as per the following table:

\begin{minipage}{0.48\textwidth}
\begin{tabularx}{\textwidth}\toprule
{}X}
\midrule
Project Scale & Laborers Needed \\
\midrule
Small & 0 \\
\midrule
Medium & 0 \\
\midrule
Large & 1 \\
\midrule
Huge & 2 \\
\midrule
Gargantuan (30x30) & 3 \\
\midrule
Colossal (40x40+) & 4+ \\
\midrule
\end{tabularx}
\end{minipage}\hfill
\begin{minipage}{0.48\textwidth}
\subsubsection*{Colossal Projects}

For colossal projects (projects bigger than Gargantuan), you need one laborer for every additional 10 square feet of the project beyond 40 by 40. For example, to build a bridge that is 200 feet long by 40 feet wide, you would need five times more laborers (20 laborers) to maintain a 2 hour crafting check.

Every laborer requires their own set of crafting tools, though they do not need proficiency with them. At a GM’s discretion, mason’s tools or woodcarver’s tools can replace carpenter’s tools for laborers where appropriate.
\end{minipage}

\section{Engineering}

\section*{Engineering Crafting Tables}

\section*{Siege Weapons}

\subsection*{Siege Weapons}

\begin{tabularx}{\textwidth}\toprule
{}XXXXXXXXXX}
\midrule
\multicolumn{2}{c}{Name} & \multicolumn{4}{c}{Materials} & Crafting Time & Checks & Difficulty & Rarity & Value \\
\midrule
\multicolumn{2}{c}{Ballista (Large)} & \multicolumn{4}{c}{2 units of lumber 
4 quality branches 
10 parts 
5 leather scraps
 rope (20 ft.)} & 8 hours & 4 & DC 16 & Common & 150 gp \\
\midrule
\multicolumn{2}{c}{Mangonel
(Large)} & \multicolumn{4}{c}{4 units of lumber 
4 quality branches 
20 parts 
2 fancy parts 
4 rawhide leather
 rope (40 ft.)} & 8 hours & 4 & DC 17 & Common & 300 gp \\
\midrule
\multicolumn{2}{c}{Ram (Large)} & \multicolumn{4}{c}{4 units of lumber 
10 rawhide leather 
10 parts 
2 ingots
 rope (20 ft.)} & 8 hours & 4 & DC 14 & Common & 65 gp \\
\midrule
\multicolumn{2}{c}{Trebuchet
(Huge)} & \multicolumn{4}{c}{8 units of lumber 
20 parts 
2 fancy parts 
10 leather scraps
 rope (60 ft.)} & 16 hours & 8 & DC 16 & Uncommon & 750 gp \\
\midrule
\multicolumn{2}{c}{Siege Tower
(Gargantuan)} & \multicolumn{4}{c}{10 units of lumber 
20 parts 
10 rawhide leather
 rope (60 ft.)} & 8 hours & 4 & DC 16 & Common & 120 gp \\
\midrule
\end{tabularx}

\section*{Fortifications}

\subsection*{Fortifications}

\begin{tabularx}{\textwidth}\toprule
{}XXXXXXXXXX}
\midrule
\multicolumn{2}{c}{Name} & \multicolumn{4}{c}{Materials} & Crafting Time & Checks & Difficulty & Rarity & Value \\
\midrule
\multicolumn{2}{c}{Low Stone Wall
(per 10 ft)} & \multicolumn{4}{c}{1 units of stone 
1 bucket of cement} & 8 hours & 4 & DC 10 & Common & 8 gp \\
\midrule
\multicolumn{2}{c}{Stone
Battlements
(per 10 ft)} & \multicolumn{4}{c}{5 units of stone 
5 bucket of cement} & 24 hours (3 days) & 12 & DC 14 & Common & 140 gp \\
\midrule
\multicolumn{2}{c}{Palisade
(per 10 ft)} & \multicolumn{4}{c}{2 units of lumber} & 2 hours & 1 & DC 10 & Common & 2 gp \\
\midrule
\multicolumn{2}{c}{Watch Tower} & \multicolumn{4}{c}{8 units of lumber 
5 parts} & 16 hours (2 days) & 8 & DC 12 & Common & 50 gp \\
\midrule
\multicolumn{2}{c}{Wooden
Battlements
(per 10 ft)} & \multicolumn{4}{c}{5 units of lumber 
4 parts 
4 metal scraps} & 4 hours & 2 & DC 12 & Common & 50 gp \\
\midrule
\end{tabularx}

\section*{Bridges}

\subsection*{Bridges}

\begin{tabularx}{\textwidth}\toprule
{}XXXXXXXXXX}
\midrule
\multicolumn{2}{c}{Name} & \multicolumn{4}{c}{Materials} & Crafting Time & Checks & Difficulty & Rarity & Value \\
\midrule
\multicolumn{2}{c}{Narrow Wooden Bridge (per 10 ft)} & \multicolumn{4}{c}{1 units of lumber 
2 parts} & 4 hours & 2 & DC 14 & Common & 25 gp \\
\midrule
\multicolumn{2}{c}{Large Wooden
Bridge (per 10 ft)} & \multicolumn{4}{c}{3 units of lumber 
6 parts} & 8 hours & 4 & DC 14 & Common & 55 gp \\
\midrule
\multicolumn{2}{c}{Huge Wooden
Bridge (per 10 ft)} & \multicolumn{4}{c}{5 units of lumber 
10 parts} & 16 hours (2 days) & 8 & DC 15 & Common & 165 gp \\
\midrule
\multicolumn{2}{c}{Narrow Stone
Bridge (per 10 ft)} & \multicolumn{4}{c}{1 units of stone 
1 bucket of cement} & 8 hours & 4 & DC 15 & Common & 80 gp \\
\midrule
\multicolumn{2}{c}{Large Stone
Bridge (per 10
ft)} & \multicolumn{4}{c}{3 units of stone 
3 buckets of cement} & 16 hours (2 days) & 8 & DC 15 & Common & 160 gp \\
\midrule
\multicolumn{2}{c}{Huge Stone
Bridge (per 10
ft)} & \multicolumn{4}{c}{5 units of stone 
5 buckets of cement} & 32 hours (4 days) & 16 & DC 16 & Common & 520 gp \\
\midrule
\end{tabularx}

\section*{Buildings}

\subsection*{Buildings}

\begin{tabularx}{\textwidth}\toprule
{}XXXXXXXXXX}
\midrule
\multicolumn{2}{c}{Name} & \multicolumn{4}{c}{Materials} & Crafting Time & Checks & Difficulty & Rarity & Value \\
\midrule
\multicolumn{2}{c}{Basic Shelter
(10 ft x 10 ft)} & \multicolumn{4}{c}{1 unit of lumber 
1 parts} & 4 hours & 2 & DC 8 & Common & 3 gp \\
\midrule
\multicolumn{2}{c}{Shack
(15ft x 15 ft)} & \multicolumn{4}{c}{2 units of lumber 
2 parts} & 4 hours
With 5 laborers & 2 & DC 10 & Common & 25 gp \\
\midrule
\multicolumn{2}{c}{Small House
(25 ft x 25 ft)} & \multicolumn{4}{c}{6 units of lumber 
6 parts} & 8 hours
With 5 laborers & 4 & DC 12 & Common & 100 gp \\
\midrule
\multicolumn{2}{c}{Small Temple
(25 ft x 35 ft)} & \multicolumn{4}{c}{8 units of lumber 
2 units of stone 
12 parts 
4 buckets of cement 
4 fancy parts} & 16 hours (2 days)
With 5 laborers & 8 & DC 14 & Common & 550 gp \\
\midrule
\multicolumn{2}{c}{Large House
(35 ft x 35 ft)} & \multicolumn{4}{c}{10 units of lumber 
5 units of stone 
10 parts 
5 buckets of cement 
5 fancy parts} & 24 hours (3 days)
With 10 laborers & 12 & DC 15 & Common & 2,400 gp \\
\midrule
\multicolumn{2}{c}{Mansion
(50 ft x 50 ft)} & \multicolumn{4}{c}{25 units of lumber 
20 units of stone 
20 parts 
20 buckets of cement 
5 fancy parts} & 80 hours (10 days) With
10 laborers & 40 & DC 16 & Common & 13,000 gp \\
\midrule
\multicolumn{2}{c}{Cathedral (50 ft x
100 ft)} & \multicolumn{4}{c}{50 units of lumber 
50 units of stone 
20 parts 
20 buckets of cement 
10 fancy parts 
2 esoteric parts} & 160 hours (20 days)
With 10 laborers & 80 & DC 17 & Common & 40,000 gp \\
\midrule
\end{tabularx}

\section*{Vehicles}

\subsection*{Vehicles}

\begin{tabularx}{\textwidth}\toprule
{}XXXXXXXXXX}
\midrule
\multicolumn{2}{c}{Name} & \multicolumn{4}{c}{Materials} & Crafting Time & Checks & Difficulty & Rarity & Value \\
\midrule
\multicolumn{2}{c}{Carriage (Large)} & \multicolumn{4}{c}{3 units of lumber 
2 quality branch 
1 tanned leather 
5 parts 
2 fancy parts} & 16 hours & 8 & DC 13 & Common & 100 gp \\
\midrule
\multicolumn{2}{c}{Cart (Large)} & \multicolumn{4}{c}{2 units of lumber 
4 metal scraps 
4 leather scraps 
2 parts} & 4 hours & 2 & DC 12 & Common & 15 gp \\
\midrule
\multicolumn{2}{c}{Chariot (Large)} & \multicolumn{4}{c}{2 units of lumber 
2 quality branches 
2 tanned leather 
4 parts 
2 fancy parts} & 20 hours & 10 & DC 14 & Common & 250 gp \\
\midrule
\multicolumn{2}{c}{Sled (Large)} & \multicolumn{4}{c}{2 units of lumber 
2 common branches 
2 parts 
4 leather scraps} & 8 hours & 4 & DC 11 & Common & 20 gp \\
\midrule
\multicolumn{2}{c}{Wagon (Large)} & \multicolumn{4}{c}{3 units of lumber 
2 quality branches 
4 parts 
4 leather scraps} & 10 hours & 5 & DC 12 & Common & 35 gp \\
\midrule
\end{tabularx}

\section*{Ships}

\subsection*{Ships}

\begin{tabularx}{\textwidth}\toprule
{}XXXXXXXXXX}
\midrule
\multicolumn{2}{c}{Name} & \multicolumn{4}{c}{Materials} & Crafting Time & Checks & Difficulty & Rarity & Value \\
\midrule
\multicolumn{2}{c}{Galley
(Gargantuan)} & \multicolumn{4}{c}{100 units of lumber 
20 quality branches 
100 parts 
10 fancy parts Rope (600 ft.)} & 160 hours (20 days)
With 10 laborers & 80 & DC 16 & Common & 26,000 gp \\
\midrule
\multicolumn{2}{c}{Keelboat
(Gargantuan)} & \multicolumn{4}{c}{20 units of lumber 
20 parts 
5 fancy parts 
10 rawhide leather Rope (200 ft.)} & 40 hours (5 days)
With 10 laborers & 15 & DC 14 & Common & 1,900 gp \\
\midrule
\multicolumn{2}{c}{Longship
(Gargantuan)} & \multicolumn{4}{c}{40 units of lumber 
10 quality branches 
100 parts 
5 fancy parts
 Rope (500 ft.)} & 80 hours (10 days)
With 10 laborers & 40 & DC 15 & Common & 8,000 gp \\
\midrule
\multicolumn{2}{c}{Rowboat
(Large)} & \multicolumn{4}{c}{3 units of lumber 
8 parts} & 8 hours & 4 & DC 13 & Common & 50 gp \\
\midrule
\multicolumn{2}{c}{Sailing Ship
(Gargantuan)} & \multicolumn{4}{c}{60 units of lumber 
10 quality branches 
100 parts 
10 fancy parts Rope (2,000 ft.)} & 80 hours (10 days)
With 10 laborers & 40 & DC 15 & Common & 8,000 gp \\
\midrule
\multicolumn{2}{c}{Warship
(Gargantuan)} & \multicolumn{4}{c}{80 units of lumber 
20 quality branch 
100 parts 
10 fancy parts Rope (2,000 ft.)} & 80 hours (10 days)
With 10 laborers & 40 & DC 17 & Common & 21,000 gp \\
\midrule
\end{tabularx}

\section*{Engineering Recipes}

\begin{minipage}{0.48\textwidth}
\subsubsection*{Bridges}

\begin{itemize}
  \item Huge Stone Bridge
  \item Huge Wooden Bridge
  \item Large Stone Bridge
  \item Large Wooden Bridge
  \item Narrow Stone Bridge
  \item Narrow Wooden Bridge
\end{itemize}

\subsubsection*{Buildings}

\begin{itemize}
  \item Basic Shelter
  \item Cathedral
  \item Large House
  \item Mansion
  \item Shack
  \item Small House
  \item Small Temple
\end{itemize}

\subsubsection*{Fortifications}

\begin{itemize}
  \item Low Stone Wall
  \item Stone Battlements
  \item Palisade
  \item Watchtower
  \item Wooden Battlements
\end{itemize}
\end{minipage}\hfill
\begin{minipage}{0.48\textwidth}
\subsubsection*{Ships}

\begin{itemize}
  \item Galley
  \item Keelboat
  \item Longship
  \item Rowboat
  \item Sailing Ship
  \item Warship
\end{itemize}

\subsubsection*{Siege Weapons}

\begin{itemize}
  \item Ballista
  \item Mangonel
  \item Ram
  \item Trebuchet
  \item Siege Tower
\end{itemize}

\subsubsection*{Vehicles}

\begin{itemize}
  \item Carriage
  \item Cart
  \item Chariot
  \item Sled
  \item Wagon
\end{itemize}
\end{minipage}

\section{Minor Branches}

\section*{Minor Branches}

Some branches of crafting tend to overlap with the needs of adventurers more than others. The ones that do not are listed here as minor branches. These are no less important to a world and an economy, but provide more basic necessities, materials, and valuable goods. The primary use of these branches of crafting are to produce valuable goods.

One common interaction with these branches of crafting is with enchanting-magic has expensive taste, and it might be hard to find, for example, boots worth 250 gp-those are some fancy boots! You can use these branches of crafting to make or calculate how such things might be made.

The following provides tool and ability score for each minor branch of crafting, as well as some examples of materials and produced items, but keep in mind that minor branches exist primarily for value added crafting, the provided examples are just reference points (instances) of the value added formula.

\section*{Weaving}

\begin{minipage}{0.48\textwidth}
Weaving allows you to create textiles from raw material. This is rarely within the scope of what adventurers will pursue, but occasionally useful. This largely exists as a background explanation for textiles. Shares a tool proficiency with tailoring.

Weaving Modifier = your Weaver's tools proficiency bonus + your Dexterity modifier
\end{minipage}\hfill
\begin{minipage}{0.48\textwidth}
\begin{tabularx}{\textwidth}\toprule
{}XX}
\midrule
\multicolumn{2}{c}{Materials} & Value \\
\midrule
\multicolumn{2}{c}{spool of thread} & 1 cp \\
\midrule
\multicolumn{2}{c}{raw wool} & 1 cp \\
\midrule
\multicolumn{2}{c}{raw cotton} & 1 cp \\
\midrule
\multicolumn{2}{c}{raw silk} & 1 sp \\
\midrule
\end{tabularx}
\end{minipage}

\subsection*{Weaving}

\begin{tabularx}{\textwidth}\toprule
{}XXXXXXXX}
\midrule
\multicolumn{2}{c}{Produced Items} & \multicolumn{4}{c}{Materials} & Checks & Difficulty & Value \\
\midrule
\multicolumn{2}{c}{Bolt of
Cloth} & \multicolumn{4}{c}{1 raw cotton 
or
 1 raw wool} & 1 & DC 8 & 1 sp \\
\midrule
\multicolumn{2}{c}{Bolt of
Silk} & \multicolumn{4}{c}{1 raw silk} & 1 & DC 10 & 2 gp \\
\midrule
\end{tabularx}

\section*{Cobbling}

\begin{minipage}{0.48\textwidth}
Cobbling is the art of making shoes and boots, typically from leather. A narrow but occasionally useful art of crafting-after all, most adventurers and common folk alike prefer to wear shoes. Often the best route of making boots fit for magical enchantment.

Cobbling Modifier = your Cobbler's tools proficiency bonus + your Dexterity modifier
\end{minipage}\hfill
\begin{minipage}{0.48\textwidth}
\begin{tabularx}{\textwidth}\toprule
{}XX}
\midrule
\multicolumn{2}{c}{Materials} & Value \\
\midrule
\multicolumn{2}{c}{scraps of leather} & 1 sp \\
\midrule
\multicolumn{2}{c}{buckles} & 2 sp \\
\midrule
\multicolumn{2}{c}{tanned leather} & 3 gp \\
\midrule
\end{tabularx}
\end{minipage}

\subsection*{Cobbling}

\begin{tabularx}{\textwidth}\toprule
{}XXXXXXXX}
\midrule
\multicolumn{2}{c}{Produced Items} & \multicolumn{4}{c}{Materials} & Checks & Difficulty & Value \\
\midrule
\multicolumn{2}{c}{Shoes} & \multicolumn{4}{c}{8 scraps of leather 
1 buckle} & 1 & DC 9 & 2 gp \\
\midrule
\multicolumn{2}{c}{Boots} & \multicolumn{4}{c}{1 tanned leather 
1 buckle} & 1 & DC 10 & 5 gp \\
\midrule
\multicolumn{2}{c}{Nice Boots} & \multicolumn{4}{c}{1 tanned leather 
1 buckle} & 3 & DC 16 & 100 gp \\
\midrule
\multicolumn{2}{c}{Fancy Boots} & \multicolumn{4}{c}{1 tanned leather 
1 buckle 
50 gp of rare materials} & 4 & DC 16 & 240 gp \\
\midrule
\end{tabularx}

\section*{Masonry}

\begin{minipage}{0.48\textwidth}
Masonry is the art of working of stone. They can provide skilled labor for large scale stoneworking projects, or create statues and small stonecarvings.

Masonry Modifier = your Mason's Tools proficiency bonus + your Strength modifier
\end{minipage}\hfill
\begin{minipage}{0.48\textwidth}
\begin{tabularx}{\textwidth}\toprule
{}XX}
\midrule
\multicolumn{2}{c}{Materials} & Value \\
\midrule
\multicolumn{2}{c}{unit of stone} & 2 gp \\
\midrule
\end{tabularx}
\end{minipage}

\subsection*{Masonry}

\begin{tabularx}{\textwidth}\toprule
{}XXXXXXXX}
\midrule
\multicolumn{2}{c}{Produced Items} & \multicolumn{4}{c}{Materials} & Checks & Difficulty & Value \\
\midrule
\multicolumn{2}{c}{Simple Statue} & \multicolumn{4}{c}{1 unit of stone} & 8 & DC 12 & 35 gp \\
\midrule
\multicolumn{2}{c}{Masterwork Statue} & \multicolumn{4}{c}{1 unit of stone} & 8 & DC 18 & 670 gp \\
\midrule
\end{tabularx}

\section*{Glassblowing}

\begin{minipage}{0.48\textwidth}
Glassblowing is the art of working glass with blown air to create bottles, vials, and small glass arts. Glassblowing typically requires glassblowers tools and a source of great heat (generally a specific heat source, or a forge, or a magical source).

Glassblowing Modifier = your Glassblower's Tools proficiency bonus + your Dexterity modifier
\end{minipage}\hfill
\begin{minipage}{0.48\textwidth}
\begin{tabularx}{\textwidth}\toprule
{}XX}
\midrule
\multicolumn{2}{c}{Materials} & Value \\
\midrule
\multicolumn{2}{c}{unformed glass} & 1 sp \\
\midrule
\end{tabularx}
\end{minipage}

\subsection*{Glassblowing}

\begin{tabularx}{\textwidth}\toprule
{}XXXXXXXX}
\midrule
\multicolumn{2}{c}{Produced Items} & \multicolumn{4}{c}{Materials} & Checks & Difficulty & Value \\
\midrule
\multicolumn{2}{c}{Glass Vial} & \multicolumn{4}{c}{1 unformed glass} & 1 & DC 9 & 1 gp \\
\midrule
\multicolumn{2}{c}{Fancy Parts
(Lenses)} & \multicolumn{4}{c}{1 unformed glass} & 3 & DC 11 & 10 gp \\
\midrule
\multicolumn{2}{c}{Esoteric Parts
(Lenses)} & \multicolumn{4}{c}{2 unformed glass} & 6 & DC 15 & 100 gp \\
\midrule
\end{tabularx}

\section*{Painting}

\begin{minipage}{0.48\textwidth}
Painting is an artistic pursuit of making art and portraits, typically on canvas. It can be used for a wide range of other purposes, such as making signs, disguises, and illustrations. Some magic items require valuable art as a basis.

Painting Modifier = your Painter's supplies proficiency bonus + your Wisdom modifier
\end{minipage}\hfill
\begin{minipage}{0.48\textwidth}
\begin{tabularx}{\textwidth}\toprule
{}XX}
\midrule
\multicolumn{2}{c}{Materials} & Value \\
\midrule
\multicolumn{2}{c}{paint} & 1 cp \\
\midrule
\multicolumn{2}{c}{canvas} & 1 cp \\
\midrule
\end{tabularx}
\end{minipage}

\subsection*{Painting}

\begin{tabularx}{\textwidth}\toprule
{}XXXXXXXX}
\midrule
\multicolumn{2}{c}{Produced Items} & \multicolumn{4}{c}{Materials} & Checks & Difficulty & Value \\
\midrule
\multicolumn{2}{c}{Painting} & \multicolumn{4}{c}{1 paint 
1 canvas} & 4 & DC 8 & 4 sp \\
\midrule
\multicolumn{2}{c}{Fancy Painting} & \multicolumn{4}{c}{1 paint 
1 canvas} & 8 & DC 15 & 150 gp \\
\midrule
\end{tabularx}

\section*{Brewing}

\begin{minipage}{0.48\textwidth}
Brewing primarily the art of fermenting things into alcoholic substances, most often forms of ales. While of little practical use, it is an art of personal enthusiasm for many of the adventuring sorts and layfolk of the world.

Brewing Modifier = your Brewer's Supplies proficiency bonus + your Wisdom modifier
\end{minipage}\hfill
\begin{minipage}{0.48\textwidth}
\begin{tabularx}{\textwidth}\toprule
{}XX}
\midrule
\multicolumn{2}{c}{Materials} & Value \\
\midrule
\multicolumn{2}{c}{supplies} & 1 gp \\
\midrule
\multicolumn{2}{c}{barrel} & 1 sp \\
\midrule
\multicolumn{2}{c}{flask} & 1 sp \\
\midrule
\multicolumn{2}{c}{oak barrel} & 5 sp \\
\midrule
\multicolumn{2}{c}{uncommon supplies} & 10 gp \\
\midrule
\end{tabularx}
\end{minipage}

\subsection*{Brewing}

\begin{tabularx}{\textwidth}\toprule
{}XXXXXXXX}
\midrule
\multicolumn{2}{c}{Produced Items} & \multicolumn{4}{c}{Materials} & Checks & Difficulty & Value \\
\midrule
\multicolumn{2}{c}{Barrel of Swill} & \multicolumn{4}{c}{1 supplies 
1 water 
1 barrel} & 2 & DC 8 & 12 sp \\
\midrule
\multicolumn{2}{c}{Barrel of Beer} & \multicolumn{4}{c}{1 supplies 
1 water 
1 barrel} & 2 & DC 12 & 10 gp \\
\midrule
\multicolumn{2}{c}{Barrel of Mead} & \multicolumn{4}{c}{1 supplies 
1 uncommon supplies 
1 water 
1 oak barrel} & 3 & DC 14 & 50 gp \\
\midrule
\multicolumn{2}{c}{Dwarven Alcohol} & \multicolumn{4}{c}{1 supplies 
1 common reactive reagents 
1 sturdy metal flask} & 4 & DC 12 & 20 gp \\
\midrule
\multicolumn{2}{c}{10 x Flask of Distilled Achohol} & \multicolumn{4}{c}{1 supplies 
1 water 
1 flask} & 2 & DC 12 & 5 sp \\
\midrule
\end{tabularx}

Brewing Time

While brewing only takes a few hours, common beer takes 3 days to ferment, and fancier ones may take weeks or months. As a very simple rule of thumb:

\begin{itemize}
  \item Beer worth less than 5 gp a barrel takes 3 days.
  \item Beer worth more than 5 gp takes 2 weeks.
  \item Achohol takes 3 days, with the 2 checks made at the start, after which the ahcohol can be collected after the brewing time.
  \item Particularly fancy or exotic beverages may take longer; up to months
\end{itemize}

\section*{Jewelcrafting}

The art of making jewelry of all forms, from cutting gems, setting them, or creating fine metal works.

Jewelcrafting Modifier = your Jeweler's Tools proficiency bonus + your Dexterity modifier

\subsection*{Cut Jewels}

\begin{tabularx}{\textwidth}\toprule
{}XXXXXXXX}
\midrule
\multicolumn{2}{c}{Produced Items} & \multicolumn{4}{c}{Materials} & Checks & Difficulty & Value \\
\midrule
\multicolumn{2}{c}{Basic Cut} & \multicolumn{4}{c}{1 rough gem} & 2(Critical) & DC 14 & x1.25* \\
\midrule
\multicolumn{2}{c}{Exotic Cut} & \multicolumn{4}{c}{1 rough gem} & 2(Critical) & DC 18 & x1.5* \\
\midrule
\end{tabularx}

\begin{itemize}
  \item (Critical) If a check is failed by 5 or more, the gem shatters and the material component is lost.
  \item * The value of cutting gems depends on the basic value of the uncut gem, and is a value multiplier. A rough gem worth 100 gp with a basic cut would become worth 125 gp. An already cut gem cannot be cut again.
\end{itemize}

\subsection*{Jewelry}

\begin{tabularx}{\textwidth}\toprule
{}XXXXXXXX}
\midrule
\multicolumn{2}{c}{Produced Items} & \multicolumn{4}{c}{Materials} & Checks & Difficulty & Value \\
\midrule
\multicolumn{2}{c}{Common Ring} & \multicolumn{4}{c}{3 silver scraps} & 4 & DC 12 & 20 gp \\
\midrule
\multicolumn{2}{c}{Valuable Ring} & \multicolumn{4}{c}{3 gold scraps} & 6 & DC 16 & 200 gp \\
\midrule
\multicolumn{2}{c}{Masterwork Ring} & \multicolumn{4}{c}{3 mithril scraps} & 8 & DC 17 & 420 gp \\
\midrule
\multicolumn{2}{c}{Socketed Ring} & \multicolumn{4}{c}{3 silver scraps} & 4 & DC 14 & 45 gp \\
\midrule
\multicolumn{2}{c}{Common Amulet} & \multicolumn{4}{c}{3 gold scraps} & 4 & DC 15 & 150 gp \\
\midrule
\multicolumn{2}{c}{Valuable Amulet} & \multicolumn{4}{c}{4 gold scraps 
1 gem worth at least 50 gp} & 6 & DC 16 & 250 gp \\
\midrule
\multicolumn{2}{c}{Masterwork Amulet} & \multicolumn{4}{c}{4 mithril scraps 
1 gem worth at least 200 gp} & 8 & DC 17 & 650 gp \\
\midrule
\end{tabularx}

\subsection*{Misc}

\begin{tabularx}{\textwidth}\toprule
{}XXXXXXXX}
\midrule
\multicolumn{2}{c}{Produced Items} & \multicolumn{4}{c}{Materials} & Checks & Difficulty & Value \\
\midrule
\multicolumn{2}{c}{Basic Glasses} & \multicolumn{4}{c}{2 parts 
1 metal scraps} & 2 & DC 12 & 10 gp \\
\midrule
\multicolumn{2}{c}{Quality Glasses} & \multicolumn{4}{c}{2 fancy parts 
1 silver scraps} & 3 & DC 14 & 50 gp \\
\midrule
\end{tabularx}

\section*{Carpentry}

\begin{minipage}{0.48\textwidth}
Carpentry makes furniture and small-scale construction. Often used to make simple mundane objects or modifications.

Carpentry Modifier = your Carpenter's tools proficiency bonus + your Dexterity modifier
\end{minipage}\hfill
\begin{minipage}{0.48\textwidth}
\begin{tabularx}{\textwidth}\toprule
{}XX}
\midrule
\multicolumn{2}{c}{Materials} & Value \\
\midrule
\multicolumn{2}{c}{Common Branch} & 1 sp \\
\midrule
\multicolumn{2}{c}{Parts} & 2 gp \\
\midrule
\end{tabularx}
\end{minipage}

\subsection*{Carpentry}

\begin{tabularx}{\textwidth}\toprule
{}XXXXXXXX}
\midrule
\multicolumn{2}{c}{Produced Items} & \multicolumn{4}{c}{Materials} & Checks & Difficulty & Value \\
\midrule
\multicolumn{2}{c}{4 x Chairs} & \multicolumn{4}{c}{4 common branch 
1 parts} & 2 & DC 8 & 1 gp \\
\midrule
\multicolumn{2}{c}{Table} & \multicolumn{4}{c}{8 common branch 
1 parts} & 2 & DC 8 & 3 gp \\
\midrule
\multicolumn{2}{c}{Door} & \multicolumn{4}{c}{4 common branch 
1 parts} & 2 & DC 10 & 6 gp \\
\midrule
\end{tabularx}

\section*{Tailoring}

\begin{minipage}{0.48\textwidth}
The art of turning textiles into clothes and cloth-based accessories, from utilitarian to ornate gowns. Ornate clothing sometimes serves as the basis of magic gear.

Tailor Modifier = your Weaver's tools proficiency bonus + your Dexterity modifier
\end{minipage}\hfill
\begin{minipage}{0.48\textwidth}
\begin{tabularx}{\textwidth}\toprule
{}XX}
\midrule
\multicolumn{2}{c}{Materials} & Value \\
\midrule
\multicolumn{2}{c}{Cloth Scraps} & 1 cp \\
\midrule
\multicolumn{2}{c}{Spool of Thread} & 1 cp \\
\midrule
\multicolumn{2}{c}{Bolt of Cloth} & 1 sp \\
\midrule
\multicolumn{2}{c}{Bolt of Silk} & 1 gp \\
\midrule
\end{tabularx}
\end{minipage}

\subsection*{Tailoring}

\begin{tabularx}{\textwidth}\toprule
{}XXXXXXXX}
\midrule
\multicolumn{2}{c}{Produced Items} & \multicolumn{4}{c}{Materials} & Checks & Difficulty & Value \\
\midrule
\multicolumn{2}{c}{Hat/Cap} & \multicolumn{4}{c}{5 cloth scraps 
1 spool of thread} & 2 & DC 8 & 2 sp \\
\midrule
\multicolumn{2}{c}{Clothes} & \multicolumn{4}{c}{2 bolts of cloth 
1 spool of thread} & 3 & DC 10 & 6 gp \\
\midrule
\multicolumn{2}{c}{Robe} & \multicolumn{4}{c}{2 bolts of cloth 
4 scraps of cloth 
1 spool of thread} & 3 & DC 10 & 6 gp \\
\midrule
\multicolumn{2}{c}{Cloak} & \multicolumn{4}{c}{2 bolts of cloth 
1 spool of thread} & 2 & DC 8 & 4 sp \\
\midrule
\multicolumn{2}{c}{Cape} & \multicolumn{4}{c}{1 bolt of cloth 
4 scraps of cloth 
1 spool of thread} & 2 & DC 8 & 3 sp \\
\midrule
\multicolumn{2}{c}{Mantle} & \multicolumn{4}{c}{1 bolt of cloth 
1 spool of thread} & 1 & DC 8 & 2 sp \\
\midrule
\multicolumn{2}{c}{Ballgown} & \multicolumn{4}{c}{2 bolts of silk 
2 spools of thread} & 4 & DC 16 & 130 gp \\
\midrule
\multicolumn{2}{c}{Noble Garb} & \multicolumn{4}{c}{2 bolts of silk 
2 spools of thread} & 4 & DC 16 & 130 gp \\
\midrule
\end{tabularx}

Why Weaver’s Tools?

Because that’s what the default system has. While I could invent a new tool, as much as possible I want to keep this compatible with the basic system, including backgrounds and proficiency. If a player wants to tailor things and consults the core books, they’ll pick weaver’s tools, so I want that to work as they’d expect

\section*{Minor Branch Recipes}

\begin{minipage}{0.48\textwidth}
\begin{itemize}
  \item Brewing
  \item Carpentry
  \item Cobbling
  \item Glassblowing
  \item Jewelcrafting
\end{itemize}
\end{minipage}\hfill
\begin{minipage}{0.48\textwidth}
\begin{itemize}
  \item Masonry
  \item Painting
  \item Tailoring
  \item Weaving
\end{itemize}
\end{minipage}

\part*{Appendices}

\chapter*{Appendix A}
\addcontentsline{toc}{chapter}{Appendix A}

\section*{Appendix A: Calculating New Items}

\begin{minipage}{0.48\textwidth}
This book is not a book about magic items, and the subject of how to make an item is beyond the scope of the appendix. This appendix will help you figure out how to craft an item you've already made. The first thing you need to know is the rarity of the item and if the item is consumable or not.
\end{minipage}\hfill
\begin{minipage}{0.48\textwidth}
\begin{tabularx}{\textwidth}\toprule
{}XXX}
\midrule
Rarity & Difficulty & Time & Estimated Labor Cost \\
\midrule
Common & 8-12 & 2-8 hours & 0.1-16 gp \\
\midrule
Uncommon & 12-15 & 4-16 hours & 8-144 gp \\
\midrule
Rare & 15-18 & 8-24 hours & 72-912 gp \\
\midrule
Very Rare & 17-20 & 16-40 hours & 3,760-3,980 gp \\
\midrule
Legendary & 20-25 & 24-40 hours & 2,388-44,140 gp \\
\midrule
\end{tabularx}
\end{minipage}

Regardless if the item is consumable or not, it should generally fall into those ranges. Consumable items should fall to the bottom of the range, while more mechanically powerful items should fall into the highest reaches. That the labor cost of an easy to make Legendary Item can be cheaper than Very Rare items is intentional; the range of item cost and difficulty is very wide within rarities, even if the final cost of the item will result in them being ordered by rarity.

The next step is to add appropriate reagents of that the appropriate crafting branch of the rarity until you reach the following costs. The easiest way to do this will be to find a similar item that is already built out, and then replace any item that doesn't make sense of your item, or to increase or decrease the cost of the item as appropriate.

\begin{minipage}{0.48\textwidth}
\subsubsection*{Default Pricing}

\begin{tabularx}{\textwidth}\toprule
{}XX}
\midrule
Rarity & Consumable Price & Price \\
\midrule
Common & 25-50 gp & 50-100 gp \\
\midrule
Uncommon & 50-250 gp & 101-500 gp \\
\midrule
Rare & 250-2,500 gp & 501-5,000 gp \\
\midrule
Very Rare & 2,500-25,000 gp & 5,001-50,000 gp \\
\midrule
Legendary & 25,000+ gp & 50,000+ gp \\
\midrule
\end{tabularx}
\end{minipage}\hfill
\begin{minipage}{0.48\textwidth}
\subsubsection*{Enchating \& Magical Items}

The most complicated (and most common) use case is Enchanting. Not only does enchanting use the basic components of its system, it adds an additional element: Scrolls. Scrolls serve as the "magic blueprint" to an item, and replace the rule of recipes or blueprints in the system. If your magic item cast spells... this step is easy. The item takes the scrolls of those spells.

If it doesn't, find appropriate allegories to what the item does. The obvious example, is a +1 weapon. It doesn't cast a spell, but it does have the same effect as a magic weapon spell, so it can use that scroll. A bag of holding is more complicated, but you can follow the same logic-the secret chest spell doesn't exactly overlap, but it shares some properties... it's close enough for these purposes.
\end{minipage}

\section*{Labor Cost Table}

The following is the labor cost table that drives the math of the system. You do not need actually use this to calculate your items, a rough estimation will suffice. This is provided for information purposes only.

\begin{itemize}
  \item Labor: The cost per check.
  \item Risk Multiplier: A multiplier applied to the material cost of the item.
  \item Skill Level: This means nothing. It is just a narrative key.
\end{itemize}

\begin{longtable}{p{2.5cm}\toprule
|p{2.5cm}|p{2.5cm}|p{2.5cm}|p{2.5cm}|p{2.5cm}|p{2.5cm}|}
\midrule
Difficulty & \multicolumn{4}{c}{Skill Level} & Labor & Risk Multiplier \\
\midrule
1 & \multicolumn{4}{c}{Novice} & 1 sp & 1 \\
\midrule
2 & \multicolumn{4}{c}{Novice} & 1 sp & 1 \\
\midrule
3 & \multicolumn{4}{c}{Novice} & 1 sp & 1 \\
\midrule
4 & \multicolumn{4}{c}{Novice} & 1 sp & 1 \\
\midrule
5 & \multicolumn{4}{c}{Novice} & 1 sp & 1 \\
\midrule
6 & \multicolumn{4}{c}{Novice} & 1 sp & 1 \\
\midrule
7 & \multicolumn{4}{c}{Novice} & 1 sp & 1 \\
\midrule
8 & \multicolumn{4}{c}{Novice} & 1 sp & 1 \\
\midrule
9 & \multicolumn{4}{c}{Apprentice} & 1 gp & 1 \\
\midrule
9 & \multicolumn{4}{c}{Apprentice} & 2 gp & 1 \\
\midrule
11 & \multicolumn{4}{c}{Journeyman} & 3 gp & 1.05 \\
\midrule
12 & \multicolumn{4}{c}{Journeyman} & 4 gp & 1.05 \\
\midrule
13 & \multicolumn{4}{c}{Journeyman} & 7 gp & 1.05 \\
\midrule
14 & \multicolumn{4}{c}{Journeyman} & 11 gp & 1.05 \\
\midrule
15 & \multicolumn{4}{c}{Master} & 18 gp & 1.05 \\
\midrule
16 & \multicolumn{4}{c}{Master} & 29 gp & 1.1 \\
\midrule
17 & \multicolumn{4}{c}{Master} & 47 gp & 1.1 \\
\midrule
18 & \multicolumn{4}{c}{Master} & 76 gp & 1.1 \\
\midrule
19 & \multicolumn{4}{c}{Master} & 123 gp & 1.1 \\
\midrule
20 & \multicolumn{4}{c}{Grandmaster} & 199 gp & 1.1 \\
\midrule
21 & \multicolumn{4}{c}{Grandmaster} & 322 gp & 1.2 \\
\midrule
22 & \multicolumn{4}{c}{Grandmaster} & 521 gp & 1.2 \\
\midrule
23 & \multicolumn{4}{c}{Grandmaster} & 843 gp & 1.2 \\
\midrule
24 & \multicolumn{4}{c}{Grandmaster} & 1,364 gp & 1.2 \\
\midrule
25 & \multicolumn{4}{c}{Grandmaster} & 2,207 gp & 1.2 \\
\midrule
26 & \multicolumn{4}{c}{Grandmaster} & 3,571 gp & 1.3 \\
\midrule
27 & \multicolumn{4}{c}{Grandmaster} & 7,778 gp & 1.3 \\
\midrule
28 & \multicolumn{4}{c}{Grandmaster} & 11,349 gp & 1.3 \\
\midrule
29 & \multicolumn{4}{c}{Grandmaster} & 19,127 gp & 1.3 \\
\midrule
30 & \multicolumn{4}{c}{Mythical} & 30,476 gp & 1.3 \\
\midrule
\end{longtable}

\chapter*{Appendix B}
\addcontentsline{toc}{chapter}{Appendix B}

\section*{Appendix B: Specific Gathering Results}

\section*{Arctic Locale}

\subsection*{Arctic}

\begin{tabularx}{\textwidth}\toprule
{}XXXXXXXX}
\midrule
\multicolumn{2}{c}{Name} & Rarity & \multicolumn{4}{c}{Description} & Properties & Purchase Price \\
\midrule
\multicolumn{2}{c}{Cold Snaps} & Common & \multicolumn{4}{c}{Frozen berries that burst violently when broken} & Reactive (Icy) & 20 gp \\
\midrule
\multicolumn{2}{c}{Ice Bamboo} & Common & \multicolumn{4}{c}{Bamboo formed entirely out of ice, does not
melt} & Curative & 15 gp \\
\midrule
\multicolumn{2}{c}{Snake Tracks} & Common & \multicolumn{4}{c}{A dead looking weed found beneath layers of
snow and ice} & Poisonous & 15 gp \\
\midrule
\multicolumn{2}{c}{Blizzard Bones} & Common & \multicolumn{4}{c}{Bones that have been through three blizzards} & Curative & 15 gp \\
\midrule
\multicolumn{2}{c}{Yeti Droppings} & Common & \multicolumn{4}{c}{Particularly potent yeti droppings} & Poisonous & 15 gp \\
\midrule
\multicolumn{2}{c}{Snowmelt Flower} & Uncommon & \multicolumn{4}{c}{A strange flower that grows up through the
snow melting a patch of it} & Reactive & 40 gp \\
\midrule
\multicolumn{2}{c}{Unicorn Fur} & Rare & \multicolumn{4}{c}{Glistening pure white stalks that grow in
sheltered spots} & Curative & 200 gp \\
\midrule
\multicolumn{2}{c}{White Lotus} & Rare & \multicolumn{4}{c}{A glistening white flower} & Curative, Poisonous & 300 gp \\
\midrule
\end{tabularx}

\section*{Desert Locale}

\subsection*{Deserts}

\begin{tabularx}{\textwidth}\toprule
{}XXXXXXXX}
\midrule
\multicolumn{2}{c}{Name} & Rarity & \multicolumn{4}{c}{Description} & Properties & Purchase Price \\
\midrule
\multicolumn{2}{c}{Dry Cough} & Common & \multicolumn{4}{c}{A gnarled plant that looks inedible} & Poisonous & 15 gp \\
\midrule
\multicolumn{2}{c}{Lighting Sand} & Common & \multicolumn{4}{c}{Burned sand that has been struck by lightning} & Reactive & 15 gp \\
\midrule
\multicolumn{2}{c}{Popping Tar} & Common & \multicolumn{4}{c}{A black tar like substance that burns violent
with crackling pops} & Reactive & 15 gp \\
\midrule
\multicolumn{2}{c}{Waterdrop Cactus} & Common & \multicolumn{4}{c}{A tiny cactus containing a single drop of water} & Curative & 15 gp \\
\midrule
\multicolumn{2}{c}{Morninglord} & Uncommon & \multicolumn{4}{c}{A cactus that produces little white flowers in
the dawn's light} & Curative & 40 gp \\
\midrule
\multicolumn{2}{c}{Oasis Bane} & Uncommon & \multicolumn{4}{c}{A small root sucks up water and looks edible} & Poisonous & 40 gp \\
\midrule
\multicolumn{2}{c}{Blacksand} & Uncommon & \multicolumn{4}{c}{A strange black sand, particularly fine grains with an odd smell} & Reactive & 40 gp \\
\midrule
\multicolumn{2}{c}{Elemental Glass} & Rare & \multicolumn{4}{c}{Fused glass flickering with primal power} & Reactive & 200 gp \\
\midrule
\end{tabularx}

\section*{Forest Locale}

\subsection*{Forests}

\begin{tabularx}{\textwidth}\toprule
{}XXXXXXXX}
\midrule
\multicolumn{2}{c}{Name} & Rarity & \multicolumn{4}{c}{Description} & Properties & Purchase Price \\
\midrule
\multicolumn{2}{c}{Elfmarks} & Common & \multicolumn{4}{c}{Small twisting vines with pale flowers} & Curative & 15 gp \\
\midrule
\multicolumn{2}{c}{Fairy Steps} & Common & \multicolumn{4}{c}{Tiny white flowers in the shape of fairy wings} & Curative & 15 gp \\
\midrule
\multicolumn{2}{c}{King's Salvation} & Common & \multicolumn{4}{c}{A golden brown root} & Curative & 15 gp \\
\midrule
\multicolumn{2}{c}{King's Damnation} & Common & \multicolumn{4}{c}{A reddish brown root} & Poisonous & 15 gp \\
\midrule
\multicolumn{2}{c}{Catfern} & Common & \multicolumn{4}{c}{Green cattail fern} & Exotic & 15 gp \\
\midrule
\multicolumn{2}{c}{Silverscale} & Uncommon & \multicolumn{4}{c}{Silvery tree bark with a scale like texture} & Curative & 40 gp \\
\midrule
\multicolumn{2}{c}{Sweetpetal} & Uncommon & \multicolumn{4}{c}{Rose-like flower petals found on the forest floor} & Poisonous & 40 gp \\
\midrule
\multicolumn{2}{c}{Dyradtears} & Rare & \multicolumn{4}{c}{Small blue flowers that grow near dead trees} & Curative & 200 gp \\
\midrule
\multicolumn{2}{c}{Divine Laurel} & Very Rare & \multicolumn{4}{c}{Golden leaves that glimmer as if gilded} & Curative & 2,000 gp \\
\midrule
\end{tabularx}

\section*{Mountain/Cave Locale}

\subsection*{Mountains/Caves}

\begin{tabularx}{\textwidth}\toprule
{}XXXXXXXX}
\midrule
\multicolumn{2}{c}{Name} & Rarity & \multicolumn{4}{c}{Description} & Properties & Purchase Price \\
\midrule
\multicolumn{2}{c}{Goldbane} & Common & \multicolumn{4}{c}{Clumpy yellow powder} & Reactive & 15 gp \\
\midrule
\multicolumn{2}{c}{Rare Earth Powders} & Common & \multicolumn{4}{c}{Dirt with traces of rare vitamins} & Curative & 15 gp \\
\midrule
\multicolumn{2}{c}{Crevice Spider Eggs} & Common & \multicolumn{4}{c}{Dried egg sacs from a small reclusive spider} & Poisonous & 15 gp \\
\midrule
\multicolumn{2}{c}{Dragongrass} & Uncommon & \multicolumn{4}{c}{Red leafy grass} & Reactive, Exotic & 40 gp \\
\midrule
\multicolumn{2}{c}{Minebane} & Uncommon & \multicolumn{4}{c}{Long black roots that give off smoke} & Reactive & 40 gp \\
\midrule
\multicolumn{2}{c}{Crystal Spider Webbing} & Uncommon & \multicolumn{4}{c}{Crystalline Webs} & Poisonous & 40 gp \\
\midrule
\end{tabularx}

\section*{Plains Locale}

\subsection*{Plains}

\begin{tabularx}{\textwidth}\toprule
{}XXXXXXXX}
\midrule
\multicolumn{2}{c}{Name} & Rarity & \multicolumn{4}{c}{Description} & Properties & Purchase Price \\
\midrule
\multicolumn{2}{c}{Lightning Roots} & Common & \multicolumn{4}{c}{Still living roots from a tree hit by lightning that have captured a fragment of primal power} & Reactive & 15 gp \\
\midrule
\multicolumn{2}{c}{Hoof Thistle} & Common & \multicolumn{4}{c}{Small snaring weeds with an unpleasant thistle} & Curative & 15 gp \\
\midrule
\multicolumn{2}{c}{Humming Berries} & Common & \multicolumn{4}{c}{Small red berries the hum slightly when held} & Reactive & 15 gp \\
\midrule
\multicolumn{2}{c}{Lich Fingers} & Common & \multicolumn{4}{c}{A small spindly white tuber root} & Poisonous & 15 gp \\
\midrule
\multicolumn{2}{c}{Dried Tar} & Common & \multicolumn{4}{c}{Black flakes of sludge like substance} & Reactive & 15 gp \\
\midrule
\multicolumn{2}{c}{Goldshine Grass} & Uncommon & \multicolumn{4}{c}{Strange grass that looks like it is gilded} & Poisonous & 40 gp \\
\midrule
\multicolumn{2}{c}{Centaur Droppings} & Uncommon & \multicolumn{4}{c}{Strange foul smelling mud} & Curative & 40 gp \\
\midrule
\multicolumn{2}{c}{Burned Belladonna} & Uncommon & \multicolumn{4}{c}{A strange plant that looks like it is burned} & Reactive, Poisonous & 60 gp \\
\midrule
\end{tabularx}

\section*{Swamp Locale}

\subsection*{Swamps}

\begin{tabularx}{\textwidth}\toprule
{}XXXXXXXX}
\midrule
\multicolumn{2}{c}{Name} & Rarity & \multicolumn{4}{c}{Description} & Properties & Purchase Price \\
\midrule
\multicolumn{2}{c}{Drooping Death} & Common & \multicolumn{4}{c}{Drooping, dead looking ferns} & Curative & 15 gp \\
\midrule
\multicolumn{2}{c}{Wartflower} & Common & \multicolumn{4}{c}{A sickly yellow flower with strange growths} & Curative & 15 gp \\
\midrule
\multicolumn{2}{c}{Swamp Oil} & Common & \multicolumn{4}{c}{A rainbow sheened oily substance} & Reactive & 15 gp \\
\midrule
\multicolumn{2}{c}{Whisp Fruit} & Common & \multicolumn{4}{c}{A slightly glowing pale round fruit} & Poisonous & 15 gp \\
\midrule
\multicolumn{2}{c}{Hag Trail} & Uncommon & \multicolumn{4}{c}{Wilted plants that grow on corpses} & Poisonous & 40 gp \\
\midrule
\multicolumn{2}{c}{Sad Salvation} & Uncommon & \multicolumn{4}{c}{A brilliantly blue flower that grows in decay} & Curative & 40 gp \\
\midrule
\multicolumn{2}{c}{Burning Sludge} & Uncommon & \multicolumn{4}{c}{Watery sludge that seems is burning hot and slightly smokes} & Reactive & 40 gp \\
\midrule
\multicolumn{2}{c}{Hag Fruit} & Rare & \multicolumn{4}{c}{A pleasantly apple looking fruit that grows on gnarled trees} & Poisonous & 200 gp \\
\midrule
\multicolumn{2}{c}{Shambling Seedling} & Rare & \multicolumn{4}{c}{A seedling that sprouted on a shambling mound} & Curative & 200 gp \\
\midrule
\multicolumn{2}{c}{Fetid Gas} & Rare & \multicolumn{4}{c}{Rare swap gas that smells of sulfur} & Reactive, Poisonous & 300 gp \\
\midrule
\end{tabularx}

\section*{Coastal Locale}

\subsection*{Coastal}

\begin{tabularx}{\textwidth}\toprule
{}XXXXXXXX}
\midrule
\multicolumn{2}{c}{Name} & Rarity & \multicolumn{4}{c}{Description} & Properties & Purchase Price \\
\midrule
\multicolumn{2}{c}{Merweed} & Common & \multicolumn{4}{c}{Always damp blue leaves} & Curative & 15 gp \\
\midrule
\multicolumn{2}{c}{Rotweed} & Common & \multicolumn{4}{c}{Seaweed like weeds that give off an unpleasant smell} & Poisonous & 15 gp \\
\midrule
\multicolumn{2}{c}{Infused Ambergris} & Common & \multicolumn{4}{c}{Whale snot from a whale that's been dining on too many magical meals} & Reactive & - \\
\midrule
\multicolumn{2}{c}{Oyster Flowers} & Uncommon & \multicolumn{4}{c}{Oyster shaped white and blue flowers with an odd smell} & Curative & 15 gp \\
\midrule
\multicolumn{2}{c}{Crystal Crustations Shells} & Uncommon & \multicolumn{4}{c}{Broken pieces shell from rare crystaline crustrations} & Reactive & 40 gp \\
\midrule
\multicolumn{2}{c}{Lunar Tracks} & Uncommon & \multicolumn{4}{c}{Small flowers that bloom only in a freshly receded tide} & Curative & 40 gp \\
\midrule
\end{tabularx}

\section*{Exotic Locale}

\subsection*{Exotic}

\begin{tabularx}{\textwidth}\toprule
{}XXXXXXXX}
\midrule
\multicolumn{2}{c}{Name} & Rarity & \multicolumn{3}{c}{Description} & Properties & Gathering Locale & Purchase Price \\
\midrule
\multicolumn{2}{c}{Elemental Earth} & Rare & \multicolumn{3}{c}{Loose soil Poisonous Plane of Earth} & Poisonous & Plane of Earth & 200 gp \\
\midrule
\multicolumn{2}{c}{Elemental Fire} & Rare & \multicolumn{3}{c}{Ever burning fire} & Reactive & Plane of Fire & 200 gp \\
\midrule
\multicolumn{2}{c}{Elemental Water} & Rare & \multicolumn{3}{c}{Water} & Curative & Plane of Water & 200 gp \\
\midrule
\multicolumn{2}{c}{Apple of Arborea} & Legendary & \multicolumn{3}{c}{A golden apple} & Curative, Exotic & Arborea & 5,000 gp \\
\midrule
\multicolumn{2}{c}{Spider Queen's Steps} & Legendary & \multicolumn{3}{c}{Dark purple flowers} & Poisonous & Underground & 5,000 gp \\
\midrule
\end{tabularx}

\chapter*{Appendix B}
\addcontentsline{toc}{chapter}{Appendix B}

\section*{Appendix B: Specific Harvesting Results}

\section*{Monstrosity}

\subsection*{Monstrosities}

\begin{tabularx}{\textwidth}\toprule
{}XXXXXXXXX}
\midrule
\multicolumn{2}{c}{Monster} & Rarity & \multicolumn{2}{c}{Organ} & \multicolumn{3}{c}{Description} & Properties & Purchase Price \\
\midrule
\multicolumn{2}{c}{Ankheg} & Common & \multicolumn{2}{c}{Ankheg Acid Gland} & \multicolumn{3}{c}{Squishy brown organ prone to leaking green stuff} & Poisonou, Reactive & 15 gp \\
\midrule
\multicolumn{2}{c}{Basilisk} & Uncommon & \multicolumn{2}{c}{Basilisk Eye} & \multicolumn{3}{c}{Beedy and hard, almost rocklike in texture} & Exotic & 40 gp \\
\midrule
\multicolumn{2}{c}{Behir} & Rare & \multicolumn{2}{c}{Pristine Behir Scale} & \multicolumn{3}{c}{Humming with static charge, grinding it can be a hazardous process} & Reactive & 200 gp \\
\midrule
\multicolumn{2}{c}{Bulette} & Uncommon & \multicolumn{2}{c}{Bulette Liver Fat} & \multicolumn{3}{c}{A giggling gelatin like substance with a grey hue} & Curative & 40 gp \\
\midrule
\multicolumn{2}{c}{Chimera} & Uncommon & \multicolumn{2}{c}{Ram Horn Marrow} & \multicolumn{3}{c}{Scrapped from the inside of the ram's horn} & Curative & 40 gp \\
\midrule
\multicolumn{2}{c}{Cockatrice} & Common & \multicolumn{2}{c}{Cockatrice Tongue} & \multicolumn{3}{c}{A hideous worm-like thing that is very tough} & Exotic & 15 gp \\
\midrule
\multicolumn{2}{c}{Darkmantle} & Common & \multicolumn{2}{c}{Darkmantle Pigment Sac} & \multicolumn{3}{c}{A small gland that changes color to whatever surface it is on} & Exotic & 15 gp \\
\midrule
\multicolumn{2}{c}{Death Dog} & Common & \multicolumn{2}{c}{Unbroken fang} & \multicolumn{3}{c}{Jagged foul smelling fangs} & Poisonous & 15 gp \\
\midrule
\multicolumn{2}{c}{Ettercap} & Common & \multicolumn{2}{c}{Webbing Mass} & \multicolumn{3}{c}{A sticky white substance that must be carefully handled} & Special & 10 gp \\
\midrule
\multicolumn{2}{c}{Gorgon} & Uncommon & \multicolumn{2}{c}{Metalized Gorgon Heart} & \multicolumn{3}{c}{A heart that has started to turn metallic with iron shot through it} & Curative & 40 gp \\
\midrule
\multicolumn{2}{c}{Grick} & Common & \multicolumn{2}{c}{Grick Beak} & \multicolumn{3}{c}{Incredibly hard surface; shiny when polished} & Poisonous & 15 gp \\
\midrule
\multicolumn{2}{c}{Harpy} & Common & \multicolumn{2}{c}{Harpy Claws} & \multicolumn{3}{c}{Sort of like very large chicken feet} & Poisonous & 15 gp \\
\midrule
\multicolumn{2}{c}{Hydra} & Uncommon & \multicolumn{2}{c}{Hydra Blood} & \multicolumn{3}{c}{Syrupy black liquid with a swamp gas smell} & Curative, Poisonous & 40 gp \\
\midrule
\multicolumn{2}{c}{Kraken} & Legendary & \multicolumn{2}{c}{Astral Grey Matter} & \multicolumn{3}{c}{A slimy material with strange properties} & Reactive & 5,000 gp \\
\midrule
\multicolumn{2}{c}{Manticore} & Common & \multicolumn{2}{c}{Pristine Tail Spike} & \multicolumn{3}{c}{A long vicious looking thing that must be carefully ground} & Poisonous & 10 gp \\
\midrule
\multicolumn{2}{c}{Medusa} & Uncommon & \multicolumn{2}{c}{Hair snake fangs} & \multicolumn{3}{c}{Tiny fangs from the snakes of a medusa's hair} & Poisonous & 40 gp \\
\midrule
\multicolumn{2}{c}{Mimic} & Common & \multicolumn{2}{c}{Mimic's "Heart"} & \multicolumn{3}{c}{An odd organ that keeps changing shape} & Exotic & 15 gp \\
\midrule
\multicolumn{2}{c}{Purple Worm} & Very Rare & \multicolumn{2}{c}{Fang Venom} & \multicolumn{3}{c}{Poison extracted from a Purple Worm's maw} & Poisonous & 2,000 gp \\
\midrule
\end{tabularx}

\section*{Elemental}

\subsection*{Elementals}

\begin{longtable}{p{2.5cm}\toprule
|p{2.5cm}|p{2.5cm}|p{2.5cm}|p{2.5cm}|p{2.5cm}|p{2.5cm}|p{2.5cm}|p{2.5cm}|p{2.5cm}|}
\midrule
\multicolumn{2}{c}{Monster} & Rarity & \multicolumn{2}{c}{Organ} & \multicolumn{3}{c}{Description} & Properties & Purchase Price \\
\midrule
\multicolumn{2}{c}{Mud Mephit} & Common & \multicolumn{2}{c}{Foul Dust} & \multicolumn{3}{c}{The crusty dried remains of its head} & Poisonous & 15 gp \\
\midrule
\multicolumn{2}{c}{Smoke Mephit} & Common & \multicolumn{2}{c}{Swirling Soot} & \multicolumn{3}{c}{Little flecks of ash that never quite settle} & Reactive & 15 gp \\
\midrule
\multicolumn{2}{c}{Steam Mephit} & Common & \multicolumn{2}{c}{Steaming Droplets} & \multicolumn{3}{c}{Droplets of water that are never quite cool} & Reactive & 15 gp \\
\midrule
\multicolumn{2}{c}{Ice Mephit} & Common & \multicolumn{2}{c}{Frozen Droplets} & \multicolumn{3}{c}{Droplets of water that never quite thaw} & Curative & 15 gp \\
\midrule
\multicolumn{2}{c}{Magma Mephit} & Common & \multicolumn{2}{c}{Burning Rocks} & \multicolumn{3}{c}{Small rock chips that are painfully hot to the touch} & Reactive & 15 gp \\
\midrule
\multicolumn{2}{c}{Magmin} & Common & \multicolumn{2}{c}{Magmin Charcoal} & \multicolumn{3}{c}{Small pieces of ever warm charcoal} & Reactive & 15 gp \\
\midrule
\multicolumn{2}{c}{Fire Snake} & Common & \multicolumn{2}{c}{Fire Snake Scales} & \multicolumn{3}{c}{Small red scales that are warm to the touch} & Reactive & 15 gp \\
\midrule
\multicolumn{2}{c}{Azer} & Common & \multicolumn{2}{c}{Flaming Beard Hairs} & \multicolumn{3}{c}{Small beard hairs made of fire; smells faintly like burning hair} & Reactive & 15 gp \\
\midrule
\multicolumn{2}{c}{Gargoyle} & Common & \multicolumn{2}{c}{Gargoyle's Stone Heart} & \multicolumn{3}{c}{A gem like heart} & Exotic & 15 gp \\
\midrule
\multicolumn{2}{c}{Water Weird} & Common & \multicolumn{2}{c}{Essence of Tainted Water} & \multicolumn{3}{c}{Brackish water with high surface tension} & Poisonous & 15 gp \\
\midrule
\multicolumn{2}{c}{Air Elemental} & Uncommon & \multicolumn{2}{c}{Uncommon Primal Essence} & \multicolumn{3}{c}{An eddy of every swirling wind magic} & Primal Essence & 40 gp \\
\midrule
\multicolumn{2}{c}{Earth Elemental} & Uncommon & \multicolumn{2}{c}{Uncommon Primal Essence} & \multicolumn{3}{c}{An unremarkable looking stone} & Primal Essence & 40 gp \\
\midrule
\multicolumn{2}{c}{Fire Elemental} & Uncommon & \multicolumn{2}{c}{Uncommon Primal Essence} & \multicolumn{3}{c}{Small sourceless fires} & Primal Essence & 40 gp \\
\midrule
\multicolumn{2}{c}{Water Elemental} & Uncommon & \multicolumn{2}{c}{Uncommon Primal Essence} & \multicolumn{3}{c}{A cup of water that never dries} & Primal Essence & 40 gp \\
\midrule
\multicolumn{2}{c}{Salamander} & Uncommon & \multicolumn{2}{c}{Salamander's Tongue} & \multicolumn{3}{c}{A long leathery tongue} & Exotic & 40 gp \\
\midrule
\multicolumn{2}{c}{Xorn} & Uncommon & \multicolumn{2}{c}{Xorn's Stomach Acid} & \multicolumn{3}{c}{A burbling very acidic substance} & Reactive, Poisonous & 40 gp \\
\midrule
\multicolumn{2}{c}{Galeb Duhr} & Uncommon & \multicolumn{2}{c}{Duhrian Heart} & \multicolumn{3}{c}{A strange rock with veins of crystal throughout} & Curative & 40 gp \\
\midrule
\multicolumn{2}{c}{Invisible Stalker} & Uncommon & \multicolumn{2}{c}{Uncommon Primal Essence} & \multicolumn{3}{c}{An eddy of wind that always swirls toward you} & Primal Essence & 40 gp \\
\midrule
\multicolumn{2}{c}{Dao} & Rare & \multicolumn{2}{c}{Heart} & \multicolumn{3}{c}{A heart shaped stone with dark veins} & Primal Essence & 200 gp \\
\midrule
\multicolumn{2}{c}{Djinni} & Rare & \multicolumn{2}{c}{Heart} & \multicolumn{3}{c}{A floating blue heart that slowly spins} & Primal Essence & 200 gp \\
\midrule
\multicolumn{2}{c}{Efreeti} & Rare & \multicolumn{2}{c}{Heart} & \multicolumn{3}{c}{An ever burning coal the size of a fist} & Primal Essence & 200 gp \\
\midrule
\multicolumn{2}{c}{Marid} & Rare & \multicolumn{2}{c}{Heart} & \multicolumn{3}{c}{A flabby heart that never stops oozing} & Primal Essence & 200 gp \\
\midrule
\end{longtable}

\section*{Dragon}

\subsection*{Dragons}

\begin{tabularx}{\textwidth}\toprule
{}XXXXXXXXX}
\midrule
\multicolumn{2}{c}{Monster} & Rarity & \multicolumn{2}{c}{Organ} & \multicolumn{3}{c}{Description} & Properties & Purchase Price \\
\midrule
\multicolumn{2}{c}{Pseudodragon} & Common & \multicolumn{2}{c}{Pseudodragon Stinger} & \multicolumn{3}{c}{A small sharp barb} & Poisonous & 15 gp \\
\midrule
\multicolumn{2}{c}{Wyrmling} & Common & \multicolumn{2}{c}{Wyrmling Heart} & \multicolumn{3}{c}{A heart in the color of the wyrmling} & Reactive, Essence & 30 gp \\
\midrule
\multicolumn{2}{c}{Faerie Dragon} & Common & \multicolumn{2}{c}{Faerie Dragon Heart} & \multicolumn{3}{c}{A small sparkling liver} & Poisonous & 15 gp \\
\midrule
\multicolumn{2}{c}{Wyvern} & Uncommon & \multicolumn{2}{c}{Wyvern Stinger} & \multicolumn{3}{c}{Pieces of the wyvern’s tail stinger} & Poisonous & 40 gp \\
\midrule
\multicolumn{2}{c}{Young Dragon} & Rare & \multicolumn{2}{c}{Dragon Heart} & \multicolumn{3}{c}{A large heart flaring with elemental power} & Reactive, Primal Essence & 300 gp \\
\midrule
\multicolumn{2}{c}{Shadow Dragon} & Rare & \multicolumn{2}{c}{Shadow Dragon Heart} & \multicolumn{3}{c}{An ethereal heart that casts a pitch black shadow} & Arcane Essence & 200 gp \\
\midrule
\multicolumn{2}{c}{Adult Dragon} & Very Rare & \multicolumn{2}{c}{Dragon Heart} & \multicolumn{3}{c}{A huge heart flaring with elemental power} & Reactive, Primal Essence & 3,000 gp \\
\midrule
\multicolumn{2}{c}{Dragon Turtle} & Very Rare & \multicolumn{2}{c}{Dragon Turtle Lungs} & \multicolumn{3}{c}{The rubbery flesh from the lung tubes} & Curative & 2,000 gp \\
\midrule
\multicolumn{2}{c}{Ancient Dragon} & Legendary & \multicolumn{2}{c}{Dragon Heart} & \multicolumn{3}{c}{A massive heart flaring with elemental power} & Reactive, Primal Essence & 7,500 gp \\
\midrule
\end{tabularx}

\chapter*{Appendix B}
\addcontentsline{toc}{chapter}{Appendix B}

\section*{Appendix B: Specific Meal Table}

\subsection*{Specific Meals}

\begin{tabularx}{\textwidth}\toprule
{}XXXXXX}
\midrule
\multicolumn{2}{c}{Creature} & \multicolumn{4}{c}{Example} & Meal Type \\
\midrule
\multicolumn{2}{c}{Ape} & \multicolumn{4}{c}{Should We Eat This? Sliced Roast} & Meat Feast \\
\midrule
\multicolumn{2}{c}{Axe Beak} & \multicolumn{4}{c}{Chicken Wing Steaks} & Meat Feast \\
\midrule
\multicolumn{2}{c}{Black Bear} & \multicolumn{4}{c}{Bean and Bear Soup} & Common Feast \\
\midrule
\multicolumn{2}{c}{Beholder} & \multicolumn{4}{c}{Eyestock Unagi Shushi} & Risky Aberration Snack \\
\midrule
\multicolumn{2}{c}{Crocodile} & \multicolumn{4}{c}{Spicy Croc Gumbo} & Common Feast \\
\midrule
\multicolumn{2}{c}{Elk} & \multicolumn{4}{c}{Elk Steaks} & Meat Feast \\
\midrule
\multicolumn{2}{c}{Giant Elk} & \multicolumn{4}{c}{Really Big Elk Steaks} & Meat Feast \\
\midrule
\multicolumn{2}{c}{Giant Fire Beetle} & \multicolumn{4}{c}{Self Cooking Beetle Bowls} & Common Feast \\
\midrule
\multicolumn{2}{c}{Giant Toad} & \multicolumn{4}{c}{Hopping Poppers} & Meat Feast \\
\midrule
\multicolumn{2}{c}{Kraken} & \multicolumn{4}{c}{Sea Aged Unagi} & Legendary Meat Feast \\
\midrule
\multicolumn{2}{c}{Kraken} & \multicolumn{4}{c}{Titan Tri tip} & Legendary Meat Feast \\
\midrule
\multicolumn{2}{c}{Roc} & \multicolumn{4}{c}{Roc Drumstick Gyros} & Superb Meat Feast \\
\midrule
\multicolumn{2}{c}{Stirge} & \multicolumn{4}{c}{Blood Sausage} & Common Feast \\
\midrule
\multicolumn{2}{c}{Tyrannosaurus Rex} & \multicolumn{4}{c}{Primeval Pot Pie} & Meat Feast \\
\midrule
\multicolumn{2}{c}{Tyrannosaurus Rex} & \multicolumn{4}{c}{Giant’s Chicken Breast} & Meat Feast \\
\midrule
\multicolumn{2}{c}{Young Hook Horror} & \multicolumn{4}{c}{Hook Turkey Sandwiches} & Common Feast \\
\midrule
\multicolumn{2}{c}{Young Hook Horror} & \multicolumn{4}{c}{Murder Chicken Tenders} & Common Feast \\
\midrule
\multicolumn{2}{c}{Young Red Dragon} & \multicolumn{4}{c}{Dragon Steak Tartare} & Elementally Fortifying Feast \\
\midrule
\multicolumn{2}{c}{Wyvern} & \multicolumn{4}{c}{Purple Poison Curry} & Meat Feast \\
\midrule
\end{tabularx}

\chapter*{Appendix C}
\addcontentsline{toc}{chapter}{Appendix C}

\section*{Appendix C: Variant Rules}

\section*{Old School}

Experience-based crafting

\begin{minipage}{0.48\textwidth}
In the olden days crafting checks took experience as a fundamentally component. This is not a feature of this sytem, but the following is a variant for those that want to run a hateful and archaic system that will cause great suffering. In this system, certain components can be (or must, as per your GM), be replaced with experience, using the following ratios:
\end{minipage}\hfill
\begin{minipage}{0.48\textwidth}
\begin{tabularx}{\textwidth}\toprule
{}XX}
\midrule
\multicolumn{2}{c}{Material} & Experience Cost \\
\midrule
\multicolumn{2}{c}{Common Reagent} & 15 \\
\midrule
\multicolumn{2}{c}{Common Essence} & 45 \\
\midrule
\multicolumn{2}{c}{Uncommon Reagent} & 65 \\
\midrule
\multicolumn{2}{c}{Uncommon Essence} & 195 \\
\midrule
\multicolumn{2}{c}{Rare Reagent} & 320 \\
\midrule
\multicolumn{2}{c}{Rare Essence} & 960 \\
\midrule
\multicolumn{2}{c}{Very Rare Reagent} & 4,250 \\
\midrule
\multicolumn{2}{c}{Very Rare Essence} & 12,750 \\
\midrule
\multicolumn{2}{c}{Legendary Reagent} & 17,750 \\
\midrule
\multicolumn{2}{c}{Legendary Essence} & 53,250 \\
\midrule
\end{tabularx}
\end{minipage}

Arcanist Crafting

This harkens back to the tales of the Arcanists of certain ancient empires investing their very life force and power in magic items, and somewhat models the system as presented in early editions. I don't necessarily recommend it for 5e, and this isn't how I run crafting, but I wouldn't to present the option here. This will make magic items more of an investment, but easier to access.

\section*{Assistance}

Group-based crafting and minions

When being assisted by a skilled craftsman (who has proficiency in the related tool and skills of the crafting branch), you gain advantage on the crafting roll.

If one roll succeeds, the check passes and the crafting continues as normal. If both of your rolls succeed, it counts as twice as much progress. If both of the rolls would be a failure, it counts as two failures and no progress is made. Too many cooks in a kitchen can be dangerous! More than one helper when crafting this method doesn't have additional benefits.

If you are using the "Take 10" approach during downtime crafting, you can pick the most skilled crafter's modifier to the crafting roll to use. Each additional person assisting during downtime crafting, the period crafting time per check is reduced by 1 hour to a minimum of 1 hour (with three helpers). People can only qualify as helpers if they have proficiency in the tool being used for the crafting project.

\section*{Actual Blacksmithing}

Actual blacksmiths have opinions

Since posting this system, I've heard from plenty of actual blacksmiths, letting me know blacksmithing is hard. This system is a model that balances game mechanics, fun, practicality, and realism in equal parts, but if you want a system that will make them happier, double all blacksmithing times, and triple armor crafting time per check. This means that making 1 check would for weapons or items would be 4 hours, and making 1 check for armor would take 6 hours.

I would recommend only using this variant when downtime is plentiful, or when combined with the "Assistance" variant rules. When combined with the assistance variant rules, the maximum number of helpers remains 3 (each reducing the time to make a check by 1 hour still).

\section*{More Common Magic}

Not just for enchanters

\begin{minipage}{0.48\textwidth}
While magical items is generally thought of as the domain of enchanters, a skilled crafter with the correct knowledge may be able to craft certain magic items with this variant rule. If you have proficiency in Arcana, you can use your crafting skill to craft items from certain enchanting tables as per the table below. When making items this way, you can only make items of types you can normally make.

If another creature proficient in arcana assists for the full duration of the craft, they can serve as a replacement for proficiency in arcana.
\end{minipage}\hfill
\begin{minipage}{0.48\textwidth}
\begin{tabularx}{\textwidth}\toprule
{}XX}
\midrule
Crafting Branch & \multicolumn{2}{c}{Enchanting Table} \\
\midrule
Blacksmithing & \multicolumn{2}{c}{Magical Armor (Metal), Magical Weapons} \\
\midrule
Leatherworking & \multicolumn{2}{c}{Magical Armor (Leather)} \\
\midrule
Jewelering & \multicolumn{2}{c}{Magical Rings, Magical Jewelry} \\
\midrule
Woodworking & \multicolumn{2}{c}{Magical Weapons (Bows)} \\
\midrule
\end{tabularx}
\end{minipage}

\section*{Crafting Skill}

Artisanal skills that aren’t tied to how good at slaying monsters you happen to be

This is an alternative to using proficiency. This was theoriginal plan, but ultimately proved unnecessarily complicated for general use, but some people prefer the sense of progression and realism from having a crafting skill that isn't tied to your combat prowess.

\subsection*{Progression and Progress}

\begin{minipage}{0.48\textwidth}
Each branch of crafting will have a different way to gain skill in that field, typically involving a variety of options for gaining each level of a skill. These are not intended to be something gained easily or quickly, and scale on a quite exponential scale. Achieving skill 3 or 4 is fairly easy for adventurers, while skill 5 and 6 would be what is achieved at the end game, and level 7 is possible, but out of reach for most adventurers. Here is a simple method of progression:
\end{minipage}\hfill
\begin{minipage}{0.48\textwidth}
\begin{tabularx}{\textwidth}\toprule
{}XXXX}
\midrule
Skill & \multicolumn{4}{c}{Gold Pieces of Items Created} \\
\midrule
1 & \multicolumn{4}{c}{10 gp} \\
\midrule
2 & \multicolumn{4}{c}{100 gp} \\
\midrule
3 & \multicolumn{4}{c}{1,000 gp} \\
\midrule
4 & \multicolumn{4}{c}{10,000 gp} \\
\midrule
5 & \multicolumn{4}{c}{100,000 gp} \\
\midrule
6 & \multicolumn{4}{c}{1,000,000 gp} \\
\midrule
7 & \multicolumn{4}{c}{10,000,000 gp} \\
\midrule
\end{tabularx}
\end{minipage}

If you combine with the innovation system, you can award double credit for any gold pieces of materials spent attempting to innovate a new item when tracking skill progression.

\section*{Innovation System}

Making it so you have to actually know how to make the things you make

Adding back in some complexity Almost all enchanting recipes use scrolls as blueprints for the magic of the item and essences to power it. It is possible that other methods exist, the scroll is fundamentally a blueprint of the magic the item uses, as well part of its magical essence that is imbued into it. A GM can opt to replace the scroll with a blueprint or innovation check (see variant rules) and additional magical essences (to replace both the knowledge and power provided by the scroll).

\subsection*{Variant: Recipes}

Materials are just stuff without a recipe. Coming in the form of techniques, blueprints, or any one of a hundred different forms of knowledge, the an essential step of making anything is know how.

A recipe alone doesn't inherently grant success-a recipe is just a path that the craftsman can walk, but final product will come down to their skills, materials, and a little luck. Even so, recipes are not created equal. The technique of swordcraft left behind by a grand master of the craft can contain knowledge that will inherently boost the skills of anyone following its techniques.

Like materials, recipes can come from three sources.

\begin{minipage}{0.48\textwidth}
Found: The world you adventure in is often vast and dotted with the legacies of those that have come before. Frequently ancient techniques and secrets can be uncovered during your adventurers, hoarded by dragons (...or maybe kobolds earlier in your adventures!).

Purchased: As with most things, money can bridge many gaps and provide many answers. Either convincing a craftsman to teach you their technique or buying a potion formula from the alchemists guild, most people in the world will understand that they stand to more to gain by selling fish than by teaching their customers how to fish, so these will won't come cheap, but can often by attained by establishing good relations... or just dropping a lot of coin.
\end{minipage}\hfill
\begin{minipage}{0.48\textwidth}
How Much Do Recipes Cost?

How much a recipe costs, or even if it's available, has a lot of factors behind it and is ultimately up to the GM, but in general be fairly expensive (to encourage innovation) but not outrageous compared to what they make. Roughly 10 times the cost of the materials to make the item.
\end{minipage}

Invented: While many craftsmen and craftswoman tread in the footsteps of others, those at the cutting edge are those that innovate and invent, stepping beyond what is known (to them). More difficult, a recipe is created through trial and effort, and will frequently leave many broken prototypes as proof of the effort.

To invent a recipe, you make an innovation check. This takes 1 hour, and once attempted cannot be attempted again until you finish a long rest. You can select materials to expend on the innovation check.

\begin{itemize}
  \item If you roll half the innovation difficult or more and did not have the correct materials, you learn the materials needed for the recipe (this doesn't require any materials to be expended).
  \item If you roll half the innovation difficult or more of the recipe and had the correct ingredients, you learn the innovation difficulty of the recipe and materials needed for the recipe.
  \item If you roll the innovation difficult with all the required ingredients spent toward the check, you learn the recipe, and this counts as the first successful crafting roll toward crafting the product of that recipe.
  \item On failure, all materials put toward the check are lost. On success, the recipe is learned, and the materials can be rolled over toward crafting the item; the first crafting check for the item automatically succeeds on that crafting attempt.
\end{itemize}

The GM can set the innovation DC of an item, or just have it default to the DC of an innovation check is the Crafting DC of the item +5.

An innovation roll is as follows:

Recipe Innovation Roll = d20 + your relevant crafting Skill + your Wisdom or Intelligence modifier (your choice)

However a player has acquired their recipes, they are encouraged to record their recipes in a book or manual.

Losing Your Recipes:

Recipes are usually going to be recorded as physical documents, and consequently must be safeguarded. If a set of Recipes are lost, it will be up to the GM to determine how many they might remember by heart.

The following is my recommendation: half their recipes + the intelligence modifier are known by heart. Recipes remembered are chosen in the order of Invented Recipes \\\&gt; Most Crafted Recipe \\\&gt; Most Recently Crafted Recipe, and then ordered by which have been used the most to the least.

Certain feats or abilities may negate any recipes lost.

\section*{Gritty Realism}

For games that move at a more deliberate pace

The normal crafting time listed in this book is balanced around a progression of minimal downtime with 8 hour long rests. For games that follow the model of 8 hour short rests and week long long rests offering more downtime, you can still use this system, but may want to consider longer crafting checks.

To accomplish this, make each crafting check take 8 hours (four times longer). This maps to one work day per crafting check. In addition:

\begin{itemize}
  \item Checks do not have to be subsequent for any crafting branch (including alchemy).
  \item Taking 10 for checks takes two consecutive workdays.
  \item Gathering checks require one week when gathering from environment.
\end{itemize}

\chapter*{Appendix E}
\addcontentsline{toc}{chapter}{Appendix E}

\section*{Appendix E: Exotic Materials}

\section*{Exotic Ingredients \& Potions}

While standard potions are made from curative, reactive, or poisonous ingredients, exotic ingredients have specialized effects. When making a potion from these ingredients, the potions effect is a combination of the effect of the exotic ingredients added.

An Exotic Potion (potion brewed entirely from exotic ingredients) doesn't need a recipe and has a crafting time of 1 hour, and a difficulty of the difficulty of all the exotic ingredients used added together, with 1 check needed per exotic ingredient added.

An exotic ingredients can be combined with a standard potion by adding the DC of the standard potion to the combined difficulty of the exotic ingredients. This can result in very powerful potions, but will frequently result in unattainable high difficulty to make it work, as adding random new components to potions typically wrecks the effect.

\section*{Exotic Effects}

\begin{minipage}{0.48\textwidth}
\subsubsection*{Apple of Arborea}

Legendary, Exotic, Difficulty +6

Consuming this apple has the effect of greater restoration cast upon the person that consumes it. If the creature that consumes it is Good aligned, they gain the benefit of death ward until they finish a long rest.

Adding it to a potion makes that potion confer the effects of
eating it, but has no alignment restrictions.

\subsubsection*{Catfern}

Common, Exotic, Difficult +1

A light and airy fern that tends to get easily caught in the wind and slightly glows.

When added to any potion you consume, you gain 30 feet of darkvision for the duration of the potion effect. If you already have darkvision, the range of your darkvision increases by 30 feet for the duration of the effect.

When added to a Potion of Climbing, it also grants you a climbing speed equal to your walking speed in addition to its normal effects.

\subsubsection*{Dragongrass}

Common, Exotic, Difficulty +2

This is a strange grass that burns very hot and tastes terrible.

When added to a Potion of Fire Breath, it allows you to replace one or more breaths with breathing fire in a cone with the effect of the spell burning hands.

When added to a Custom Potion that would deal damage to a target area, it allows you to instead drink the potion and breath of a 15-foot cone of the damaging effect the potion would have had
\end{minipage}\hfill
\begin{minipage}{0.48\textwidth}
\subsubsection*{Basilisk Eye}

Common, Exotic, Difficulty +3

At first glance, it looks like a stone.

When this and 1 common divine essence is added to any healing potion, that healing potion also removes the Petrified condition when used.

\subsubsection*{Gargoyle's Heart}

Common, Exotic, Difficulty +3

A gem like heart that forms inside gargoyles that have been animated for a certain number of years.

When you add this to a potion, any creature that consumes the potion develops are tough rock-like skin. Their AC can't be less than 16, regardless of what kind of armor they are wearing, and they become immune to critical strikes. These effects fade when the effects of the potion fade, or last 1 hour if the potion would otherwise not have duration.

\subsubsection*{Mimic Heart}

Common, Exotic, Difficulty +2

This strange ever shifting fleshy organ has potent shifting
properties that can make the following Exotic Potions.

When added to a Potion of Climbing, it turns it into a Potion
of Alter Self, granting the effect of the spell alter self for 1
hour (no concentration required).

When added to a Potion of Growth along with at least one
other rare reactive ingredient, it becomes a Potion of
Polymorph granting the effect of the polymorph spell for 10
minutes (no concentration required)
\end{minipage}

\chapter*{Appendix F}
\addcontentsline{toc}{chapter}{Appendix F}

\section*{Appendix F: Camp Actions}

\begin{minipage}{0.48\textwidth}
Camp Actions are things you can do during a long rest that make the most of your time-while adventurers need their beauty sleep (well, most of them), there's always a few spare hours during a Long Rest you can spend in one of the following ways to better prepare yourself for the harrowing times to come.

A long rest is 8 hours long, and most adventurers need 6 hours of sleep. This leaves 2 hours of light activity in which to take a camp action from the following list.
\end{minipage}\hfill
\begin{minipage}{0.48\textwidth}
Elves, Constructs, and More

Some 5e races have unique sleep requirements. They consequently spend less of their time sleeping or sleep in unique ways. They can take the "Take a Watch" action as many times as they have available time for after taking care of whatever resting needs they have, but may take only one other Camp Action while gaining the benefits of a long rest.
\end{minipage}

\section*{Take a Watch}

Adventuring is dangerous, and adventurers often decide to long rest in strange places-sometimes it's best to set a watch.

\section*{Craft}

An adventurer that takes this action can make 2 hours of progress toward Crafting during a long rest. This progress is made at the end of the long rest. In order to take this action, the adventurer must have the related crafting tools on hand.

Requires a campfire, and any Wisdom Perception checks during this time are made with disadvantage.

\section*{Cook}

A special form of the crafting Camp Action that can be taken with cook's utensils. A hearty meal sits better than any trail rations... even when it is cooked from the simplest of ingredients. You and all willing creatures (willing to eat your cooking) regain an additional Hit Die from the long rest when it is finished.

If you have proficiency with cook's utensils, creatures regain additional Hit Dice equal to your proficiency bonus.

Requires a campfire, and any Wisdom (Perception) checks during this time are made with disadvantage

\section*{Prepare}

The life of an adventure has many challenges and it is only natural a cautious adventurer would want to prepare for them. Select one ability score to prepare for the upcoming day and perform 2 hours of an activity that hones it for the challenges ahead (you could prepare Strength or Dexterity through stretches or exercises, Intelligence through studying, Wisdom through meditation, etc).

After you finish the long rest, you gain a Preparation die, starting at as a d6. Each time you make an ability check related to your chosen ability score, roll the Preparation die and add it to the result. The Preparation die decreases by one step each time it is rolled until depleted (d6, d4, d2, depleted) May require a campfire, and any Wisdom (Perception) checks during this time are made with disadvantage.

\section*{Slumber}

Sometimes a hard day of adventuring deserves a little extra shut eye. Taking this Camp Action is more akin to a camp inaction, and you get the full recommended 8 hours of sleep. During this deep slumber, automatically fail Wisdom (Perception) checks and your passive Perception is 0, however you reduce any levels of Exhaustion by 2 and awake with 1 inspiration.

Requires a campfire, and any Wisdom (Perception) checks during this time are made with disadvantage.

\section*{Task}

Sometimes you will have a task that requires your time, but doesn't fit into the above options. For example, copying spells to your spell book as a wizard. When engaging in such a task, you can replace your Camp Action with making 2 hours of progress toward that task.

May require a campfire, and any Wisdom (Perception) checks during this time are made with disadvantage.

\chapter*{Appendix M}
\addcontentsline{toc}{chapter}{Appendix M}

\section*{Appendix M: Crafting Magic \& Items}

\section*{Crafting Magic Spells}

Crafting magic is a unique kind of utilitarian magic. Almost always ritual based, it is not intended for use in combat, and comes with some special rules. In most games, crafting magic spells will not count against spells known, but can only be learned by being taught or purchased, perhaps known from a background.

These spells exist for GMs to award to players to make it easier to use crafting-they are not required, but help integrate crafting into adventuring, and fulfill a narrative niche. Though these spells are arcane spells that are primarily the domain of Wizards and Inventors, a GM can assign them to any player they would make sense for (for example, perhaps a Cleric who follows a god of craft, creation, or forging).

\begin{minipage}{0.48\textwidth}
\subsubsection*{Forge Fire}

1st-level transmutation (ritual)

Classes: Inventor, Wizard 
Casting Time: 1 minute 
Range: 15 feet 
Components: V, S 
Duration: Concentration, up to 8 hours

A Medium-sized fire within range, such as a campfire, burns with unnatural heat within its embers for the duration. It gives off intense heat, dealing twice as much damage to any creature that takes damage from it and consumes materials twice as fast as normal. For the duration, it can serve as a forge to smelt nonmagical metals, and counts as a forge for the purposes of blacksmithing.

\begin{itemize}
  \item Spell: Forge Fire
\end{itemize}

\subsubsection*{Process Hide}

1st-level transmutation (ritual)

Classes: Inventor, Wizard 
Casting Time: 1 minute 
Range: Touch 
Components: V, S, M (a hide which is consumed by the spell) 
Duration: Concentration, up to 1 hour

You work magic into a hide, turning it into processed leather. If you concentrate for the duration, you can turn a single hide into a rawhide leather, boiled leather, or tanned leather. Alternatively, you can turn hide scraps into leather scraps.

\begin{itemize}
  \item Spell: Process Hide
\end{itemize}

\subsubsection*{Change Material}

2nd-level transmutation (ritual)

Classes: Inventor, Wizard 
Casting Time: 2 hours 
Range: Touch 
Components: V, S, M (a common material used in the casting of the spell, which is consumed by the spell upon turning into the new material) 
Duration: Instantaneous

You convert one common material into another material of equal or lesser mass and value.

At Higher Levels. When you cast this spell using a 4th- or 5th-level spell slot, you can convert a uncommon material the same way. When you cast this spell using a 6th- or 7th-level spell slot, you can convert a rare material the same way. When you cast this spell using an 8th- or 9th-level spell slot, you can convert a very rare material the same way.

\begin{itemize}
  \item Spell: Change Material
\end{itemize}

\subsubsection*{Conjure Tool}

2nd-level conjuration (ritual)

Classes: Inventor, Wizard 
Casting Time: 10 minutes 
Range: 5 feet 
Components: V, S 
Duration: 4 hours

You conjure an artisan tool of your choice. It is magical in nature, but serves as a nonmagical version of that artisan tool for the duration.

\begin{itemize}
  \item Spell: Conjure Tool
\end{itemize}

\subsubsection*{Dimensional Toolbox}

3rd-level conjuration (ritual)

Classes: Inventor, Wizard 
Casting Time: 10 minutes 
Range: 5 feet 
Components: V, S, M (a key worth 50 gp) 
Duration: 4 hours

You conjure an opening to a dimensional toolbox. This is stocked with every artisan tool and up to 10 gp worth of common materials. As an action, you can reach in and take out a tool or material contained within. Any tool or material that is not used when the spell ends fades away (materials used in crafting recipes remain).

At Higher Levels. When you cast this spell using a spell slot of 5th level or higher, the toolbox contains +1 tools (tools that add +1 to the crafting check made). When you cast this spell using a spell slot of 9th level, the toolbox contains +2 tools (tools that add +2 to the crafting check made).

\begin{itemize}
  \item Spell: Dimensional Toolbox
\end{itemize}
\end{minipage}\hfill
\begin{minipage}{0.48\textwidth}
\subsubsection*{Unseen Crafter}

5th-level conjuration (ritual)

Classes: Inventor, Wizard 
Casting Time: 1 minute 
Range: 60 feet 
Components: V, S, M (a bit of string and wood) 
Duration: 8 hours

This spell creates an invisible, mindless, shapeless, Medium force that performs simple tasks at your command until the spell ends. The crafter springs into existence in an unoccupied space on the ground within range. It has AC 10, 1 hit point, and a Strength of 2, and it can't attack. If it drops to 0 hit points, the spell ends.

Once on each of your turns as a bonus action, you can mentally command the crafter to move up to 15 feet and interact with an object. The crafter can perform simple tasks that a human servant could do, such as fetching things, cleaning, mending, folding clothes, lighting fires, serving food, and pouring wine. Additionally, it can perform skilled crafting checks. It always uses the taking 10 rule for crafting checks, completing one check every 4 hours. Its crafting modifier is +5. When you summon it, you can summon it with one artisan tool of your choice, which only it can use. It can use any other artisan tools otherwise available. If you cast this spell again while the crafter still exists, the previous one vanishes.

Once you give the command, the servant performs the task to the best of its ability until it completes the task, then waits for your next command. If you command the servant to perform a task that would move it more than 60 feet away from you, the spell ends.

At Higher Levels. When you cast this spell using a spell slot of 6th level or higher, its crafting modifier increases by 1.

\begin{itemize}
  \item Spell: Unseen Crafter
\end{itemize}

\subsubsection*{Dimensional Workshop}

6th-level conjuration

Classes: Inventor, Wizard 
Casting Time: 10 minutes 
Range: 5 feet 
Components: V, S 
Duration: 4 hours

You conjure an extradimensional dwelling in range that lasts for the duration. You choose where its one entrance is located. The entrance shimmers faintly and is 5 feet wide and 10 feet tall. You and any creature you designate when you cast the spell can enter the extradimensional dwelling as long as the portal remains open. You can open or close the portal if you are within 30 feet of it. While closed, the portal is invisible.

Beyond the portal is a magnificent foyer with numerous chambers beyond. The atmosphere is clean, fresh, and warm.

You can create any floor plan you like, but the space can't exceed 20 cubes, each cube being 10 feet on each side. The place is furnished with every artisan tool and typical crafting accessory you could need: anvils, furnaces, looms, etc. Furnishings and other objects created by this spell dissipate into smoke if removed from the workshop. When the spell ends, any creatures inside the extradimensional space are expelled into the open spaces nearest to the entrance.

At Higher Levels. When you cast this spell using a spell slot of 7th level or higher, the dwelling lasts for 24 hours.

\begin{itemize}
  \item Spell: Dimensional Workshop
\end{itemize}
\end{minipage}

\section*{Crafting Utility Magic Items}

\begin{minipage}{0.48\textwidth}
\subsubsection*{Blacksmith's Bellows}

Wonderous item, common

These are a small set of portable bellows. Can be used to cast
 forge fire K without expending a spell slot. Once used, it can't be used again until the next dawn.

\begin{itemize}
  \item Item: Blacksmith's Bellows
\end{itemize}

\subsubsection*{Tanning Bag}

Wondrous item, common

A magical satchel. If you put an unprocessed hide into this bag, and speak one of three magic words. Depending on the word, the hide will be processed into rawhide leather, boiled leather, or tanned leather. This process takes 4 hours, and you can process one piece of hide at a time.

\begin{itemize}
  \item Item: Tanning Bag
\end{itemize}

\subsubsection*{Boundless Toolbelt}

Wondrous item, uncommon

While wearing this toolbelt, you can cast conjure tool(B) as an action, drawing forth the tool from your toolbelt.

\begin{itemize}
  \item Item: Boundless Toolbelt
\end{itemize}

\subsubsection*{Potion Fridgebox}

Wondrous item, uncommon

This resembles a small wooden box that can fit a container carrying up to a gallon of liquid in it. You can place unfinished potions, brews, or similar items in it, pausing the crafting time on it, allowing you to resume crafting it the item at a later time.

You can also store perishable liquids in this fridge, and they will not spoil.

\begin{itemize}
  \item Item: Potion Fridgebox
\end{itemize}
\end{minipage}\hfill
\begin{minipage}{0.48\textwidth}
\subsubsection*{Lucky Material Pouch}

Wondrous item, uncommon

A small magical bag that always seems to have the material you need at the bottom of it. As an action, you can reach out and draw out any crafting material worth 1 gp or less. After each time you do so, roll a d4. On a 1, the bag is empty and no more materials can be retrieved until the next dawn.

\begin{itemize}
  \item Item: Lucky Material Pouch
\end{itemize}

\subsubsection*{Ring of Labor}

Wondrous item, very rare

A simple copper band. You can use this ring to cast unseen crafter(B). Once used, it can't be used again until the next dawn.

\begin{itemize}
  \item Item: Ring of Labor
\end{itemize}

\subsubsection*{Workshop Key}

Wondrous item, very rare (requires attunement)

While attuned to this key, you can can use it to cast dimensional workshop(B) without expending a spell slot. Once you use it, you cannot use it again until dawn the next day.

\begin{itemize}
  \item Item: Workshop Key
\end{itemize}
\end{minipage}

\end{document}